\newcommand{\opal}{\textsc{OPAL}}
\newcommand{\opalcycl}{\textsc{OPAL-cycl}}
\newcommand{\opalmap}{\textsc{OPAL-map}}
\newcommand{\opalt}{\textsc{OPAL-t}}
\newcommand{\opalenv}{\textsc{OPAL-envelope}}
\newcommand{\ippl}{\textsc{$IP^{2}L$}}
\newcommand{\mcfour}{\textsc{$MC^{4}$}}

\newcommand{\mad}{\textsc{mad}}
\newcommand{\madnine}{\textsc{mad9}}
\newcommand{\madninep}{\textsc{mad9p}}
\newcommand{\madeight}{\textsc{mad8}}

\newcommand{\classic}{\textsc{classic}}
\newcommand{\hfifehut}{\textsc{H5hut}}
\newcommand{\hfifepart}{\textsc{H5Part}}
\newcommand{\hfifeblock}{\textsc{H5Block}}
\newcommand{\hfifefe}{\textsc{H5FED}}

\newcommand{\hipa}{\textsc{PSI-HIPA}}
\newcommand{\oursolver}{\textsc{SAAMG-PCG}}

\renewcommand{\epsilon}{\varepsilon} 
\renewcommand{\vec}[1]{{\bf #1}} 
\newcommand{\dt}[1]{\frac{\partial #1}{\partial t}}
\newcommand{\dtt}[1]{\frac{\partial^2 #1}{\partial t^2}}
\newcommand{\dtvec}[1]{\frac{\partial {\mathbf #1}}{\partial t}}
\newcommand{\dttvec}[1]{\frac{\partial^2 {\mathbf #1}}{\partial t^2}}
\newcommand{\rot}{\vec{\nabla} \wedge }
\renewcommand{\div}{\vec{\nabla} \cdot }

\def\vec#1{\mathbf{#1}}
\def\vecg#1{\boldsymbol{#1}}
\def\norm#1{\| #1 \|} 
\def\tr{^{\!\top}}

\def\curl{{\bf curl}\,}
\def\curlp{{\rm curl}_p\,}
\def\div{{\rm div}\,}
\def\grad{\nabla}
\def\gradp{\nabla_p}
\def\dotp#1#2{\langle#1,#2\rangle}
\def\eps{\varepsilon}

\newcommand{\mat}[1]{\ensuremath{\boldsymbol{#1}}}
\newcommand{\vect}[1]{\ensuremath{\mathbf{#1}}}
\newcommand{\iprod}[2]{\ensuremath{\langle#1,#2\rangle}}
\newcommand{\abs}[1]{\ensuremath{|#1|}}

\newcommand{\Nedelec}{N\'{e}d\'{e}lec}

\newcommand{\id}[1]{\structure{#1}}

\newcommand {\Co}{{\mathbb{C}}}
\newcommand {\Int}{{\mathbb{Z}}}
\newcommand {\Nat}{{\mathbb{N}}}
%
%
\newcommand {\Hcurl}{{H(\mathbf{curl};\Omega)}}
\newcommand {\Hocurl}{{H_0(\mathbf{curl};\Omega)}}
\newcommand {\Hdiv}{{H(\mathrm{div};\Omega)}}
\newcommand {\Hodiv}{{H_0(\mathbf{div};\Omega)}}
%
\renewcommand {\Re}{{\rm I \kern-2pt R}}
\newcommand{\vc}[1]{\mbox{\boldmath $#1$}}
\newcommand {\RM}[1]{\mathrm{#1}}


\newcommand{\bs}[1]{\mathbf #1}
\newcommand{\rms}[1]
{\overset{\sim}{#1}}

%--------------
% Draft helpers
%--------------

\usepackage{suffix}

% A simple colored box inlined with the text
%  #1: color to use
%  #2: text to put
\newcommand{\INLINEBOX}[2]{%
   \begin{center}%
    \fcolorbox{#1!60!black}{#1}{%
      \addtolength{\linewidth}{-0.6cm}%  fixed value, works for normal article text
      \begin{minipage}{\linewidth} #2 \end{minipage}%
    }%
   \end{center}\vspace{1pt}%
}

% A box at the margin containing the given text
\newcommand{\MARGINBOX}[1]{%
  \mbox{}%
  \marginpar%
   [\tiny\raggedleft\hspace{0pt}#1]%
   {\tiny\raggedright\hspace{0pt}#1}%
}

% mark specific elements: starred versions use inline boxes
\newcommand{\TODO}[2][]{\MARGINBOX{\textcolor{red!80!black}{\emph{ToDo (#1):}} #2}}
\WithSuffix\newcommand\TODO*[2][]{\INLINEBOX{red!20!white}{\emph{ToDo (#1):} #2}}

\newcommand{\FIXME}[2][]{\MARGINBOX{\textcolor{blue!80!black}{\emph{FixMe (#1):}} #2}}
\WithSuffix\newcommand\FIXME*[2][]{\INLINEBOX{blue!20!white}{\emph{FixMe (#1):} #2}}

\newcommand{\NOTE}[2][]{\MARGINBOX{\textcolor{green!80!black}{\emph{Note (#1):}} #2}}
\WithSuffix\newcommand\NOTE*[2][]{\INLINEBOX{green!20!white}{\emph{Note (#1):} #2}}


\graphicspath{
{figures/figures-phys/},
{figures/figures-arch/},
{figures/figures-cycl/},
{figures/figures-cycl/Inj2/},
{figures/figures-general/},
{figures/figures-general-accel/},
{figures/figures-hpc/},
{figures/figures-logo/},
{figures/figures-mad9p/},
{figures/figures-num/},
{figures/figures-slides/},
{figures/figures-xfel/},
{figures/figuresjang/}
}

%------------------
% OLD STUFF
%------------------

%\setbeamertemplate{footline}
%{
%	\leavevmode%
%	\hbox{%
%	\begin{beamercolorbox}[wd=.14\paperwidth,ht=2.25ex,dp=1ex,center]{author in head/foot}%
%		\usebeamerfont{author in head/foot}\insertshortauthor
%	\end{beamercolorbox}%
%	\begin{beamercolorbox}[wd=.48\paperwidth,ht=2.25ex,dp=1ex,center]{title	in head/foot}%
%		\usebeamerfont{title in head/foot}\insertshorttitle
%	\end{beamercolorbox}%
%	\begin{beamercolorbox}[wd=.23\paperwidth,ht=2.25ex,dp=1ex,right]{date in head/foot}%
%		\usebeamerfont{date in head/foot}\insertshortdate{}\hspace*{2em}
%		\insertframenumber{} /
%		\inserttotalframenumber\hspace*{2ex} 
%	\end{beamercolorbox}}%
%	\vskip0pt%
%}

\newcommand{\rnb}{radially neighboring bunches }
\newcommand{\nbe}{neighboring bunch effects }
\newcommand{\sce}{space charge effects }
\newcommand{\scf}{space charge force }






