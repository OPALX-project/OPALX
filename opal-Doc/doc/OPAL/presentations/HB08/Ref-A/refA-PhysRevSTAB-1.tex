\documentclass{article}
 \usepackage{latexsym}
  \setlength{\parskip}{6.0mm}
 \setlength{\parindent}{0pt}

 \begin{document}
Villigen \& Beijing \today \\

Respected Referee

we have carefully answered all you question and implemented many
of your suggestions. Our response is clearly stated, below each of your raised points.

We however disagree on your initial statement: "The single major concern of this referee is whether or not the subject 
matter of this paper is appropriate for a PRSTAB" and citing PRST-AB's mission: 
  
 {\bf Mission and description of the journal}: {\em Physical Review Special Topics - Accelerators and Beams (PRST-AB) aims to publish papers that contain advances in 
the science and {\bf technology} of accelerators and beams. }
 
General in science it is accepted that computing and especially high performance computing is the third leg of science. In particle accelerator science in particular this fact
is reflected in many technical publications and programs for example the Scientific Discovery through Advanced Computing (SciDAC) with a strong section on accelerator science. If we look in the archives of 
PRST-AB we find publications with a similar scope than ours. Please take note of two recently published papers in PRST-AB with a similar scope than ours: 
{\em Three-dimensional quasistatic model for high brightness beam dynamics simulation Phys. Rev. ST Accel. Beams 9, 044204 (2006)} and 
{ \em Parallelization of a beam dynamics code and first large scale radio frequency quadrupole simulations Phys. Rev. ST Accel. Beams 10, 014201 (2007)} .

With the best regards

Andreas Adelmann \& Jianjun Yang
  
\pagebreak  

 REFEREE A - ZC10034 \\


 \underline {General assessment of paper: }
 
 This paper presents a new code for simulating intense cyclotron beams, 
 including space charge effects between radially neighboring bunches. The 
 referee feels that this is a unique tool and will be of value to the 
 cyclotron community. To the best of the referee's knowledge, the work 
 presented in the paper is accurate. 
 
 The single major concern of this referee is whether or not the subject 
 matter of this paper is appropriate for a PRSTAB. The bulk of the paper is 
 dedicated to the computational method, which introduces a new implementation 
 of standard PIC tools. The balance of the paper is focused on benchmarking 
 the code and some applications. However, this latter portion, which 
 represents the physics in the paper, is less than 25{\%} of the total paper 
 length. In fact, the most interesting and the really unique portion of this 
 work -- the neighboring bunch effects -- represents only 1 page ($\sim 
 $5{\%}) of the total 20 page paper. 
 
 This referee feels that this paper would make a good contribution to PRSTAB 
 if the physics and applications sections of the paper could be ``bulked 
 up''. A comparison with experimental data would certainly justify it's 
 publication, but if this is not possible than perhaps a few more simulations 
 of neighboring bunch effects for different machines or configurations would 
 be appropriate. The referee encourages the authors to explore and present 
 more of the physics that their model exposes, esp. for neighboring bunch 
 effects, so that the paper is clearly a physics paper and not a 
 computational paper. 
 
 If the authors do not feel that this can be accomplished, then the referee 
 would suggest that the paper be submitted to an alternate journal which is 
 more accepting of computational papers, such as Nuclear Instr. and Methods. 
 
 Other specific concerns about the paper are given below. 
 
 \underline {Specific topical comments:}
 
 1. It is stated in Introduction that for PSI upgrade, ``when the beam 
 current increases from 2mA to 3mA the mutual space charge effects between 
 radially neighboring bunches{\ldots}need to be seriously considered. 
 However, later on in the paper at the beginning of Section III on page 8, 
 the authors state, ``{\ldots}$\backslash $DeltaR stays sufficiently 
 large{\ldots}neighboring effects are negligible''. This seems to invalidate 
 the statement made in the Introduction stating that neighboring effects 
 should be considered. The point needs clarification as it is confusing for 
 the reader. 
 
  \vspace{+2mm}
 {\it {\bf Our response}  We clarify:  the Injector II has separated turns (discussed in Section III on page 8) and
 hence neighbouring turn simulations are not of utmost importance, while in the case of 
 the Ring cyclotron (discussed in the Introduction)  with overlapping tuns such simulations are needed in order to 
 refine the physical model. In detail:
 
Introduction:  For high intensity cyclotrons, {\bf such as PSI 590\,MeV Ring cyclotron}, with small turn separation .... 

Beginning of Section III on page 8: Generally, in cyclotrons, $\Delta R$ reduces gradually  with increasing beam energy.
For machines like the PSI Injector\,II, $\Delta R$ stays sufficiently large from injection to extraction, and in such cases,  neighboring bunch effects are negligible. 
For others, {\bf  like the PSI 590\,MeV Ring cyclotron under consideration in this section}, $\Delta R$ decreases strongly during the course of acceleration resulting in the need to consider neighboring bunch effects in order to obtain a precise description of the beam dynamics.
 \vspace{+2mm}
 
 2. Two methods for obtaining magnetic field data are presented in Section 
 II, but it is unclear which of these methods is used for the simulations 
 presented later in the paper. 
 
 \vspace{+2mm}
 {\it {\bf Our response} We added: For all calculations in this paper, we use the method by Gordon and Taivassalo.
 \vspace{+2mm}
 
 
 3. The parameter M presented in Section III needs more discussion. It is 
 unclear to the reader what a reasonable choice of M would be and why one 
 choice would be more reasonable than another. On page 11 it is stated that 
 convergence is reached for N{\_}b=9 and M=4.5. However, this leads to the 
 question of whether another combination of N{\_}b and M might also produce 
 convergence, and if so, would it be equally valid and would the result of 
 the simulation be the same? Does the value of a reasonable M depend on 
 N{\_}b or is it independent? A more physical explanation for the choice of M 
 should be included. 

 \vspace{+2mm}
 {\it {\bf Our response} 
 We added a 2D model to estimate the empirical parameters $N_b$ and $M$.
 However we have to point out, that  $N_b$ and $M$ must be chosen
   in order to balance computation accuracy and computation expenses (time).
   In our model, when the turn separation is less than the 1/M times of beam's rms size, 
   we start to inject new bunches, namely, start to take the effects of neighboring bunch into account. 
   This means the effects before this time point is neglected. 
   So with $M$ we define the turn at which we {\bf start} to include the neighboring bunche effects. 
   On the other hand, $N_b$ represent the total bunches we inject.
   Theoretically speaking, the more bunches we use, the more self-consiscent result we obtain on the of computation. 
   Seen from above analysis, these two parameters are independent. Both of them can affect the simulation result significantly.
 } 
 \vspace{+2mm}

 
 4. The discussion of energy binning in section III needs to be placed within 
 the context of the computational algorithm listed in section II, either by 
 moving the discussion to this section, or by addressing the point 
 specifically of how the energy binning and spatial binning are occurring 
 simultaneously. Specifically, it is somewhat confusing to the reader for how 
 particles are being binned by energy, which may allow them to cross spatial 
 bunches (or not? Another confusing point), but are still giving the accurate 
 spatial density grid. The referee is confident that the authors have thought 
 this through and have come up with an accurate procedure, but the method is 
 not clearly reflected in the text. 
 
 \vspace{+2mm}
 {\it {\bf Our response} 
 
 We place the section on energy binning as sugested and add more explanation w.r.t. the 
 procedure of the energy binning.
 }
 
 
 
 
 
 
 \underline {Grammar and phraseology: }
 
 \textit{Abstract: }
 
 1. ``Space charge being the most significant of collective effect{\ldots}'' 
 Should be changed 2. to ``Space charge being one of the most significant of 
 collective effects{\ldots}'', since for some machines other collective 
 effects such as wakefields and impedances may be more of a concern than 
 direct space charge. 
 

\vspace{+2mm}
 {\it {\bf Our response} We changed the wording according to the suggestion.
\vspace{+2mm}

 3. ``{\ldots} in the light{\ldots}.'' Should be changed to ``{\ldots} in the 
 context{\ldots} `` 
 
 \vspace{+2mm}
 {\it {\bf Our response} We changed the wording according to the suggestion.
\vspace{+2mm}
 
 
 \textit{Introduction: }
 
 1. ``3th'' should be changed to ``3$^{rd}$''
 
  \vspace{+2mm}
 {\it {\bf Our response} We changed the wording according to the suggestion.
\vspace{+2mm}

 \textit{Section II:}
 
 1. ``{\ldots} median plane date{\ldots}'' should be changed to 
 ``{\ldots}median plane data{\ldots}''
 
   \vspace{+2mm}
 {\it {\bf Our response} We changed the typo.
\vspace{+2mm}
 
 The writing in this paper is good in general, but there are some small typos 
 and grammatical errors throughout the paper that should be addressed. 
 
 \end{document}
 