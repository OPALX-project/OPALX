\documentclass[10pt]{report}
 \setlength{\parskip}{6.0mm}
 \setlength{\parindent}{0pt}
 \begin{document}
 Villigen \& Beijing \today \\
 
 Respected Referee
 
 we have carefully answered all you question and implemented many
 of your suggestions. Our response is clearly stated, below each of your raised points.
 
 We however disagree on your initial statement: "This is much more of a technical computing note than a note on accelerator or
 beam physics" and citing PRST-AB's  mission: 
 
 {\bf Mission and description of the journal}:
{\em Physical Review Special Topics - Accelerators and Beams (PRST-AB) aims to publish papers that contain advances in 
the science and {\bf technology} of accelerators and beams. }
 
General in science it is accepted that computing and especially high performance computing is the third leg of science. In particle accelerator science in particular this fact
is reflected in many technical publications and programs for example the Scientific Discovery through Advanced Computing (SciDAC) with a strong section on accelerator science. If we look in the archives of 
PRST-AB we find publications with a similar scope than ours. Please take note of two recently published papers in PRST-AB with a similar scope than ours: 
{\em Three-dimensional quasistatic model for high brightness beam dynamics simulation Phys. Rev. ST Accel. Beams 9, 044204 (2006)} and 
{ \em Parallelization of a beam dynamics code and first large scale radio frequency quadrupole simulations Phys. Rev. ST Accel. Beams 10, 014201 (2007)} .

With the best regards
  
Andreas Adelmann \& Jianjun Yang
  
\pagebreak  
REFEREE B - ZC10034
 
 This is much more of a technical computing note than a note on accelerator or
 beam physics. An admittedly very complex and sophisticated computer code has
 been built and seems to be producing new and exciting results for the case of
 strong space charge in cyclotrons. But the results are not explained
 physically, or if they are, the explanations are not convincing, perhaps even
 incorrect. The authors may wish to re-consider submitting this paper unless
 more physics is included.
 
 Some major comments:
 
 \begin{enumerate}
  \item P.\ 2: \textit{``..., however in [5] it is shown that
 under some special condition even a zero current beam is showing such a
 behavior...''} The authors seem to misunderstand the claims of reference 5.
 Certainly, if there is no space charge, there is no tendency towards round beam
 bunches. (Ref. 5, by the way, incorrectly gives Catania as the conference
 venue.)
 
 \vspace{+2mm}
 {\it {\bf Our response}  We removed the zero current claim and have fixed Ref. 5.}
 \vspace{+2mm}

 
  \item P.\ 3, bottom: \textit{``.... when the
 beam current increases from 2 mA to 3 mA, the mutual space charge effects
 between radially neighboring bunches will have more impact on the beam
 dynamics...''} It is implied that the neighboring turn effect is more important
 in the upgraded machine. This does not seem correct since before upgrade, turns
 are closer together. The two effects -- higher current but larger turn
 separation -- would seem to cancel.
 
 
 \vspace{+2mm}
 {\it {\bf Our response}  We highlight the fact that in case of the PSI 590\,MeV Ring cyclotron
 the turns {\em after the resonator upgrade} are not fully separated as one can see in FIG 1. The not fully separated turns, together with the
 intensity upgrade are the motivating factors for the presented attempt to study/understad neighbouring bunch effects. }
 \vspace{+2mm}
 
%{\it {\bf jjyang} After upgrade, the turn separation can be reduced by 25\%, and the beam intensity will increased by 50\%. In addition, the dependence of this two factors 
%are quite different. Therefore the cancellation of the two effects can only be partial.}
  
\vspace{+2mm} 

  \item The condition equation 11 (p.9) is that the radius gain be less than
 something 
 written as the 
 product of two user-chosen parameters. Is this a typo? These parameters are not
 defined. It seems tp me that $r_{\rm rms}$ is meant to be the bunch width and
 only $M$ is user-chosen, but that is not what is stated.
 
  \vspace{+2mm}
 {\it {\bf Our response} Yes, this is a typo and is fixed now. $M$ is a user-chosen parameter, $r_{rms}$ is the rms beam size calcuated after each intergration step.   
 }
 \vspace{+2mm}

 \item P.9: \textit{...The underlying assumption for this is that all bunches
 have the same phase space distribution when they reach a certain position.}
 This cannot be true: the first bunch that was injected, having no neighbor at
 higher energy, behaves differently than later ones that do. The statement
 applies only to the steady state. I find this section describing the
 multi-bunch algorithm extremely confusing. For example, why is the bunch that
 has been tracked for one turn any more worthy of being stored than the initial
 beam (which presumably is already stored anyway)? Does something irreversible
 happen to the bunch over one turn? Then why is one turn sufficient?
 
  \vspace{+2mm}
 {\it {\bf Our response} We clarify: the model in this paper is to study the neighbouring bunch effects under the steady state condition.
   This means that the  {\bf i-1}th bunch beheaves exactly the same as the {\bf i}th bunch with the delay of revolution time. 
   There is one sentence just after the sentence cited by referee: 
   {\bf ...This is realistic and reasonable when the machine is running in a steady state...}

   As we mentioned on P. 10,  the reason why we only use a small amount of bunches rather than all, is explained by two facts: 
   {\bf a) ...The scale of particle number and the dimensions of the grid are beyond the capability of today's supercomputer resources...}
   { \bf b) The distance of the bunches in azimuthal direction is much larger than in radial and hence does not influence the results }

   The phase space data is stored only when the Eq.11 is fulfilled.
   We did not state that the second bunch will surely be injected after the first turn. 
   If Eq.11 is not fulfilled after the first turn, it don't store data. It will run for the second turn, and then check whether Eq.11 is fulfilled or not, if not, it will run for 
   the third trun ...., until Eq.11 is fulfilled, the phase space is stored in order to inject a new bunch from this data after one revolution time.
 }
 \vspace{+2mm}
 
\item P.10: Algorithm 2 is to me no help at all; I find it totally
 incomprehensible. 
 What do the \# signs mean? Symbols like $T_r\rightarrow R_1,r_{\rm rms}$: what
 do they mean? What does it mean that $R_0=R_1$? Etc. It also seems to assume
 some familiarity with computer coding terms \textit{procedure, while,} etc. I
 think it is a good idea to describe the algorithm systematically, but
 introducing new symbols and words is not helpful. Algorithm 1 on the other hand
 is very readable.
 
 \vspace{+2mm}
 {\it {\bf Our response} Algorithm 2 is changed into a more easy-to-understand way.
 }
 \vspace{+2mm}

  \item Equation 12: The sentence before this says that $\overline{\gamma}_i$ is
 the 
 average $\gamma$ for bunch $i$. The sentence after it states \textit{Then every
 particle is grouped into the energy bin whose $\overline{\gamma}_i$ is closest
 to its $\gamma$.}, implying that $i$ labels bins, not bunches. So which is it?
 Or is there one bin per bunch? If this is the case, it needs stating. If not,
 then I am far from understanding the binning technique.

 
 \vspace{+2mm}
 {\it {\bf Our response}  Yes, there is one bin per bunch, this is written in a more clear form 
 in the paper. }
 \vspace{+2mm}

 \item P.19: In the last paragraph, \textit{``...particles with a radius smaller
 than the center particle, the outward force is stronger than the inward force,
 so they will move outwards;...''}. In an isochronous machine, radial space
 charge forces result in azimuthal motion and vice versa. Yet here is stated
 that radial forces result in radial motion. Why?
 
 \vspace{+2mm}
 {\it {\bf Our response} As we stated on P.2 {\bf ...Secondly, there is a strong radial-longitudinal coupling which is influenced by non-linear radial and
longitudinal space charge forces...}, we understand radial space charge forces not only result in azimuthal motion but also result in radial motion, 
this is so called  ``vortex motion'' by Mr. Gordon in Ref[10]. Therefore when the outward force is stronger than the inward force the particle will move 
in both radial and azimuthal direction. However we want to emphasise that the phenomenon we observed should be attributed to radial motion rather than azimuthal motion.   
 }
 \vspace{+2mm} 

 \end{enumerate}

 There are also some rather more trivial comments:
 
 \begin{enumerate}
  \item P.\ 2: \textit{``...is developing a spatial
 stationary spherical beam distribution.''} The stationary distribution is not
 spherical except in the very special case where the vertical tune is as large
 as the radial. In general, since vertical tune is smaller than radial, the
 distribution in 3D is cigar-shaped.


 \vspace{+2mm}
 {\it {\bf Our response}  
 We write this more general by stating that the stationary distribution is 3D ellipsoidal.}
 \vspace{+2mm}

 
  \item P.\ 3, bottom: \textit{``.... The turn separation is consequently
 increased, but 
 remains at the same order of magnitude 
 as the radial bunch size, as shown in Fig. 1.''} but the radial bunch size is 
 not plotted on Fig.\ 1. As well, this statement begs the question Why increase
 the turn spacing if the radial bunch sizes get larger anyway? I know the answer
 is that the radial sizes become larger because the current is higher, but this
 is nowhere stated. 


 \vspace{+2mm}
 {\it {\bf Our response}  fixed following referee's hints.
 We clarified this in the text.}
 \vspace{+2mm}

 
  \item I find the notation confusing since all vectors are bolded except
 $\beta$, so I
 continually forget that it is a vector. Should use boldmath as in {\boldmath
 $\beta$}.
 
 \vspace{+2mm}
 {\it {\bf Our response} fixed.
 }
 \vspace{+2mm}
 

  \item Page 5 states that the magnet gap is large, while on page 7 it states
 that 
 cyclotron magnet gaps are narrow.


 \vspace{+2mm}
 {\it {\bf Our response}  
 Our gap is not narrow, we made that more clear in the text.}
 \vspace{+2mm}
 
  \item The paragraph just after equation 12 explains very well some details and
 
 motivates the calculation, I suggest placing it at the start of section III,
 rather than at the end.


 \vspace{+2mm}
 {\it {\bf Our response}  
 We have changed slightly the order of presentation and would like to stick with the location of the mentioned paragraph.}
 \vspace{+2mm}
 
  \item There is an interesting systematic difference between OPAL and FIXPO
 indicated in 
 Figure 4. It is small, but still deserves a comment.


 \vspace{+2mm}
 {\it {\bf Our response} We found the reason of difference was the initial conditions seting for reference orbit were  the same for the two simulations.
with this fixed, the eigen-ellipses agree very well now!
   
 }
 \vspace{+2mm}
  

  \item P.15: \textit{``Thereafter, the charge distribution stays at an
 extremely narrow 
 phase width (about 2$^\circ$)''} implies that the beam continues at a fixed
 phase width. It doesn't, rather it continues at a fixed length so the phase
 width decreases as $1/R$.


 \vspace{+2mm}
 {\it {\bf Our response}  fixed following referee's suggestion
 }
 \vspace{+2mm}
 
 \item P.16: The explanation is unconvincing. There is no reason to suspect the
 different treatment of the vertical motion results in a different
 radial-longitudinal character. An odd thing is that the Sphere model of Ref.28
 gives results that qualitatively look more similar to the OPAL results than the
 Needle model does. See especially turn 5.


 \vspace{+2mm}
 {\it {\bf Our response}   
   We insist that coupling effects between the axial and radial motion can be one of the reason for the difference in radial-longitudinal character.
   In the Needle model,{\bf ... to separate the vertical motion from the horizontal one,...}(P.448 in Ref 28), 
   this means the coupling effects are not considered in the Needle model.  
 
   Because the difference of physical model, there are differences between Sphere model and Needle model, (as shown in  Fig.2 and Fig.3 in Ref 28).
   Similarly we can not expect OPAL-CYCL code can give out exactly the same result as Sphere or Needel model. 
 }
 \vspace{+2mm}

 \item P.17: It is not clear to the reader that there are 4 main resonators and
 only one 3$^{\rm rd}$ harmonic.
 
 \item There is no table summarizing the main parameters of the cyclotrons in
 question, so it is hard for the reader to judge what is meant by a 2$^\circ$
 phase width. The harmonic number $h=6$ is nowhere mentioned.

 \item Nothing is said regarding the details of the acceleration. It is known
 that if the rf voltage increases with $R$, the bunches will shorten. This could
 explain the Fig.8 result of steadily decreasing bunch lengths. But because of
 lack of information, the reader cannot judge.


 \vspace{+2mm}
 {\it {\bf Our response to 9, 10 and 11}  

We add a table summarizing parameter of Injector\,II and the Ring Cyclotron and
add more explanation of the machines at the beginning of section V 

 }
 \vspace{+2mm}
 
 \end{enumerate}
 
 I counted 22 typographical and grammatical errors. This paper needs a good
 editor to correct the English.
 \end{document}