\documentclass[10pt]{report}
\usepackage{graphicx}
 \setlength{\parskip}{6.0mm}
 \setlength{\parindent}{0pt}
 \begin{document}
 Villigen \& Beijing \today \\
 
 Respected Referee
 
 we have carefully answered all you question and implemented many
 of your suggestions. Our response is clearly stated, below each of your raised points.
 
 With the best regards

Andreas Adelmann \& Jianjun Yang

Point 1.1 
{\it p.9: Delta R is first the radial gain per turn. Then it is $R_e-R_s$. Are
these also neighboring turns? I think not, since it states "the single
bunch is tracked for MAYBE several revolution periods". (My emphasis on
MAYBE.) So how to make sense of eq.11? At first I thought it was a test
that the single circulating bunch of distribution $f_0$ (What is $f_0$?
Uniform? Gaussian?) had been tracked with no energy gain to see if it
had stabilized. But now I think it is checking with energy gain to find
the point where the radial gain per turn is less than M times the bunch
transverse rms size. But then I'm bothered by the word "several" instead
of "one". But this interpretation may also be wrong, since according to it, M is
only a condition of closeness parameter, whereas in Fig. 4 it appears to
be a parameter that actually CAUSES the electric field to
change. Perhaps Fig.4 is for bunches separated by $M*r_{rms}$, i.e., it
applies exactly when the condition 11 is fulfilled as if it were an
equality. This is not stated. I am still uncertain.}

{\it {\bf Our response} to Point 1.1: } In order to make the algorithm more clear,
we described the procedure of multi-bunch injection more in detail. 
We hope all the above questions are answered accordingly.

Point 1.2
{\it Furthermore, Delta R varies with azimuth around the cyclotron, so is an
arbitrary azimuth chosen or is it the average? If so, why is it found by
tracking? The average radius gain per turn can be calculated simply
analytically? }

{\it {\bf Our response} to Point 1.2: } In order to decide if a neighboring bunch has to be injected condition (11) is evaluated at the injection azimuth of the cyclotron. 
Of course an average over azimuthal angle could be considered as well but this will add much overhead of execution time.
Considering the imperfection of the field map and the non constant RF voltage along radius in many real cyclotrons, 
it is not easy to obtain the precise value of analytic average radius gain per turn, and this of course need more work before doing simulation.
that is the reason why we calculate everything from the tracked particle distribution.

Point 2 
{\it p.9: I puzzled a long time till I realized that in the sentence before
eqn.11, "A new..." should be "The new...". At least, I think so. I am
assuming that $R_e$ is the average R over the bunch. This is stated
nowhere; the sentence in question mentions only that it is stored, not
from whence it comes. 
}

{\it {\bf Our response} to Point 2: }  We adopt the suggestion of the referee and change {\it {\bf A new} } to {\it {\bf The new }} 
and change {\it {\bf radial position $R_e$}} to {\it {\bf average radial position of the bunch $R_e$}}

Point 3
{\it p.9: The Ansatz that "all bunches have the same phase space distribution
when they reach a certain position" eliminates all coherent
oscillations. Neighboring bunch effects may or may not cause coherent
instability, but this code will not discover such. This could be
mentioned. }

{\it {\bf Our response} to Point 3: } It is correct that no coherent oscillation between the bunches can occur before
we take neighboring bunches into account. We clarify this fact by writing: 
{\it The underlying assumption for this Ansatz is that all bunches have the same phase space distribution when they reach a certain position, 
i.e. when $f_{R_{e}}$ is saved. This is realistic and reasonable when the machine is running in a steady state. 
It needs to be mentioned that up to that position the coherent instabilities which might be caused by neighboring bunch effects, 
are not covered by this model.}

Point 4
{\it Alg.2 still uses computer jargon. The authors don't seem to understand
the meaning of the word "example". In my previous report, I gave one
example of this problem of Alg.2, and they corrected only that one
example. Here is another: Step 6: $R_s=R_e$ cannot be true as this means
$R_s-R_e=0$. Again there is a confusion of memory locations and physical
parameters. }

{\it {\bf Our response} to Point 4: } $R_s=R_e$ here means the value of the variable $R_s$ is set to $R_e$.
In order to avoid further confusion we changed this slightly in Alg. 2.
We insist that it is common and reasonable in a paper to describe a algorithm in a code 
with the help of pseudo-code.

Point 5
{\it Alg.2, step 7: Why is the calculation continually re-started with the
initial distribution $f_0$? Should you not use the tracked distribution?}

{\it {\bf Our response} to Point 5: }
We change the typo $f_0$ to $f_{R_{e}}$ .

Point 6

{\it p.10: Seem to confuse the words "beam" and "bunch". A beam is a steady stream
of particles. A bunch is a localized assembly of particles.}

{\it {\bf Our response} to Point 6: }
We adopt the suggestion of the referee and distinguish "beam" and "bunch" as suggested.


Point 7

{\it p.12, line 3: not {\tt -> }nor 

p.12, line 8: are {\tt ->} is 

p.12, consistently mis-typeset. i should be in math mode but is not, th
as a superscript should NOT be in math mode, but is. Should be 
$i^{\rm th}$. 
}

{\it {\bf Our response} to Point 7: }
We corrected all mistakes in point 7.

Point 8

{\it p.18 and elsewhere: Still using the expression "spherical beam". Firstly,
it is a bunch, not a beam, and secondly, the bunches are not spherical, 
but circular in the median plane.
}

{\it {\bf Our response} to Point 8: } We change beam to bunch use circular instead of spherical.

\end{document}
