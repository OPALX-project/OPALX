                 %% ****** Start of file template.aps ****** %
%%
%%
%%   This file is part of the APS files in the REVTeX 4 distribution.%!TEX encoding = UTF-8 Unicode
%%   Version 4.0 of REVTeX, August 2001
%%
%%
%%   Copyright (c) 2001 The American Physical Society.
%%
%%   See the REVTeX 4 README file for restrictions and more information.
%%
%
% This is a template for producing manuscripts for use with REVTEX 4.0
% Copy this file to another name and then work on that file.
% That way, you always have this original template file to use.
%
% Group addresses by affiliation; use superscriptaddress for long
% author lists, or if there are many overlapping affiliations.
% For Phys. Rev. appearance, change preprint to twocolumn.
% Choose pra, prb, prc, prd, pre, prl, prstab, or rmp for journal
%  Add 'draft' option to mark overfull boxes with black boxes
%  Add 'showpacs' option to make PACS codes appear
%  Add 'showkeys' option to make keywords appear
\documentclass[aps,prstab,twocolumn,superscriptaddress,showpacs]{revtex4}
\usepackage{graphicx}
\usepackage{url}
%\usepackage[center]{caption2}
\usepackage[colorlinks,linkcolor=blue,anchorcolor=blue,citecolor=blue]{hyperref} % hyper reference to contents 
\usepackage{algorithm,algorithmic}


%\documentclass[aps,prl,preprint,superscriptaddress]{revtex4}
%\documentclass[aps,prl,twocolumn,groupedaddress]{revtex4}

% You should use BibTeX and apsrev.bst for references
% Choosing a journal automatically selects the correct APS
% BibTeX style file (bst file), so only uncomment the line
% below if necessary.
\bibliographystyle{apsrev}
\newcommand{\opal}{\textsc{OPAL}}
\newcommand{\opalt}{\textsc{OPAL-t }}
\newcommand{\opalcycl}{\textsc{OPAL-cycl}}
\newcommand{\opalmap}{\textsc{OPAL-map }}
\newcommand{\opalenv}{\textsc{OPAL-envelop}}

\newcommand{\mad}{\textsc{mad }}
\newcommand{\madnine}{\textsc{mad9 }}
\newcommand{\madninep}{\textsc{mad9p }}
\newcommand{\madeight}{\textsc{mad8 }}

\newcommand{\classic}{\textsc{classic }}
\newcommand{\hfifepart}{\textsc{H5Part }}
\newcommand{\hfifefe}{\textsc{H5FED }}

\renewcommand{\epsilon}{\varepsilon} 
\renewcommand{\vec}[1]{{\bf #1}} 
\newcommand{\dt}[1]{\frac{\partial #1}{\partial t}}
\newcommand{\dtt}[1]{\frac{\partial^2 #1}{\partial t^2}}
\newcommand{\dtvec}[1]{\frac{\partial {\mathbf #1}}{\partial t}}
\newcommand{\dttvec}[1]{\frac{\partial^2 {\mathbf #1}}{\partial t^2}}
\newcommand{\rot}{\vec{\nabla} \wedge }
\renewcommand{\div}{\vec{\nabla} \cdot }

\def\vec#1{\mathbf{#1}}
\def\vecg#1{\boldsymbol{#1}}
\def\norm#1{\| #1 \|} 
\def\tr{^{\!\top}}

\def\curl{{\bf curl}\,}
\def\curlp{{\rm curl}_p\,}
\def\div{{\rm div}\,}
\def\grad{\nabla}
\def\gradp{\nabla_p}
\def\dotp#1#2{\langle#1,#2\rangle}
\def\eps{\varepsilon}

\newcommand{\mat}[1]{\ensuremath{\boldsymbol{#1}}}
\newcommand{\vect}[1]{\ensuremath{\mathbf{#1}}}
\newcommand{\iprod}[2]{\ensuremath{\langle#1,#2\rangle}}
\newcommand{\abs}[1]{\ensuremath{|#1|}}

\newcommand{\Nedelec}{N\'{e}d\'{e}lec}

\newcommand{\id}[1]{\structure{#1}}

\newcommand {\Co}{{\mathbb{C}}}
\newcommand {\Int}{{\mathbb{Z}}}
\newcommand {\Nat}{{\mathbb{N}}}
%
%
\newcommand {\Hcurl}{{H(\mathbf{curl};\Omega)}}
\newcommand {\Hocurl}{{H_0(\mathbf{curl};\Omega)}}
\newcommand {\Hdiv}{{H(\mathrm{div};\Omega)}}
\newcommand {\Hodiv}{{H_0(\mathbf{div};\Omega)}}
%
\renewcommand {\Re}{{\rm I \kern-2pt R}}
\newcommand{\vc}[1]{\mbox{\boldmath $#1$}}
\newcommand {\RM}[1]{\mathrm{#1}}


\newcommand{\bs}[1]{\mathbf #1}
\renewcommand{\baselinestretch}{2.0}
\begin{document}

\title{Note on space charge fields of 2D round Gaussian distribution}

\maketitle

For the 2D round beam with Gaussian distribution, the charge density can be expressed by:
\begin{equation}\label{eq:1}
  \rho(r,\theta, z)=\left\{
  \begin{array}{ll}
    \frac{\lambda}{\sqrt{2\pi}\sigma}e^{-\frac{r^2}{2\sigma^2}} & r \le a \\
    0 & r > a
  \end {array}
  \right.
\end{equation}  

Let's consider a cylinder along the axis of the DC beam. Its height is equal to the volicity of beam $v$.
we can get the the total charge in this cylinder by volume integral. 
\begin{equation}\label{eq:11}
  Q(r)  =  \int_V\rho(r,\theta, z)dV \nonumber
\end{equation}  
If $r \le a$
\begin{eqnarray}\label{eq:12}
  Q(r) & = & \int_0^v\int_0^r \frac{\lambda}{\sqrt{2\pi}\sigma}e^{-\frac{r^2}{2\sigma^2}} 2\pi r drdz \nonumber \\ 
  & = & \sqrt{2\pi}\lambda v\sigma\left(1-e^{{-\frac{r^2}{2\sigma^2}}}\right) 
\end{eqnarray}
and if $r > a$
\begin{eqnarray}\label{eq:13}
  Q(r) & = & \int_0^v\int_0^a \frac{\lambda}{\sqrt{2\pi}\sigma}e^{-\frac{r^2}{2\sigma^2}} 2\pi r drdz \nonumber \\ 
  & = & \sqrt{2\pi}\lambda v\sigma\left(1-e^{{-\frac{a^2}{2\sigma^2}}}\right) 
\end{eqnarray}

According to the defination of beam current, the beam current can be expressed as 
\begin{equation}\label{eq:3}
I_0 = Q(a) = \sqrt{2\pi}\lambda v\sigma \left(1-e^{{-\frac{a^2}{2\sigma^2}}}\right)
\end{equation}
Apply Gauss Law of electric field for any closed surface 
\begin{equation}\label{eq:4}
  \oint_A \bs{E}\cdot d\bs{A} = \frac{Q(r)}{\epsilon_0}
\end{equation}
Considering the 2D axisymetric distribution, There is only radial component $E_{sc}(r)$ exists, and on the side face $E_{sc}(r)$ is constant.
Apply Eq.{\ref{eq:4}} onto the cylinder surface, we get
\begin{equation}\label{eq:5}
  E_{sc}(r)\cdot 2\pi r v = \frac{\sqrt{2\pi}\lambda v\sigma}{\epsilon_0}\left(1-e^{{-\frac{r^2}{2\sigma^2}}}\right)
\end{equation}

By combining Eq.{\ref{eq:11} to \ref{eq:5}}, the electric field generated by such a single axisymmetric DC beam with Gaussian charge distribution can be expressed by 
\begin{equation}\label{eq:Esc}
  \bs{E_{\RM{sc}}(r)} = \left\{
    \begin{array}{ll}
      \frac{I_0}{2\pi\varepsilon_0 v r}\frac{1-e^{-\frac{r^2}{2\sigma^2}}}{1-e^{-\frac{a^2}{2\sigma^2}}}\bs{e}_r & r \le a \\
      \frac{I_0}{2\pi\varepsilon_0 v r}\bs{e}_r & r > a
    \end{array}
    \right.
\end{equation}  
where $a$ is truncated radius, $I_0$ beam current. $\bs{e}_r$ unit vector on radial direction. 

Note that relativsitic effects is not included above.

\end{document}

