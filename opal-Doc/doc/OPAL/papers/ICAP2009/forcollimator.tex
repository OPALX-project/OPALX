The physics models in collimator include energy loss and Coulomb scattering. 
The nuclear scattering is not included for particles of hundreds of MeV since it has negligible contribution compared to Coulomb scattering. 
The energy loss is simulated using the Bethe-Bloch equation. 
Comparing the stopping power with the PSTAR program of National Institute of Standards and Technology (NIST),
the error is within 10\% from several MeV to 10 GeV for copper. Especially, the error is within 3\% in the region from 50 MeV to 1 GeV. 
In general, there is energy straggling when a beam passes through the material. 
For relatively thick absorbers such that the number of collisions is large, the energy loss distribution is shown to be Gaussian in form, with the spread $\sigma$ given by \cite{William}. 
The Coulomb scattering is treated as two independent events: the multiple Coulomb scattering and the large angle Rutherford scattering,
 using the distribution given in Classical Electrodynamics \cite{Jackson}.

A 72 MeV cold Gaussian beam with $\sigma_x=\sigma_y=5$ mm meets a copper slit with the half aperture of $3$ mm from 0.01 m to 0.1 m. 
Figure \ref{fig:trace3} shows the trajectory of particles which didn't pass through the collimator directly. 
Most of the partiles in collimator were absorbed with a range of about 7 mm, except a few ones were deflected by the collimator. 
As a benchmarking of the collimator models in OPAL, the energy spectrum and angle deviation is compared with two general-purpose Monte Carlo codes: 
MCNPX \cite{MCNPX} and FLUKA \cite{FLUKA1,FLUKA2} as shown in Fig. \ref{fig:spectandscatter}. 
A ellipse collimator with the half aperture of $3$ mm in both x and y direction is selected. 
The deflected particles contribute to the energy spectrum and angle deviation after a collimator, and these partilces may be lost downstream.



@book{William,
  Address =	 {Berlin Heidelberg New York},
  Author =	 {William R. Leo},
  Edition =	 {2nd},
  Publisher =	 {Springer-Verlag},
  Title =	 {Techniques for nuclear and particle physics experiments},
  Year =	 1994
}
@book{Jackson,
  Address =	 {New York},
  Author =	 {J. D. Jackson},
  Edition =	 {3rd},
  Publisher =	 {John Wiley &. Sons},
  Title =	 {Classical Electrodynamics},
  Year =	 1998
}
@inproceedings{FLUKA1,
  author =	 "G. Battistoni, S. Muraro, P.R. Sala, F. Cerutti, A. Ferrari,
S. Roesler, A. Fasso`, J. Ranft",
  title =	 "The FLUKA code: Description and benchmarking",
  booktitle =	 "Proceedings of the Hadronic Shower Simulation Workshop 2006",
  address =	 "Fermilab",
  year =	 2006,
  note =	 "M. Albrow, R. Raja eds., AIP Conference Proceeding 896, 31-49, (2007)"
}
@techreport{FLUKA2,
  author =	 "A. Fasso`, A. Ferrari, J. Ranft, and P.R. Sala",
  title =	 "FLUKA: a multi-particle transport code",
  type =	 "Tech. Report",
  year =	 2005,
  number =	 "CERN-2005-10 (2005), INFN/TC_05/11, SLAC-R-773"
}
