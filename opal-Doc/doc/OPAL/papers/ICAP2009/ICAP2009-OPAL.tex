\documentclass[a4paper,10pt,3p,preprint,pdftex]{elsarticle}

\usepackage{amsmath}
\usepackage{amsfonts}
\usepackage{graphicx}
\usepackage{algorithm,algorithmic}
\usepackage{url}
\usepackage{color}
\usepackage{tikz}
\usepackage{multirow}


\newcommand {\RM}[1]{\mathrm{#1}}
\renewcommand{\Re}{\mathbb{R}}

\newcommand{\opal}{\textsc{OPAL}}
\newcommand{\ippl}{\textsc{IP$^2$L}}


\graphicspath{{figures/}}

\title{The Object Oriented Parallel Accelerator Library (OPAL), Design, Implementation and Application}

\author[psi]{A.~Adelmann\corref{cor}}
\ead{andreas.adelmann@psi.ch}
\author[psi]{Ch.~Kraus}
\author[psi]{Y.~Ineichen}

\author[lanl]{S.~Russel}
\author[ciae]{J. J. Yang}


\address[psi]{Paul Scherrer Institut, CH-5234 Villigen, Switzerland}
\address[eth]{ETH Z\"urich, Computer Science Department, Universit\"atsstrasse 6, CH-8092 Z\"urich, Switzerland}
\address[lanl]{Los Alamos National Laboratory}
\address[ciae]{China Institute of Atomic Energy, Beijing, 102413, China \&  Department of Engineering Physics, Tsinghua University, Beijing, 100084, China}

\cortext[cor]{\color{red}Corresponding author}
\journal{ICAP2009 San Fransisco}


\begin{document}

\begin{keyword}
  %Parallel HDF5 \sep FastBit \sep Particle Accelerator Science
\end{keyword}

\begin{abstract}
OPAL (Object Oriented Parallel Accelerator Library) is a tool for charged-particle optics in accelerator structures and beam lines including 3D space charge, 
short range wake-fields and 1D coherent synchrotron radiation. Built from first principles as a parallel application, OPAL admits simulations of any scale, 
from the laptop to the largest HPC clusters available today. Simulations, in particular HPC (High Performance Computing) simulations, form the third pillar of science, 
complementing theory and experiment. In this paper we present numerical and HPC capabilities such as fast direct and iterative solvers together with timings up to 
several thousands of processors. The application of OPAL to the PSI-XFEL project as well as to the ongoing high power cyclotron upgrade will demonstrate OPAL's versatile capabilities.
Plans for future developments w.r.t. a 3D finite element time domain Maxwell solver and 3D synchrotron radiation will be discussed.
\end{abstract}

\maketitle

\end{document}
