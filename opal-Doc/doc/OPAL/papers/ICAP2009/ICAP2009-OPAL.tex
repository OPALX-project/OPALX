\documentclass[acus]{JAC2003}

%%
%%  This file was updated in April 2009 by J. Poole to be in line with Word tempaltes
%%
%%  Use \documentclass[boxit]{JAC2003}
%%  to draw a frame with the correct margins on the output.
%%
%%  Use \documentclass[acus]{JAC2003}
%%  for US letter paper layout
%%
\usepackage{graphicx}
\usepackage{booktabs}
\usepackage{url}
\usepackage[colorlinks,linkcolor=blue,anchorcolor=blue,citecolor=blue]{hyperref} % hyper reference to contents 
\usepackage{algorithm,algorithmic}
\usepackage{tikz}
\usepackage{pgflibraryshapes}  
\usepackage{textcomp}
\usepackage{amsmath}
\usepackage{amsfonts}

\newcommand{\opal}{\textsc{OPAL}}
\newcommand{\opalt}{\textsc{OPAL-t }}
\newcommand{\opale}{\textsc{OPAL-e }}
\newcommand{\opalcycl}{\textsc{OPAL-cycl}}
\newcommand{\opalmap}{\textsc{OPAL-map }}
\newcommand{\opalenv}{\textsc{OPAL-envelop}}

\newcommand{\mad}{\textsc{mad }}
\newcommand{\madnine}{\textsc{mad9 }}
\newcommand{\madninep}{\textsc{mad9p }}
\newcommand{\madeight}{\textsc{mad8 }}

\newcommand{\classic}{\textsc{classic }}
\newcommand{\hfifepart}{\textsc{H5Part }}
\newcommand{\hfifefe}{\textsc{H5FED }}

\renewcommand{\epsilon}{\varepsilon} 
\renewcommand{\vec}[1]{{\bf #1}} 
\newcommand{\dt}[1]{\frac{\partial #1}{\partial t}}
\newcommand{\dtt}[1]{\frac{\partial^2 #1}{\partial t^2}}
\newcommand{\dtvec}[1]{\frac{\partial {\mathbf #1}}{\partial t}}
\newcommand{\dttvec}[1]{\frac{\partial^2 {\mathbf #1}}{\partial t^2}}
\newcommand{\rot}{\vec{\nabla} \wedge }
\renewcommand{\div}{\vec{\nabla} \cdot }

\def\vec#1{\mathbf{#1}}
\def\vecg#1{\boldsymbol{#1}}
\def\norm#1{\| #1 \|} 
\def\tr{^{\!\top}}

\def\curl{{\bf curl}\,}
\def\curlp{{\rm curl}_p\,}
\def\div{{\rm div}\,}
\def\grad{\nabla}
\def\gradp{\nabla_p}
\def\dotp#1#2{\langle#1,#2\rangle}
\def\eps{\varepsilon}

\newcommand{\mat}[1]{\ensuremath{\boldsymbol{#1}}}
\newcommand{\vect}[1]{\ensuremath{\mathbf{#1}}}
\newcommand{\iprod}[2]{\ensuremath{\langle#1,#2\rangle}}
\newcommand{\abs}[1]{\ensuremath{|#1|}}

\newcommand{\Nedelec}{N\'{e}d\'{e}lec}

\newcommand{\id}[1]{\structure{#1}}

\newcommand {\Co}{{\mathbb{C}}}
\newcommand {\Int}{{\mathbb{Z}}}
\newcommand {\Nat}{{\mathbb{N}}}
%
%
\newcommand {\Hcurl}{{H(\mathbf{curl};\Omega)}}
\newcommand {\Hocurl}{{H_0(\mathbf{curl};\Omega)}}
\newcommand {\Hdiv}{{H(\mathrm{div};\Omega)}}
\newcommand {\Hodiv}{{H_0(\mathbf{div};\Omega)}}
%
\renewcommand {\Re}{{\rm I \kern-2pt R}}
\newcommand{\vc}[1]{\mbox{\boldmath $#1$}}
\newcommand {\RM}[1]{\mathrm{#1}}



% A simple colored box inlined with the text
%  #1: color to use
%  #2: text to put
\newcommand{\INLINEBOX}[2]{%
   \begin{center}%
    \fcolorbox{#1!60!black}{#1}{%
      \addtolength{\linewidth}{-0.6cm}%  fixed value, works for normal article text
            \begin{minipage}{2\linewidth} #2 \end{minipage}%
  \begin{minipage}{\linewidth} #2 \end{minipage}%
    }%
   \end{center}\vspace{1pt}%
}

% A box at the margin containing the given text
\newcommand{\MARGINBOX}[1]{%
  \mbox{}%
  \marginpar%
   [\tiny\raggedleft\hspace{0pt}#1]%
   {\tiny\raggedright\hspace{0pt}#1}%
}

% mark specific elements: starred versions use inline boxes
\newcommand{\TODO}[2][]{\MARGINBOX{\textcolor{red!80!black}{\emph{ToDo (#1):}} #2}}
\WithSuffix\newcommand\TODO*[2][]{\INLINEBOX{red!20!white}{\emph{ToDo (#1):} #2}}

\newcommand{\FIXME}[2][]{\MARGINBOX{\textcolor{blue!80!black}{\emph{FixMe (#1):}} #2}}
\WithSuffix\newcommand\FIXME*[2][]{\INLINEBOX{blue!20!white}{\emph{FixMe (#1):} #2}}

\newcommand{\NOTE}[2][]{\MARGINBOX{\textcolor{green!80!black}{\emph{Note (#1):}} #2}}
\WithSuffix\newcommand\NOTE*[2][]{\INLINEBOX{green!20!white}{\emph{Note (#1):} #2}}

\newcommand{\DRAFT}[2][]{\MARGINBOX{\textcolor{blue!80!black}{\textsc{Draft (#1):}} #2}}
\WithSuffix\newcommand\DRAFT*[2][]{\INLINEBOX{blue!20!white}{\textsc{Draft (#1):} #2}}

\newcommand{\bs}[1]{\mathbf #1}

%%
%%   VARIABLE HEIGHT FOR THE TITLE BOX (default 35mm)
%%


\setlength{\titleblockheight}{40mm}


\begin{document}
\title{DRAFT --- THE OBJECT ORIENTED PARALLEL ACCELERATOR LIBRARY (OPAL), DESIGN, IMPLEMENTATION AND APPLICATION}

\author{A.~Adelmann\thanks{andreas.adelmann@psi.ch}, Ch.~Kraus, Y. Ineichen, PSI, Villigen Switzerland \\
S.~Russell, LANL, Los Alamos, USA,\\ Y. Bi, J.J~Yang, CIAE, Beijing, China}

\maketitle

\begin{abstract}
   \opal\  (Object Oriented Parallel Accelerator Library) is a tool for charged-particle optic calculations in accelerator structures and beam lines including 3D space charge, 
short range wake-fields and 1D coherent synchrotron radiation and particle matter interaction. Built from first principles as a parallel application, OPAL admits simulations of any scale, 
from the laptop to the largest High Performance Computing (HPC) clusters available today. Simulations, in particular HPC simulations, form the third pillar of science, 
complementing theory and experiment. \opal\ has a fast FFT based direct solver and an iterative solver, able to handle efficiently exact boundary conditions on complex geometries. 
We present timings of \opalt using the FFT based space charge solver with up to several thousands of cores. 
%The application of OPAL to the PSI-XFEL project as well as to the ongoing high power cyclotron upgrade will demonstrate OPAL's versatile capabilities.
%Plans for future developments towards a 3D finite element time domain Maxwell solver for large structures and simulation capabilities for 3D synchrotron radiation will be discussed.
\end{abstract}

\section{OPAL In A NUTSHELL}
\opal\ is a tool for charged-particle optics in
accelerator structures and beam lines. 
Using the \mad language with extensions, \opal\ is derived from \madninep and is based 
on the CLASSIC class library,
which was started in 1995 by an international collaboration.  The Independent Parallel Particle Layer (\ippl) is
the framework which provides parallel particles and fields using data parallel ansatz, together with Trilinos for linear solvers and preconditioners. Parallel input/output is
provided by H5Part/Block a special purpose API on top of HDF5. For some special numerical algorithms we use the Gnu Scientific Library (GSL).


\opal\ is built from the ground up as a parallel application exemplifying the fact that HPC (High Performance Computing) 
is the third leg of science, complementing theory and experiment. 
HPC is now made possible through the increasingly sophisticated mathematical models and evolving computer power available on the desktop
and in super computer centres. \opal\ runs on your laptop as well as on the largest HPC clusters available today.

The state-of-the-art software design philosophy based on design patterns, makes it easy  to add new features into \opal, in the form of new \texttt{C++}~classes.
Figure \ref{fig:opalstr} presents a more detailed view into the complex architecture of \opal. 

\begin{figure}[htb]
\begin{center}
 \begin{tikzpicture}[scale=0.8, transform shape]
    \footnotesize
      \begin{scope}[shape=rectangle,rounded corners,minimum width=3.0cm,minimum height=0.5cm,fill=yellow,text centered]
      
      \draw[rounded corners, draw=green!40, thick, fill=green!25, opacity=0.5, text centered] (-1.55, 1.31) rectangle (8.55,-0.31) node[black, thick, anchor=center, opacity=1., font=\Large] at (3.5, 0.5) {\opal};
      \node[fill= green!40] (0_00) at (0.0,1.0) {MAD-Parser};
      \node[fill= green!40] (0_00) at (3.5,1.0) {Flavors: t,Cycl};
      \node[fill= green!40] (0_00) at (7.0,1.0) {Optimization};
      \node[fill= green!40] (0_00) at (0,0.0)   {Solvers: Direct, Iterative};
      \node[fill= green!40] (0_00) at (3.5,0.0) {Integrators};
      \node[fill= green!40] (0_00) at (7.0,0.0) {Distributions};

       \draw[rounded corners, draw=red!45, thick, fill=red!25, opacity=0.5, text centered] (-1.55, -0.69) rectangle (8.55,-3.81) node[black, thick, anchor=center, opacity=1.0, font=\Large] at (3.5, -3.5) {\ippl};
       \node[fill= red!45] (q_00) at (0,-1) {FFT};
       \node[fill= red!45] (q_01) at (3.5,-1) {D-Operators};
       \node[fill= red!45] (q_02) at (7,-1) {NGP,CIC, TSI};
       \node[fill= red!45] (q_10) at (0,-1.75) {Fields};
       \node[fill= red!45] (q_11) at (3.5,-1.75) {Mesh};
       \node[fill= red!45] (q_12) at (7,-1.75) {Particles};
       \node[fill=red!45] (q_20) at (0,-2.75) {Load Balancing};
       \node[fill=red!45] (q_21) at (3.5,-2.75) {Domain Decomp.};
       \node[fill=red!45] (q_22) at (7,-2.75) {Communication};
       \node[fill=red!45] (q_21) at (0,-3.5) {PETE};
       \node[fill=red!45] (q_22) at (7,-3.5) {Polymorphism};

       \node[rotate=90,minimum width=1.7cm,fill=gray] (bla) at (-1.9,0.49){\textcolor{white} {\classic}};
       \node[rotate=90,minimum width=3.15cm,fill= magenta] (bla) at (-1.9,-2.225){\textcolor{white}{H5Part and H5FED}};
       \node[fill=blue!65,minimum width=10.75cm] (q_23) at (3.25,-4.25) {\textcolor{white}{Trilinos \& GSL}};

      \end{scope}
 \end{tikzpicture}
\caption{The \opal\ software structure}
\label{fig:opalstr}
\end{center}
\end{figure}


OPAL comes in the following flavors:
\begin{itemize}
\item \opalt
\item \opalcycl 
\item \opalmap\ (not yet fully released)
\item \opalenv\ (not yet fully released)
\end{itemize}

\opalt  tracks particles with time as the independent variable and can be used to model beam lines, dc guns, photo guns and complete XFEL's excluding the undulator. 
Collective effects such as space charge (3D solver), coherent synchrotron radiation (1D solver) and longitudinal and transverse wake fields are considered.
When comparing simulation results to measured data, collimators (at the moment without secondary effects) and pepper pot elements are important devices. 
%\opalt is a superset of IMPACT-T \cite{qiang2005} and can be used to model guns, injectors and complete XFEL's excluding the undulator. 
\opalcycl is another flavor which tracks particles with 3D space charge including neighboring turns in cyclotrons, with time as the independent variable. Both flavors can be used in sequence, hence
full start-to-end cyclotron simulations are possible.  
\opalmap tracks particles with 3D space charge using split operator techniques.  \opalenv\ is based on the 3D-envelope equation (\`a la HOMDYN) and can be used to design XFEL's

Documentation and quality assurance are given our highest attention since we are convinced that adequate documentation 
is a key factor in the usefulness of a code like \opal\ to study present and future particle accelerators. 
Using tools such as a source code version control system (subversion), and source code documentation (Doxygen) together with an extensive user manual
we are committed to provide users as well as co-developers with state-of-the-art documentation for \opal. Rigorous quality control is realized by means of daily build and regression tests.

In the sequel we will only discuss features of \opalt based on the current production version 1.1.5.

\section{MODELS}
In recent years, precise beam dynamics simulations in the design of
high-current low-energy hadron machines as well as of 4th generation
light sources have become a very important research topic.  Hadron
machines are characterized by high currents and hence require excellent
control of beam losses and/or keeping the emittance of the beam in narrow ranges.  This is a challenging
problem which requires the accurate modeling of the dynamics of a large
ensemble of macro or real particles subject to complicated external
focusing, accelerating fields and wake fields, particle-matter interaction, as well as the self-fields
caused by Coulomb interaction of the particles.  In general the
geometries of particle accelerators are large and complicated which has
a direct impact on the numerical solution method.

Some of the effects can be studied by using a low dimensional model,
i.e., envelope equations~\cite{sach:68, sach:71, stru-reis:1984,
  gluckstern1}.  These are a set of ordinary differential equations for
the second-order moments of a time-dependent particle distribution.
They can be calculated fast, however the level of detail is mostly not
sufficient for quantitative studies.  Furthermore, a priori knowledge of
critical beam parameters such as the emittance is required with the consequence that the
envelope equations cannot be used as a self-consistent method.

One way to overcome these limitations is by considering the
Vlasov-Poisson description of the phase space, including external fields and
self-fields and, if needed, other effects such as wakes.  To that end
let $f(\mathbf{x},\mathbf{v},t)$ be the density of the particles in the
phase space, i.e., the position-velocity $(\mathbf{x}, \mathbf{v})$
space.  Its evolution is determined by the collisionless \emph{Vlasov
  equation},
%%
% Neglecting
% particles collisions, because typical bunch densities are low, the
% evolution of the beam's distribution function
% $f(\mathbf{x},\mathbf{v},t)$ can be considered as a Cauchy problem.
% Denoting the computational domain $\Omega \subset \Re^d$ with $d=2,3$,
% and using $\mathbf {x} \in \Re^d$ to describe a position in
% configuration and $\mathbf {v} \in \Re^d$ a point in velocity space, the
% collision-less Vlasov-Poisson equation can be expressed by
\begin{equation}\label{eq:Vlasov}
  \frac{df}{dt}=\partial_t f + \mathbf{v} \cdot \nabla_{\mathbf{x}} f
  +\frac{q}{m_0}(\mathbf{E}+ \mathbf{v}\times\mathbf{B})\cdot
  \nabla_{\mathbf{v}} f  =  0,
\end{equation}
where $m_0$, $q$ denote particle mass and charge, respectively.  The
electric and magnetic fields $\mathbf{E}$ and $\mathbf{B}$ are
superpositions of external fields and self-fields (space charge),
%%and other collective effects such as wake fields,
\begin{equation}\label{eq:allfield}
%%  \begin{aligned}
    \mathbf{E} =
    \mathbf{E_{\RM{ext}}} + \mathbf{E_{\RM{self}}}  + \mathbf{E_{\RM{wake}}}, \quad
    \mathbf{B} =
    \mathbf{B_{\RM{ext}}} + \mathbf{B_{\RM{self}}}.
%%  \end{aligned}
\end{equation}
%%
If $\mathbf{E}$ and $\mathbf{B}$ are known, then each particle can be
propagated according to the equation of motion for charged particles in an
electromagnetic field,
\begin{equation*}\label{eq:motion}
  \frac{d\mathbf{x}(t)}{dt}  = \mathbf{v},
  \quad
  \frac{d\mathbf{v}(t)}{dt}  = \frac{q}{m_0}\left(\mathbf{E} +
    \mathbf{v}\times \mathbf{B}\right).
\end{equation*}
%%which is numerically integrated for each particle.  
%%
% We assume for the rest of the paper that the external fields and the
% wake fields are known and furthermore treat the physical system
% electrostatic (which we obtain without loss of generality by an
% appropriate Lorentz transformation).

After the movement of the particles 
$\mathbf{E_{\RM{self}}}$ and $\mathbf{B_{\RM{self}}}$ have to be updated.  
%%
To that end we change the coordinate system into one moving with the
particles.  By means of the appropriate \emph{Lorentz
  transformation}~\cite{lali:84} we arrive at a (quasi-) static
approximation of the system in which the transformed magnetic field
becomes negligible, $\hat{\mathbf{B}}\! \approx\! \mathbf{0}$.  The
transformed electric field is obtained from
\begin{equation}\label{eq:e-field}
  \hat{\mathbf{E}}=\hat{\mathbf{E}}_{\RM{self}}=-\nabla\hat{\phi},
\end{equation}
where the electrostatic potential $\hat{\phi}$ is the solution of the
\emph{Poisson problem}
\begin{equation}\label{eq:poisson0}
  - \Delta \hat{\phi}(\mathbf{x}) =
  \frac{\hat{\rho}(\mathbf{x})}{\varepsilon_0},
\end{equation}
equipped with appropriate boundary conditions. Here, $\hat{\rho}$ denotes the spatial charge
density and $\varepsilon_0$ is the dielectric constant.
% in the (beam rest) frame
% If we can make an appropriate Lorentz transformation to arrive at a
% (quasi-)static approximation of the system, the space charge fields can
% be obtained by solving a Poisson problem with appropriate boundary
% conditions:
% \begin{equation}\label{eq:Poisson}
%   - \nabla^{2} \phi(\mathbf{x}) = \frac{\rho(\mathbf{x})}{\varepsilon_0},
% \end{equation}
% where $\phi$ and $\rho$ are the electrostatic potential and the spatial
% charge density in the appropriate (beam rest) frame.  The electric field
% can then be calculated by
% \begin{equation}\label{eq:Efield}
%   \mathbf{E}=\mathbf{E_{\RM{self}}}=-\nabla\phi.
% \end{equation}
By means of the inverse Lorentz transformation the electric field
$\hat{\mathbf{E}}$ can then be transformed back to yield both the
electric and the magnetic fields in~\eqref{eq:allfield}.
% corresponding to the static grid.  

In \opal\ the discretized Poisson equation is either solved by a combination of a Green function and FFT or by a conjugate gradient algorithm, preconditioned
with algebraic multi-grid using smoothed aggregation (SA-AMG PCG). This 3D solver has the unique capability to include the exact
boundary geometry. %\cite{Hockney:1} 
%The Poisson problem~\eqref{eq:poisson0} discretized by finite
%differences can efficiently be solved on a rectangular grid by a
%Particle-In-Cell (PIC) approach~\cite{qiry:01}.  
The right hand side in~\eqref{eq:poisson0} is discretized by sampling the particles at the
grid points.  In~\eqref{eq:e-field}, $\hat{\phi}$ is interpolated at the
particle positions from its values at the grid points. We also note that
the FFT-based Poisson solvers and similar
approaches~\cite{qiry:01,qigl:04} are usually restricted to box-shaped or open domains in order to obtain
good performance.


%\subsection{Time Integration}

\subsection{Field Solver}
\subsubsection{A Direct FFT Based Poisson Solver}
In our implementation of the PIC method, firstly a rectangular 3D grid containing all particles is constructed.  Subsequently, the charges 
are interpolated onto the grid points. Then the discretized Poisson equation is solved on the grid to obtain the scalar field at the grid points. 
The electric field is calculated on the grid and interpolated back on to the positions of the particles .


%At each time step, in order to solve for the space charge fields, the frames $\bs{S}_{\RM{local}}$ and $\bs{S}_{\RM{beam}}$ are redefined according to current 6D 
%phase space distribution, and all particles are transformed from $\bs{S}_{\RM{lab}}$ to $\bs{S}_{\RM{local}}$.
%Then a Lorentz transformation is performed to transform all particles to $\bs{S}_{\RM{beam}}$.
%The Poisson equation is then solved in the frame $\bs{S}_{\RM{beam}}$. 
In 3D Cartesian coordinates, the solution of the Poisson equation at point $\bs{x}$ can be expressed by 
\begin{equation}\label{eq:Poten}
  \phi(\bs{x})= \frac{1}{4\pi\varepsilon_0}\int{G(\bs{x},\bs{x}')\rho(\bs{x},\bs{x}')d\bs{x}'}
\end{equation}
with $G$ the 3D Green function 
\begin{equation}\label{eq:Green}
  G(\bs{x},\bs{x}')= \frac{1}{\sqrt{(\bs{x}-\bs{x}')^2}}
\end{equation}
assuming open boundary conditions.
The typical steps of calculating space charge fields using Hockney's FFT algorithm is sketched in Algorithm \ref{alg1:sc3d},
where the quantities with superscript $D$ (discrete) refer to grid quantities.

\begin{algorithm}
  \caption{3D Space Charge Calculation} 
  \label{alg1:sc3d}
  \begin{algorithmic}[1]
    \STATE \textbf{procedure} 3DSpaceCharge(In: $\rho$, $G$, Out: $\bs{E_{sc}}$,$\bs{B_{sc}}$)
       \STATE Create 3D rectangular grid which contains all particles, % and doubled it in each dimension, 
       \STATE Interpolate the charge $q$ of each macro-particle to nearby mesh points to obtain $\rho^D$, 
       \STATE Lorentz transformation to obtain $\rho^D$ in the beam rest frame $\bs{S}_{\RM{beam}}$,
       \STATE FFT $\rho^D$ and $G^D$ to obtain $\widehat{\rho}^D$ and $\widehat{G}^D$,
       \STATE Determine $\widehat{\phi}^D$ on the grid using $\widehat{\phi}^D = \widehat{\rho}^D \cdot \widehat{G}^D$,
       \STATE Use FFT$^{-1}$ of $\widehat{\phi }^D$ to obtain $\phi^D$,
       \STATE Compute $\bs{E}^D= -\nabla \phi^D$,
       \STATE Interpolate $\bs{E}$ at the particle positions $\bs{x}$ from $\bs{E}^D$,
       \STATE Perform Lorentz back transform to obtain $\bs{E_{\RM{sc}}}$ and $\bs{B_{\RM{sc}}}$ in  frame $\bs{S}_{\RM{local}}$ and transform back  to $\bs{S}_{\RM{lab}}$.
       \STATE \textbf{end procedure}
  \end{algorithmic}
\end{algorithm}

The image charge of a beam near a cathode is not negligible, hence
open boundary conditions are not justified in such a case. To find the space-charge forces on the beam from
the image charge by the standard Green function method,
we need to solve the Poisson equation with a computational
domain containing both the image charge and the beam.
We are using a shifted-Green function \cite{Qiang:2006p95} technique in order to efficiently compute the correct potential at the cathode.
With this technique, the FFT is used to calculate the cyclic convolution
and the previous algorithm can be used to calculate the
potential in the shifted field domain.

At  emission from a dc gun, or when calculating neighboring turns in a cyclotron, the electrostatic approximation is not valid anymore. To overcome this problem
we divide the beam into $n$ energy bins. The space charge solver uses now $n$ separate Lorentz transformations. 

To show the parallel performance of \opalt  we consider two problems, the first one has $5 \cdot 10^6$ particles on a $64 \times 64\times128$ mesh and 200 time steps are considered.
The used CPU time as a function of cores is shown in Figure \ref{fig:parperfsmall}. 
We obtain in the order of $5 \cdot 10^6$ particle pushes  per second on a 16 nodes (HP BL460c blades) cluster, each node having a dual-socket quad core Intel Xeon E5450 $3.0$ GHz with 16GB ECC RAM.
For a high-bandwidth low-latency communication the InfiniBand interconnect is used.
\begin{figure}[htb]
   \centering
  \includegraphics*[scale=0.25]{drift1}
  \vspace{-5mm}
   \caption{CPU time of a production run showing the scaling of the most important parts of \opalt on a 128 core HP Cluster.}
   \label{fig:parperfsmall}
\end{figure}

The second problem consists of $1 \cdot 10^{8}$ particles on a $512^3$ mesh and timings for 3 integration steps are shown in Figure \ref{fig:parperfbig}. 
The timings where obtained on the Cray XT5 cluster of the Swiss Supercomputing Center (CSCS) in Manno.
Each of the 1844 compute nodes consists of 2 quad-core AMD Opteron 2.4 GHz Shanghai processors giving 8 cores in total per node with 16 GBytes of memory. 
The high-speed network based on a SeaStar 2.2 communications processor which is able to provide 2 GBytes/s of injection bandwidth for the node, with a theoretical 
peak of 9.6 GBytes/s of bandwidth in each direction for the through-flow of packets out on the network.
\begin{figure}[htb]
   \centering
  \includegraphics*[scale=0.25]{drift2b}
  \vspace{-5mm}
   \caption{CPU time of a test run showing the scaling of the most important parts of \opalt on a Cray XT5.}
   \label{fig:parperfbig}
\end{figure}

\subsubsection{A Fast Iterative Parallel Poisson Solver on Irregular Domains} 
 The problem is discretized by finite differences.  Depending on the treatment of the
  Dirichlet boundary the resulting system of equations is symmetric or
  `mildly' nonsymmetric positive definite.  In all cases, the system is
  solved by the preconditioned conjugate gradient algorithm with
  smoothed aggregation (SA) based algebraic multigrid (AMG)
  preconditioning.  Additionally we investigated variants of the implementation of
  SA-AMG that lead to considerable improvements in the execution times.
  We demonstrate good scalability of the solver on distributed memory
  parallel processor with up to 2048 processors in \cite{Adelmann:2009p543}. In this paper we also compare our
  \oursolver\ solver with the FFT-based solver described in the preceding paragraph.
  
\subsection{Particle Matter Interaction}
The physics models describing particle matter interaction includes energy loss and Coulomb scattering. 
The nuclear scattering is not yet included for particles in the order of  hundreds of MeVs. Their contribution 
is  negligible compared to Coulomb scattering. The energy loss model is based on the Bethe-Bloch equation. 
Comparing the stopping power with the PSTAR program of National Institute of Standards and Technology (NIST), we find
errors in the order of 10\% for copper, from several MeV to 10 GeV. Important for our immediate application at PSI, the error is within 3\% in the region from 50 MeV to 1 GeV. 
In general, there is energy straggling when a beam passes through the material. 
For relatively thick absorbers such that the number of collisions is large, the energy loss distribution is Gaussian \cite{William}. 
The Coulomb scattering is treated as two independent events: the multiple Coulomb scattering and the large angle Rutherford scattering, 
using the distribution given in \cite{Jackson}.

\subsubsection{Validation}
A 72 MeV cold Gaussian beam with $\sigma_x=\sigma_y=5$ mm is send through a copper slit with the half aperture of $3$ mm from 0.01 to 0.1 m. 
Figure \ref{fig:trace3} shows some trajectories of particles which are either absorbed or deflected by the collimator. 
\begin{figure}[htb]
   \centering
  \includegraphics*[width=90mm]{trace3}
   \caption{Trajectories of particles which are either absorbed or deflected by the collimator.}
   \label{fig:trace3}
\end{figure}
Most of the particles  were absorbed within a range of about 7.4 mm, except for a few which were deflected by the collimator. 
As a benchmark of the elliptic collimator models in OPAL, the energy spectrum and angle deviation is compared against two general-purpose Monte Carlo codes, 
MCNPX \cite{MCNPX} and FLUKA \cite{FLUKA1,FLUKA2}, as shown in Fig. \ref{fig:spectandscatter}. 
%A ellipse collimator with the half aperture of $3$ mm in both x and y direction is selected. 
The deflected particles contribute to the energy spectrum and angle deviation after a collimator. These particles may be lost downstream.
\begin{figure}[htb]
   \centering
  \includegraphics*[width=90mm]{spectandscatter}
   \caption{Energy spectrum and angle deviation (small plot)}
   \label{fig:spectandscatter}
\end{figure}





%\section{EXAMPLES}

%\subsection{Cyclotron including Neighboring Turns}

%\subsection{Electron Source Modeling}


\section{ACKNOWLEDGMENTS}
The majority of computations have been performed on the Cray XT5 in the
framework of the PSI CSCS ``Horizon'' collaboration.  We acknowledge the
help of the XT5 support team at CSCS, most notably Timothy Stitt and D. Kiselev 
for the MCNPX simulations and fruitful discussions w.r.t. the particle matter interaction models.

\bibliography{paper}
\bibliographystyle{unsrt}
%\bibliographystyle{elsarticle-num}

%\begin{thebibliography}{9}   % Use for  1-9  references
%\begin{thebibliography}{99} % Use for 10-99 references

%\bibitem{accelconf-ref}
%C. Petit-Jean-Genaz and J. Poole, ``JACoW, A service to the Accelerator Community'',
%EPAC'04, Lucerne, July 2004, THZCH03,  p.~249, \texttt{http://www.JACoW.org}.

%\bibitem{jacow-help} A. Name and D. Person, Phys. Rev. Lett. 25 (1997) 56.

%\bibitem{exampl-ref}
%A.N. Other, ``A Very Interesting Paper'', EPAC'96, Sitges, June 1996, MOPCH31, p. 7984 (1996),
%\texttt{http://www.JACoW.org}.


%\end{thebibliography}

\end{document}
