\documentclass[acus]{JAC2003}

%%
%%  This file was updated in April 2009 by J. Poole to be in line with Word tempaltes
%%
%%  Use \documentclass[boxit]{JAC2003}
%%  to draw a frame with the correct margins on the output.
%%
%%  Use \documentclass[acus]{JAC2003}
%%  for US letter paper layout
%%

\usepackage{graphicx}
\usepackage{booktabs}
\usepackage{url}
\usepackage[colorlinks,linkcolor=blue,anchorcolor=blue,citecolor=blue]{hyperref} % hyper reference to contents 
\usepackage{algorithm,algorithmic}
\usepackage{tikz}
\usepackage{pgflibraryshapes}  

\newcommand{\opal}{\textsc{OPAL}}
\newcommand{\opalt}{\textsc{OPAL-t }}
\newcommand{\opale}{\textsc{OPAL-e }}
\newcommand{\opalcycl}{\textsc{OPAL-cycl}}
\newcommand{\opalmap}{\textsc{OPAL-map }}
\newcommand{\opalenv}{\textsc{OPAL-envelop}}

\newcommand{\mad}{\textsc{mad }}
\newcommand{\madnine}{\textsc{mad9 }}
\newcommand{\madninep}{\textsc{mad9p }}
\newcommand{\madeight}{\textsc{mad8 }}

\newcommand{\classic}{\textsc{classic }}
\newcommand{\hfifepart}{\textsc{H5Part }}
\newcommand{\hfifefe}{\textsc{H5FED }}

\renewcommand{\epsilon}{\varepsilon} 
\renewcommand{\vec}[1]{{\bf #1}} 
\newcommand{\dt}[1]{\frac{\partial #1}{\partial t}}
\newcommand{\dtt}[1]{\frac{\partial^2 #1}{\partial t^2}}
\newcommand{\dtvec}[1]{\frac{\partial {\mathbf #1}}{\partial t}}
\newcommand{\dttvec}[1]{\frac{\partial^2 {\mathbf #1}}{\partial t^2}}
\newcommand{\rot}{\vec{\nabla} \wedge }
\renewcommand{\div}{\vec{\nabla} \cdot }

\def\vec#1{\mathbf{#1}}
\def\vecg#1{\boldsymbol{#1}}
\def\norm#1{\| #1 \|} 
\def\tr{^{\!\top}}

\def\curl{{\bf curl}\,}
\def\curlp{{\rm curl}_p\,}
\def\div{{\rm div}\,}
\def\grad{\nabla}
\def\gradp{\nabla_p}
\def\dotp#1#2{\langle#1,#2\rangle}
\def\eps{\varepsilon}

\newcommand{\mat}[1]{\ensuremath{\boldsymbol{#1}}}
\newcommand{\vect}[1]{\ensuremath{\mathbf{#1}}}
\newcommand{\iprod}[2]{\ensuremath{\langle#1,#2\rangle}}
\newcommand{\abs}[1]{\ensuremath{|#1|}}

\newcommand{\Nedelec}{N\'{e}d\'{e}lec}

\newcommand{\id}[1]{\structure{#1}}

\newcommand {\Co}{{\mathbb{C}}}
\newcommand {\Int}{{\mathbb{Z}}}
\newcommand {\Nat}{{\mathbb{N}}}
%
%
\newcommand {\Hcurl}{{H(\mathbf{curl};\Omega)}}
\newcommand {\Hocurl}{{H_0(\mathbf{curl};\Omega)}}
\newcommand {\Hdiv}{{H(\mathrm{div};\Omega)}}
\newcommand {\Hodiv}{{H_0(\mathbf{div};\Omega)}}
%
\renewcommand {\Re}{{\rm I \kern-2pt R}}
\newcommand{\vc}[1]{\mbox{\boldmath $#1$}}
\newcommand {\RM}[1]{\mathrm{#1}}



% A simple colored box inlined with the text
%  #1: color to use
%  #2: text to put
\newcommand{\INLINEBOX}[2]{%
   \begin{center}%
    \fcolorbox{#1!60!black}{#1}{%
      \addtolength{\linewidth}{-0.6cm}%  fixed value, works for normal article text
            \begin{minipage}{2\linewidth} #2 \end{minipage}%
  \begin{minipage}{\linewidth} #2 \end{minipage}%
    }%
   \end{center}\vspace{1pt}%
}

% A box at the margin containing the given text
\newcommand{\MARGINBOX}[1]{%
  \mbox{}%
  \marginpar%
   [\tiny\raggedleft\hspace{0pt}#1]%
   {\tiny\raggedright\hspace{0pt}#1}%
}

% mark specific elements: starred versions use inline boxes
\newcommand{\TODO}[2][]{\MARGINBOX{\textcolor{red!80!black}{\emph{ToDo (#1):}} #2}}
\WithSuffix\newcommand\TODO*[2][]{\INLINEBOX{red!20!white}{\emph{ToDo (#1):} #2}}

\newcommand{\FIXME}[2][]{\MARGINBOX{\textcolor{blue!80!black}{\emph{FixMe (#1):}} #2}}
\WithSuffix\newcommand\FIXME*[2][]{\INLINEBOX{blue!20!white}{\emph{FixMe (#1):} #2}}

\newcommand{\NOTE}[2][]{\MARGINBOX{\textcolor{green!80!black}{\emph{Note (#1):}} #2}}
\WithSuffix\newcommand\NOTE*[2][]{\INLINEBOX{green!20!white}{\emph{Note (#1):} #2}}

\newcommand{\DRAFT}[2][]{\MARGINBOX{\textcolor{blue!80!black}{\textsc{Draft (#1):}} #2}}
\WithSuffix\newcommand\DRAFT*[2][]{\INLINEBOX{blue!20!white}{\textsc{Draft (#1):} #2}}

\newcommand{\bs}[1]{\mathbf #1}

%%
%%   VARIABLE HEIGHT FOR THE TITLE BOX (default 35mm)
%%

\setlength{\titleblockheight}{37mm}

\begin{document}
\begin{center}
\begin{tikzpicture}[scale=0.7, transform shape]
    \footnotesize
      \begin{scope}[shape=rectangle,rounded corners,minimum width=3.0cm,minimum height=0.5cm,fill=yellow,text centered]
      
      \draw[rounded corners, draw=green!40, thick, fill=green!25, opacity=0.5, text centered] (-1.55, 1.31) rectangle (8.55,-0.31) node[black, thick, anchor=center, opacity=1., font=\Large] at (3.5, 0.5) {\opal};
      \node[fill= green!40] (0_00) at (0.0,1.0) {MAD-Parser};
      \node[fill= green!40] (0_00) at (3.5,1.0) {Flavors: t,Cycl};
      \node[fill= green!40] (0_00) at (7.0,1.0) {Optimization};
      \node[fill= green!40] (0_00) at (0,0.0)   {Solvers: Direct,MG};
      \node[fill= green!40] (0_00) at (3.5,0.0) {Integrators};
      \node[fill= green!40] (0_00) at (7.0,0.0) {Distributions};

       \draw[rounded corners, draw=red!45, thick, fill=red!25, opacity=0.5, text centered] (-1.55, -0.69) rectangle (8.55,-3.81) node[black, thick, anchor=center, opacity=1.0, font=\Large] at (3.5, -2.25) {\ippl};
       \node[fill= red!45] (q_00) at (0,-1) {FFT};
       \node[fill= red!45] (q_01) at (3.5,-1) {D-Operators};
       \node[fill= red!45] (q_02) at (7,-1) {NGP,CIC, TSI};
       \node[fill= red!45] (q_10) at (0,-1.75) {Fields};
       \node[fill= red!45] (q_11) at (3.5,-1.75) {Mesh};
       \node[fill= red!45] (q_12) at (7,-1.75) {Particles};
       \node[fill=red!45] (q_20) at (0,-2.75) {Load Balancing};
       \node[fill=red!45] (q_21) at (3.5,-2.75) {Domain Decomp.};
       \node[fill=red!45] (q_22) at (7,-2.75) {Message Passing};
       \node[fill=red!45] (q_20) at (0,-3.5) {STL};
       \node[fill=red!45] (q_21) at (3.5,-3.5) {PETE};
       \node[fill=red!45] (q_22) at (7,-3.5) {Polymorphism};

       \node[rotate=90,minimum width=1.7cm,fill=gray] (bla) at (-1.9,0.49){\textcolor{white} {\classic}};
       \node[rotate=90,minimum width=3.15cm,fill= magenta] (bla) at (-1.9,-2.225){\textcolor{white}{H5Part and H5FED}};
       \node[fill=blue!65,minimum width=10.75cm] (q_23) at (3.25,-4.25) {\textcolor{white}{Trilinos}};

      \end{scope}
 \end{tikzpicture}
\end{center}
% \begin{itemize}
%   \item {\bf\color{green}  { \bf \opal Object Oriented Parallel Accelerator Library}}
%   \item { $IP^{2}L$ {\bf Independent Parallel Particle Layer}}
%   \item {\color{gray}  {\bf Class Library for Accelerator Simulation System and Control}}
%   \item  {\bf\color{magenta} {\bf \hfifepart and \hfifefe for parallel particle and field I/O (HDF5)}}
%   \item {\bf\color{blue!65} {\bf Trilinos} \texttt{http://trilinos.sandia.gov/}}
% \end{itemize}
\end{document}
