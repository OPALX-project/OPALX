\documentclass[a4paper,10pt,twocolumn,5p,preprint,pdftex]{elsarticle}

\biboptions{sort&compress}

\usepackage{amsmath}
\usepackage{amsfonts}
\usepackage{graphicx}
\usepackage{algorithm,algorithmic}

\newdefinition{remark}{Remark}
\newcommand {\RM}[1]{\mathrm{#1}}
\renewcommand{\Re}{\mathbb{R}}

\graphicspath{{figures/}}

%TODO:
\title{A Fast Parallel Poisson Solver on Irregular Domains for Beam
  Dynamic Simulations}

%\tnotetext[t1]{This document is a collaborative effort.}

\author[psi]{A.~Adelmann}
\ead{andreas.adelmann@psi.ch}

\author[eth]{P.~Arbenz}
\ead{arbenz@inf.ethz.ch}

\author[eth]{Y.~Ineichen}
\ead{ineichen@inf.ethz.ch}

\address[psi]{Paul Scherrer Institut, CH-5234 Villigen, Switzerland}
\address[eth]{ETH Z\"urich, Computer Science Department,
  Universit\"atsstrasse 6, CH-8092 Z\"urich, Switzerland}

\journal{Computer Physics Communications}

\begin{document}

%MAX: 6 keywords
\begin{keyword}
  Algebraic multigrid \sep Poisson equation \sep Irregular Domains \sep
  Space-Charge \sep Beam Dynamics
\end{keyword}

\begin{abstract}
  We discuss the scalable parallel solution of Poisson equation within a
  Particle-In-Cell (PIC) approach for the simulation of electron beams
  in particle accelerators of complicated shape.  Finite difference
  discretization with various boundary treatments.  Preconditioned
  conjugate gradient method with SA-AMG preconditioning.  Results on
  distributed memory parallel processor.
\end{abstract}

\maketitle



%\section{Introduction and Motivation}
\section{Introduction}
\label{sec:intro}

In recent years, precise beam dynamics calculations in high-current
low-energy hadron machines and in the design of 4th generation light
sources have become a very important research subject.  Hadron machines
are characterized by high currents and hence require excellent control
of beam losses and/or keeping the emittance (a measure of the phase
space) of the beam in narrow ranges.  This is a challenging problem
which requires the accurate modeling of the dynamics of a large ensemble
of macro or real particles subject to complicated external focusing,
accelerating and wake-fields, as well as the self-fields caused by
Coulomb interaction of the particles.  In general the geometries of
particle accelerators are large and complicated which has a direct
impact on the numerical solution method.

Some of the effects can be studied by using a low dimensional model
i.e.\ envelope equations~\cite{Sacherer68, Sa71, stru-reis:1984,
  gluckstern1}.  These are a set of ordinary differential equations for
the second-order moments of a time-dependent particle distribution.
They can be calculated fast, however the level of detail is mostly not
sufficient for quantitative studies.  Furthermore, a priori knowledge of
critical beam parameters such as the emittance is required such that
envelope equations cannot be used as a self-consistent method.

One way to overcome these limitations is by considering the
Vlasov-Poisson description of the phase space, including external and
self-fields and if needed other effects such as wakes.  To that end let
$f(\mathbf{x},\mathbf{v},t)$ be the density of the particles in the
phase space (i.e.\ the position-velocity $(\mathbf{x}, \mathbf{v})$
space).  Its evolution is determined by the collision-less \emph{Vlasov
  equation}.
%%
% Neglecting
% particles collisions, because typical bunch densities are low, the
% evolution of the beam's distribution function
% $f(\mathbf{x},\mathbf{v},t)$ can be considered as a Cauchy problem.
% Denoting the computational domain $\Omega \subset \Re^d$ with $d=2,3$,
% and using $\mathbf {x} \in \Re^d$ to describe a position in
% configuration and $\mathbf {v} \in \Re^d$ a point in velocity space, the
% collision-less Vlasov-Poisson equation can be expressed by
\begin{equation}\label{eq:Vlasov}
  \frac{df}{dt}=\partial_t f + \mathbf{v} \cdot \nabla_{\mathbf{x}} f
  +\frac{q}{m_0}(\mathbf{E}+ \mathbf{v}\times\mathbf{B})\cdot
  \nabla_{\mathbf{v}} f  =  0,
\end{equation}
where $m_0$, $q$ denote particle mass and charge.  The electric and
magnetic fields $\mathbf{E}$ and $\mathbf{B}$ are superpositions of
external fields and space charge fields,
%%and other collective effects such as wake fields,
\begin{equation}\label{eq:allfield}
%%  \begin{aligned}
    \mathbf{E} =
    \mathbf{E_{\RM{ext}}} + \mathbf{E_{\RM{self}}}, \quad
    \mathbf{B} =
    \mathbf{B_{\RM{ext}}} + \mathbf{B_{\RM{self}}}.
%%  \end{aligned}
\end{equation}
%%
If $\mathbf{E}$ and $\mathbf{B}$ are known, then each particle can be
moved according to the equation of motion for charged particles in an
electromagnetic field,
\begin{equation*}\label{eq:motion}
  \frac{d\mathbf{x}(t)}{dt}  = \mathbf{v},
  \quad
  \frac{d\mathbf{v}(t)}{dt}  = \frac{q}{m_0}\left(\mathbf{E} +
    \mathbf{v}\times \mathbf{B}\right).
\end{equation*}
%%which is numerically integrated for each particle.  

% We assume for the rest of the paper that the external fields and the
% wake fields are known and furthermore treat the physical system
% electrostatic (which we obtain without loss of generality by an
% appropriate Lorentz transformation).

After the movement of the particles the space charge portions of
$\mathbf{E}$ and $\mathbf{B}$ have to updated.  

To that end we change the coordinate system into the one moving with the
particles.  By means of the appropriate Lorentz
transformation~\cite{lali:84} we arrive at a (quasi-) static
approximation of the system in which the transformed magnetic field
becomes negligible, $\hat{\mathbf{B}}\! \approx\! \mathbf{0}$.  The
transformed electric field is obtained by
\begin{equation}\label{eq:e-field}
  \hat{\mathbf{E}}=\hat{\mathbf{E}}_{\RM{self}}=-\nabla\hat{\phi}
\end{equation}
where the electrostatic potential $\hat{\phi}$ is the solution of the
\emph{Poisson problem}
\begin{equation}\label{eq:poisson0}
  - \Delta \hat{\phi}(\mathbf{x}) =
  \frac{\hat{\rho}(\mathbf{x})}{\varepsilon_0},
\end{equation}
equipped with appropriate boundary conditions, see
section~\ref{sec:method}.  Here, $\hat{\rho}$ denotes the spatial charge
density and $\varepsilon_0$ is the dielectric constant.
% in the (beam rest) frame
% If we can make an appropriate Lorentz transformation to arrive at a
% (quasi-)static approximation of the system, the space charge fields can
% be obtained by solving a Poisson problem with appropriate boundary
% conditions:
% \begin{equation}\label{eq:Poisson}
%   - \nabla^{2} \phi(\mathbf{x}) = \frac{\rho(\mathbf{x})}{\varepsilon_0},
% \end{equation}
% where $\phi$ and $\rho$ are the electrostatic potential and the spatial
% charge density in the appropriate (beam rest) frame.  The electric field
% can then be calculated by
% \begin{equation}\label{eq:Efield}
%   \mathbf{E}=\mathbf{E_{\RM{self}}}=-\nabla\phi.
% \end{equation}
By means of the inverse Lorentz transformation the electric field
$\hat{\mathbf{E}}$ can then be transformed back to yield both the
electric and the magnetic fields in~\eqref{eq:allfield}.  

The Poisson problem~\eqref{eq:poisson0} discretized by finite
differences can efficiently be solved on a rectangular grid by a
particle-in-cell (PIC) approach~\cite{qiry01}.  The right hand side
in~\eqref{eq:poisson0} is obtained by sampling the particles at the grid
points.  In~\eqref{eq:e-field}, $\hat{\phi}$ is interpolated at the
particle positions from its values at the grid points.

The FFT-based Poisson solvers and similar approaches~\cite{qiry01,
  qigl04} are restricted to boxed-shaped domains.  In
Section~\ref{sec:method} we describe how the Poisson equation on a
``general'' domain $\Omega \subset \Re^3$ can be solved by finite
differences and the PIC approach.  We treat the boundary in a similar
way as McCorquodale \textit{et al.}~\cite{mcgv:04}.  The system of
equation is solved by the conjugate gradient algorithm preconditioned by
smoothed aggregation AMG.  In Section~\ref{sec:impl} we deal with
details of the implementation, in particular its parallelization.  In
Section~\ref{sec:results} we report on numerical experiments that we
conducted on the Cray XT/4 at the Swiss Supercomputing Center CSCS.
In Section~\ref{sec:concl} we draw our conclusions.

% , and thanks to
% high performance computing a three-dimensional (3D) model with
% sufficient resolution is feasible.

% A very attractive approach to solve the Poisson equation is by a fast
% direct solver based on FFT's~\cite{qiry01}.  However, this approach is
% limited to simple geometries such as boxes and cylinders.  Other fast
% and accurate methods are known~\cite{qigl04} but they are also limited
% to simple geometries.  Because $\rho$ is varying slowly in
% time~\eqref{eq:poisson0}, it is desirable to use solutions from previous
% time step(s).  In the sequel of the paper we consider iterative solvers
% for~\eqref{eq:poisson0} on general but simply connected geometries
% $\Omega \subset \Re^d$.

%%% Local Variables: 
%%% mode: latex
%%% TeX-master: "paper"
%%% End: 

%+ motivate irregular boundaries



%\section{Methodik}
\section{The discretization}
\label{sec:discr}

In this section we discuss the solution of th Poisson equation in a
domain 
%%
% Widely used methods for the calculation of the space-charge fields are
% the particle-mesh method and the particle-particle method.  The
% particle-mesh method, based on solving Poisson's equation for the
% electrostatic potential, is typically much faster than the
% particle-particle method.  Furthermore, it provides better numerical
% results for sufficiently 'smooth' distributed particles.
%%
\begin{figure}[htb]
  \centering
  \input{figures/cylinder.pdf_t}
  \caption{Sketch of a typical domain}
  \label{fig:domain}
\end{figure}
%%
$\Omega \subset \Re^3$ as indicated in Fig.~\ref{fig:domain}.  The
boundary of the domain is composed of two parts, a curved, smooth
surface $\Gamma_1$ and two planar portions at $z=-d$ and $z=+d$ that
form together $\Gamma_2$.  In physical terms $\Gamma_1$ forms the casing
of the pipe, while $\Gamma_2$ is the open boundary at the inlet and
outlet of the beam pipe.  The centroid of the particle bunch is at the
origin of the coordinate system.  The Poisson problem that we are going
to solve is given by
\begin{equation} \label{eq:poisson}
  \begin{aligned}
    -\Delta \phi &= \frac{\rho}{\epsilon_0}\ \text{in}\ \Omega, \\
    \phi &= g \equiv 0\ \text{on}\ \Gamma_1,   \\
    \frac{\partial \phi}{\partial \mathbf{n}} + \frac {1}{d} \phi &= 0\
    \text{on}\ \Gamma_2.
  \end{aligned}
\end{equation}
The parameter $d$ in the Robin boundary condition is introduced by the
Lorentz transformation mentioned earlier (CORRECT???).

We discretize~\eqref{eq:poisson} by a second order finite difference
scheme defined on a rectangular lattice (grid)
\begin{displaymath}
  \Omega_h:=\left\{ \mathbf{x} \in {\Omega}\cup\Gamma_2 \ |\ x_i/h_i \in
    \mathbb{Z} \ \mbox{for}\ i=1,2,3. \right\},
\end{displaymath}
where $h_i$ is the grid spacing in the $i$-th coordinate direction. The
grid is arranged in a way that the two portions of $\Gamma_2$ lie in
grid planes.  A lattice point is called an \emph{interior} point if all
its direct neighbours are in $\Omega$.  All other grid points are called
\emph{near-boundary} points.  At interior points $\mathbf{x}$ we
approximate $\Delta u (\mathbf{x})$ by the well known 7-point difference
star
\begin{equation}  \label{eq:7pt-star}
  -\Delta_h u(\mathbf{x}) = 
  \sum_{i=1}^3
  \frac{-u(\mathbf{x}\!-\!h_i\mathbf{e}_i) + 2 u(\mathbf{x})
  - u(\mathbf{x}\!+\!h_i\mathbf{e}_i)}{h_i^2}.
\end{equation}
At grid points near the boundary we have to take the boundary conditions
in~\eqref{eq:poisson} into account.  To explain the schemes on the
Dirichlet (or PEC) boundary $\Gamma_1$ let $\mathbf{x}$ be a
near-boundary point.  Let $\mathbf{x}' := \mathbf{x} - h_i\mathbf{e}_i$
for some $i$ be outside $\Omega$ and let $\mathbf{x}^* := \mathbf{x} - s
h_i\mathbf{e}_i$, $0<s\le1$, be the boundary point between $\mathbf{x}$
and $\mathbf{x}'$ that is closest to $\mathbf{x}$,
cf.~Fig.~\ref{fig:boundary}.


% We are dealing with the finite difference method for the Poisson
% equation
% \begin{equation}  \label{eq:poisson}
%   -\Delta u (\mathbf{x}) = f(\mathbf{x}), \qquad \mathbf{x}\in\Omega.
% \end{equation}
% on irregular domains $\Omega$ in two or three space dimensions.  For the
% presentation we restrict ourselves to the 2-dimensional case.  We are
% given a \emph{regular} mesh $\Omega_h$ with grid size $h$.  $\Delta u
% (\mathbf{x})$ is approximated at the interior point $O=\mathbf{x} \in
% \Omega_h$ by the well known 5-point difference star
% \begin{equation}  \label{eq:5pt-star}
%   -\Delta_h u(x,y) = 
%   \frac{1}{h^2}
%   \left( 4 u_O - u_W - u_S - u_E - u_N \right)
% \end{equation}
% Here, $u_N$, $u_W$, $u_S$, and $u_E$, are the values of the FD
% approximation at the grid points in a distance $h$ of the central point
% $O$ in the north, west, south, and east direction, respectively.  This
% formula cannot be applied if one of the grid points is outside $\Omega$.
% We discuss three approaches for dealing with near-boundary points%
%%
\begin{figure}[htb]
  \centering
  \input{figures/bdry.pdf_t}
  \caption{1-dimensional sketch of a near-boundary point $x$}
  \label{fig:boundary}
\end{figure}
%%
If $s=1$, i.e., if $\mathbf{x}'\in \partial\Omega$ then $u(\mathbf{x}')$
in~\eqref{eq:7pt-star} is replaced by the prescribed boundary value.
%%
Otherwise, we proceed in one of the three following ways~\cite{fowa:60,
  hack:94}
\begin{enumerate}
\item In \emph{constant extrapolation} the boundary value prescribed at
  $\mathbf{x} - s h_i\mathbf{e}_i \in \Gamma_1$ is assigned to
  $u(\mathbf{x}')$,
  \begin{equation}
    \label{eq:const_extrapol}
    u(\mathbf{x}') = u(\mathbf{x} - h_i\mathbf{e}_i) := g(\mathbf{x}^*).
  \end{equation}
  % Insertion in the five-point difference approximation gives
  % \begin{equation}  \label{eq:5pt-star0}
  %   -\Delta_h^0 u(x,y) = 
  %   \frac{1}{h^2} \left( 4 u_O - g(W^*) - u_S - u_E - u_N \right)
  % \end{equation}
  % This boundary treatment leads to global matrices that are symmetric
  % and positive definite.
\item In \emph{linear extrapolation} the value at $\mathbf{x}'$ is
  obtained by means of the values $u$ at $\mathbf{x}$ and at $\mathbf{x}
  - sh_i\mathbf{e}_i$,
  \begin{equation}
    \label{eq:lin_extrapol}
    u(\mathbf{x}') := (1-\frac{1}{s})\, u(\mathbf{x}) + \frac{1}{s}\,
    g(\mathbf{x}^*).
  \end{equation}
  % This modifies the five-point difference approximation to
  % \begin{equation}  \label{eq:5pt-star1}
  %   \begin{aligned}
  %     -\Delta_h^1 u(x,y) &= \frac{1}{h^2} \left((3+\frac{1}{s})\, u_O -
  %     \right. \\ 
  %     &\quad \left. \frac{1}{s}\, g(W^*) - u_S - u_E - u_N \right).
  %   \end{aligned}
  % \end{equation}
  % \begin{equation}  \label{eq:5pt-star1}
  %   \begin{split}
  %     -\Delta_h^1 u(x,y) =  \\
  %     \frac{1}{h^2} \left((3+\frac{1}{s})\, u_O - \frac{1}{s}\, \right.\\
  %     \left. g(W^*) - u_S - u_E - u_N \right).
  %   \end{split}
  % \end{equation}
  % \begin{equation}  \label{eq:5pt-star1}
  %   -\Delta_h^1 u(x,y) = 
  %   \frac{1}{h^2} \left((3+\frac{1}{s})\, u_O - \frac{1}{s}\,
  %     g(W^*) - u_S - u_E - u_N \right).
  % \end{equation}

% (Note that possibly $\mathbf{x}'\in \partial\Omega$.)

\item In \emph{quadratic extrapolation} we assume that $\mathbf{x}'' :=
  \mathbf{x} + h_i\mathbf{e}_i \in \Omega_h$ which requires a
  sufficiently fine grid and a sufficiently smooth boundary.  Then the
  value $ u(\mathbf{x}')$ is obtained by means of the values of $u$ at
  $\mathbf{x}$, $\mathbf{x}''$, and the boundary point $\mathbf{x}^*$,
  \begin{equation} \label{eq:quad_extrapol}
    %% \begin{aligned}
    u(\mathbf{x}') := \frac{2(s\!-\!1)}{s} u(\mathbf{x}) -
    \frac{s\!-\!1}{s\!+\!1} u(\mathbf{x''}) + %%\\
    %% &\hspace*{10mm}
    \frac{2}{s(s\!+\!1)} g(\mathbf{x}^*),
    %% \end{aligned}
  \end{equation}
%   Substituting $u_W$ in~\eqref{eq:5pt-star} gives
%   \begin{equation}  \label{eq:5pt-star2}
%     \begin{aligned}
%       -\Delta_h^2 u(x,y) &=
%       \frac{1}{h^2} \left( (2+\frac{2}{s})\,  u_O -\frac{2}{s + 1}\, u_E
%       \right. \\ 
%       &\quad\ \left. -\frac{2}{(s + 1)s}\, g(W^*) - u_N - u_S \right),
%     \end{aligned}
%   \end{equation}
  %\begin{equation}  \label{eq:5pt-star2}
  %  -\Delta_h^2 u(x,y) = 
  %  \frac{1}{h^2} \left(
  %    (2+\frac{2}{s})\,  u_O
  %    -\frac{2}{s + 1}\, u_E
  %    -\frac{2}{(s + 1)s}\, g(W^*)
  %    - u_N - u_S \right),
  %\end{equation}
  which leads to the Shortley-Weller approximation~\cite{shwe:39,mcgv:04}.
\end{enumerate}
In all extrapolation formulae above $g(\mathbf{x}^*) =
g(\mathbf{x}\!-\!sh_i\mathbf{e}_i) = 0$.  The value on the right side
of~\eqref{eq:const_extrapol}--\eqref{eq:quad_extrapol} substitutes
$u(\mathbf{x}\! -\! h_i\mathbf{e}_i)$ in~\eqref{eq:7pt-star}.

Let us now look at a grid point $\mathbf{x}$ on the open boundary
$\Gamma_2$.  If $\mathbf{x}$ is located on the inlet of the beam pipe
then $\mathbf{x}'':=\mathbf{x}\! +\! h_3\mathbf{e}_3 \in \Omega$ and
$\mathbf{x}':=\mathbf{x}\! -\! h_3\mathbf{e}_3 \not\in \Omega$.  The
Robin boundary condition is approximated by a central difference,
\begin{displaymath}
  - \frac{u(\mathbf{x}'') - u(\mathbf{x}')}{2h_3}
  + \frac{1}{d}u(\mathbf{x}) = 0,
\end{displaymath}
which leads to
\begin{equation}  \label{eq:inlet}
  u(\mathbf{x}') = u(\mathbf{x}'') - \frac{2h_3}{d}u(\mathbf{x}) = 0.
\end{equation}
The same formula holds on the outlet boundary portion if $\mathbf{x}'$
denotes the virtual grid point outside $\Omega$.

Notice that some lattice points may be close to the boundary with regard
to more than one coordinate direction.  Then, the
procedures~\eqref{eq:const_extrapol}--\eqref{eq:quad_extrapol} must be
applied in all of them.  This holds in particular for points on the open
boundary $\Gamma_2$.

The boundary treatment just described leads to systems of equations  
\begin{equation} \label{eq:lin-syst}
  A \mathbf{x} = \mathbf{b},
\end{equation}
where $\mathbf{x}$ is the vector of unknown values of the potential and
$\mathbf{b}$ is the vector of the given charge density at the grid
points.

The Poisson matrix $A$ is an $M$-matrix irrespective of the boundary
treatment~\cite{hack:94}.  Constant and linear extrapolation lead to a
\emph{symmetric} positive definite $A$ while quadratic extrapolation
yields a \emph{nonsymmetric} but still positive definite Poisson matrix.

Notice that the boundary extrapolation introduces large diagonal
elements in $A$ if $s$ gets close to zero.  In order to avoid numerical
troubles it is advisable to scale the system matrix.  If $D =
\mbox{diag}{A}$, then we replace $A$ by $D^{-1/2} A D^{-1/2}$ and adapt
the right-hand side accordingly.

% \begin{remark}
%   $\Delta u-\Delta_h^0u = \mathcal{O}(h)$,
%   $\Delta u-\Delta_h^1u = \mathcal{O}(h^2)$,
%   $\Delta u-\Delta_h^2u = \mathcal{O}(h^2)$
% \end{remark}


%%% Local Variables: 
%%% mode: latex
%%% TeX-master: "paper"
%%% End: 
%+ motivate irregular domains

\section{The solution method}
\label{sec:method}

In this section we discuss the solution of the Poisson
problem~\eqref{eq:lin-syst} discretized by finite differences as
described in the previous section.

\subsection{The conjugate gradient algorithm}

For symmetric positive definite (spd) systems, the conjugate gradient
(CG) algorithm~\cite{hest:52} provides a fast and memory efficient
solver.  The CG algorithm minimizes the quadratic functional
\begin{equation} \label{eq:cg-funct}
  \varphi(\mathbf{x}) = \frac{1}{2}\mathbf{x}^T A \mathbf{x} - \mathbf{x}^T
  \mathbf{b}
\end{equation}
in the Krylov space that is implicitely constructed in the iteration.
In the $k$-th iteration step the CG algorithm minimizes the quadratic
functional along a search direction $\mathbf{d}_k$.  The search
directions turn out to by pairwise conjugate, $\mathbf{d}_k^T A
\mathbf{d}_j = 0$ for all $k\neq j$.

If we use the quadratic extrapolation~\eqref{eq:quad_extrapol} at the
boundary then $A$ in~\eqref{eq:lin-syst} is not symmetric positive
definite anymore.  Nevertheless, the solution of~\eqref{eq:lin-syst} is
still a minimizer of $\varphi(\mathbf{x})$.  The CG algorithm can be
used to solve~\eqref{eq:lin-syst}.  However, the orthogonality among the
search directions is lost.  Only consecutive search directions are
conjugate, $\mathbf{d}_k^T A \mathbf{d}_{k-1} = 0$, reflecting the fact
that $\varphi(\mathbf{x})$ is minimized only locally.  So, in the
nonsymmetric positive definite case, CG is essentially steepest descent,
or, equivalently, \textsc{Orthomin}(1)~\cite{yoje:80}.
\textsc{Orthomin}($s$) enforces conjugacy among $\mathbf{d}_k, \ldots,
\mathbf{d}_{k-s}$.  GMRES enforces orthogonality among all previous
search directions.  All the \textsc{Orthomin} variants and
GMRES~\cite{sasc:86} converge for nonsymmetric positive definite
systems.  The memory consumption increases with $s$ and is maximal with
GMRES that is designed for solving arbitrary nonsymmetric linear
systems.  To save memory space bi-conjugate methods like, e.g., the
stabilized biconjugate gradient (BiCGStab) method~\cite{vors:92} are
good candidates for solving~\eqref{eq:lin-syst}.  When considering
computational costs we note that BiCGStab requires two matrix-vector
products in contrast to CG that reqires only one.

Although $A$ is nonsymmetric it is so `only in the boundary points'.
Therefore, one may hope that the conjugate gradient method still
performes reasonably well.

\subsection{Preconditioning}

To improve the convergence properties of CG methods one can try to
improve the condition number of the matrix by preconditioning the
system.  This is due to the fact that usually the convergence of an
iterative solver degrades when the condition number of a system
increases.  A preconditioned system has the following form
\begin{equation*}
  \underbrace{{M}^{-1}{A}}_{\overline{{A}}}
  \mathbf{x} =
  \underbrace{{M}^{-1}\mathbf{b}}_{\overline{{b}}},
\end{equation*}
where the positive definite matrix $\mathbf{M}^{-1}$ is called
preconditioner.  When the preconditioner is well chosen the new system
has a smaller condition number.

In this paper we are concerned with multilevel preconditioners.
Multigrid or multilevel preconditioners are the most effective
preconditioners, in particular for the Poisson problems that we want to
solve~\cite{hack:85,tros:00}.  Multigrid methods are based on the
observation that a smooth error on a fine grid can be well approximated
on a coarser grid.  When this coarser grid is chosen to be a sufficient
factor smaller than the fine grid the resulting problem is smaller and
thus cheaper to solve.  Obviously one can continue coarsening the grid
until one arrives at a problem size that can be cheaply solved by a
direct solver.  This observation suggests an algorithm similar to
Algorithm~\ref{alg:mg_algo}.

\begin{algorithm}
  \caption{Multigrid V-cycle Algorithm} \label{alg:mg_algo}
  \begin{algorithmic}[1]
    \STATE \textbf{procedure} MultiGridSolve($A_\ell$, $b_\ell$, $x_\ell$, $\ell$)
    
    \IF{$\ell$ = maxLevel-1}
    \STATE DirectSolve $A_\ell \mathbf{x}_\ell = \mathbf{b}_\ell$
    \ELSE
    \STATE $\mathbf{x}_\ell$ $\leftarrow$ $S^{pre}_\ell$($A_\ell$, $\mathbf{b}_\ell$, $0$)
    \STATE $\mathbf{r}_\ell$ $\leftarrow$ $\mathbf{b}_\ell$ - $A_\ell
    \mathbf{x}_\ell$ \COMMENT{calculate residual}
    \STATE $\mathbf{b}_{\ell+1}$ $\leftarrow$ $R_\ell \mathbf{r}_\ell$
    \COMMENT{Restriction} 
    \STATE $\mathbf{v}_{\ell+1}$ $\leftarrow$ $\mathbf{0}$
    \STATE MultiGridSolve($A_{\ell+1}$, $\mathbf{b}_{\ell+1}$,
    $\mathbf{v}_{\ell+1}$, $\ell+1$) 
    \STATE $\mathbf{x}_\ell$ $\leftarrow$ $\mathbf{x}_\ell$ + $P_\ell
    \mathbf{v}_{\ell+1}$ \COMMENT{coarse grid correction} 
    \STATE $\mathbf{x}_\ell$ $\leftarrow$ $S^{post}_\ell$($A_\ell$,
    $\mathbf{b}_\ell$, $\mathbf{x}_\ell$) 
    \ENDIF
    \STATE \textbf{end procedure}
  \end{algorithmic}
\end{algorithm}

The procedure starts on the finest level ($\ell\!=\!0$) and repeatedly
coarsens the grid until the coarsest level is reached
(\texttt{maxLevel}) on which a direct solver is used to solve the
problem at hand.  On all other levels $\ell$ the algorithm starts by
presmoothing $S_\ell^{pre}$ the problem to damp high frequency
components of the error (line 5).  Subsequently the fine grid on level
$\ell$ can be restricted with the restriction operator $R_\ell$ to a
coarser grid on level $\ell+ 1$ (line $7$).  This essentially
``transfers'' the low frequency components on the fine grid to high
frequency components on the coarse grid.  After the recursion has
reached the coarsest level and used the direct solver to solve the
coarse level problem the solution can be prolongated back to a finer
grid.  This is achieved with the prolongation operator $P_\ell$
(line~10).  Often a postsmoother $S_\ell^{post}$ is used to remove
artifacts caused by the prolongation operator.  Usually these operators
(for every level $\ell$) are defined in a setup phase preceding the
execution of the actual Multigrid algorithm.  Lastly $A_\ell$ denotes
the matrix of the discretized system in level $\ell$.

The described Multigrid algorithm has order $O(n)$ and its convergence
should be independent of the grid-spacings.  Multigrid methods work well
for symmetric positive definite systems obtained from elliptic PDE's.

\subsection{A Multigrid Preconditioner for Space Charge
  Calculations} \label{sec:mgprec}

As described in~\cite{oowa:98}, Multigrid methods provide a less
sensitive convergence behavior to parameter changes when used as a
preconditioner applied to an iterative solver.  Additionally the
algebraic Multigrid (AMG) performs better when used as preconditioner
opposed the application as stand-alone solver.  This provides a
sufficient motivation to use an AMG (described in this subsection) as
preconditioner for an iterative solver (CG, BiCGStab).

Independent of the application of Multigrid methods the performance
profoundly depends on the choices of the operators (introduced in the
previous subsection) and their interplay.  To ensure that the resulting
preconditioner is symmetric we use the same pre- and postsmoother $S_l$
and the restriction operator is chosen to be the transpose of the
prolongation operator $R_l = P_l^T$.  This leaves us with two operators
($P_l$ and $S_l$) that have to be defined for every level.

The prolongation and restriction operator can be defined in various
ways.  When using a Geometric Multigrid (GMG) the restriction and
prolongation operator is defined by using geometric information about
the grid.   Another approach is to solely base these operators on the
algebraic information contained in the system of equations without
considering the und erlaying grid hierarchy.  This describes the basic
property of operators utilized in AMG's.  The independence of the
underlying geometric grid makes the AMG very robust in the presence of
complicated structured or unstructured grids.  Since we are dealing with
an anisotropic grid this property makes the AMG a valuable choice,
saving us a great deal of trouble and additional effort.  Normally when
applying a GMG to an anisotropic grid special transfer operators are
needed (i.   e.  semi-coarsening).  Aside the more costly transfer
operators we will experience that the standard smoothers applied to an
anisotropic problem will perform badly.  The reason for the bad
performance is that the smoother has no averaging effect in directions
that are anisotropic (or extremely stretched).  Therefore the smoothing
effect in these direction is inexistent or very weak causing the
Multigrid method to converge very slowly.  In contrast AMG's work
efficiently on all error components.  In comparison the convergence rate
of an AMG is about the same as for a GMG.

\paragraph{Prolongation Operator $P_l$} Aggregation based methods
cluster the fine grid unknowns to aggregates (of a specific form, size,
etc.) as representation for the unknowns on the coarse grid.  First the
discretization matrix $A_l$ is converted into a graph $G_l$.  Each vertex
of $G_l$ is then assigned to one aggregate of the disjoint aggregate
set.  In a next step a tentative prolongation operator matrix is formed
where matrix rows correspond to vertices and matrix columns to
aggregates.  A matrix entry $(i,j)$ has a value of $1$ if the $it^{th}$
vertex is contained in $j^{th}$ aggregate and $0$ otherwise.  This
prolongation operator basically corresponds to a piecewise constant
interpolation operation.  To improve the robustness one can additionally
smooth (normally with a damped Jacobi smoother) the tentative
prolongation operator.  In general applying the smoother results in
better interpolation properties opposed to the piecewise constant
polynomials.  The application of the smoother is beneficial for spd
problems arising from elliptic problems improving convergence.
In~\cite{tuto:00} various strategies are proposed how this process
can be parallelized.  One of the simplest parallel method is to let each
processor aggregate its portion of the grid.  This method is called
``decoupled'' since every processor act independently of others.  Usually
the aggregates have a size of $3^d$ in $d$ dimensions.  For our 3D
problems aggregates therefore have a size of $3 \times 3 \times 3 = 27$
forming small cubic aggregates with no overlapping elements.

\paragraph{Smoothing Operator $S_l$} As advised in~\cite{abht:03} we
choose a Chebyshev polynomial smoother.  The choice is motivated by the
result that polynomial smoothers perform better than (e.g.) Gauss-Seidel
smoothers for a parallel solver.  Advantages are e.g.\ that polynomial
smoother do not need special matrix kernels and formats for optimal
performance and generally polynomial methods can profit of architecture
optimized matrix vector products.  Additionally polynomial smoothers are
much easier to parallelize than e.g.\ Gauss-Seidel methods.

\paragraph{Coarse Level Solver} The coarse level equation is solved by a
direct LU based solver.  In particular Trilinos (the software framework
employed) provides a package (Amesos) offering various direct
solvers.  The employed solver (KLU) ships the coarse level problem to
node $0$ and solves it by means of a LU decomposition.  Once the solution
has been calculated it is broadcast to all nodes.

\subsubsection*{Implementation}

The Multigrid preconditioner and solver is implemented with help of the
Trilinos \cite{Trilinos-TOMS} library.  Trilinos provides
state-of-the-art tools for numerical computation in various
packages.  Amongst others there is a package called ML \cite{gsht:06}
providing a Multigrid preconditioner.  We utilized this package to create
the AMG preconditioner using a smoothed aggregation-based transfer
operator.   A summary of the essential AMG preconditioner parameters
discussed in the previous subsection are presented in
Table~\ref{tab:sa_setup}.  Currently the solver is parallelized in $z$
direction.

\begin{table}[h!b!p!]
  \begin{center}
    \begin{tabular}{l|l}
      \hline
      name & value \\
      \hline
      \hline
      max levels & 5 \\
      increasing or decreasing & decreasing \\
      prec type & MGW \\
      aggregation: type & Uncoupled \\
      smoother: type & Chebyshev \\
      smoother: sweeps & 3 \\
      smoother: pre or post & both \\
      coarse: type & Amesos-KLU \\
      \hline
    \end{tabular}
    \caption{Parameters for multilevel
      preconditioner ML}
    \label{tab:sa_setup}
  \end{center}
\end{table}

For the embedding in the physical simulation code we utilized
Independent Parallel Particle Layer (IPPL).  This library is an
object-oriented framework for particle based applications in
computational science requiring high-performance parallel computers,
hence it is used by \textsc{OPAL} to a great extent to handle particles,
fields, operators on grids and fields and communication.  In the context
of this thesis IPPL is only relevant because \textsc{OPAL} also uses
IPPL to represent the grid with the interpolated charges of the
particles.




%%% Local Variables: 
%%% mode: latex
%%% TeX-master: "paper"
%%% End: 
%+ motivate irregular domains


%\section{Implementation}
\section{Implementation}
\label{sec:impl}

%+ parallel

%%% Local Variables: 
%%% mode: latex
%%% TeX-master: "paper"
%%% End: 


%\section{Numerical Experiments and Results}
\section{Numerical Experiments and Results}
\label{sec:results}

%+ (evt.) FFT vs. iterativ solver: impact on space-charges

%%% Local Variables: 
%%% mode: latex
%%% TeX-master: "paper"
%%% End:


%\section{Conclusion}
\section{Conclusion}
\label{sec:concl}

We have presented a scalable Poisson solver suitable to handle domains
with irregular boundaries as they arise, for example, in beam dynamics
simulations.  PEC and open-space boundary approximations have been
discussed.  A real world example where the solver was used in an beam
dynamics code (OPAL) shows the relevance of this approach.
{\color{red}Why?}  The code exhibits excellent scalability up to 2048
processors with cylindrial tubes embedded in meshes with up to $1024^3$
nodes.  This will enable precise beam dynamics simulations of a new
level of detail in the very near future.

In real particle simulations (and other test cases encountered) system
matrices are only mildly nonsymmetric such that PCG could be applied..
{\color{red}Shortley-Weller bc}

Variants reducing time to solution..

\subsection{Future work}

We already possess a boundary treatment for simply connected geometries.
In the future we will conduct the same benchmarks mentioned here to
measure the performance of ``arbitrary'' boundaries.  An interesting
study will concern the effect on the physics of a simulation run or even
studies requiring a larger level of detail.

Another improvement will be to implement adaptive mesh refinement to
reduce the number of gridpoints in regions less relevant for
space-charge calculations.

Study: meshsize...

%%% Local Variables: 
%%% mode: latex
%%% TeX-master: "paper"
%%% End: 



\section*{Acknowledgments}

%references

\bibliography{bib,literature}
\bibliographystyle{elsarticle-num}

\end{document}
