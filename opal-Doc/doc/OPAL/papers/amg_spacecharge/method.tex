\section{The solution method}
\label{sec:method}

In this section we discuss the solution of~\eqref{eq:lin-syst}, the
Poisson problem~\eqref{eq:poisson} discretized by finite differences as
described in the previous section.

\subsection{The conjugate gradient algorithm}

The matrix $A$ in~\eqref{eq:lin-syst} is symmetric positive definite
(spd) if the boundary conditions are treated by constant or linear
extrapolation.  For symmetric positive definite systems, the conjugate
gradient (CG) algorithm~\cite{hack:94,hest:52} provides a fast and
memory efficient solver.  The CG algorithm minimizes the quadratic
functional
\begin{equation} \label{eq:cg-funct}
  \varphi(\mathbf{x}) = \frac{1}{2}\mathbf{x}^T A \mathbf{x} - \mathbf{x}^T
  \mathbf{b}
\end{equation}
in the Krylov space that is implicitly constructed in the iteration.  In
the $k$-th iteration step the CG algorithm minimizes the quadratic
functional $\varphi$ along a search direction $\mathbf{d}_k$.  The search
directions turn out to by pairwise conjugate, $\mathbf{d}_k^T A
\mathbf{d}_j = 0$ for all $k\neq j$, and $\varphi(\mathbf{x})$ is
minimized in the whole $k$-dimensional Krylov space.

If we use the quadratic extrapolation~\eqref{eq:quad_extrapol} at the
boundary then $A$ in~\eqref{eq:lin-syst} is not symmetric positive
definite anymore.  Nevertheless, the solution of~\eqref{eq:lin-syst} is
still a minimizer of $\varphi(\mathbf{x})$.  The CG algorithm can be
used to solve~\eqref{eq:lin-syst}.  It is known to
converge~\cite{gree:97}.  However, the finite termination property of CG
is lost as the search directions are not mutually conjugate any more.
Only consecutive search directions are conjugate, $\mathbf{d}_k^T A
\mathbf{d}_{k-1} = 0$, reflecting the fact that $\varphi(\mathbf{x})$ is
minimized only locally.
% So, in the nonsymmetric positive definite case, the finite termination
% property of CG is lost.
Young \& Jea~\cite{yoje:80} investigated generalizations of the
conjugate gradient algorithm for nonsymmetric positive definite
matrices, in which conjugacy is enforced among $\mathbf{d}_k, \ldots,
\mathbf{d}_{k-s}$ for some $s>1$.  We do not pursue these approaches
here.  As GMRES~\cite{sasc:86}, they consume much more memory space than
the straightforward CG method that turned out to be extremely efficient
for our application.  Although $A$ is nonsymmetric it is so only
`mildly', i.e., there are some deviations from symmetry only at some of
the boundary points.  Therefore, one may hope that the conjugate
gradient method still performs reasonably well.  This is what we
actually did observe in our experiments.

% or, equivalently, \textsc{Orthomin}(1)~\cite{yoje:80}.
% \textsc{Orthomin}($s$) enforces conjugacy among $\mathbf{d}_k, \ldots,
% \mathbf{d}_{k-s}$.  GMRES enforces orthogonality among all previous
% search directions.  All the \textsc{Orthomin} variants and
% GMRES~\cite{sasc:86} converge for nonsymmetric positive definite
% systems.  The memory consumption increases with $s$ and is maximal with
% GMRES that is designed for solving arbitrary nonsymmetric linear
% systems.  
Methods that are almost as memory efficient as CG like, e.g., the
stabilized biconjugate gradient (BiCGStab) method~\cite{vors:92} could
be used for solving~\eqref{eq:lin-syst}, also.  However, when
considering computational costs we note that BiCGStab requires two
matrix-vector products per iteration step, in contrast to CG that
requires only one.

\subsection{Preconditioning}

To improve the convergence behavior of the CG methods we
precondition~\eqref{eq:lin-syst}.  The preconditioned system has the
form
\begin{equation*}
  {M}^{-1}{A} \mathbf{x} = {M}^{-1}\mathbf{b},
\end{equation*}
where the positive definite matrix $M$ is the preconditioner.
A good choice of the preconditioner reduces the condition of the system
and thus the number of steps the iterative solver takes until
convergence~\cite{hack:94,gree:97}.  Preconditioning is inevitable for systems
originating in finite difference discretizations of elliptic equations since
their condition number increases as $h^{-2}$ where $h$ is the mesh width.

In this paper we are concerned with multilevel preconditioners.
Multigrid or multilevel preconditioners are the most effective
preconditioners, in particular for the Poisson
problem~\cite{hack:85,tros:00}.  Multigrid methods make use of the
observation that a smooth error on a fine grid can be well approximated
on a coarser grid.  When this coarser grid is chosen to be a sufficient
factor smaller than the fine grid the resulting problem is smaller and
thus cheaper to solve.  We can continue coarsening the grid until we
arrive at a problem size that can be solved cheaply by a direct solver.
This observation suggests an algorithm similar to
Algorithm~\ref{alg:mg_algo}.

\begin{algorithm}
  \caption{Multigrid V-cycle Algorithm} \label{alg:mg_algo}
  \begin{algorithmic}[1]
    \STATE \textbf{procedure} MultiGridSolve($A_\ell$, $\mathbf{b}_\ell$, $\mathbf{x}_\ell$, $\ell$)
    
    \IF{$\ell = \mbox{maxLevel}-1$}
    \STATE DirectSolve $A_\ell \mathbf{x}_\ell = \mathbf{b}_\ell$
    \ELSE
    \STATE $\mathbf{x}_\ell$ $\leftarrow$ $S^{pre}_\ell$($A_\ell$, $\mathbf{b}_\ell$, $0$) \COMMENT{presmoothing}
    \STATE $\mathbf{r}_\ell$ $\leftarrow$ $\mathbf{b}_\ell$ - $A_\ell
    \mathbf{x}_\ell$ \COMMENT{calculate residual}
    \STATE $\mathbf{b}_{\ell+1}$ $\leftarrow$ $R_\ell \mathbf{r}_\ell$
    \COMMENT{restriction} 
    \STATE $\mathbf{v}_{\ell+1}$ $\leftarrow$ $\mathbf{0}$
    \STATE MultiGridSolve($A_{\ell+1}$, $\mathbf{b}_{\ell+1}$,
    $\mathbf{v}_{\ell+1}$, $\ell\!+\!1$) 
    \STATE $\mathbf{x}_\ell$ $\leftarrow$ $\mathbf{x}_\ell$ + $P_\ell
    \mathbf{v}_{\ell+1}$ \COMMENT{coarse grid correction} 
    \STATE $\mathbf{x}_\ell$ $\leftarrow$ $S^{post}_\ell$($A_\ell$,
    $\mathbf{b}_\ell$, $\mathbf{x}_\ell$) 
    \ENDIF
    \STATE \textbf{end procedure}
  \end{algorithmic}
\end{algorithm}

The procedure starts on the finest level ($\ell\!=\!0$) and repeatedly
coarsens the grid until the coarsest level is reached
(\texttt{maxLevel}) on which a direct solver is used to solve the
problem at hand.  On all other levels $\ell$ the algorithm starts by
presmoothing $S_\ell^{pre}$ the problem to damp high frequency
components of the error (line~5).  Subsequently the fine grid on level
$\ell$ can be restricted with the restriction operator $R_\ell$ to a
coarser grid on level $\ell+ 1$ (line~7).  This essentially
``transfers'' the low frequency components on the fine grid to high
frequency components on the coarse grid.  After the recursion has
reached the coarsest level and used the direct solver to solve the
coarse level problem the solution can be prolongated back to a finer
grid.  This is achieved with the prolongation operator $P_\ell$
(line~10).  Often a postsmoother $S_\ell^{post}$ is used to remove
artifacts caused by the prolongation operator.  Usually these operators
(for every level $\ell$) are defined in a setup phase preceding the
execution of the actual multigrid algorithm.  Lastly, $A_\ell$ denotes
the matrix of the discretized system in level $\ell$. 

%The described 
%algorithm has order $O(n)$ and its convergence should be independent 
%of the grid-spacings. 

%\subsubsection{Multigrid Parameters} \label{sec:mgparams}

%As described in~\cite{oowa:98}, Multigrid methods provide a less
%sensitive convergence behavior to parameter changes when used as a
%preconditioner applied to an iterative solver.  Additionally the
%algebraic Multigrid (AMG) performs better when used as preconditioner
%opposed the application as stand-alone solver.  This provides a
%sufficient motivation to use an AMG (described in this subsection) as
%preconditioner for an iterative solver (CG, BiCGStab).

The performance of multigird methods profoundly depends on the choices and
interplay of the smoothing and restriction operators. To ensure that the
resulting preconditioner is symmetric we use the same pre- and postsmoother
$S_\ell$ and the restriction operator is chosen to be the transpose of the
prolongation operator $R_\ell = P_\ell^T$. This leaves us with two operators,
$P_\ell$ and $S_\ell$, that have to be defined for every level.

%The prolongation and restriction operator can be defined in various
%ways.  When using a Geometric Multigrid (GMG) the restriction and
%prolongation operator is defined by using geometric information about
%the grid.   Another approach is to solely base these operators on the
%algebraic information contained in the system of equations without
%considering the und erlaying grid hierarchy.  This describes the basic
%property of operators utilized in AMG's.  The independence of the
%underlying geometric grid makes the AMG very robust in the presence of
%complicated structured or unstructured grids.  Since we are dealing with
%an anisotropic grid this property makes the AMG a valuable choice,
%saving us a great deal of trouble and additional effort.  Normally when
%applying a GMG to an anisotropic grid special transfer operators are
%needed (i. e. semi-coarsening).  Aside the more costly transfer
%operators we will experience that the standard smoothers applied to an
%anisotropic problem will perform badly.  The reason for the bad
%performance is that the smoother has no averaging effect in directions
%that are anisotropic (or extremely stretched).  Therefore the smoothing
%effect in these direction is inexistent or very weak causing the
%Multigrid method to converge very slowly.  In contrast AMG's work
%efficiently on all error components.  In comparison the convergence rate
%of an AMG is about the same as for a GMG.

%TODO: cite aggregation based methods?
\paragraph{Prolongation Operator $P_\ell$} Aggregation based methods
cluster the fine grid unknowns to aggregates (of a specific form, size,
etc.) as representation for the unknowns on the coarse grid.  First,
each vertex of $G_\ell$, the adjacency graph of $A_\ell$, is assigned to
one of the pairwise disjoint aggregates.  Then, a tentative prolongation
operator matrix is formed where matrix rows correspond to vertices and
matrix columns to aggregates.  A matrix entry $(i,j)$ has a value of $1$
if the $i^{th}$ vertex is contained in $j^{th}$ aggregate and $0$
otherwise.  This prolongation operator basically corresponds to a
piecewise constant interpolation operation.  To improve robustness one
can additionally smooth the tentative prolongation operator. This is
normally done with a damped Jacobi smoother.  In general applying a
smoother results in better interpolation properties opposed to the
piecewise constant polynomials and improves convergence properties.
Tuminaro \& Tong~\cite{tuto:00} propose various strategies how to
parallelize this process.  The simplest strategy is to have each
processor aggregate its portion of the grid.  This method is called
``decoupled'' since the processors act independently of each other.
Usually the aggregates are formed as cubes of $3^d$ vertices in $d$
dimensions.  Since the domains under consideration are close to
rectangular the decoupled scheme seems to be an appropriate strategy.
In the interior of our domain we get regular cubes covering the
specified number of vertices.  Only a few aggregates near subgrid
interfaces and domain boundary contain fewer vertices resulting in a
non-optimal aggregate size.  The overhead introduced is small.  On the
coarsest level each processor holds at least one degree of freedom.

\paragraph{Smoothing Operator $S_\ell$} As advised in~\cite{abht:03} we
choose a Chebyshev polynomial smoother.  The choice is motivated by the
observation that polynomial smoothers perform better than Gauss-Seidel
smoothers in parallel.  Advantages are, e.g., that polynomial smoother
do not need special matrix kernels and formats for optimal performance
and generally polynomial methods can profit of architecture optimized
matrix vector products.  Nevertheless, routines are needed that yield
bounds for the spectrum.

\paragraph{Coarse Level Solver} The employed coarse level solver
(Amesos-KLU) ships the coarse level problem to node~0 and solves it
there by means of an LU factorization.  Once the solution has been
calculated it is broadcast to all nodes.  To gather and scatter data a
substantial amount of communication is required.  Moreover the actual
solve can be expensive if the matrix possesses a large amount of
nonzeros per row.

An alternative is to apply a few steps of an iterative solver (e.g.
Gauss-Seidel) at the coarsest level.  A small number of iteration steps
decreases the quality of the preconditioner and thus increases the PCG
iteration count.  A large number of iteration steps increases the time
for applying the AMG preconditioner.  We found three Gauss-Seidel
iteration steps to be a good choice for our application.


%%% Local Variables: 
%%% mode: latex
%%% TeX-master: "paper"
%%% End: 
