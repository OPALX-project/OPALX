\section{Introduction}
\label{sec:intro}

In recent years, precise beam dynamics simulations in the design of
high-current low-energy hadron machines as well as of 4th generation
light sources have become a very important research topic.  Hadron
machines are characterized by high currents and hence require excellent
control of beam losses and/or keeping the emittance (a measure of the
phase space) of the beam in narrow ranges. This is a challenging problem
which requires the accurate modeling of the dynamics of a large ensemble
of macro or real particles subject to complicated external focusing,
accelerating and wake-fields, as well as the self-fields caused by the
Coulomb interaction of the particles.  In general the geometries of
particle accelerators are large and complicated. The discretization of
the computational domain is time dependent due to the relativistic
nature of the problem.
%%time and space dilatation. 
Both phenomena have a direct impact on the numerical solution method.
% A particular choice of boundary conditions is already worth to mention
% at this point, and is motivated by modeling needs.  All boundary points
% of the third kind (Robin) falling on a planar plane (named inlet and
% outlet in the paper) which is perpendicular to the z-axis as indicated
% in Figure~\ref{fig:domain}.
 
The solver described in this paper is part of a general accelerator
modeling tool \opal~(Object Oriented Parallel Accelerator Library)
\cite{opal}.  \opal\ allows to tackle the most challenging problems in
the field of high precision particle accelerator modeling.  These
include the simulation of high power hadron accelerators and of next
generation light sources.

Some of the effects can be studied by using a low dimensional model,
i.e., envelope equations~\cite{sach:68, sach:71, stru-reis:1984,
  gluckstern1}.  These are a set of ordinary differential equations for
the second-order moments of a time-dependent particle distribution.
They can be calculated fast, however the level of detail is mostly 
insufficient for quantitative studies.  Furthermore, a priori knowledge of
critical beam parameters such as the emittance is required with the consequence that the
envelope equations cannot be used as a self-consistent method.

One way to overcome these limitations is by considering the
Vlasov-Poisson description of the phase space, including external and
self-fields and, if needed, other effects such as wakes.  To that end
let $f(\mathbf{x},\mathbf{v},t)$ be the density of the particles in the
phase space, i.e., the position-velocity $(\mathbf{x}, \mathbf{v})$
space.  Its evolution is determined by the collisionless \emph{Vlasov
  equation},
%%
% Neglecting
% particles collisions, because typical bunch densities are low, the
% evolution of the beam's distribution function
% $f(\mathbf{x},\mathbf{v},t)$ can be considered as a Cauchy problem.
% Denoting the computational domain $\Omega \subset \Re^d$ with $d=2,3$,
% and using $\mathbf {x} \in \Re^d$ to describe a position in
% configuration and $\mathbf {v} \in \Re^d$ a point in velocity space, the
% collision-less Vlasov-Poisson equation can be expressed by
\begin{equation}\label{eq:Vlasov}
  \frac{df}{dt}=\partial_t f + \mathbf{v} \cdot \nabla_{\mathbf{x}} f
  +\frac{q}{m_0}(\mathbf{E}+ \mathbf{v}\times\mathbf{B})\cdot
  \nabla_{\mathbf{v}} f  =  0,
\end{equation}
where $m_0$, $q$ denote particle mass and charge, respectively.  The
electric and magnetic fields $\mathbf{E}$ and $\mathbf{B}$ are
superpositions of external fields and self-fields (space charge),
%%and other collective effects such as wake fields,
\begin{equation}\label{eq:allfield}
%%  \begin{aligned}
    \mathbf{E} =
    \mathbf{E_{\RM{ext}}} + \mathbf{E_{\RM{self}}}, \quad
    \mathbf{B} =
    \mathbf{B_{\RM{ext}}} + \mathbf{B_{\RM{self}}}.
%%  \end{aligned}
\end{equation}
%%
If $\mathbf{E}$ and $\mathbf{B}$ are known, then each particle can be
propagated according to the equation of motion for charged particles in an
electromagnetic field,
\begin{equation*}\label{eq:motion}
  \frac{d\mathbf{x}(t)}{dt}  = \mathbf{v},
  \quad
  \frac{d\mathbf{v}(t)}{dt}  = \frac{q}{m_0}\left(\mathbf{E} +
    \mathbf{v}\times \mathbf{B}\right).
\end{equation*}
%%which is numerically integrated for each particle.  
%%
% We assume for the rest of the paper that the external fields and the
% wake fields are known and furthermore treat the physical system
% electrostatic (which we obtain without loss of generality by an
% appropriate Lorentz transformation).

After the movement of the particles 
$\mathbf{E_{\RM{self}}}$ and $\mathbf{B_{\RM{self}}}$ have to be updated.  
%%
To that end we change the coordinate system into the one moving with the
particles.  By means of the appropriate \emph{Lorentz
  transformation}~\cite{lali:84} we arrive at a (quasi-) static
approximation of the system in which the transformed magnetic field
becomes negligible, $\hat{\mathbf{B}}\! \approx\! \mathbf{0}$.  The
transformed electric field is obtained from
\begin{equation}\label{eq:e-field}
  \hat{\mathbf{E}}=\hat{\mathbf{E}}_{\RM{self}}=-\nabla\hat{\phi},
\end{equation}
where the electrostatic potential $\hat{\phi}$ is the solution of the
\emph{Poisson problem}
\begin{equation}\label{eq:poisson0}
  - \Delta \hat{\phi}(\mathbf{x}) =
  \frac{\hat{\rho}(\mathbf{x})}{\varepsilon_0},
\end{equation}
equipped with appropriate boundary conditions, see
section~\ref{sec:discr}.  Here, $\hat{\rho}$ denotes the spatial charge
density and $\varepsilon_0$ is the dielectric constant.
% in the (beam rest) frame
% If we can make an appropriate Lorentz transformation to arrive at a
% (quasi-)static approximation of the system, the space charge fields can
% be obtained by solving a Poisson problem with appropriate boundary
% conditions:
% \begin{equation}\label{eq:Poisson}
%   - \nabla^{2} \phi(\mathbf{x}) = \frac{\rho(\mathbf{x})}{\varepsilon_0},
% \end{equation}
% where $\phi$ and $\rho$ are the electrostatic potential and the spatial
% charge density in the appropriate (beam rest) frame.  The electric field
% can then be calculated by
% \begin{equation}\label{eq:Efield}
%   \mathbf{E}=\mathbf{E_{\RM{self}}}=-\nabla\phi.
% \end{equation}
By means of the inverse Lorentz transformation the electric field
$\hat{\mathbf{E}}$ can then be transformed back to yield both the
electric and the magnetic fields in~\eqref{eq:allfield}.
% corresponding to the static grid.  

The Poisson problem~\eqref{eq:poisson0} discretized by finite
differences can efficiently be solved on a rectangular grid by a
Particle-In-Cell (PIC) approach~\cite{qiry:01}.  The right hand side
in~\eqref{eq:poisson0} is discretized by sampling the particles at the
grid points.  In~\eqref{eq:e-field}, $\hat{\phi}$ is interpolated at the
particle positions from its values at the grid points. We also note that
the FFT-based Poisson solvers and similar
approaches~\cite{qiry:01,qigl:04} are restricted to box-shaped or open domains.

Serafini et al.~\cite{serafini_2005} report on a state-of-the-art
conventional FFT-based algorithm for solving the Poisson equation with
`infinite-domain', i.e., open boundary conditions for large problems in
accelerator modeling.  The authors show improvements in both accuracy
and performance, by combining several techniques: the method of local
corrections, the James algorithm, and adaptive mesh refinement.

However with the quest of high intensity, high brightness beams together
with ultra low particle losses, there is a high demand to consider the
true geometry of the beam-pipe in the numerical model. This assures that
the image charge components are taken properly into account. This
results in a more exact modeling of the non-linear beam dynamics which
is indispensable for the next generation of particle accelerators.

In a related paper by P{\"o}plau et
al.~\cite{poplau_self-adaptive_2008}, an iterative solver preconditioned
by geometric multigrid is used to calculate space-charge forces.  The
authors employ a mesh with adaptive spacings to reduce the workload of
the BiCGStab solver used to solve the nonsymmetric system arising from
quadratic extrapolation at the boundary.  The geometric multigrid solver
used in their approach is much more sensitive to anisotropic grids
arising in beam dynamic simulations (e.g.\ special coarsening operators
have to be defined).  With smoothed aggregation-based algebraic multigrid
(AMG) preconditioning as used in this paper the aggregation smoother takes
care of anisotropies and related issues and leads to a robustness superior
to geometric multigrid, see~\cite{trcl:09} for a discussion.  The
preconditioner easily adapts to the elongation  of the computational domain
that happens during our simulation.

In Section~\ref{sec:discr} we describe how the Poisson equation on a
`general' domain $\Omega \subset \Re^3$ can be solved by finite
differences and the PIC approach.  We treat the boundary in three
different ways, by constant, by linear, and by quadratic extrapolation,
the latter being similar to the approach of McCorquodale \textit{et
  al.}~\cite{mcgv:04}.  The system of equations is solved by the
conjugate gradient algorithm preconditioned by smoothed
aggregation-based algebraic multigrid (SA-AMG)~\cite{vamb:96a, tuto:00},
see Section~\ref{sec:method}.  The preconditioned conjugate
gradient~(PCG) algorithm is also used if the system is `mildly'
nonsymmetric.  In Section~\ref{sec:impl} we deal with details of the
implementation, in particular its parallelization.  In
Section~\ref{sec:results} we report on numerical experiments including a
physical application from beam dynamics.  In Section~\ref{sec:concl} we
draw our conclusions.
% , and thanks to
% high performance computing a three-dimensional (3D) model with
% sufficient resolution is feasible.

% A very attractive approach to solve the Poisson equation is by a fast
% direct solver based on FFT's~\cite{qiry:01}.  However, this approach is
% limited to simple geometries such as boxes and cylinders.  Other fast
% and accurate methods are known~\cite{qigl:04} but they are also limited
% to simple geometries.  Because $\rho$ is varying slowly in
% time~\eqref{eq:poisson0}, it is desirable to use solutions from previous
% time step(s).  In the sequel of the paper we consider iterative solvers
% for~\eqref{eq:poisson0} on general but simply connected geometries
% $\Omega \subset \Re^d$.

%%% Local Variables: 
%%% mode: latex
%%% TeX-master: "paper"
%%% End: 

%+ motivate irregular boundaries

