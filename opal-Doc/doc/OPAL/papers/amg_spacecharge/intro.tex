\section{Introduction}
\label{sec:intro}

In recent years, precise beam dynamics calculations in high-current
low-energy hadron machines and in the design of 4th generation light
sources have become a very important research subject.  Hadron machines
are characterized by high currents and hence require excellent control
of beam losses and/or keeping the emittance (a measure of the phase
space) of the beam in narrow ranges.  This is a challenging problem
which requires the accurate modeling of the dynamics of a large ensemble
of macro or real particles subject to complicated external focusing,
accelerating and wake-fields, as well as the self-fields caused by
Coulomb interaction of the particles.  In general the geometries of
particle accelerators are large and complicated which has a direct
impact on the numerical solution method.

Some of the effects can be studied by using a low dimensional model
i.e.\ envelope equations~\cite{Sacherer68, Sa71, stru-reis:1984,
  gluckstern1}.  These are a set of ordinary differential equations for
the second-order moments of a time-dependent particle distribution.
They can be calculated fast, however the level of detail is mostly not
sufficient for quantitative studies.  Furthermore, a priori knowledge of
critical beam parameters such as the emittance is required such that
envelope equations cannot be used as a self-consistent method.

One way to overcome these limitations is by considering the
Vlasov-Poisson description of the phase space, including external and
self-fields and if needed other effects such as wakes.  To that end let
$f(\mathbf{x},\mathbf{v},t)$ be the density of the particles in the
phase space (i.e.\ the position-velocity $(\mathbf{x}, \mathbf{v})$
space).  Its evolution is determined by the collision-less \emph{Vlasov
  equation}.
%%
% Neglecting
% particles collisions, because typical bunch densities are low, the
% evolution of the beam's distribution function
% $f(\mathbf{x},\mathbf{v},t)$ can be considered as a Cauchy problem.
% Denoting the computational domain $\Omega \subset \Re^d$ with $d=2,3$,
% and using $\mathbf {x} \in \Re^d$ to describe a position in
% configuration and $\mathbf {v} \in \Re^d$ a point in velocity space, the
% collision-less Vlasov-Poisson equation can be expressed by
\begin{equation}\label{eq:Vlasov}
  \frac{df}{dt}=\partial_t f + \mathbf{v} \cdot \nabla_{\mathbf{x}} f
  +\frac{q}{m_0}(\mathbf{E}+ \mathbf{v}\times\mathbf{B})\cdot
  \nabla_{\mathbf{v}} f  =  0,
\end{equation}
where $m_0$, $q$ denote particle mass and charge.  The electric and
magnetic fields $\mathbf{E}$ and $\mathbf{B}$ are superpositions of
external fields and space charge fields,
%%and other collective effects such as wake fields,
\begin{equation}\label{eq:allfield}
%%  \begin{aligned}
    \mathbf{E} =
    \mathbf{E_{\RM{ext}}} + \mathbf{E_{\RM{self}}}, \quad
    \mathbf{B} =
    \mathbf{B_{\RM{ext}}} + \mathbf{B_{\RM{self}}}.
%%  \end{aligned}
\end{equation}
%%
If $\mathbf{E}$ and $\mathbf{B}$ are known, then each particle can be
moved according to the equation of motion for charged particles in an
electromagnetic field,
\begin{equation*}\label{eq:motion}
  \frac{d\mathbf{x}(t)}{dt}  = \mathbf{v},
  \quad
  \frac{d\mathbf{v}(t)}{dt}  = \frac{q}{m_0}\left(\mathbf{E} +
    \mathbf{v}\times \mathbf{B}\right).
\end{equation*}
%%which is numerically integrated for each particle.  

% We assume for the rest of the paper that the external fields and the
% wake fields are known and furthermore treat the physical system
% electrostatic (which we obtain without loss of generality by an
% appropriate Lorentz transformation).

After the movement of the particles the space charge portions of
$\mathbf{E}$ and $\mathbf{B}$ have to updated.  

To that end we change the coordinate system into the one moving with the
particles.  By means of the appropriate Lorentz
transformation~\cite{lali:84} we arrive at a (quasi-) static
approximation of the system in which the transformed magnetic field
becomes negligible, $\hat{\mathbf{B}}\! \approx\! \mathbf{0}$.  The
transformed electric field is obtained by
\begin{equation}\label{eq:e-field}
  \hat{\mathbf{E}}=\hat{\mathbf{E}}_{\RM{self}}=-\nabla\hat{\phi}
\end{equation}
where the electrostatic potential $\hat{\phi}$ is the solution of the
\emph{Poisson problem}
\begin{equation}\label{eq:poisson0}
  - \Delta \hat{\phi}(\mathbf{x}) =
  \frac{\hat{\rho}(\mathbf{x})}{\varepsilon_0},
\end{equation}
equipped with appropriate boundary conditions, see
section~\ref{sec:method}.  Here, $\hat{\rho}$ denotes the spatial charge
density and $\varepsilon_0$ is the dielectric constant.
% in the (beam rest) frame
% If we can make an appropriate Lorentz transformation to arrive at a
% (quasi-)static approximation of the system, the space charge fields can
% be obtained by solving a Poisson problem with appropriate boundary
% conditions:
% \begin{equation}\label{eq:Poisson}
%   - \nabla^{2} \phi(\mathbf{x}) = \frac{\rho(\mathbf{x})}{\varepsilon_0},
% \end{equation}
% where $\phi$ and $\rho$ are the electrostatic potential and the spatial
% charge density in the appropriate (beam rest) frame.  The electric field
% can then be calculated by
% \begin{equation}\label{eq:Efield}
%   \mathbf{E}=\mathbf{E_{\RM{self}}}=-\nabla\phi.
% \end{equation}
By means of the inverse Lorentz transformation the electric field
$\hat{\mathbf{E}}$ can then be transformed back to yield both the
electric and the magnetic fields in~\eqref{eq:allfield}.  

The Poisson problem~\eqref{eq:poisson0} discretized by finite
differences can efficiently be solved on a rectangular grid by a
particle-in-cell (PIC) approach~\cite{qiry01}.  The right hand side
in~\eqref{eq:poisson0} is obtained by sampling the particles at the grid
points.  In~\eqref{eq:e-field}, $\hat{\phi}$ is interpolated at the
particle positions from its values at the grid points.

The FFT-based Poisson solvers and similar approaches~\cite{qiry01,
  qigl04} are restricted to boxed-shaped domains.  In
Section~\ref{sec:method} we describe how the Poisson equation on a
``general'' domain $\Omega \subset \Re^3$ can be solved by finite
differences and the PIC approach.  We treat the boundary in a similar
way as McCorquodale \textit{et al.}~\cite{mcgv:04}.  The system of
equation is solved by the conjugate gradient algorithm preconditioned by
smoothed aggregation AMG.  In Section~\ref{sec:impl} we deal with
details of the implementation, in particular its parallelization.  In
Section~\ref{sec:results} we report on numerical experiments that we
conducted on the Cray XT/4 at the Swiss Supercomputing Center CSCS.
In Section~\ref{sec:concl} we draw our conclusions.

% , and thanks to
% high performance computing a three-dimensional (3D) model with
% sufficient resolution is feasible.

% A very attractive approach to solve the Poisson equation is by a fast
% direct solver based on FFT's~\cite{qiry01}.  However, this approach is
% limited to simple geometries such as boxes and cylinders.  Other fast
% and accurate methods are known~\cite{qigl04} but they are also limited
% to simple geometries.  Because $\rho$ is varying slowly in
% time~\eqref{eq:poisson0}, it is desirable to use solutions from previous
% time step(s).  In the sequel of the paper we consider iterative solvers
% for~\eqref{eq:poisson0} on general but simply connected geometries
% $\Omega \subset \Re^d$.

%%% Local Variables: 
%%% mode: latex
%%% TeX-master: "paper"
%%% End: 

%+ motivate irregular boundaries

