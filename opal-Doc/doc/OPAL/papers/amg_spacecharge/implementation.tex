\section{Implementation details}
\label{sec:impl}

The multigrid preconditioner and iterative solver are implemented with
the help of the Trilinos framework~\cite{Trilinos-Web-Site,
  Trilinos-TOMS}.  Trilinos provides state-of-the-art tools for
numerical computation in various packages.  Aztec, e.g., provides
iterative solvers and ML~\cite{gsht:06} multilevel preconditioners.  By
means of ML, we created our smoothed aggregation-based AMG
preconditioner.  The essential parameters of the preconditioner
discussed above are listed in Table~\ref{tab:sa_setup}.
%%
\begin{table}[htb]
  \begin{center}
    \begin{tabular}{l|l}
      \hline
      name & value \\
      \hline
      preconditioner type & MGV \\
      aggregation type & Uncoupled \\
      smoother type & Chebyshev \\
      smoother & pre and post \\
      coarse level solver & Amesos-KLU \\
      maximal coarse level size & 1000 \\
      \hline
    \end{tabular}
    \caption{Parameters for multilevel
      preconditioner ML}
    \label{tab:sa_setup}
  \end{center}
\end{table}

To embed the solver in the physical simulation code (\opal~\cite{opal})
we utilized the Independent Parallel Particle Layer (\ippl~\cite{ippl}).
This library is an object-oriented framework for particle based
applications in computational science designed for the use on
high-performance parallel computers.  In the context of this paper
\ippl\ is only relevant because \opal\ uses \ippl\ to represent and
interpolate the particles at grid points with a charge conserving
Cloud-in-Cell area weighting scheme.  ML requires an
\texttt{Epetra\_Map} handling the parallel decomposition to create
parallel distributed matrices and vectors. To avoid additional
communication the \texttt{Epetra\_Map} and the \ippl\ field are determined
to have the same parallel decomposition. In this special case the task
of converting the \ippl\ field to an \texttt{Epetra\_Vector} is as
simple as looping over local indices and assigning values.  

A particle based domain decomposition technique based on recursive coordinate
bisection is used (in section \ref{sec:physrun}) to parallelize the computation
on a distributed memory environment.  One, two and three-dimensional
decompositions are available.  For problems of beam dynamics with highly
nonuniform and time dependent particle distributions, a dynamic load balancing
is necessary to preserve the parallel efficiency of the particle integration
scheme.  Here, we rely on the fact that \ippl\ attains a good load balance of
the data.

%{\color{red}TODO: distribution scheme.}

%Since we are using ML with Epetra vectors and matrices it is necessary to
%convert IPPL grids to Epetra vectors and visa versa.  Epetra uses maps to
%describe the parallel distribution of vectors and matrices.  To avoid any
%communication the distribution of Epetra vectors correspond to the
%distribution of IPPL grid points.  Once both representations have the
%%same parallel layout the task of converting one into the other becomes
%simple.  Here we rely on the fact that IPPL already attained a good load
%balancing of the data.  {\color{red}TODO: distribution scheme.}

We use the solution of one time step as the initial guess for the next
time step.

% We store the solution vector so that it can be reused as an initial
% guess for the next execution of the iterative solver in the
% consecutive time-step.

\subsection{Domains}

The simplest domains under consideration are regular, rectangular
domains.  This domains are used by the FFT solver with so-called
open-space boundary conditions as described in \cite{hoea:88}. 

%{\color{red}These are Dirichlet
%  boundary conditions where the boundary values are obtained from the
 % Green's function with a pole of strength ??? at ???.  REFERENCE
 % needed!}
% \begin{equation*}
%   \phi \rightarrow 0 \; \text{ as } \; \vert \mathbf{x} \vert
%   \rightarrow \infty.
% \end{equation*}
% {\color{red}(how do you derive from this the boundary conditions on the cube?)}

Our \oursolver\ solver is not restricted to rectangular domains.  It can handle
irregular domains as the ones introduced in the next section.

\subsection*{Non-rectangular domains}

To properly handle emerging irregular domains we implemented an abstract
class providing an interface to query the discretization near the
boundary.  Every implementation of an irregular domain has to identify
boundary points and provide the stencil for \textit{near-boundary}
points given one of the extrapolation schemes discussed in
Section~\ref{sec:discr}.  Boundary points are stored in \texttt{STL}
containers. Essentially the coordinate value of a gridline is mapped to
its intersection values, providing a fast look-up table for a given
gridline.

In this work we use the \oursolver\ solver mainly for cylindrical domains with
an elliptic base area.  These domains can be characterized by means of
two parameters: the semi-major and semi-minor axis.  We compute the
intersection points of the grid with the elliptical domain boundary by
using its implicit representation and subsequently store them into a
\texttt{STL} container.  These intersections have to be recomputed
whenever the parameters of the ellipse or the mesh spacings change.

%For the rest of this paper the ``default'' domain for AMG is considered to be
%cylindrical.

%\subsubsection*{Arbitrary Domains}

%TODO: finish and refine
%In the following we will define an arbitrary boundary as a continuous,
%bijective, computable function $r=f(z, \theta)$ where $z$ is the coordinate of
%the cross-section and $\theta \in [0,2\pi]$ denotes the angle {\color{red}TODO:
%def angle}. Consequently the boundary can take arbitrary shapes as long as the
%properties of $f$ are retained. Therefore each $x$ and $y$ gridline has
%exactly two intersection points $(r_1,r_2)$ with the boundary at a given
%cross-section $z$ and $(\theta_1, \theta_2)$.

%Compared to the cylindrical domain the boundary treatment for arbitrary
%domains is slightly more complex. The difficulties mainly arise because
%we possess no algebraic closed from describing the boundary. To this end
%we generate surface meshes of beam line elements, like, i.e. cavities and
%solenoids.

%Subsequently OPAL has to read surface meshes and compute the boundary points by
%intersecting gridlines and surface mesh. Both tasks require a good share of
%cycles (computing time) and memory. Once the mesh has loaded the intersections
%can be computed and stored. This is accomplished when creating the boundary
%object by checking if and where a given grid line intersects a triangle of the
%surface mesh. The utilized algorithm is described to a great extent
%in~\cite{motr:97}.
%{\color{red}TODO: more details}

%, are
%designed in a CAD like software product that can export STEP files.
%With the help of \textsc{Heronion}~\cite{oswald2007c}, a surface mesh
%can be computed from STEP files which subsequently can be dumped in
%\textsc{H5Fed} \cite{H5FED-Web} files.

%In a second step the triangular surface mesh (in form of a
%\textsc{H5Fed} file) has to be imported in OPAL and the intersection of
%the grid with the surface mesh has to be computed.  This is done by an
%algorithm~\cite{motr:97} using the barycentric coordinate system,
%essentially shifting the triangle to the origin and projecting it down
%onto a plane.  Once the triangle is in this position its very easy to
%determine if a grid-line intersects the triangle and to compute the
%intersection.

%%% Local Variables: 
%%% mode: latex
%%% TeX-master: "paper"
%%% End: 
