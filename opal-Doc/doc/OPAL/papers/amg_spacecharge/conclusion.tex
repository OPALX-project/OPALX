\section{Conclusion}
\label{sec:concl}

We have presented a scalable Poisson solver suitable to handle domains
with irregular boundaries as they arise, for example, in beam dynamics
simulations.  The solver employs the conjugate gradient algorithm
preconditioned by smoothed aggregation based AMG (SAAMG-PCG).  PEC and
open-space boundary approximations have been discussed.  A real world
example where the solver was used in a beam dynamics code (OPAL) shows
the relevance of this approach by observing up to $~40\%$ difference in
the RMS beam size when comparing to the FFT-based solver with open
domains.  The code exhibits excellent scalability up to 2048 processors
with cylindrial tubes embedded in meshes with up to $1024^3$ grid
points.  In the very near future, this approach will enable precise beam
dynamics simulations in large particle accelerator structures with a
level of detail not obtained before.

In real particle simulations (and other test cases encountered) system
matrices arising from quadratic boundary treatment are only `mildly'
nonsymmetric such that PCG can be applied.

Planed future work includes adaptive mesh refinement in order to reduce
the number of grid points in regions that are less relevant for the
space charge calculation.  The boundary treatment for simply connected
geometries will be extended to cope with more realistic geometries.  This
new method, designed for accurate 3D space charge calculations, will be
used in the beam dynamics simulations for the SwissFEL project, a next
generation light source foreseen to be built in Switzerland.

%%% Local Variables: 
%%% mode: latex
%%% TeX-master: "paper"
%%% End: 
