\section{Conclusion}
\label{sec:concl}

We have presented a scalable Poisson solver suitable to handle domains
with irregular boundaries as they arise, for example, in beam dynamics
simulations.  PEC and open-space boundary approximations have been
discussed.  A real world example where the solver was used in an beam
dynamics code (OPAL) shows the relevance of this approach.
{\color{red}Why?}  The code exhibits excellent scalability up to 2048
processors with cylindrial tubes embedded in meshes with up to $1024^3$
nodes.  This will enable precise beam dynamics simulations of a new
level of detail in the very near future.

In real particle simulations (and other test cases encountered) system
matrices are only mildly nonsymmetric such that PCG could be applied..
{\color{red}Shortley-Weller bc}

Variants reducing time to solution..

\subsection{Future work}

We already possess a boundary treatment for simply connected geometries.
In the future we will conduct the same benchmarks mentioned here to
measure the performance of ``arbitrary'' boundaries.  An interesting
study will concern the effect on the physics of a simulation run or even
studies requiring a larger level of detail.

Another improvement will be to implement adaptive mesh refinement to
reduce the number of gridpoints in regions less relevant for
space-charge calculations.

Study: meshsize...

%%% Local Variables: 
%%% mode: latex
%%% TeX-master: "paper"
%%% End: 
