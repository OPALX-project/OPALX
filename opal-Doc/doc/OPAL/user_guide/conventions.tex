\chapter{Conventions}

\label{chp:definitions}
\index{conventions|(}
\index{global!reference}
\index{reference!global}
The accelerator and/or beam line to be studied with \opalmap is described as
a sequence of beam elements placed sequentially along a reference or
design orbit.
The global reference orbit, also known as the \figref{design orbit}{global} 
is the path of a charged particle having the central design momentum 
of the accelerator through idealised magnets with no fringe fields.

In case of \opalt and \opalcycl with time as the independent variable no such 
explicit design orbit exists. The rest of this chapter is only relevant to 
\opalmap except Section \ref{sec:units} on the physical units. 


\section{Design or Reference Orbit}
\index{design!orbit}
\index{orbit!design}
\index{reference!orbit}
\index{orbit!reference}
The reference orbit consists of a series of
straight line segments and circular arcs.
It is defined under the assumption that all elements are
perfectly aligned along the design orbit.
The accompanying tripod of the reference orbit spans
a local curvilinear right handed coordinate system $(x,y,s)$.
The local $s$-axis is the tangent to the reference orbit.
The other two axes are perpendicular to the reference orbit and
are labelled~$x$ (in the bend plane)
and~$y$ (perpendicular to the bend plane).

\section{Closed Orbit}
\index{closed!orbit}
\index{orbit!closed}
\index{misalignment}
\index{field!error}
\index{error!field}
\index{error!misalignment}
\index{fringe field}
\index{field!fringe}
Due to various errors like misalignment errors, field errors,
fringe fields etc.,
the closed orbit does not coincide with the design orbit.
It also changes with the momentum error.
The closed orbit is described with respect to the reference orbit, 
using the \figref{local reference system}{local}.
It is evaluated including any nonlinear effects.

\begin{figure}[ht]
  \begin{center}
    \setlength{\unitlength}{1pt}
    \begin{picture}(400,250)
                                % axes
      \thicklines
      \put(180,100){\vector(0,1){90}}
      \put(180,0){\line(0,1){33.3}}
      \put(187,185){\makebox(0,0){$y$}}
      \put(280,0){\vector(-1,1){160}}
      \put(120,150){\makebox(0,0){$x$}}
      \thinlines
      \put(180,100){\circle*{4}}
                                % coordinates
      \put(150,130){\line(0,1){60}}
      \put(155,125){\line(0,1){60}}
      \put(160,120){\line(0,1){60}}
      \put(165,115){\line(0,1){60}}
      \put(170,110){\line(0,1){60}}
      \put(175,105){\line(0,1){60}}
      \put(180,160){\line(-1,1){30}}
      \put(180,150){\line(-1,1){30}}
      \put(180,140){\line(-1,1){30}}
      \put(180,130){\line(-1,1){30}}
      \put(180,120){\line(-1,1){30}}
      \put(180,110){\line(-1,1){30}}
                                % radius of curvature and centre
      \put(280,0){\vector(-3,1){172}}
      \put(160,30){\makebox(0,0){$\rho$}}
      \put(280,0){\vector(0,1){142}}
      \put(290,60){\makebox(0,0){$\rho$}}
      \put(280,0){\circle*{4}}
      \put(295,15){\vector(-1,-1){12}}
      \put(295,15){\makebox(0,0)[bl]{\shortstack{centre of \\curvature}}}
      \path(270,3.3)(280,9)
      \path(260,6.7)(280,18)
      \path(250,10)(280,27)
      \path(240,13.3)(280,36)
      \path(230,16.7)(280,45)
      \path(220,20)(280,54)
      \path(210,23.3)(280,63)
      \path(200,26.7)(280,72)
      \path(190,30)(280,81)
      \path(180,33.3)(280,90)
      \path(170,36.7)(280,99)
      \path(160,40)(280,108)
      \path(150,43.3)(280,117)
      \path(140,46.7)(280,126)
      \path(130,50)(280,135)
      \path(120,53.3)(278,142)
      \path(110,56.7)(218,117.9)
                                % actual orbit
      \thicklines
      \bezier{150}(0,100)(75,165)(150,190)
      \bezier{150}(150,190)(240,220)(320,220)
      \put(320,220){\vector(1,0){4}}
      \put(80,180){\vector(1,-1){12}}
      \put(80,180){\makebox(0,0)[br]{\shortstack{actual \\orbit}}}
                                % reference orbit
      \bezier{150}(40,0)(100,60)(180,100)
      \bezier{150}(180,100)(260,140)(340,160)
      \put(340,160){\vector(4,1){4}}
      \put(330,165){\makebox{$s$}}
      \put(320,135){\vector(-1,1){12}}
      \put(320,135){\makebox(0,0)[tl]{\shortstack{reference \\orbit}}}
    \end{picture}
    \caption{Local Reference System}
    \label{fig:local}
  \end{center}
\end{figure}

\index{betatron motion}
\index{synchrotron motion}
\opalmap computes the betatron and synchrotron oscillations
with respect to the closed orbit.
Results are given in the local $(x, y, s)$-system
defined by the reference orbit.

\section{Global Reference System}
\index{global!reference}
\index{reference!global}

The \figref{global reference orbit}{global} 
of the accelerator is uniquely defined by the sequence of physical elements.
The local reference system $(x, y, s)$
may thus be related to a global Cartesian coordinate system $(X, Y, Z)$.
The positions between beam elements are numbered $0, \ldots , i, \ldots n$.
The local reference system $(x_i, y_i, s_i)$ at position $i$,
i.e.\ the displacement and direction of the reference orbit
with respect to the system $(X, Y, Z)$ are defined by three displacements 
$(X_i, Y_i, Z_i)$ and three angles $(\Theta_i, \Phi_i, \Psi_i)$.

\begin{figure}[ht]%                                         1.2
  \begin{center}
    \setlength{\unitlength}{1pt}
    \begin{picture}(400,270)
                                % global axes
      \thicklines
      \put(20,150){\line(2,-1){40}}
      \dashline{3}(60,130)(100,110)
      \put(100,110){\line(2,-1){60}}
      \put(193,63){\vector(2,-1){90}}
      \put(300,20){\makebox(0,0){$Z$}}
      \put(20,100){\line(3,1){30}}
      \dashline{3}(50,110)(248,176)
      \put(248,176){\vector(3,1){132}}
      \put(373,210){\makebox(0,0){$X$}}
      \put(80,0){\vector(0,1){270}}
      \put(70,265){\makebox(0,0){$Y$}}
                                %local axes
      \put(133.3,0){\vector(1,3){90}}
      \put(215,270){\makebox(0,0){$x$}}
      \put(300,150){\vector(-2,1){160}}
      \put(140,220){\makebox(0,0){$y$}}
      \put(0,100){\vector(2,1){270}}
      \put(260,240){\makebox(0,0){$s$}}
                                % projection of s onto ZX
      \thinlines
      \put(100,110){\circle*{4}}
      \put(200,200){\circle*{4}}
      \path(0,110)(170,110)
      \dashline{3}(170,110)(240,110)
      \path(240,110)(290,110)
      \put(280,90){\makebox(0,0)[t]{\shortstack{projection of $s$ \\
            onto $(Z,X)$-plane}}}
      \put(280,95){\vector(0,1){13}}
      \put(20,110){\circle*{4}}
      \put(50,110){\circle*{4}}
                                % hashing of xy plane
      \path(210,195)(160,45)
      \path(220,190)(169.7,39.7)
      \path(230,185)(187.5,57.5)
      \path(240,180)(205,75)
      \path(250,175)(222.5,92.5)
      \path(260,170)(240,110)
      \path(270,165)(257.5,127.5)
      \path(280,160)(275,145)

      \path(196.7,190)(278.9,148.9)
      \path(193.3,180)(271.1,141.1)
      \path(190,170)(263.3,133.3)
      \path(186.7,160)(255.5,125.5)
      \path(183.3,150)(247.7,117.7)
      \path(180,140)(239.9,109.9)
      \path(176.7,130)(232.1,101.1)
      \path(173.3,120)(224.3,94.3)
      \path(170,110)(216.5,86.5)
      \path(166.7,100)(208.7,78.7)
      \path(163.3,90)(200.9,70.9)
      \path(160,80)(193.1,63.1)
      \path(156.7,70)(185.3,55.3)
      \path(153.3,60)(177.5,47.5)
      \path(150,50)(169.7,39.7)
                                % hashing of projection plane
      \put(70,110){\line(0,1){25}}
      \put(80,110){\line(0,1){30}}
      \put(90,110){\line(0,1){35}}
      \put(100,110){\line(0,1){40}}
      \put(110,110){\line(0,1){45}}
      \put(120,110){\line(0,1){50}}
      \put(130,110){\line(0,1){55}}
      \put(140,110){\line(0,1){60}}
      \put(150,110){\line(0,1){65}}
      \put(160,110){\line(0,1){70}}
      \put(170,110){\line(0,1){75}}
      \put(180,140){\line(0,1){50}}
      \put(190,170){\line(0,1){25}}
                                % intersection of xy and ZX
      \put(130,0){\line(1,1){230}}
      \put(135,5){\circle*{4}}
      \put(193.3,63.3){\circle*{4}}
      \put(240,110){\circle*{4}}
      \put(286.7,156.7){\circle*{4}}
      \put(335,205){\circle*{4}}
      \thicklines
      \put(200,25){\vector(-1,1){20}}
      \put(200,20){\makebox(0,0)[t]{\shortstack{intersection of \\
            $(x,y)$ and $(Z,X)$ planes}}}
                                % reference orbit
      \bezier{80}(140,150)(170,185)(200,200)
      \bezier{80}(200,200)(230,215)(260,220)
      \put(260,220){\makebox(0,0)[l]{\shortstack{reference \\orbit}}}
                                % roll angle
      \bezier{30}(160,30)(160,40)(150,50)
      \put(152,48){\vector(-1,1){2}}
      \put(150,30){\makebox(0,0){$\psi$}}
      \put(140,30){\makebox(0,0)[br]{roll angle}}
                                % pitch angle
      \bezier{20}(60,110)(60,120)(55,125)
      \put(57,123){\vector(-1,2){2}}
      \put(50,118){\makebox(0,0){$\phi$}}
      \put(40,125){\makebox(0,0)[br]{pitch angle}}
                                % azimuth
      \bezier{20}(130,95)(140,100)(140,110)
      \put(140,105){\vector(0,1){5}}
      \put(130,105){\makebox(0,0){$\theta$}}
      \put(115,95){\makebox(0,0)[t]{azimuth}}
    \end{picture}
    \caption{Global Reference System}
    \label{fig:global}
  \end{center}
\end{figure}

\index{displacement}
The above quantities are defined more precisely as follows:
\begin{description}
\item[X]
  \index{horizontal displacement}
  Displacement of the local origin in $X$-direction.
\item[Y]
  \index{vertical displacement}
  Displacement of the local origin in $Y$-direction.
\item[Z]
  \index{longitudinal displacement}
  Displacement of the local origin in $Z$-direction.
  \index{angle}
\item[THETA]
  \index{azimuth angle}
  Angle of rotation (azimuth) about the global $Y$-axis,
  between the global $Z$-axis and the projection
  of the reference orbit onto the $(Z, X)$-plane.
  A positive angle \texttt{THETA} forms a right-hand screw with the $Y$-axis.
\item[PHI]
  \index{pitch angle}
  Pitch angle, i.e. the angle between the reference orbit and its projection
  onto the $(Z, X)$-plane.
  A positive angle \texttt{PHI} correspond to $Y$ increasing with $s$.
  If only horizontal bends are present,
  the reference orbit remains in the ($Z$, $X$)-plane.
  In this case \texttt{PHI} is always zero.
\item[PSI]
  \index{roll angle}
  Roll angle about the local $s$-axis,
  i.e. the angle between the intersection $(x, y)$- and
  $(Z, X)$-planes and the local $x$-axis.
  A positive angle \texttt{PSI} forms a right-hand screw with the $s$-axis.
\end{description}
The angles \texttt{(THETA, PHI, PSI)} are \textbf{not} the Euler angles.
The reference orbit starts at the origin and points by default
in the direction of the positive $Z$-axis.
The initial local axes $(x, y, s)$ 
coincide with the global axes $(X, Y, Z)$ in this order.
The six quantities $(X_0, Y_0, Z_0, \Theta_0, \Phi_0, \Psi_0)$
thus all have zero initial values by default.
The program user may however specify different initial conditions.

Internally the displacement is described by a vector $V$
and the orientation by a unitary matrix $W$.
The column vectors of $W$ are the unit vectors spanning 
the local coordinate axes in the order $(x, y, s)$.
$V$ and $W$ have the values:
\[
V=\left(\begin{array}{c}
    X \\
    Y \\
    Z
  \end{array}\right),
\qquad
W=\Theta\Phi\Psi
\]
where
\[
\Theta=\left(\begin{array}{ccc}
    \cos\theta &  0 &  \sin\theta \\
    0         &  1 &   0 \\
    -\sin\theta &  0 &  \cos\theta
  \end{array}\right),
\quad
\Phi=\left(\begin{array}{ccc}
    1 &  0        &  0 \\
    0 &  \cos\phi &  \sin\phi \\
    0 & -\sin\phi &  \cos\phi
  \end{array}\right),
\quad
\]
\[
\Psi=\left(\begin{array}{ccc}
    \cos\psi & -\sin\psi &  0 \\
    \sin\psi &  \cos\psi &  0 \\
    0        &  0        &  1
  \end{array}\right).
\]

The reference orbit should be closed and it should not be twisted.
This means that the displacement of the local reference system
must be periodic with the revolution frequency of the accelerator,
while the position angles must be periodic $\pmod{2\pi}$
with the revolution frequency.
If \texttt{PSI} is not periodic $\pmod{2\pi}$, 
coupling effects are introduced.
When advancing through a beam element,
\opalmap computes $V_i$ and $W_i$
by the recurrence relations
\[
V_i = W_{i-1}R_i + V_{i-1}, \qquad
W_i = w_{i-1}S_i.
\]

The vector $R_i$ is the displacement and the matrix
$S_i$ is the rotation of the local reference system 
at the exit of the element $i$ with respect to the entrance 
of the same element.
The values of $R_i$ and $S_i$
are listed below for each physical element type.

\section{Local Reference Systems}
\index{local!reference}
\index{reference!local}
\subsection{Reference System for Straight Beam Elements}
\label{sec:straight}
In straight elements the \figref{local reference system}{straight} 
is simply translated by the length of the element along the local $s$-axis.
This is true for
\begin{itemize}
\item \secref{Drift spaces}{drift}
\item \secref{Quadrupoles}{quadrupole}
\item \secref{Sextupoles}{sextupole}
\item \secref{Octupoles}{octupole}
\item \secref{Multipoles}{octupole}
\item \secref{Solenoids}{solenoid}
\item \secref{RF cavities}{cavity}
\item \secref{Electrostatic separators}{separator}
\item \secref{Closed orbit correctors}{corrector}
\item \secref{Beam position monitors}{monitors}
\end{itemize}
The corresponding $R$, $S$ are
\[
R=\left(\begin{array}{c}
    0 \\
    0 \\
    L
  \end{array}\right),
\qquad
S=\left(\begin{array}{ccc}
    1 & 0 & 0 \\
    0 & 1 & 0 \\
    0 & 0 & 1 \\
  \end{array}\right).
\]
A rotation of the element about the $S$-axis has no effect
on $R$ and $S$, since the rotations of the reference system 
before and after the element cancel.

\begin{figure}[ht]
  \begin{center}
    \setlength{\unitlength}{1pt}
    \begin{picture}(400,100)
      \thinlines
                                % axes
      \put(150,50){\circle{8}}\put(150,50){\circle*{2}}
      \put(140,40){\makebox(0,0){$y_1$}}
      \put(250,50){\circle{8}}\put(250,50){\circle*{2}}
      \put(260,40){\makebox(0,0){$y_2$}}
      \put(100,50){\line(1,0){46}}
      \put(154,50){\line(1,0){92}}
      \put(254,50){\vector(1,0){46}}
      \put(290,40){\makebox(0,0){$s$}}
      \put(150,0){\line(0,1){46}}
      \put(150,54){\vector(0,1){46}}
      \put(140,90){\makebox(0,0){$x_1$}}
      \put(250,0){\line(0,1){46}}
      \put(250,54){\vector(0,1){46}}
      \put(260,90){\makebox(0,0){$x_2$}}
                                % magnet outline
      \thicklines
      \put(150,54){\line(0,1){26}}
      \put(150,46){\line(0,-1){26}}
      \put(250,54){\line(0,1){26}}
      \put(250,46){\line(0,-1){26}}
      \put(150,20){\line(1,0){100}}
      \put(150,80){\line(1,0){100}}
      \put(200,2){\vector(1,0){50}}
      \put(200,2){\vector(-1,0){50}}
      \put(200,10){\makebox(0,0){L}}
    \end{picture}
    \caption{Reference System for Straight Beam Elements}
    \label{fig:straight}
  \end{center}
\end{figure}

\subsection{Reference System for Bending Magnets}
\label{rbend}
Both \figref{rectangular}{rbend} and \figref{sector}{sbend}
bending magnets have a curved reference orbit.
For both types of magnets
\[
R=\left(\begin{array}{c}
    \rho(\cos\alpha-1) \\
    0 \\
    \rho\sin\alpha
  \end{array}\right),
\qquad
S=\left(\begin{array}{ccc}
    \cos\alpha & 0 & -\sin\alpha \\
    0          & 1 &  0 \\
    \sin\alpha & 0 &  \cos\alpha
  \end{array}\right),
\]
where $\alpha$ is the bend angle.
A positive bend angle represents a bend to the right,
i.e. towards negative $x$ values.
For sector bending magnets,
the bend radius is given by $\rho$,
and for rectangular bending magnets it has the value
\[
\rho = L / 2 \sin(\alpha/2).
\]
If the magnet is rotated about the $s$-axis by an angle psi,
$R$ and $S$ are transformed by
\[
R^{*} = T R, \qquad S^{*} = T S T^{-1}.
\]
where $T$ is the orthogonal rotation matrix
\[
T=
\begin{pmatrix}
    \cos\psi & -\sin\psi &  0 \\
    \sin\psi &  \cos\psi &  0 \\
    0        &  0        &  1 
\end{pmatrix}.
\]
The special value $\psi = \pi/2$ represents a bend down.

\begin{figure}[ht]
  \begin{center}
    \setlength{\unitlength}{1pt}
    \begin{picture}(400,215)
                                % axes
      \thinlines
      \put(150,150){\circle{8}}\put(150,150){\circle*{2}}
      \put(160,140){\makebox(0,0){$y_1$}}
      \put(250,150){\circle{8}}\put(250,150){\circle*{2}}
      \put(240,140){\makebox(0,0){$y_2$}}
      \put(74,124.7){\vector(3,1){72}}
      \put(84,135){\makebox(0,0){$s_1$}}
      \put(254,148.7){\vector(3,-1){72}}
      \put(316,135){\makebox(0,0){$s_2$}}
      \put(200,0){\vector(-1,3){48.7}}
      \put(165,75){\makebox(0,0){$\rho$}}
      \put(148.7,154){\vector(-1,3){18}}
      \put(118,206){\makebox(0,0){$x_1$}}
      \put(200,0){\vector(1,3){48.7}}
      \put(235,75){\makebox(0,0){$\rho$}}
      \put(251.3,154){\vector(1,3){18}}
      \put(282,206){\makebox(0,0){$x_2$}}
      \bezier{20}(190.5,28.5)(200,31.7)(209.5,28.5)
      \put(200,20){\makebox(0,0){$\alpha$}}
      \put(154,150){\line(1,0){92}}
      \put(200,150){\circle*{4}}
      \put(200,150){\vector(0,1){60}}
      \put(210,200){\makebox(0,0){$x$}}
      \put(150,154){\line(0,1){44}}
      \put(150,146){\line(0,-1){46}}
      \put(151,154){\line(1,4){11}}
      \put(250,154){\line(0,1){44}}
      \put(250,146){\line(0,-1){46}}
      \put(249,154){\line(-1,4){11}}
                                % magnet outline
      \thicklines
      \put(200,102){\vector(-1,0){50}}
      \put(200,102){\vector(1,0){50}}
      \put(200,110){\makebox(0,0){L}}
      \put(151,154){\line(1,4){6}}
      \put(149,146){\line(-1,-4){6}}
      \put(249,154){\line(-1,4){6}}
      \put(251,146){\line(1,-4){6}}
      \put(157,178){\line(1,0){86}}
      \put(143,122){\line(1,0){114}}
      \bezier{10}(150,195)(155.5,195)(160.9,193.7)
      \put(155.5,195){\vector(3,-1){5.4}}
      \put(150,205){\makebox(0,0)[l]{$e_1$}}
      \bezier{10}(250,195)(244.5,195)(239.1,193.7)
      \put(244.5,195){\vector(-3,-1){5.4}}
      \put(250,205){\makebox(0,0)[r]{$e_2$}}
    \end{picture}
    \caption[Reference System for a Rectangular Bending Magnet]%
    {Reference System for a Rectangular Bending Magnet;
      the signs of pole-face rotations are positive as shown.}
    \label{fig:rbend}
  \end{center}
\end{figure}

\begin{figure}[ht]
  \begin{center}
    \setlength{\unitlength}{1pt}
    \begin{picture}(400,215)
                                % axes
      \thinlines
      \put(150,150){\circle{8}}\put(150,150){\circle*{2}}
      \put(160,140){\makebox(0,0){$y_1$}}
      \put(250,150){\circle{8}}\put(250,150){\circle*{2}}
      \put(240,140){\makebox(0,0){$y_2$}}
      \put(74,124.7){\vector(3,1){72}}
      \put(84,135){\makebox(0,0){$s_1$}}
      \put(254,148.7){\vector(3,-1){72}}
      \put(316,135){\makebox(0,0){$s_2$}}
      \put(200,0){\vector(-1,3){48.7}}
      \put(165,75){\makebox(0,0){$\rho$}}
      \put(148.7,154){\vector(-1,3){18}}
      \put(118,206){\makebox(0,0){$x_1$}}
      \put(200,0){\vector(1,3){48.7}}
      \put(235,75){\makebox(0,0){$\rho$}}
      \put(251.3,154){\vector(1,3){18}}
      \put(282,206){\makebox(0,0){$x_2$}}
      \bezier{20}(190.5,28.5)(200,31.7)(209.5,28.5)
      \put(200,20){\makebox(0,0){$\alpha$}}
      \put(200,158.8){\circle*{4}}
      \put(200,158.8){\vector(0,1){50}}
      \put(210,200){\makebox(0,0){$r$}}
      \put(151,154){\line(1,4){10}}
      \put(249,154){\line(-1,4){10}}
                                % magnet outline
      \thicklines
      \bezier{100}(154,151.3)(200,166.7)(246,151.3)
      \put(162,154){\vector(-3,-1){8}}
      \put(238,154){\vector(3,-1){8}}
      \put(210,168){\makebox(0,0){L}}
      \put(151,154){\line(1,4){6}}
      \put(149,146){\line(-1,-4){6}}
      \put(249,154){\line(-1,4){6}}
      \put(251,146){\line(1,-4){6}}
      \bezier{90}(157,178)(200,188.4)(243,178)
      \bezier{110}(143,122)(200,148.6)(257,122)
      \bezier{20}(137.4,187.9)(149.1,191.5)(159.7,188.8)
      \put(153.7,190.8){\vector(3,-1){6}}
      \put(150,180){\makebox(0,0){$e_1$}}
      \bezier{20}(262.6,187.9)(250.9,191.5)(240.3,188.8)
      \put(246.3,190.8){\vector(-3,-1){6}}
      \put(250,180){\makebox(0,0){$e_2$}}
    \end{picture}
    \caption[Reference System for a Sector Bending Magnet]%
    {Reference System for a Sector Bending Magnet;
      the signs of pole-face rotations are positive as shown.}
    \label{fig:sbend}
  \end{center}
\end{figure}

\subsection{Rotation of the Reference System}
\label{sec:refrot}
For a 
\figref{rotation of the reference system by an angle $\psi$ about the
  beam ($s$) axis}{srot}: 
\[
S=\left(\begin{array}{ccc}
    \cos\psi & -\sin\psi &  0 \\
    \sin\psi &  \cos\psi &  0 \\
    0        &  0        &  1 \\
  \end{array}\right),
\]
while for a 
\figref{rotation of the reference system by an angle $\theta$ about
  the vertical ($y$) axis}{yrot}:
\[
S=\left(\begin{array}{ccc}
    \cos\theta &  0 & -\sin\theta \\
    0          &  1 &  0 \\
    \sin\theta &  0 &  \cos\theta
  \end{array}\right).
\]
In both cases the displacement $R$ is zero.

\begin{figure}[ht]%
  \begin{center}
    \setlength{\unitlength}{1pt}
    \begin{picture}(400,200)
      \thinlines
      \put(200,100){\circle{8}}\put(200,100){\circle*{2}}
      \put(190,90){\makebox(0,0){$s$}}
      \put(100,100){\line(1,0){96}}
      \put(204,100){\vector(1,0){96}}
      \put(290,90){\makebox(0,0){$x_1$}}
      \put(200,0){\line(0,1){96}}
      \put(200,104){\vector(0,1){96}}
      \put(190,210){\makebox(0,0){$y_1$}}
      \put(103,75.75){\line(4,1){93}}
      \put(204,101){\vector(4,1){93}}
      \put(287,134){\makebox(0,0){$x_2$}}
      \put(224.25,3){\line(-1,4){23.25}}
      \put(199,104){\vector(-1,4){23.25}}
      \put(166,187){\makebox(0,0){$y_2$}}
      \bezier{20}(260,100)(260,107.5)(258,114.5)
      \put(260,106.5){\vector(-1,4){2}}
      \put(250,106.25){\makebox(0,0){$\psi$}}
      \put(220,150){\circle{8}}\put(220,150){\circle*{2}}
      \put(220,140){\makebox(0,0){beam}}
    \end{picture}
    \caption{Reference System for a Rotation Around the s-Axis}
    \label{fig:srot}
  \end{center}
\end{figure}

\begin{figure}[ht]%
  \begin{center}
    \setlength{\unitlength}{1pt}
    \begin{picture}(400,200)
      \thinlines
      \put(200,100){\circle{8}}\put(200,100){\circle*{2}}
      \put(190,90){\makebox(0,0){$y$}}
      \put(100,100){\line(1,0){96}}
      \put(204,100){\vector(1,0){96}}
      \put(290,110){\makebox(0,0){$s_1$}}
      \put(200,0){\line(0,1){96}}
      \put(200,104){\vector(0,1){96}}
      \put(190,190){\makebox(0,0){$x_1$}}
      \put(103,124.25){\line(4,-1){93}}
      \put(204,99){\vector(4,-1){93}}
      \put(287,66){\makebox(0,0){$s_2$}}
      \put(175.75,3){\line(1,4){23.25}}
      \put(201,104){\vector(1,4){23.25}}
      \put(234,187){\makebox(0,0){$x_2$}}
      \bezier{20}(260,100)(260,92.5)(258,85.5)
      \put(260,93.5){\vector(-1,-4){2}}
      \put(250,93.75){\makebox(0,0){$\theta$}}
      \thicklines
      \put(100,130){\vector(1,0){200}}
      \put(290,140){\makebox(0,0){beam}}
    \end{picture}
    \caption{Reference System for a Rotation Around the y-Axis}
    \label{fig:yrot}
  \end{center}
\end{figure}

\subsection{Elements which do not Change the Local Reference}
The following elements do not affect the
reference orbit and are ignored for geometry calculations:
\begin{itemize}
\item Beam-beam interactions %\secref{Beam-beam interactions}{sec:beambeam}
\item \secref{Marker}{marker}
\end{itemize}

\section{Sign Conventions for Magnetic Fields}
\label{sec:sign}
\index{sign conventions for fields}
\index{field!signs}
The \opalmap program uses the following Taylor expansion for the normal and
skewed field components respectively in the mid-plane $y$=0, 
described in \bibref{SLAC-75}{matrix}:
\[
B_x(x,0)=\sum_{k=0}^{\infty}\frac{B_{kn}x^k}{k!}, \qquad
B_x(x,0)=\sum_{k=0}^{\infty}\frac{B_{ks}x^k}{k!}.
\]
Note the factorial in the denominator.
The field coefficients have the following meaning:
\begin{description}
\item[$B_{0n}$] Normal dipole field component.
  The component is positive if the field points in the positive $y$
  direction. 
  A positive field bends a positively charged particle travelling in
  positive $s$-direction to the right.
\item[$B_{0s}$] Skew dipole field component.
  The component is positive if the field points in the negative $y$
  direction. 
  A positive field bends a positively charged particle travelling in
  positive $s$-direction down.
\item[$B_{1n}$] Normal quadrupole field component
  $B_{1n}=\partial B_y/\partial x$.
  The component is positive if $B_y$ is positive on the positive $x$-axis.
  A positive value corresponds to horizontal focussing of a positively
  charged particle.
\item[$B_{1s}$] Skew quadrupole field component
  $B_{1s}=\partial B_x/\partial x$.
  The component is positive if $B_x$ is negative on the positive $x$-axis.
\item[$B_{2n}$] Normal sextupole field component
  $B_{2n}=\partial^2 B_y/\partial x^2$.
  The component is positive if $B_y$ is positive on the $x$-axis.
\item[$B_{2s}$] Skew sextupole field component
  $B_{2s}=\partial^2 B_x/\partial x^2$.
  The component is negative if $B_x$ is positive on the $x$-axis.
\item[$B_{3n}$] Normal octupole field component
  $B_{3n}=\partial^3 B_y/\partial x^3$.
  The component is positive if $B_y$ is positive on the positive $x$-axis.
\item[$B_{3s}$] Skew octupole field component
  $B_{3s}=\partial^3 B_x/\partial x^3$.
  The component is negative if $B_x$ is positive on the $x$-axis.
\end{description}
All derivatives are taken on the $x$-axis.
Using this expansion and the curvature $h$ of the reference orbit,
the longitudinal component of the vector potential for a magnet with
mid-plane symmetry is to order~4:
\index{field!vector potential}
\index{vector potential}

\begin{align*} 
  A_s \quad = \quad 
  & B_{0n}\left(x-\frac{hx^2}{2(1+hx)}\right) \\
  &+B_{1n}\left(\frac{1}{2}(x^2-y^2)-\frac{h}{6}x^{3}
  +\frac{h^2}{24}(4x^{4}-y^{4})+\ldots\right) \\
  &+B_{2n}\left(\frac{1}{6}(x^{3}-3xy^2)-\frac{h}{24}(x^{4}-y^{4})
   +\ldots\right) \\
  &+B_{3n}\left(\frac{1}{24}(x^{4}-6x^2y^2+y^{4})+\ldots\right)
   + \ldots
\end{align*}
Taking $\mathrm{curl} A$ in curvilinear coordinates,
the field components can be computed as
\index{field!components}
\begin{align*} 
  B_x(x,y) \quad = \quad
  & B_{1n}\left(y+\frac{h^2}{6}y^{3}+\ldots\right) 
   +B_{2n}\left(xy-\frac{h^{3}}{6}y^{3}+\ldots\right) \\
  &+B_{3n}\left(\frac{1}{6}(3x^2y-y^{3})+\ldots\right) + \ldots \\
  B_y(x,y) \quad = \quad 
  & B_{0n}
   +B_{1n}\left(x-\frac{h}{2}y^2+\frac{h^2}{2}xy^2+\ldots\right) \\
  &+B_{2n}\left(\frac{1}{2}(x^2-y^2)-\frac{h}{2}xy^2+\ldots\right) \\
  &+B_{3n}\left(\frac{1}{6}(x^{3}-3xy^2)+\ldots\right) + \ldots
\end{align*}
One can easily verify that both $\mathrm{curl} B$ and $\mathrm{div} B$
are zero to the order of the $B_3$ term.
Introducing the magnetic rigidity $B \rho$,
the multipole coefficients are computed as
\[
K_{kn}=eB_{kn}/p_0=B_{kn}/B\rho,\qquad
K_{ks}=eB_{ks}/p_0=B_{ks}/B\rho.
\]
Note that the $K_k$ have the \textbf{same sign} as the corresponding
field components $B_k$.
The signs will be changed due to the sign of particle charges and
the direction of travel of the beam.

\section{Variables in \opalt}
\label{sec:variablesopalt}
\index{variablesopalt}
For each variable the physical units are listed in square brackets.

\subsection{Variables Describing Orbits}
\label{sec:opalt:canon}
\index{canonical variables}
\index{variables!canonical}
\opalt uses the following canonical variables
to describe the motion of particles:

\begin{description}
\item[X]
  Horizontal position $x$ of a particle relative to the axis of the element [m].

\item[PX]
  Horizontal canonical momentum:
  $\mathtt{PX} = \beta_x \gamma$.

\item[Y]
  Horizontal position $y$ of a particle relative to the axis of the element [m].

\item[PY]
  Horizontal canonical momentum:
  $\mathtt{PY} = \beta_y \gamma$.

\item[Z]
  Longitudinal position $z$ of a particle in floor co-ordinates [m].

\item[PZ]
 Longitudinal canonical momentum: 
 $\mathtt{PZ} = \beta_z \gamma$.

 \end{description}

The independent variable is:
\begin{description}
\item[t]
  Time [s].
\end{description}

\section{Variables in \opalcycl}
\label{sec:variablesopalcycl}
\index{variablesopalcycl}

\subsection{Variables Describing Orbits}
\label{sec:opalcycl:canon}
\index{canonical variables}
\index{variables!canonical}

\opalcycl uses the following canonical variables to describe the motion of particles:

\begin{description}
\item[X]
  Horizontal position $x$ of a particle in given global Cartesian coordinates [m].

\item[PX]
  Horizontal canonical momentum:
  $\mathtt{PX} = \beta_x \gamma$.

\item[Y]
  Horizontal position $x$ of a particle in global Cartesian coordinates [m].

\item[PY]
  Horizontal canonical momentum:
  $\mathtt{PY} = \beta_y \gamma$.

\item[Z]
  Vertical position $z$ of a particle in global Cartesian coordinates [m].

\item[PZ]
  Vertical canonical momentum:
  $\mathtt{PZ} = \beta_z \gamma$.

\end{description}

The independent variable is: \textbf{t} [s].


\subsection{Local frame initial distribution}
\label{sec:opalcycl:localframe}
To ensure compatibility with \opalt and \opalmap 
the initial distribution of the bunch,
either read from file or generated by specifying rms beam quantities,
is specified in the local Cartesian coordinates.
During run time,the 6 phase space variables \((x, y, z, p_x, p_y, p_z)\) 
are transformed to the global Cartesian coordinates
using initial reference $r_0$,$\theta_0$ and $p_{r0}$, $p_{\theta 0}$.
These 4 parameters are defined at the \secref{\texttt{CYCLOTRON} command }{cyclotron}
before particle tracking starts. 

\begin{align*}  
X &= x\cos(\theta_0) - y\sin(\theta_0) + r_0\cos(\theta_0)  \\
Y &= x\sin(\theta_0) + y\cos(\theta_0) + r_0\sin(\theta_0)  \\
Z &= z 
\end{align*}
\begin{align*}  
PX &= (p_x+p_{r0})\cos(\theta_0) - (p_y+p_{\theta 0})\sin(\theta_0) \\
PY &= (p_x+p_{r0})\sin(\theta_0) + (p_y+p_{\theta 0})\cos(\theta_0) \\
PZ &= p_z 
\end{align*}    

\section{Variables in \opalmap}
\label{sec:variables}
\index{variables}
For each variable the physical units are listed in square brackets.

\subsection{Canonical Variables Describing Orbits}
\label{sec:canon}
\index{canonical variables}
\index{variables!canonical}
\opalmap uses the following canonical variables
to describe the motion of particles:

\begin{description}
\item[X]
  Horizontal position $x$ of the (closed) orbit,
  referred to the ideal orbit [m].

\item[PX]
  Horizontal canonical momentum of the (closed) orbit referred 
  to the ideal orbit, divided by the reference momentum:
  $\mathtt{PX} = p_x / p_0$.

\item[Y]
  Vertical position $y$ of the (closed) orbit,
  referred to the ideal orbit [m].

\item[PY]
  Vertical canonical momentum of the (closed) orbit referred 
  to the ideal orbit, divided by the reference momentum:
  $\mathtt{PY} = p_y / p_0$.

\item[T]
  The negative time difference, 
  multiplied by the instantaneous velocity of the particle [m]:
  $\mathtt{T} = - v \delta(t)$.
  A positive \texttt{T} means that the particle arrives ahead of the 
  \textbf{reference particle}.
  \texttt{T} describes the deviation of the particle from the orbit of 
  a fictitious reference particle having the constant 
  \textbf{reference momentum} $p_s$ and the 
  \textbf{reference velocity} $v_s$.
  $v_s$ defines the revolution frequency.
  The velocities have the values 
  \[
  v = c p / \sqrt{p^2 + m^2 c^2}, \qquad
  v_s = c p_s / \sqrt{p_s^2 + m^2 c^2},
  \]
  where $c$ is the velocity of light, $m$ is the particle rest mass, 
  and $p$ is the instantaneous momentum of the particle.

\item[PT]
  Momentum error, divided by the reference momentum:
  $\mathtt{PT} = \delta p / p_s$.
  This value is only non-zero when synchrotron motion is present.
  It describes the deviation of the particle from the orbit of a
  particle with the reference momentum $p_s$.
\end{description}

The independent variable is:
\begin{description}
\item[S]
  Arc length $s$ along the reference orbit [m].
\end{description}

The longitudinal variables is in the limit of fully relativistic particles
($\gamma \gg 1, v = c, p c = E$),
the variables \texttt{T, PT} used here agree with the longitudinal variables
used in \bibref{TRANSPORT}{transport}.
This means that \texttt{T} becomes the negative path length difference,
while \texttt{PT} is the fractional momentum error.
The reference momentum must be constant in order to keep the system 
canonical.

\subsection{Normalised Variables and other Derived Quantities}
\label{sec:normal}
\index{normalised variables}
\index{variables!normalised}
\begin{description}
\item[XN]
  The normalised horizontal displacement [$\mathrm{m}^{1/2}$]:
  $\mathtt{XN} = x_n = \Re(E_1^T S Z)$.

\item[PXN]
  The normalised horizontal transverse momentum [$\mathrm{m}^{1/2}$]:
  $\mathtt{PXN} = p_{xn} = \Im(E_1^T S Z$).

\item[WX]
  The horizontal Courant-Snyder invariant [m]:
  $\mathtt{WX} = \sqrt{x_n^2 + p_{xn}^2}$.

\item[PHIX]
  The horizontal phase:
  $\mathtt{PHIX} = - \arctan(p_{xn} / x_n) / 2 \pi$.

\item[YN]
  The normalised vertical displacement [$\mathrm{m}^{1/2}$]:
  $\mathtt{YN} = y_n = \Re(E_2^T S Z)$.

\item[PYN]
  The normalised vertical transverse momentum [$\mathrm{m}^{1/2}$]:
  $\mathtt{PYN} = p_{yn} = \Im(E_2^T S Z)$.

\item[WY]
  The vertical Courant-Snyder invariant [m]:
  $\mathtt{WY} = \sqrt{y_n^2 + p_{yn}^2}$.

\item[PHIY]
  The vertical phase:
  $\mathtt{PHIY} = - \arctan(p_{yn} / y_n) / 2 \pi$.

\item[TN]
  The normalised longitudinal displacement [$\mathrm{m}^{1/2}$]:
  $\mathtt{TN} = t_n = \Re(E_3^T S Z)$.

\item[PTN]
  The normalised longitudinal transverse momentum [$\mathrm{m}^{1/2}$]:
  $\mathtt{PTN} = p_{tn} = Im(E_3^T S Z)$.

\item[WT]
  The longitudinal invariant [m]:
  $\mathtt{WT} = \sqrt{t_n^2 + p_{tn}^2}$.

\item[PHIT]
  The longitudinal phase:
  $\mathtt{PHIT} = + \arctan(p_{tn} / t_n) / 2 \pi$.

\end{description}
in the above formulas $Z$ is the phase space vector
$Z = (x, p_x, y, p_y, t, p_t)^T$.
The matrix $S$ is the ``symplectic unit matrix''
\[
S = 
\begin{pmatrix} 
    0 & 1 & 0 & 0 & 0 & 0 \\
   -1 & 0 & 0 & 0 & 0 & 0 \\
    0 & 0 & 0 & 1 & 0 & 0 \\
    0 & 0 &-1 & 0 & 0 & 0 \\
    0 & 0 & 0 & 0 & 0 & 1 \\
    0 & 0 & 0 & 0 &-1 & 0
  \end{pmatrix},
\]
and the vectors $E_i$ are the three complex eigenvectors.
The superscript $T$ denotes the transpose of a vector or matrix.

\section{Physical Units}
\label{sec:units}
\index{units}
\index{physical units}
Throughout the computations \opal uses \tabref{international units}{units},
as defined by SI (Syst\`eme International).

\begin{table}[ht] \footnotesize
  \begin{center}
    \caption{Physical Units}
    \label{tab:units}
    \begin{tabular}{|l|l|} 
      \hline
      quantity                         & dimension \\
      \hline
      Length                           & m (metres) \\
      Angle                            & rad (radians) \\
      Quadrupole coefficient           & $\mathrm{m}^{-2}$ \\
      Multipole coefficient, 2n poles  & $\mathrm{m}^{-n}$ \\
      Electric voltage                 & MV (Megavolts) \\
      Electric field strength          & MV/m \\
      Frequency                        & MHz (Megahertz) \\
      Phase angles                     & $2\pi$ \\
      Particle energy                  & GeV \\
      Particle mass                    & GeV/$c^2$ \\
      Particle momentum                & GeV/c \\
      Beam current                     & A (Amperes) \\
      Particle charge                  & e (elementary charges) \\
      Impedances                       & M$\Omega$ (Megohms) \\
      Emittances                       & $\pi$ m mrad \\
      Emittances \opalt             & m rad \footnote{(normalized and un-normalized)}\\
      RF power                         & MW (Megawatts) \\
      Higher mode loss factor          & V/pc \\
      \hline
    \end{tabular}
  \end{center}
\end{table}

\index{conventions|)}
