\chapter{Control Statements}
\label{sec:control}
\index{control statements|(}
\index{program control}

\section{Getting Help}

\subsection{HELP Command}
\label{sec:help}
\index{HELP}
A user who is uncertain about the attributes of a command
should try the command \texttt{HELP}, which has three formats:
\begin{verbatim}
HELP;                 // Give help on the "HELP" command
HELP,NAME=label;      // List funct. and attr. types of "label"
HELP,label;           // Shortcut for the second format
\end{verbatim}
\texttt{label} is an \secref{\texttt{identifier}}{label}.
If it is non-blank,
\opal prints the function of the object \texttt{label} and lists its
attribute types.
Entering \texttt{HELP} alone displays help on the \texttt{HELP}
command. 

\noindent Examples:
\begin{verbatim}
HELP;
HELP,NAME=TWISS;
HELP,TWISS;
\end{verbatim}

\subsection{SHOW Command}
\label{sec:show}
\index{SHOW}
The \texttt{SHOW} statement displays the current attribute values
of an object.
It has three formats:
\begin{verbatim}
SHOW;                 // Give help on the "SHOW" command
SHOW,NAME=pattern;    // Show names matching of "pattern"
SHOW,pattern;         // Shortcut for the second format
\end{verbatim}
\texttt{pattern} is an \secref{\texttt{regular expression}}{wildcard}.
If it is non-blank,
\opal displays all object names matching the \texttt{pattern}.
Entering \texttt{SHOW} alone displays help on the \texttt{SHOW}
command. 
%If the \secref{\texttt{OPTION,MAD8}}{option} is active,
%the format is MAD-8 format, otherwise it is \opal format.
\noindent Examples:
\begin{verbatim}
SHOW;
SHOW,NAME="QD.*\.L*";
SHOW,"QD.*\.L*";
\end{verbatim}

\subsection{WHAT Command}
\label{sec:what}
\index{WHAT}
The \texttt{WHAT} statement displays all object names matching a given
regular expression.
It has three formats:
\begin{verbatim}
WHAT;                   // Give help on the "WHAT" command
WHAT,NAME=label;        // Show definition of "label"
WHAT,label;             // Shortcut for the second format
\end{verbatim}
\texttt{label} is an \secref{\texttt{identifier}}{label}.
If it is non-blank,
\opal displays the object \texttt{label} in a format similar to the
input statement that created the object. 
Entering \texttt{WHAT} alone displays help on the \texttt{WHAT}
command. 
\noindent Examples:
\begin{verbatim}
WHAT;
WHAT,NAME=QD;
WHAT,QD;
\end{verbatim}

\section{STOP / QUIT Statement}
\label{sec:stop}
\index{STOP / QUIT}
The statement
\begin{verbatim}
STOP or QUIT;
\end{verbatim}
terminates execution of the \opal program,
or, when the statement occurs in a \secref{\texttt{CALL} file}{call},
returns to the calling file.
Any statement following the \texttt{STOP} or  \texttt{QUIT} statement is ignored.

\section{OPTION Statement}
\label{sec:option}
\index{OPTION}
The \texttt{OPTION} command controls global command execution and sets
a few global quantities:
\begin{verbatim}

OPTION,ECHO=logical,INFO=logical,TRACE=logical,
VERIFY=logical,WARN=logical,ADDERROR=logical,
SEED=real,TELL=logical,PSDUMPFREQ=integral, STATDUMPFREQ=integral
SPTDUMFREQ=integral,REPARTFREQ=integral,
REBINFREQ=integral;
\end{verbatim}
The first five logical flags activate or deactivate execution options:
\begin{description}
\item[ECHO]
  \index{ECHO}
  Controls printing of an echo of input lines on the standard error file.
\item[INFO]
  \index{INFO}
  If this option is turned off, \opal suppresses all information messages. 
%\item[\opal8]
%  \index{\opal8}
%  When the \opal8 option is on,
%  all output of the \secref{\texttt{SHOW}}{show},
%  \secref{\texttt{SAVE}}{save}, 
%  \secref{\texttt{ESAVE}}{errorsave}, and
%  \secref{\texttt{MAKESEQ}}{makeseq} is written in \opal-8 format.
%  \opal can thus serve as a translator from \opal to \opal-8 format.
\item[TRACE]
  \index{TRACE}
  When the TRACE option is on,
  \opal writes additional trace information on the standard error file 
  for each executable command. 
  This information includes the command name
  and elapsed CPU time before and after the command.
\item[VERIFY]
  \index{VERIFY}
  If this option is on, \opal gives a message for each undefined variable
  or element in a beam line.
\item[WARN]
  \index{WARN}
  If this option is turned off, \opal suppresses all warning messages.
\item[ADDERROR]
  \index{ADDERROR}
  If this logical flag is set,
  an \texttt{EALIGN}, \texttt{EFIELD}, or \texttt{EFCOMP}, causes the errors to
  be added on top of existing ones.
  If it is not set, new errors overwrite any previous definitions.
\item[SEED]
  \index{SEED}
  Selects a particular sequence of random values.
  A SEED value is an integer in the range [0...999999999] (default: 123456789).
  SEED can be an expression.
  See also: \secref{random values}{adefer}.
\item[PSDUMPFREQ]
 \index{PSDUMPFREQ}
   Defines after how many time steps we dump the phase space into the H5Part file. Its default value is 10. 
\item[STATDUMPFREQ]
 \index{STATDUMPFREQ}
 	Defines after how many time steps we dump statistical data, such as RMS beam emittiance, to the .stat file.
 	The default value is 10. Currently only available for \opalt.
\item[SPTDUMPFREQ]
 \index{SPTDUMPFREQ}
   Defines after how many time steps we dump the phase space of single particle.
   It is always useful to record the trajectory of reference particle
   or some specified particle for primary study. Its default value is 1.  
\item[REPARTFREQ]
 \index{REPARTFREQ}
   Defines after how many time steps we do particles repartition to balance the computational load of  
   the computer nodes. Its default value is 10.  
   
\item[REBINFREQ]
 \index{REBINFREQ}
   Defines after how many time steps we update the energy Bin ID of each particle. For the time being. 
   Only available for multi-bunch simulation in \opalcycl. Its default value is 100.  

\item[PSDUMPEACHTURN]
 \index{PSDUMPEACHTURN}
 Control option of phase space dumping. If true, dump phase space after each turn. 
 For the time being, this is only use for multi-bunch simulation in \opalcycl. Its default set is false. 

\item[PSDUMPLOCALFRAME]
 \index{PSDUMPLOCALFRAME}
 Control option whether the phase space data is dumped in global Cartesian frame or local Cartesian frame.
 If true, in local frame, otherwise in global Cartesian frame.Only available for OPAL-cycl. Its default set is false.
 Note that restarting run cannot be launched by reading in phase space data in local frame.
     
\end{description}
The last attribute requests listing of the current settings:
\begin{description}
\item[TELL]
  \index{TELL}
  If true, the current settings are listed.
\item[SCAN]
\index{SCAN}  
 If true one can simulate in a loop several machines where some variables can be random variables. Find an
 example at \ref{sec:randmach}. 
\end{description}

\noindent Examples:
\begin{verbatim}
OPTION,ECHO=FALSE,TELL;
OPTION,SEED=987456321
\end{verbatim}

\begin{table}[ht] \footnotesize
  \caption{Default Settings for Options}
  \label{tab:option}
  \begin{center}
    \begin{tabular}{|ll|ll|ll|ll|}
      \hline
      \texttt{ECHO}     & = \texttt{true}  &
      \texttt{INFO}     & = \texttt{true}  &
      \texttt{TRACE}    & = \texttt{false} \\
%      \texttt{PSDUMPFREQ}   & = \texttt{10} \\
      \hline
      \texttt{WARN}     & = \texttt{true } &
      \texttt{ADDERROR} & = \texttt{false} &
      \texttt{SEED}     & = 123456789      \\
      \hline
       \texttt{PSDUMPFREQ}     & = \texttt{10 } &
      \texttt{SPTDUMPFREQ} & = \texttt{1} &
      \texttt{REPARTFREQ}     & = 10  \\
      \hline
            \texttt{ }     & = \texttt{-1 } &
      \texttt{ }     &  = 64 &
        \texttt{TELL}     & = \texttt{false} \\
       \hline
         \texttt{VERIFY}   & = \texttt{false} &
         \texttt{STATDUMPFREQ}     & =10  &
           \texttt{}     & & \\
             \hline
    \end{tabular}
  \end{center}
\end{table}

\section{Parameter Statements}
\label{sec:parameter}
\index{parameter}
\index{variable}

\subsection{Variable Definitions}
\label{sec:variable}
\opal recognises several types of variables.

\subsubsection{Real Scalar Variables}
\index{variable!real}
\index{real!variable}
\begin{verbatim}
REAL variable-name=real-expression;
\end{verbatim}
The keyword \texttt{REAL} is optional.
For backward compatibility the program also accepts the form
\begin{verbatim}
variable-name:=real-expression;
\end{verbatim}
This statement creates a new global variable \texttt{variable-name}
and discards any old variable with the same name.
Its value depends on all quantities occurring
in \secref{\texttt{real-expression}}{areal}.
Whenever an operand changes in \texttt{real-expression},
a new value is calculated.
The definition may be thought of as a mathematical equation.
However, \opal is not able to solve the equation for a quantity on the
right-hand side.

An assignment in the sense of the FORTRAN or C languages can be achieved
by using the \secref{\texttt{EVAL} function}{eval}.

A reserved variable is the value \texttt{P0} which is used as the
global reference momentum for normalising all magnetic field coefficients.
\noindent Example:
\begin{verbatim}
REAL GEV=100;
P0=GEV;
\end{verbatim}
Circular definitions are not allowed:
\begin{verbatim}
X=X+1;    // X cannot be equal to X+1
A=B;
B=A;      // A and B are equal, but of unknown value
\end{verbatim}
However, redefinitions by assignment are allowed:
\begin{verbatim}
X=EVAL(X+1);
\end{verbatim}

\subsubsection{Real Vector Variables}
\index{real vector}
\index{variable!vector}
\index{vector!real}
\begin{verbatim}
REAL VECTOR variable-name=vector-expression;
\end{verbatim}
The keyword \texttt{REAL} is optional.
This statement creates a new global variable \texttt{variable-name}
and discards any old variable with the same name.
Its value depends on all quantities occurring
in \secref{\texttt{vector-expression}}{anarray} on the right-hand side.
Whenever an operand changes in \texttt{vector-expression},
a new value is calculated.
The definition may be thought of as a mathematical equation.
However, \opal is not able to solve the equation for a quantity on the
right-hand side.

\noindent Example:
\begin{verbatim}
REAL VECTOR A = TABLE(10, #);
REAL VECTOR B = { 1, 2, 3, 4, 5, 6, 7, 8, 9, 10 };
\end{verbatim}
Circular definitions are not allowed, but redefinitions by assignment 
are allowed.

\subsubsection{Logical Variables}
\index{variable!logical}
\index{logical!variable}
\begin{verbatim}
BOOL variable-name=logical-expression;
\end{verbatim}
This statement creates a new global variable \texttt{variable-name}
and discards any old variable with the same name.
Its value depends on all quantities occurring
in \secref{\texttt{logical-expression}}{alogical}.
Whenever an operand changes in \texttt{logical-expression},
a new value is calculated.
The definition may be thought of as a mathematical equation.
However, \opal is not able to solve the equation for a quantity on the
right-hand side.

\noindent Example:
\begin{verbatim}
BOOL FLAG = X != 0;
\end{verbatim}
Circular definitions are not allowed, but redefinitions by assignment 
are allowed.

\subsection{Symbolic Constants}
\label{sec:constant}
\index{constant!real}
\index{real constant}
\opal recognises a few build-in built-in \tabref{mathematical and physical
constants.}{constant} 
Additional constants can be defined by the command
\begin{verbatim}
REAL CONST label:CONSTANT=<real-expression>;
\end{verbatim}
which defines a constant with the name \texttt{label}.
The keyword \texttt{REAL} is optional, and \texttt{label} must be unique.
An existing symbolic constant can never be redefined.
The \texttt{real-expression} is evaluated at the time the
\texttt{CONST} definition is read, and the result is stored as the
value of the constant.

\noindent Example:
\begin{verbatim}
CONST IN=0.0254; // conversion of inches to metres
\end{verbatim}

\subsection{Vector Values}
\label{sec:vector}
\index{vector!real}
\index{real vector}
A vector of expressions is established by a statement
\begin{verbatim}
REAL VECTOR vector-name=vector-expression;
\end{verbatim}
The keyword \texttt{REAL} is optional.
It creates a new global vector \texttt{vector-name}
and discards any old vector with the same name.
Its value depends on all quantities occurring in 
\secref{\texttt{vector-expression}}{anarray}.
Whenever an operand changes in \texttt{vector-expression},
a new value is calculated.
The definition may be thought of as a mathematical equation.
However, \opal is not able to solve the equation for a quantity on the
right-hand side.

\noindent Example:
\begin{verbatim}
VECTOR A_AMPL={2.5e-3,3.4e-2,0,4.5e-8};
VECTOR A_ON=TABLE(10,1);
\end{verbatim}
Circular definitions are not allowed.

\subsection{Assignment to Variables}
\label{sec:eval}
\index{variable!assignment}
\index{assignment}
A value is assigned to a variable or vector by using the function
\texttt{EVAL(real-expression)}.
When seen, this function is immediately evaluated and replaced by the
result treated like a constant.
\begin{verbatim}
variable-name=EVAL(real-expression);
\end{verbatim}
This statement acts like a FORTRAN or C assignment.
The \texttt{real-expression} or \texttt{vector-expression} is
\textbf{evaluated}, 
and the result is assigned as a constant to the variable or vector on
the left-hand side. 
Finally the expression is discarded.
The \texttt{EVAL} function can also be used within an expression, e.~g.:
\begin{verbatim}
vector-name=TABLE(range,EVAL(real-expression));
vector-name={...,EVAL(real-expression),...);
\end{verbatim}
A sequence like the following is permitted:
\begin{verbatim}
...                 // some definitions
X=0;                // create variable X with value zero
WHILE (X <= 0.10) {
  TWISS,LINE=...;   // uses X=0, 0.01, ..., 0.10
  X=EVAL(X+.01);    // increment variable X by 0.01
                    // CANNOT use: X=X+.01;
}
\end{verbatim}

\subsection{VALUE: Output of Expressions}
\label{sec:value}
\index{variable!value}
\index{value of variable}
The statement
\begin{verbatim}
VALUE,VALUE=expression-vector;
\end{verbatim}
\index{VALUE}
evaluates a set of expressions using the most recent values of
any operands and prints the results on the standard error file.

\noindent Example:
\begin{verbatim}
A=4;
VALUE,VALUE=TABLE(5,#*A);
P1=5;
P2=7;
VALUE,VALUE={P1,P2,P1*P2-3};
\end{verbatim}
These commands give the results:
\begin{verbatim}
value: {0*A,1*A,2*A,3*A,4*A} = {0,4,8,12,16}
value: {P1,P2,P1*P2-3} = {5,7,32}
\end{verbatim}
This commands serves mainly for printing one or more quantities
which depend on matched attributes.
It also allows use of \opal as a programmable calculator.
One may also tabulate functions.


\section{H5Part Export}
\label{sec:h5partexport}
\index{H5Part}
By default \opal writes all data into the H5Part (parallel) file format.
The extension of the input-file will be replaced by {\em .h5}.


\section{Restart Mode}
\label{sec:Restart}
\index{H5Part}
There are two ways to restart from existing file. 
If you want to always restart from and dump phase space data into a single $.h5$ file, 
restart new job using following command.
\begin{verbatim}
opal input.in -restart N --commlib mpi
\end{verbatim}
Here N is the dumpping step from which the job restart.  

On some cluster, if the $.h5$ is very big, it may cause some problems when open the file or
append new data into it. So as another option,you cant create a new $.h5$ file to dump the 
new phase space data for each restart run. In this case you can restart new job using following command.  
\begin{verbatim}
opal input.in -restart N -restartfn H5filename 
     --commlib mpi
\end{verbatim}
Here H5filename is $.h5$ file name (include extension). The rule for the new file is like this:
If H5filename is $input.h5$, then the new file will be $input\_Part00002.h5$,
If H5filename is $input\_Part00002.h5$, then the new file will be $input\_Part00003.h5$.
The rest may be deduced by analogy.

\section{DOOM: Interact with the DOOM Data Base}
\label{sec:doom}
\index{DOOM}
\index{data base}
\opal can now be interfaced to the 
\htmladdnormallink{DOOM}{http://wwwslap.cern.ch/act/doom/} 
(\opal Object-Oriented Database)
data base; this allows the user to transmit most objects known by \opal
to and from user-written programs.
We assume that \opal has been launched with:
\begin{verbatim}
opal -db data-base-name
\end{verbatim}
It will then open the data base \texttt{data-base-name} and read all
data stored therein.
Whenever the user requests it, or when a \texttt{STOP} statement is 
executed, \opal will write all modified objects to the data base and
(temporarily) close it.
Another program can then access the data base and optionally modify it.
\opal regains access to the data base by explicitely re-opening it.

Please note this is not and will never be support for \noopalt and \noopalcycl .

Interaction with the data base is done with the command
\index{DOOM!command}
\begin{verbatim}
DOOM,OPEN=logical,CLOSE=logical,SHUT=logical,
	NOUPDATE=logical,DEBUG=integer;
\end{verbatim}
which has the following attributes:
\begin{description}
\item[CLOSE]
  \index{CLOSE}
  Close the database for good. Afterwards one may continue with \opal without 
  affecting the database contents. At the \texttt{STOP;} command a \\
  \texttt{DOOM,CLOSE;} is executed by default.
\item[SHUT]
  \index{SHUT}
  Close the database temporarily before a launching a user program by a
  \secref{\texttt{SYSTEM} call}{system}.
\item[OPEN]
  \index{OPEN}
  Reopen the database after \texttt{DOOM,CLOSE;} or \texttt{DOOM,SHUT;} 
  and possible execution of a system call. 
  The first \texttt{DOOM,OPEN;} is performed automatically when \opal 
  is launched with a data base name and need not be requested explicitly. 
\item[NOUPDATE]
  \index{NOUPDATE}
  After this command, the database is not updated any more, be it at
  a \texttt{DOOM,SHUT;}, \texttt{DOOM,CLOSE;}, or at the \texttt{STOP;} 
  command. 
  This can come handy when you do not want to spoil the contents of the 
  database. 
  You can revoke the effect (i.e. enable updating) with
\begin{verbatim}
DOOM,NOUPDATE=false;
\end{verbatim}
\item[DEBUG]
  \index{DEBUG}
  Enables debugging, if the \texttt{<integer>} is non-zero.
\end{description}
\noindent Example:
\begin{verbatim}
// Open the data base by default when launching \opal.
// Read some input.
...
DOOM,CLOSE; // Allow access to another program
// Do something which does not require 
// access to the data base
...
SYSTEM,"..."; // Use the data base from outside \opal
// Do another thing which does not require 
// access to the data base
...
DOOM,OPEN; // Re-gain access to the data base
// Modify something.
...
STOP; // Data base is closed by default.
\end{verbatim}

\section{Miscellaneous Commands}

\subsection{ECHO Statement}
\label{sec:echo}
\index{echo of commands}
\index{ECHO}
The \texttt{ECHO} statement has two formats:
\begin{verbatim}
ECHO,MESSAGE=message;
ECHO,message;           // shortcut
\end{verbatim}
\texttt{message} is a \secref{string value}{astring}.
It is immediately transmitted to the \texttt{ECHO} stream.

\subsection{SYSTEM: Execute System Command}
\label{sec:system}
\index{system command}
\index{SYSTEM}
During an interactive \opal session the command \texttt{SYSTEM}
allows to execute operating system commands.
After execution of the system command, successful or not,
control returns to \opal.
At present this command is only available under UNIX or VM/CMS.
It has two formats:
\begin{verbatim}
SYSTEM,CMD=string;
SYSTEM,string;         // shortcut
\end{verbatim}
The \secref{string}{astring} \texttt{string} must be a valid operating
system command. 

\subsection{SYSTEM Command under UNIX}
Most UNIX commands can be issued directly.

\noindent Example:
\begin{verbatim}
SYSTEM,"ls -l"
\end{verbatim}
causes a listing of the current directory in long form on the terminal.

\section{TITLE Statement}
\label{sec:title}
\index{page title}
\index{TITLE}
The \texttt{TITLE} statement has three formats:
\begin{verbatim}
TITLE,STRING=page-header;   // define new page header
TITLE,page-header;          // shortcut for first format
TITLE,STRING="";            // clear page header
\end{verbatim}
\texttt{page-header} is a \secref{string value}{astring}.
It defines the page header which will be used as a title for
subsequent output pages. 
Before the first \texttt{TITLE} statement is encountered, 
the page header is empty.
It can be redefined at any time.

\section{File Handling} 
 
\subsection{CALL Statement}
\label{sec:call}
\index{CALL}
The \texttt{CALL} command has two formats:
\begin{verbatim}
CALL,FILE=file-name;
CALL,file-name;
\end{verbatim}
\texttt{file-name} is a \secref{string}{astring}.
The statement causes the input to switch to the named file.
Input continues on that file until a \texttt{STOP} or an end of file
is encountered. 
\noindent Example:
\begin{verbatim}
CALL,FILE="structure";
CALL,"structure";
\end{verbatim}

\subsection{SAVE Statement}
\label{sec:save}
\index{SAVE}
The \texttt{SAVE} command has two formats:
\begin{verbatim}
SAVE,FILE=file-name
\end{verbatim}
\texttt{file-name} is a \secref{string}{astring}.
The command causes all beam element, beam line, and parameter definitions
to be written on the named file.
%If the \secref{\texttt{OPTION,\opal8}}{option} is active,
%the format is \opal-8 format, otherwise it is \opal format.
%The file may be read again in the same run.
\noindent Examples:
\begin{verbatim}
SAVE,FILE="structure";
SAVE,"structure";
\end{verbatim}

\subsection{MAKESEQ Statement}
\label{sec:makeseq}
\index{MAKESEQ}
A file containing a machine sequence can be generated in \opal by the
command 
\begin{verbatim}
MAKESEQ,LINE=string,NAME=string,FILE=string;
\end{verbatim}

Please note this is not yet supported for \noopalt and \noopalcycl .


The named \secref{beam line}{line} or \secref{sequence}{sequence} is
written as a flat \secref{\texttt{SEQUENCE}}{sequence} with the given
name on the named file. 
%If the \secref{\texttt{OPTION,\opal8}}{option} is active,
%the format is \opal-8 format, otherwise it is \opal format.
All required elements and parameters are also written.
All expressions are evaluated and only their values appear in the
output. 
The command has the following attributes:
\begin{description}
\item[LINE]
  The line for which a flat sequence is to be written.
\item[NAME]
  The name to be given to the sequence written.
\item[FILE]
  The name of the file to receive the output.
\end{description}

\section{IF: Conditional Execution}
\label{sec:if}
\index{IF}
Conditional execution can be requested by an \texttt{IF} statement.
It allows usages similar to the C language \texttt{if} statement:
\begin{verbatim}
IF (logical) statement;
IF (logical) statement; ELSE statement;
IF (logical) { statement-group; }
IF (logical) { statement-group; } ELSE { statement-group; }
\end{verbatim}
Note that all statements must be terminated with semicolons (\texttt{;}),
but there is no semicolon after a closing brace.
The statement or group of statements following the \texttt{IF} is
executed if the condition is satisfied.
If the condition is false, and there is an \texttt{ELSE},
the statement or group following the \texttt{ELSE} is executed.

\section{WHILE: Repeated Execution}
\label{sec:while}
\index{WHILE}
Repeated execution can be requested by a \texttt{WHILE} statement.
It allows usages similar to the C language \texttt{while} statement:
\begin{verbatim}
WHILE (logical) statement;
WHILE (logical) { statement-group; }
\end{verbatim}
Note that all statements must be terminated with semicolons (\texttt{;}),
but there is no semicolon after a closing brace.
The condition is re-evaluated in each iteration.
The statement or group of statements following the \texttt{WHILE} is
repeated as long as the condition is satisfied.
Of course some variable(s) must be changed within the \texttt{WHILE} group 
to allow the loop to terminate.

\section{MACRO: Macro Statements (Subroutines)}
\label{sec:macro}
\index{MACRO}
Subroutine-like commands can be defined by a \texttt{MACRO} statement.
It allows usages similar to C language function call statements.
A macro is defined by one of the following statements:
\begin{verbatim}
name(formals): MACRO { token-list }
name(): MACRO { token-list }
\end{verbatim}
A macro may have formal arguments, which will be replaced by actual arguments 
at execution time. An empty formals list is denoted by \texttt{()}.
Otherwise the \texttt{formals} consist of one or more names, 
separated by commas.
The \texttt{token-list} consists of input tokens 
(strings, names, numbers, delimiters etc.) 
and is stored unchanged in the definition.

A macro is executed by one of the statements:
\begin{verbatim}
name(actuals);
name();
\end{verbatim}
Each actual consists of a set of tokens which replaces all occurrences of
the corresponding formal name.
The actuals are separated by commas.
\noindent Example:
\begin{verbatim}
// macro definitions:
SHOWIT(X): MACRO {
   SHOW, NAME = X;
}
DOIT(): MACRO {
   DYNAMIC,LINE=RING,FILE="DYNAMIC.OUT";
}

// macro calls:
SHOWIT(PI);
DOIT();
\end{verbatim}

\index{control statements|)}
