\chapter{Physics Commands}
\label{sec:physics}
 
\section{Global Reference Momentum}
\label{sec:P0}
\index{P0}
Before any physics computations are attempted the following command
should be entered:
\begin{verbatim}
P0=real;
\end{verbatim}
This command sets the global reference momentum in GeV/c,
which is used to compute the magnetic fields from the normalised
multipole coefficients.
The \secref{\texttt{BEAM}~command}{beam} then renormalises the
multipole coefficients.
This mechanism allows sending a beam with a momentum different from the
design momentum through a beam line.

\section{BEAM Command}
\label{sec:beam}
\index{BEAM}
All OPAL commands working on a beam require the setting of various
quantities related to this beam. 
These are entered by a \texttt{BEAM} command:
\begin{verbatim}
label:BEAM,PARTICLE=name,MASS=real,CHARGE=real,
      ENERGY=real,PC=real,GAMMA=real,BCURRENT=real,
      EX=real,EXN=real,EY=real,EYN=real,ET=real,
      KBUNCH=integer,NPART=real,BUNCHED=logical,
      RADIATE=logical,DAMP=logical,QUANTUM=logical;
\end{verbatim}
The \texttt{label} is opional, it defaults to \texttt{UNNAMED\_BEAM}.
\index{UNNAMED\_BEAM}
The particle mass and charge are defined by:
\begin{description}
\item{PARTICLE}
  \index{PARTICLE}
  The name of particles in the machine.
  OPAL knows the mass and the charge for the following particles:
  \begin{description}
  \item{POSITRON}
    \index{POSITRON}
    The particles are positrons (\texttt{MASS}=$m_e$,
    \texttt{CHARGE}=1),
  \item{ELECTRON}
    \index{ELECTRON}
    The particles are electrons (\texttt{MASS}=$m_e$,
    \texttt{CHARGE}=-1),
  \item{PROTON}
    \index{PROTON}
    The particles are protons (default, \texttt{MASS}=$m_p$,
    \texttt{CHARGE}=1),
  \item{ANTIPROTON}
    \index{ANTIPROTON}
    The particles are anti-protons (\texttt{MASS}=$m_p$,
    \texttt{CHARGE}=-1).
  \end{description}
\end{description}
For other particle names one may enter:
\begin{description}
\item{MASS}
  \index{MASS}
  \index{particle!mass}\index{mass}
  The particle mass in GeV.
\item{CHARGE}
  \index{CHARGE}
  \index{particle!charge}\index{charge}
  The particle charge expressed in elementary charges.
\end{description}
By default the particle momentum is \secref{P0}{P0}.
A different value can be defined by one of the following:
\begin{description}
\item{ENERGY}
  \index{ENERGY}
  \index{particle!energy}\index{energy}
  \index{total energy}
  The total energy per particle in GeV.
  If given, it must be greater then the particle mass.
\item{PC}
  \index{PC}
  \index{momentum}
  \index{particle!momentum}
  The momentum per particle in GeV/c.
  If given, it must be greater than zero.
\item{GAMMA}
  \index{GAMMA}
  The ratio between total energy and rest energy of the particles
  $\gamma = E / m_0$.
  If given, it must be greater than one.
  If the mass is changed a new value for the energy should be entered.
  Otherwise the energy remains unchanged,
  and the momentum and $\gamma$ are recalculated.
\end{description}
The emittances are defined by:
\begin{description}
\item{EX}
  \index{EX}
  \index{emittance!horizontal}
  The horizontal emittance
  $E_x=\sigma_x^2/\beta_x$
  (default:~1~m).
\item{EY}
  \index{EY}
  \index{emittance!vertical}
  The vertical emittance
  $E_y=\sigma_y^2/\beta_y$
  (default:~1~m).
\item{ET}
  \index{ET}
  \index{emittance!longitudinal}
  The longitudinal emittance
  $E_t=\sigma_e/(p_0c) \cdot c\sigma_t$
  (default:~1~m).
  The emittances can be replaced
  by the normalised emittances and the energy spread:
\item{EXN}
  \index{EXN}
  \index{normalised emittance}\index{emittance!normalised}
  The normalised horizontal emittance [m]:
  $E_{xn}=4\beta\gamma E_x$
  (ignored if $E_x$ is given).
\item{EYN}
  \index{EYN}
  The normalised vertical emittance [m]:
  $E_{yn}=4\beta \gamma E_y$
  (ignored if $E_y$ is given).
\item{SIGT}
  \index{SIGT}
  \index{bunch!length}
  The bunch length $c\sigma_t$ [m].
\item{SIGE}
  \index{SIGE}
  \index{energy!spread}
  The {\em relative} energy spread $\sigma_e/p_0 c$ [1].
\end{description}
For the time being, only the particle definition
(\texttt{PARTICLE,MASS,CHARGE,PC,
ENERGY,GAMMA}) is used.
The other parameters will be implemented as needed when new commands
become available.
%Certain commands compute the synchrotron tune $Q_s$
%from the RF cavities.
%If $Q_s\neq 0$,
%the relative energy spread $\sigma_e/p_0c$
%and the bunch length $c\sigma_t$ are
%\[
%\frac{\sigma_e}{p_0c}=\sqrt{\frac{2\pi Q_s E_t}{\eta C}},
%\qquad
%c\sigma_t=\sqrt{\frac{\eta C E_t}{2\pi Q_s}},
%\]
%where $C$ is the machine circumference, and
%$\eta = (1/\gamma^2) - (1/\gamma_{tr}^2)$.
%Finally, the \texttt{BEAM} command accepts
%\begin{description}
%\item{KBUNCH}
%  \index{bunch!number}
%  The number of particle bunches in the machine (default:~1).
%\item{NPART}
%  \index{bunch}
%  The number of particles per bunch (default:~0).
%\item{BCURRENT}
%  \index{beam!current}\index{bunch!current}
%  \index{current}
%  The bunch current (default:~0~A).
%\item{BUNCHED}
%  \index{bunch}
%  A logical flag.
%  If set, the beam is treated as bunched whenever this makes sense.
%\item{RADIATE}
%  \index{radiation}
%  \index{synchrotron!radiation}
%  A logical flag.
%  If set, synchrotron radiation is considered in all bipolar magnets.
%\end{description}
