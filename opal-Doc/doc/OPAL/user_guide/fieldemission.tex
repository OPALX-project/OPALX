\chapter{Field Emission}
\label{chp:femiss}
\index{FieldEmission(}
Field emission is a major source of both dark current particles and primary
incident particles in secondary emission. The Fowler-Nordheim (F-N) formula
we use here to predict the emitted current density is given in (\ref{eq:units})
\cite{BC} \cite{FN}
%
\begin{equation}\label{eq:units}
    J(\mathbf{r},t) = \frac{A(\beta E)^2}{\varphi t(y)^2}
                      \exp{\left(\frac{-B v(y)\varphi^{3/2}}{\beta E}\right)}
                      \left[\hbox{A/$m^2$}\right]
\end{equation}
%
where $J(\mathbf{r},t)$ stands for emitted electric current density in position
$\mathbf{r}$ and time $t$. The Greek letters $\varphi$ and $\beta$ denote the
work function of the surface material and the local field enhancement factor
respectively. The parameter $E$ is the electric field in the normal direction
of surface. The parameters $A$ and $B$ are empirical constants. The functions
$v(y)$ and $t(y)$ representing the image charge effects \cite{BC} as a function
of the Fowler-Nordheim parameter $y$ with the following definition\cite{DE}
%
\begin{equation}\label{eq:imagecharge}
    y = \sqrt{\frac{e^3}{4\pi\varepsilon}}\frac{\sqrt{\beta E}}{\varphi} 
      = 3.795\times10^{-5}\frac{\sqrt{\beta E}}{\varphi} \text{.}
\end{equation}
%
In our model, we have chosen a simpler approximation originated by J. H. Han\cite{DE}
\begin{eqnarray*}
v(y) &=& a-by^2 \\
t(y) &\approx& 1 \text{.}
\end{eqnarray*}
These approximations are valid for a large range of $y$, corresponding to
typical applied electric field ranges in RF guns.\\

Whenever the normal components of an electric field are strong enough the field
emission current density will be limited by space charge effect\cite{BC}. 
To cover this situation we incorporated the 1D Child-Langmuir law
%
\begin{align}\label{eq:SpaceCharge}
    J(\mathbf{r},t) & =\frac{4\varepsilon_0}{9}\sqrt{2\frac{e}{m}}\left(\frac{V^{3/2}}{d^2}\right)\notag\\
    &
    =\frac{4\varepsilon_0}{9}\sqrt{2\frac{e}{m}}\left(\frac{E^{3/2}}{d^{1/2}}\right)
    \left[\hbox{A/$m^2$}\right]
\end{align}
%
into our field emission model. $J(\mathbf{r},t)$ denotes space charge limited emission 
current density in position $\mathbf{r}$ and time $t$, $\varepsilon_0$ the
permittivity in vacuum, $E$ the normal component of electric field on the surface
and $d$ the distance from the position where $E$ is evaluated. Currently we 
choose $d$ to be equal to the distance traveled by emitted particles in one
time step, i.e., $d=\frac{\displaystyle eE\Delta{t}^2}{\displaystyle 2m_0}$ where $\Delta{t}$ is simulation
time step.

\section{Field Emission Command}
\label{sec:FieldEmissionCmd}
\index{FIELDEMISSIONCMD}
To perform field emission related simulation, a triangulated surface geometry defined by GEOMETRY command (see  Chapter \ref{chp:geometry})  should be specified and attached to the elements (currently only "RFCavity" element is valid for field emission). A "SURFACEEMISSION" type of distribution, defined in DISTRIBUTION command should be attached to the GEOMETRY command. And users can customize dark current simulation by specifying the value of the work
function $\varphi$, local field enhancement factor $\beta$ and other parameters
present in (\ref{eq:units}) and (\ref{eq:imagecharge}) in the "SURFACEEMISSION" type of distribution definition in input file. See the following example input file and Table \ref{tab:fieldfmissioncmd} for a summary of the field emission related command in the "SURFACEEMISSION" type of distribution definition.\\
\begin{alltt}
DistSurf: DISTRIBUTION, DISTRIBUTION = "SURFACEEMISSION", 
          NPDARKCUR = 0, INWARDMARGIN=0.0, 
          FNBETA = 30, FNMAXEMI = 2,  
          FNFIELDTHR = -0.1;
ge:       GEOMETRY, FGEOM="../New_Gun.h5", 
          S=0.0, DISTR=DistSurf, 
          ZSHIFT=0.0; 
FINSSGUN: RFCavity, L = 0.175, 
          VOLT = 100.0, FMAPFN = "../RF_GUN_PSI-fieldmap.T7" , 
          GEOMETRY = ge, ELEMEDGE =0.0, 
          TYPE = "STANDING", FREQ = 2997.922938148;
\ldots
\end{alltt}

\begin{table}[ht] \footnotesize
  \begin{center}
    \caption{Field Emission Command summary}
    \label{tab:fieldfmissioncmd}
    \begin{tabular}{|l|p{0.8\textwidth}|l|}
      \hline
      Command &Purpose (Default) \\
      \hline
      FNA&Empirical constant A for F-N emission model ($1.54\times10^{-6}$)\\
      FNB&Empirical constant B for F-N emission model ($6.83\times10^9$) \\
      FNY&Constant for image charge effect parameter y(E) ($3.795\times10^{-5}$) \\
      FNVYZERO&Zero order constant for v(y) function ($0.9632$) \\
      FNVYSECOND&Second order constant for v(y) function ($1.065$)\\
      FNPHIW&Work function of gun surface material ($4.65$ eV) \\
      FNBETA&Field enhancement factor $\beta$ for F-N emission ($50.0$) \\
      FNFIELDTHR&Field threshold for F-N emission ($30.0$ MV/m) \\
      FNMAXEMI&Maximum Number of electrons emitted from a single triangle in each time step ($10$) \\

      \hline
    \end{tabular}
  \end{center}
\end{table}



