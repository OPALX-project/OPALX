\begin{thebibliography}{10}
%1 
\bibitem{bib:GKS}
{\sl The Graphical Kernel System (GKS)}.
ISO, Geneva, July 1985.
International Standard ISO 7942.
%2 
\bibitem{bib:autin}
B. Autin and Y. Marti.
{\sl Closed Orbit Correction of Alternating Gradient Machines
  using a small Number of Magnets}.
CERN/ISR-MA/73-17, CERN, 1973.
%3 
\bibitem{bib:barber} D.P.~Barber, K. Heinemann, H.~Mais and G.~Ripken,
  {\sl A Fokker--Planck Treatment of Stochastic Particle Motion within
    the Framework of a Fully Coupled 6-dimensional Formalism for
    Electron-Positron Storage Rings including Classical Spin Motion in
    Linear Approximation}, DESY report 91-146, 1991.
%4
\bibitem{bib:scandale} R.~Bartolini, A.~Bazzani, M.~Giovannozzi,
  W.~Scandale and E. Todesco, {\sl Tune evaluation in simulations and
    experiments}, CERN SL/95-84 (AP) (1995).
%5 
\bibitem{bib:bjorken}
J.~D.~Bjorken and S.~K.~Mtingwa.
{\sl Particle Accelerators} {\bf 13}, pg. 115.
%6 
\bibitem{bib:bollt} E.M.~Bollt and J.D.~Meiss, {\sl Targeting chaotic
    orbits to the Moon through recurrence}, Phys.~Lett. A{\bf 204},
  373 (1995).
%7 
\bibitem{B-bib:bramham}
P. Bramham and H. Henke.
private communication and LEP Note LEP-70/107, CERN.
%8 
\bibitem{bib:matrix}
Karl~L. Brown.
{\sl A First-and Second-Order Matrix Theory for the Design
  of Beam Transport Systems and Charged Particle Spectrometers}.
SLAC 75, Revision 3, SLAC, 1972.
%9 
\bibitem{bib:transport}
Karl~L. Brown, D.~C. Carey, Ch. Iselin, and F. Rothacker.
{\sl TRANSPORT --- A Computer Program for Designing Charged
  Particle Beam Transport Systems}.
CERN 73-16, revised as CERN 80-4, CERN, 1980.
%10 
\bibitem{bib:chao}
A. Chao.
{\sl Evaluation of beam distribution parameters in an electron
  storage ring}.
Journal of Applied Physics, 50:595--598, 1979.
%11
\bibitem{bib:touschek}
A.~W. Chao and M.~J. Lee.
{\sl SPEAR II Touschek lifetime}.
SPEAR-181, SLAC, October 1974.
%12 
\bibitem{bib:conte}
M.~Conte and M.~Martini.
{\sl Particle Accelerators} {\bf 17}, 1 (1985).
%13
\bibitem{bib:courant}
E.~D. Courant and H.~S. Snyder.
{\sl Theory of the alternating gradient synchrotron}.
Annals of Physics, 3:1--48, 1958.
%14 
\bibitem{bib:TFS}
Ph. Defert, Ph. Hofmann, and R. Keyser.
{\sl The Table File System, the C Interfaces}.
LAW Note 9, CERN, 1989.
%15 
\bibitem{bib:HARMON}
M. Donald and D. Schofield.
{\sl A User's Guide to the {\tt HARMON} Program}.
LEP Note 420, CERN, 1982.
%16 
\bibitem{bib:dragt}
A. Dragt.
{\sl Lectures on Nonlinear Orbit Dynamics, 1981 Summer School on High
  Energy Particle Accelerators, Fermi National Accelerator Laboratory, July
  1981}.
American Institute of Physics, 1982.
%17 
\bibitem{bib:edwards}
D.~A. Edwards and L.~C. Teng.
{\sl Parametrisation of linear coupled motion in periodic systems}.
IEEE Trans. on Nucl. Sc., 20:885, 1973.
%18  
\bibitem{bib:giovannozzi} M.~Giovannozzi, {\sl Analysis of the stability
    domain of planar symplectic maps using invariant manifolds},
  CERN/PS 96-05~(PA) (1996).
%19 
\bibitem{bib:gxplot}
H. Grote.
{\sl GXPLOT User's Guide and Reference Manual}.
LEP TH Note 57, CERN, 1988.
%20 
\bibitem{bib:LEP}
LEP~Design Group.
{\sl Design Study of a 22 to 130 GeV e\({}^{+}\)e\({}^{-}\) Colliding Beam
  Machine (LEP)}.
CERN/ISR-LEP/79-33, CERN, 1979.
%21 
\bibitem{bib:BMPM}
M. Hanney, J.~M. Jowett, and E. Keil.
{\sl BEAMPARAM --- A program for computing beam dynamics and
  performance of $e^{+}e^{-}$ storage rings}.
CERN/LEP-TH/88-2, CERN, 1988.
%22 
\bibitem{bib:helm}
R.~H. Helm, M.~J. Lee, P.~L. Morton, and M. Sands.
{\sl Evaluation of synchrotron radiation integrals}.
IEEE Trans. Nucl. Sc., NS-20, 1973.
%23 
\bibitem{bib:MINUIT}
F. James.
{\sl MINUIT, A package of programs to minimise a function of n
  variables, compute the covariance matrix, and find the true errors}.
program library code D507, CERN, 1978.
%24 
\bibitem{bib:keil}
E. Keil.
{\sl Synchrotron radiation from a large electron-positron storage ring}.
CERN/ISR-LTD/76-23, CERN, 1976.
%25 
\bibitem{bib:knuth}
D.~E. Knuth.
{\sl The Art of Computer Programming}.
Volume~2, Addison-Wesley, second edition, 1981.
Semi-numerical Algorithms.
%26 
\bibitem{bib:laskar} J.~Laskar, C.~Froeschl\'{e} and A.~Celletti, {\sl The
    measure of chaos by the numerical analysis of the fundamental
    frequencies. Application to the standard mapping}, Physica D{\bf
    56}, 253 (1992).
%27
\bibitem{bib:mais}
H.~Mais and G.~Ripken,
{\sl Theory of Coupled Synchro-Betatron Oscillations}.
DESY internal Report, DESY M-82-05, 1982.
%28
\bibitem{bib:meddahi} M.~Meddahi, {\sl Chromaticity correction for the
    $108^{\circ}/60^{\circ}$ lattice}, CERN SL/Note 96-19 (AP) (1996).
%29
\bibitem{bib:milutinovic}
J. Milutinovic and S. Ruggiero.
{\sl Comparison of Accelerator Codes for a RHIC Lattice}.
AD/AP/TN-9, BNL, 1988.
%30 
\bibitem{bib:montague}
B.~W. Montague.
{\sl Linear Optics for Improved Chromaticity Correction}.
LEP Note 165, CERN, 1979.
%31
\bibitem{bib:ripken}
Gerhard Ripken,
{\sl Untersuchungen zur Strahlf\"uhrung und Stabilit\"at der
Teilchenbewegung in Beschleunigern und Storage-Ringen unter strenger
Ber\"ucksichtigung einer Kopplung der Betatronschwingungen}.
DESY internal Report R1-70/4, 1970.
%32 
\bibitem{bib:chamonix} F.~Ruggiero, {\sl Dynamic Aperture for LEP~2 with
    various optics and tunes}, Proc. Sixth Workshop on LEP
  Performance, Chamonix, 1996, ed.~J.~Poole (CERN SL/96-05 (DI),
  1996), pp.~132--136.
%33 
\bibitem{bib:teng}
L.~C. Teng.
{\sl Concerning n-Dimensional Coupled Motion}.
FN 229, FNAL, 1971.
%34 
\bibitem{bib:volkel}
U. V\"olkel.
{\sl Particle loss by Touschek effect in a storage ring}.
DESY 67-5, DESY, 1967.
%35 
\bibitem{bib:walker}
R.~P. Walker.
{\sl Calculation of the Touschek lifetime in electron storage rings}.
1987.
Also SERC Daresbury Laboratory preprint, DL/SCI/P542A.
%36 
\bibitem{bib:wilson}
P.~B. Wilson.
{\sl Proc. 8th Int. Conf. on High-Energy Accelerators}.
Stanford, 1974.
%37 
\bibitem{bib:wrulich}
A. Wrulich and H. Meyer.
{\sl Life time due to the beam-beam bremsstrahlung effect}.
PET-75-2, DESY, 1975.
%
\bibitem{langdon}
  Birdsall, Langdon.
  \newblock {\em Plasma Physics via computer simulation}.
  \newblock Page 340-341

  \bibitem{fubiani2006}
  G.\ Fubaiani, J.\ Qiang, E.\ Esarey, W.\ P.\ Leemans, G.\ Dugan.
  \newblock {\em Space charge modeling of dense electron beams with large energy spreads}.
  \newblock Acclerators and Beams, 9, 064402 (2006) 
 
 \bibitem{qiang2005}
 Ji Qiang, Steve Lidia, Robert D. Ryne, and Cecile Limborg-Deprey
  \newblock {\em A Three-Dimensional Quasi-Static Model for High Brightness Beam Dynamics Simulation}
    \newblock Lawrence Berkeley National Laboratory. Paper LBNL-59098. http://repositories.cdlib.org/lbnl/LBNL-59098
 
 \bibitem{qiang2006-2}
 J. Qiang, S. Lidia, R. D. Ryne, C. Limborg-Deprey, 
  \newblock Phys. Rev. Special Topics - Accel. Beams 9, 044204, (2006).
 
  \bibitem{JohoDist}
 W. Joho
  \newblock private discussions
 
\bibitem{superfish}
J. H. Billen, L. M. Young
{\sl POISSON SUPERFISH}.
LA-UR-96-1834, Los Alamos National Laboratory, 2004
 
\bibitem{SanzSerna}
J.M. Sanz-Sern and M.P. Calvo
\newblock {\em Numerical Hamiltonian Problems}
Chapman and Hall 1994

\bibitem{forestall}
 E. Forest. Phys. Lett. A 158, p99, 1991

% @Article{arbenzetal2001,
%   author = 	 {Arbenz, P. and Geus, R. and Adam, S.},
%   title = 	 {Solving Maxwell eigenvalue problems for accelerating cavities},
%   journal = 	 {Physical Review Special Topics - Accelerators and Beams},
%   year = 	 {2001},
%   volume = 	 {4},
%   pages = 	 {022001-1:022001-10},
%   annote = 	 {Abstract: we investigate algorithms for computing steady state electromagnetic waves
%                   in cavities. The Maxwell equations for the strength of the electric field are solved by
%                   a mixed method with quadratic finite edge (N{\'e}d{\'e}lec) elements for the field
%                   values and corresponding node-based finite elements for the Lagrange multiplier, This
%                   approach avoids so-called spurious modes which are introduced if the divergence-free
%                   condition for the electric field is not treated properly. To compute a few of the smalles
%                   positive eigenvalues and corresponding eigenmodes of the resulting large sparse-matrix
%                   eigenvalue problems, two algorithms have been used: the implicitly restarted Lanczos
%                   algorithm and the Jacobi-Davidson algorithm, both with shit-and-invert spectral transformation.
%                   Two-level hierarchical basis preconditioners have been employed for the iterative solution
%                   of the resulting systems of equations.}
% }

 \bibitem{bib:arbenzetal2001}
P. Arbenz, R. Geus and S. Adam
{Solving Maxwell eigenvalue problems for accelerating cavities,
Physical Review Special Topics - Accelerators and Beams,
4, pp. 022001-1:022001-10, 2001.}

% @article{arbenzetal2006,
%   author =       {P. Arbenz and M. Be\v{c}ka and R. Geus and
%                   U. Hetmaniuk and T. Mengotti},
%   title =        {On a Parallel Multilevel Preconditioned {M}axwell
%                   Eigensolver},
%   journal =      {PARCO},
%   volume =       {32},
%   number =       {2},
%   pages =        {157-165},
%   year =         {2006},
%   doi =          {10.1016/j.parco.2005.06.005}
% }
\bibitem{bib:arbenzetal2006}
P. Arbenz, M. Be\v{c}ka, R. Geus, U. Hetmaniuk and T. Mengotti\\
\newblock{ \em On a Parallel Multilevel Preconditioned {M}axwell Eigensolver}\\
PARCO, 32(2), pp. 157-165, 2006, doi: 10.1016/j.parco.2005.06.005
 
\end{thebibliography}
