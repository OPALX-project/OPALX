\chapter{Tables}
\label{sec:tables}
\index{tables|(}

\section{Introduction}
\label{sec:tabintro}

Please note Tables are not yet supported in \noopalt and \noopalcycl . They probable does not make sense in a non map based computation.  A typical computation in \opal  follows \figref{these steps}{tables}:

\begin{enumerate}
\item
  Define a beam line using a \secref{\texttt{LINE}}{line}
  or \secref{\texttt{SEQUENCE}}{sequence} command.
\item
  Define a beam with using \secref{\texttt{BEAM}}{beam} command.
\item
  If desired, define imperfections on the defined beam line,
  using one of the \secref{error commands}{error}.
\item
  If desired, attach special integrators to selected elements of the
  defined beam line,
  using \secref{\texttt{SETINTEGRATOR}}{setint} commands.
\item
  If desired, perform an orbit correction on the defined beam line,
  using \secref{\texttt{THREADBPM}}{thread} or
  \secref{\texttt{MICADO}}{micado} command.
\item
  Define a ``Table Object'' using one of the \secref{table}{tables} commands. 
  This selects a \secref{\texttt{RANGE}}{arange} of a beam line,
  and defines an algorithm to fill the table,
  e.~g. with the lattice functions, 
  the accumulated transfer map from the beginning of the range to the
  current position, or the \secref{\texttt{SURVEY} data}{survey}.
  A table remains in memory and is recomputed as required.
  It is erased only if one of the contained elements or beam lines is changed. 
\item
  If desired, apply the \secref{match module}{match} to adjust parameters
  in the machine. The match module can work on any number of tables
  simultaneously. 
\item
  Use a \secref{\texttt{LIST}}{list} command to print the contents of any
  table. 
\end{enumerate}

\begin{figure}[ht]
  \begin{center}
    \begin{picture}(400,400)
      \thinlines
      \put(90,370){\vector(0,-1){30}}
      \put(150,370){\vector(0,-1){30}}
      \put(80,310){\framebox(80,30){beam line}}
      \put(210,325){\vector(-1,0){50}}
      \put(240,325){\circle{60}}
      \put(240,325){\makebox(0,0){\shortstack{Error\\Generator}}}
      \put(320,325){\vector(-1,0){50}}
      \put(120,310){\vector(0,-1){30}}
      \put(120,250){\circle{60}}
      \put(120,250){\makebox(0,0){\shortstack{Table\\Generator}}}
      \put(320,250){\vector(-1,0){170}}
      \put(120,220){\vector(0,-1){30}}
      \put(80,160){\framebox(80,30){Table object}}
      \put(210,175){\vector(-1,0){50}}
      \put(240,175){\circle{60}}
      \put(240,175){\makebox(0,0){\shortstack{Match\\Module}}}
      \put(320,175){\vector(-1,0){50}}
      \put(120,100){\circle{60}}
      \put(120,160){\vector(0,-1){30}}
      \put(120,100){\makebox(0,0){\shortstack{Table\\Lister}}}
      \put(320,100){\vector(-1,0){170}}
      \put(120,70){\vector(0,-1){30}}
      \put(80,10){\framebox(80,30){Output}}

      \Thicklines
      \put(20,370){\framebox(80,30){\shortstack{LINE or\\SEQUENCE}}}
      \put(140,370){\framebox(80,30){BEAM}}
      \put(320,310){\framebox(80,30){Error Command}}
      \put(320,235){\framebox(80,30){Table Command}}
      \put(320,160){\framebox(80,30){Match Command}}
      \put(320,85){\framebox(80,30){LIST Command}}
    \end{picture}
    \caption{Schematic view of the Interaction between Objects in \opal}
    \label{fig:tables}
  \end{center}
\end{figure}

\section{Element Selection}
\label{sec:select}
\index{element!selection}
\index{select!element}

Many \opal commands allow for the possibility to process
a subset of the elements occurring in a \secref{beam line}{lines}.
For this purpose each \secref{beam line}{lines} and
\secref{table}{tables} has a selection flag for each element.
The \texttt{SELECT} command may be used to manipulate these flags.
It affects the following commands:
\begin{itemize}
\item \secref{\texttt{EALIGN}}{erroralign},
\item \secref{\texttt{EFIELD}}{errorfield},
\item \secref{\texttt{EFCOM}}{errorfield},
\item \secref{\texttt{EPRINT}}{errorprint},
\item \secref{\texttt{ESAVE}}{errorsave},
\item \secref{\texttt{SURVEY}}{survey},
\item \secref{\texttt{TWISS}}{twiss},
\item \secref{\texttt{ATTLIST}}{attlist},
\item \secref{\texttt{LIST}}{list},
\item \secref{\texttt{EIGEN}}{canned},
\item \secref{\texttt{ENVELOPE}}{canned},
\item \secref{\texttt{MATRIX}}{canned},
\item \secref{\texttt{TWISS3}}{canned}.
\end{itemize}
Three formats are recognised for the command:
\begin{verbatim}
SELECT,LINE=name,FULL;
SELECT,LINE=name,CLEAR;
SELECT,LINE=name,RANGE=range,CLASS=name,TYPE=name,
                PATTERN=regular_expression;
\end{verbatim}
\index{SELECT}
All three forms require the parameter
\begin{description}
\item[LINE]
  \index{LINE}
  The label of a previously defined \secref{beam line}{lines} or
  \secref{table}{tables} on which the selection is to be done (no default).
\end{description}
The first format sets the selection flag for all elements in the beam line
and the second format clears the selection flags for all elements.

The third format keeps all existing selections and additionally marks those
elements as selected which belong to the intersection of the following four
sets: 
\begin{description}
\item[RANGE]
  \index{RANGE}
  If \texttt{RANGE} is omitted, the first set contains all elements;
  if it is given, the first set is limited to the named range in the line. 
  The default is equivalent to \texttt{RANGE=\#S/\#E}.
\item[CLASS]
  \index{CLASS}
  The \secref{\texttt{label}}{label} of an element class (default: blank).
  If \texttt{CLASS} is omitted, the second set contains all elements;
  if it is given, the set contains only the elements derived directly or
  indirectly from the named class.
\item[TYPE]
  \index{TYPE}
  If \texttt{TYPE} is omitted, the third set contains all elements;
  if it is given, the set contains only the elements whose \texttt{TYPE}
  attribute is equal to the given name. 
\item[PATTERN]
  \index{PATTERN}
  If \texttt{PATTERN} is omitted, the fourth set contains all elements;
  if it is given, the \secref{\texttt{regular expression}}{wildcard}
  is applied to all element names, and the set contains only the matching
  elements.
\end{description}
The effect of subsequent \texttt{SELECT} commands produces the union of all
selections. 
If a fresh selection is desired, precede the new \texttt{SELECT} command
with the command
\begin{verbatim}
SELECT,LINE=name,CLEAR;
\end{verbatim}
Example:
\begin{verbatim}
SELECT,LINE=X,RANGE=IP1/IP2;                            // (1)
SELECT,LINE=X,CLASS=BB;                                 // (2)
SELECT,LINE=X,PATTERN=".*\.L1";                         // (3)
SELECT,LINE=X,RANGE=IP1/IP2,CLASS=BB,PATTERN=".*\.L1";  // (4)
SELECT,LINE=LHC,CLASS=IP;                               // (5)
T:TWISS,LINE=LHC;                                       // (6)
SELECT,LINE=T,CLASS=QUADRUPOLE;                         // (7)
TWISS3,TABLE=T,FILE=name,TFS;                           // (8)
U:TWISS,LINE=LHC;                                       // (9)
\end{verbatim}
Command (1) selects all elements from \texttt{IP}1 to \texttt{IP2}, 
both included.
Command (2) selects all elements in the ring which belong to or are 
derived from class \texttt{BB}.
Command (3) selects all elements in the ring whose names end \texttt{.L1}.
Command (4) is the most restrictive, it selects all elements
between \texttt{IP1} and \texttt{IP2} which are derived from class 
\texttt{BB} and whose names end in \texttt{.L1}.

Command (5) sets the selections flags on all interaction points in
line \texttt{LHC}.
The selection is transmitted to command (6), which generates the
\texttt{TWISS} table \texttt{T} lists the default columns for the
interaction points. 
Command (7) sets additional selection flags on the table \texttt{T}
for all quadrupoles,
and command (8) lists the Mais-Ripken functions for all interaction
points and all quadrupoles.
Command (9) gets the selection from line \texttt{LHC}, i.~e. the
selection concerns only the interaction points.

\section{Attach Special Integrators}
\label{sec:setint}
\index{SETINTEGRATOR}
\index{integrator}

By default the algorithm applied to each element in the machine is
defined by the table command.
It may be overridden for selected elements by applying a \texttt{SETINTEGRATOR}
command to attach a special integrator.
It has the form
\begin{verbatim}
SETINTEGRATOR,LINE=name,TYPE=name,SLICES=integer;
\end{verbatim}
with the meaning
\begin{description}
\item[LINE]
  \index{LINE}
  The name of the beam line to be affected.
\item[TYPE]
  \index{TYPE}
  The integrator type name.
  For the time being, only the name \texttt{MPSPLITINTEGRATOR} is
  known.
  \index{MPSPLITINTEGRATOR}
  This integrator type replaces any multipole-like element (dipole,
  quadrupole, sextupole, octupole, general multipole) by one or
  several thin lenses.
\item[SLICES]
  \index{SLICES}
  The number of thin lenses to be used.
  For one to four slices, the TEAPOT method of thin multipole slicing
  \index{TEAPOT}
  is used, for more slices, the thin lenses are placed in equidistant
  places. 
\end{description}
The command runs through all elements in the beam line,
and attaches an integrator as specified by \texttt{TYPE} to all
selected elements.


\section{Compute Geometric Layout}
\label{sec:survey}
\index{SURVEY}

The \texttt{SURVEY} command builds a table containing the geometric layout
of the machine.
The table can be referred to by its label for later manipulation.
It has the following read/write attributes:
\begin{verbatim}
label:SURVEY,LINE=name,RANGE=range,REVERSE=logical,
	STATIC=logical, X0=real,Y0=real,Z0=real,THETA0=real,
    	PHI0=real,PSI0=real;
label:SURVEY,LINE=name,RANGE=range,REVERSE=logical,
	STATIC=logical,INIT=table-row;
\end{verbatim}
Its parameter list specifies the initial position and orientation
of the reference orbit in the \figref{global coordinate system}{global} 
$(X,Y,Z)$.
Omitted attributes assume zero values.
Valid attributes are:
\begin{description}
\item[LINE]
  \index{LINE}
  The label of a previously defined beam line (no default).
\item[RANGE]
  \index{RANGE}
  If this attribute is given, the table is restricted to the range given.
\item[STATIC]
  \index{STATIC}
  If this attribute is true, the table is filled at definition time,
  and it is then treated as frozen, even if parameters of the machine change.
\item[INIT]
  \index{INIT}
  If this attribute is given, the initial position and direction is taken
  from the specified row of another \texttt{SURVEY} table.
  If this attribute is omitted, the initial position and direction is
  constructed from the attributes listed below.
\item[X0]
  \index{X0}
  The initial X coordinate [m].
\item[Y0]
  \index{Y0}
  The initial Y coordinate [m].
\item[Z0]
  \index{Z0}
  The initial Z coordinate [m].
\item[THETA0]
  \index{THETA0}
  The initial angle theta [rad].
\item[PHI0]
  \index{PHI0}
  The initial angle phi [rad].
\item[PSI0]
  \index{PSI0}
  The initial angle psi [rad].
\end{description}

Example:
\begin{verbatim}
GEOMETRY:SURVEY,LINE=RING;
\end{verbatim}
This example computes the machine layout for the line \texttt{RING} with
zero initial conditions and saves the result under the name
\texttt{GEOMETRY}. 

The \texttt{SURVEY} table has some \tabref{read-only attributes}{survey_glob}:
\begin{table}[Ht]
  \caption{Global read-only values in a \texttt{SURVEY} table}
  \label{tab:survey_glob}
  \begin{center}
    \begin{tabular}{|l|l|l|}
      \hline
      Variable & Unit & Name \\
      \hline
      Total arc length & m & \texttt{L}\index{L} \\
      Final X coordinate & m & \texttt{X}\index{X} \\
      Final Y coordinate & m & \texttt{Y}\index{Y} \\
      Final Z coordinate & m & \texttt{Z}\index{Z} \\
      Final angle theta & rad & \texttt{THETA}\index{THETA} \\
      Final angle phi & rad & \texttt{PHI}\index{PHI} \\
      Final angle psi & rad & \texttt{PSI}\index{PSI} \\
      \hline
    \end{tabular}
  \end{center}
\end{table}

In each position there are \tabref{table values}{survey_col}, 
each of which is accessible with the \secref{syntax}{acell}
\begin{verbatim}
table-name "@" place "->" column-name
\end{verbatim}
The azimuth, elevation angle, and roll angle are defined as the
components of a vector.
This vector points in the direction of the rotation axis and has the
rotation angle as its length.
The direction cosines are the components of the unit vectors along the
local coordinate axes, expressed in the global coordinate system.

\begin{table}[Ht]
  \caption{Column values available in a \texttt{SURVEY} table}
  \label{tab:survey_col}
  \begin{center}
    \begin{tabular}{|l|l|l|}
      \hline
      Variable & Unit & Name \\
      \hline
      Arc length from beginning & m & \texttt{S}\index{S} \\
      Global X coordinate & m & \texttt{X}\index{X} \\
      Global Y coordinate & m & \texttt{Y}\index{Y} \\
      Global Z coordinate  & m & \texttt{Z}\index{Z} \\
      Azimuth & rad & \texttt{THETA}\index{THETA} \\
      Elevation angle & rad & \texttt{PHI}\index{PHI} \\
      Roll angle & rad & \texttt{PSI}\index{PSI} \\
      \hline
    \end{tabular}

    \begin{tabular}{|l|l|l|l|}
      \hline
      Direction cosines for local axes & \multicolumn{3}{c|}{Local axis} \\
      Global direction & \texttt{x} & \texttt{y} & \texttt{s} \\
      \hline
      \texttt{X} & \texttt{W11}\index{W11} &
      \texttt{W12}\index{W12} & \texttt{W13}\index{W13} \\
      \texttt{Y} & \texttt{W21}\index{W21} &
      \texttt{W22}\index{W22} & \texttt{W23}\index{W23} \\
      \texttt{Z} & \texttt{W31}\index{W31} &
      \texttt{W32}\index{W32} & \texttt{W33}\index{W33} \\
      \hline
    \end{tabular}
  \end{center}
\end{table}
\clearpage
The default column set for the command
\secref{\texttt{LIST,ALL}}{list} contains the element name,
the arc length, the global positions and the three angles.

\section{Orbit Threader}
\label{sec:thread}
\index{orbit threader}
\index{THREADBPM}
\index{threader}
When the closed orbit search in a \secref{\texttt{MICADO}}{micado} or
\secref{\texttt{TWISS}}{twiss} command fails due to machine imperfections, 
it may be possible to thread the orbit through the machine using the command 
\begin{verbatim}
THREADBPM,LINE=name,BEAM=name,RANGE=range,TOL=real,
                           LISTC=logical,LISTM=logical;
\end{verbatim}
This command has the attributes:
\begin{description}
\item[LINE]
  \index{LINE}
  The label of a previously defined beam line (no default).
\item[BEAM]
  \index{BEAM}
  The label of a previously defined beam (default \texttt{UNNAMED\_BEAM}).
  This is used to define the type and reference momentum for the
  particles to be sent through the line.
\item[RANGE]
  \index{RANGE}
  If this attribute is given, the table is restricted to the range given.
\item[TOL]
  \index{TOL}
  The orbit is observed at all beam position monitors (BPM).
  When the orbit deviation exceeds \texttt{TOL},
  a correction is attempted.
  \opal backtracks to find another monitor and two correctors and tunes the
  latter so as to adjust the readings of the two monitors to zero.
\item[LISTC]
  \index{LISTC}
  If true, list the corrector settings after correction.
\item[LISTM]
  \index{LISTM}
  If true, list the monitor readings after correction.
\end{description}

\section{Closed Orbit Correction}
\label{sec:micado}
\index{MICADO}
\index{closed orbit}
\index{orbit correction}
Closed orbit correction with the MICADO algorithm is initiated by the command 
\begin{verbatim}
MICADO,LINE=name,BEAM=name,RANGE=range,METHOD=name,
	TOL=real, ITERATIONS=integer,CORRECTORS=integer,
	PLANE=name, LISTC1=logical,LISTM1=logical,
	LISTC2=logical,LISTM2=logical;
\end{verbatim}
This command has the attributes:
\begin{description}
\item[LINE]
  \index{LINE}
  The label of a previously defined beam line (no default).
\item[BEAM]
  \index{BEAM}
  The label of a previously defined beam (default \texttt{UNNAMED\_BEAM}).
  This is used to define the type and reference momentum for the
  particles to be sent through the line.
\item[RANGE]
  \index{RANGE}
  If this attribute is given, the table is restricted to the range given.
\item[METHOD]
  \index{METHOD}
  This attribute specifies the method to be used for tracking the orbit.
  Known names are:
  \begin{description}
  \item[THIN]
    \index{THIN}
    Use thin lens approximations.
  \item[LINEAR]
    \index{LINEAR}
    Use linear map approximation (around the closed orbit).
    This is the default.
  \item[THICK]
    \index{THICK}
    Use finite-length lenses.
  \end{description}
\item[CORRECTORS]
  \index{CORRECTORS}
  The maximum number of correctors to be used in each iteration.
\item[TOL]
  \index{TOL}
  The tolerance for the r.m.s. closed orbit.
\item[ITERATIONS]
  \index{ITERATIONS}
  The number of iterations desired.
  When this number is zero,
  the orbit is only checked but not corrected.
  Each iteration is terminated when the imposed number of correctors has been
  used or when the tolerance is reached.
\item[PLANE]
  \index{PLANE}
  The plane to be corrected.
  Known names are:
  \begin{description}
  \item[X]
    \index{X}
    Correct the horizontal plane only.
  \item[Y]
    \index{Y}
    Correct the vertical plane only.
  \item[BOTH]
    \index{BOTH}
    Correct both planes (default).
  \end{description}
\item[LISTC1]
  \index{LISTC1}
  If true, list the corrector settings before each iteration.
\item[LISTM1]
  \index{LISTM1}
  If true, list the monitor readings after each iteration.
\item[LISTC2]
  \index{LISTC2}
  If true, list the corrector settings after correction.
\item[LISTM2]
  \index{LISTM2}
  If true, list the monitor readings after correction.
\end{description}

\section{Lattice Function Tables}
\label{sec:twiss}
\index{lattice functions}
\index{TWISS}
\index{TWISSTRACK}

Two commands are available to generate tables of lattice functions:
\begin{verbatim}
TWISS,LINE=name,BEAM=name,RANGE=range,STATIC=logical,METHOD= name,
      REVBEAM=logical,REVTRACK=logical;
TWISSTRACK,BEAM=name,RANGE=range,STATIC=logical,METHOD= name,
      INIT=table-row,REVBEAM=logical,REVTRACK=logical;
\end{verbatim}

Either command builds a ``twiss'' table containing the lattice functions for
a range of a previously defined \secref{beam line}{lines}.
The first command, \texttt{TWISS}, uses the periodic solution for that
range, while the second command, \texttt{TWISSTRACK} a selected line
of another twiss table.
The table can be referred to by its label for later manipulation.
Both commands have the following read/write attributes:
\begin{description}
\item[LINE]
  \index{LINE}
  The label of a previously defined beam line (no default).
\item[BEAM]
  \index{BEAM}
  The label of a previously defined beam (default \texttt{UNNAMED\_BEAM}).
  This is used to define the type and reference momentum for the
  particles to be sent through the line.
\item[RANGE]
  \index{RANGE}
  If this attribute is given, the table is restricted to the range given.
\item[STATIC]
  \index{STATIC}
  If this attribute is true, the table is filled at definition time,
  and it is then treated as frozen, even if parameters of the machine change.
\item[METHOD]
  \index{METHOD}
  This attribute specifies the method to be used for filling the table.
  Known names are:
  \begin{description}
  \item[THIN]
    \index{THIN}
    Use thin lens approximations.
  \item[LINEAR]
    \index{LINEAR}
    Use linear map approximation (around the closed orbit).
    This is the default.
  \item[THICK]
    \index{THICK}
    Use finite-length lenses.
  \end{description}
\item[REVBEAM]
  \index{REVBEAM}
  If true, the beam is assumed to run backwards through the beam line
  (from $s=C$ to $s=0$).
\item[REVTRACK]
  \index{REVTRACK}
  If true, the calculation proceeds in the direction opposite to the beam
  direction.
  This attribute may be used in matching, when tracking lattice
  functions from a known position back to a position to be used to
  adjust an insertion.
\end{description}
The \texttt{TWISS} command has the following read-only attributes:
\begin{description}
\item[L]
  \index{L}
  The total arc length [m].
\item[FREQ0]
  \index{FREQ0}
  The computed revolution frequency in MHz.
\item[QX]
  \index{QX}
  The computed tune for mode 1.
\item[QY]
  \index{QY}
  The computed tune for mode 2.
\item[QS]
  \index{QS}
  The computed tune for mode 3.
\item[U0]
  \index{U0}
  The computed energy loss per turn [MeV].
\item[JX]
  \index{JX}
  The computed damping partition number for mode 1.
\item[JY]
  \index{JY}
  The computed damping partition number for mode 2.
\item[JE]
  \index{JE}
  The computed damping partition number for mode 3.
\end{description}
The values related to damping exist only when the flag \texttt{DAMP}
has been set.

The \texttt{TWISSTRACK} can be initialised by the following attribute:
\begin{description}
\item[INIT]
  \index{INIT}
  If this attribute is given, the initial position and direction is taken
  from the specified row of another \texttt{TWISS} table.
\end{description}

Example:
\begin{verbatim}
LATFUN:TWISS,LINE=CELL;
\end{verbatim}
This example computes the lattice functions for \texttt{CELL} with
zero initial conditions and saves the result under the name
\texttt{LATFUN}. 

A table generated by \texttt{TWISS} (not \texttt{TWISSTRACK})
also has some \tabref{read-only attributes}{twiss-glob}:
\begin{table}[Ht]
  \caption{Global read-only values in a \texttt{TWISS} table}
  \label{tab:twiss-glob}
  \begin{center}
    \begin{tabular}{|l|l|l|}
      \hline
      Variable & Unit & Name \\
      \hline
      Revolution frequency & Hz & \texttt{FREQ}\index{FREQ} \\
      Tune for mode 1 & 1 & \texttt{QX}\index{QX} \\
      Tune for mode 2 & 1 & \texttt{QY}\index{QY} \\
      Tune for mode 3 & 1 & \texttt{QS}\index{QS} \\
      Energy loss per turn & MeV & \texttt{U0}\index{U0} \\
      Damping partition number for mode 1 & 1 & \texttt{JX}\index{JX} \\
      Damping partition number for mode 2 & 1 & \texttt{JY}\index{JY} \\
      Damping partition number for mode 3 & 1 & \texttt{JE}\index{JE} \\
      \hline
    \end{tabular}
  \end{center}
\end{table}

In each the \texttt{TWISS} table stores the closed orbit and the
transfer map from the beginning to this position.
Several \tabref{column values}{twiss-col},
as well as the \tabref{eigenvectors}{twiss-eig},
\tabref{second moments}{twiss-sig},
and \tabref{transfer matrices}{twiss-mat} can be derived from this
information
These quantities are accessible with the \secref{syntax}{acell}
\begin{verbatim}
table-name "@" place "->" column-name
\end{verbatim}

\begin{table}[Ht] \footnotesize
  \caption{Column values in a \texttt{TWISS} table}
  \label{tab:twiss-col}
  \begin{center}
    \begin{tabular}{|l|l|l|}
      \hline
      Variable & Unit & Name \\
      \hline
      Arc length from beginning & m & \texttt{S}\index{S} \\
      \hline
      Horizontal position for closed orbit & m & \texttt{XC}\index{XC} \\
      Horizontal momentum for closed orbit & 1 & \texttt{PXC}\index{PXC} \\
      Vertical position for closed orbit & m & \texttt{YC}\index{YC} \\
      Vertical momentum for closed orbit & 1 & \texttt{PYC}\index{PYC} \\
      Longitudinal position for closed orbit & m & \texttt{TC}\index{TC} \\
      Longitudinal momentum for closed orbit & 1 & \texttt{PTC}\index{PTC} \\
      \hline
      Dispersion for horizontal position & m & \texttt{DX}\index{DX} \\
      Dispersion for horizontal momentum & 1 & \texttt{DPX}\index{DPX} \\
      Dispersion for vertical position & m & \texttt{DY}\index{DY} \\
      Dispersion for vertical momentum & 1 & \texttt{DPY}\index{DPY} \\
      Dispersion for longitudinal position & m & \texttt{DT}\index{DT} \\
      Dispersion for longitudinal momentum & 1 & \texttt{DPT}\index{DPT} \\
      \hline                  
    \end{tabular}

    \begin{tabular}{|l|l|l|l|l|}
      \hline
      & & \multicolumn{3}{c|}{Plane} \\
      Courant-Snyder Functions & Unit & \texttt{X} & \texttt{Y} & \texttt{T} \\
      \hline
      Beta-function  & m & \texttt{BETX}\index{BETX} &
      \texttt{BETY}\index{BETY} & \texttt{BETT}\index{BETT} \\
      Alpha-function & 1 & \texttt{ALFX}\index{ALFX} &
      \texttt{ALFY}\index{ALFY} & \texttt{ALFT}\index{ALFT} \\
      Phase          & 1 & \texttt{MUX}\index{MUX} &
      \texttt{MUY}\index{MUY}   & \texttt{MUT}\index{MUT} \\
      \hline
    \end{tabular}
    
  \end{center}
\end{table}
\begin{table}[Ht] 
  \caption{Mais-Ripken Functions  in a \texttt{TWISS} table}
  \label{tab:twiss-col-mais}
  \begin{center}
    \begin{tabular}{|l|l|l|l|l|l|}
      \hline
      & & &
        \multicolumn{3}{c|}{Plane} \\
      Mais-Ripken Functions & Unit & Mode &
      \texttt{X} & \texttt{Y} & \texttt{T} \\
      \hline
      & & 1 &
      \texttt{BETX1}\index{BETX1} & \texttt{BETY1}\index{BETY1} &
      \texttt{BETT1}\index{BETT1} \\
      Beta-function & m & 2 &
      \texttt{BETX2}\index{BETX2} & \texttt{BETY2}\index{BETY2} &
      \texttt{BETT2}\index{BETT2} \\
      & & 3 &
      \texttt{BETX3}\index{BETX3} & \texttt{BETY3}\index{BETY3} &
      \texttt{BETT3}\index{BETT3} \\
      \hline
      & & 1 &
      \texttt{ALFX1}\index{ALFX1} & \texttt{ALFY1}\index{ALFY1} &
      \texttt{ALFT1}\index{ALFT1} \\
      Alpha-function & 1 & 2 &
      \texttt{ALFX2}\index{ALFX2} & \texttt{ALFY2}\index{ALFY2} &
      \texttt{ALFT2}\index{ALFT2} \\
      & & 3 &
      \texttt{ALFX3}\index{ALFX3} & \texttt{ALFY3}\index{ALFY3} &
      \texttt{ALFT3}\index{ALFT3} \\
      \hline
      & & 1 &
      \texttt{GAMX1}\index{GAMX1} & \texttt{GAMY1}\index{GAMY1} &
      \texttt{GAMT1}\index{GAMT1} \\
      Gamma-function & 1/m & 2 &
      \texttt{GAMX2}\index{GAMX2} & \texttt{GAMY2}\index{GAMY2} &
      \texttt{GAMT2}\index{GAMT2} \\
      & & 3 &
      \texttt{GAMX3}\index{GAMX3} & \texttt{GAMY3}\index{GAMY3} &
      \texttt{GAMT3}\index{GAMT3} \\
      \hline
    \end{tabular}
  \end{center}
\end{table}
\clearpage
The Courant-Snyder lattice functions are calculated in a way analogous
\index{Courant-Snyder}
to the lattice functions given by \bibref{Edwards and Teng}{edwards}.
\index{Edwards-Teng}
After partitioning the eigenvector matrix into two~by~two blocks it
can be factored as
\[ 
E = \left(
  \begin{array}{lll}
    E_{11} & E_{12} & E_{13} \\
    E_{21} & E_{22} & E_{23} \\
    E_{31} & E_{32} & E_{33}
  \end{array}
\right) = R W = \left(
  \begin{array}{lll}
    r_1 I  & R_{12} & R_{13} \\
    R_{21} & r_2 I  & R_{23} \\
    R_{31} & R_{32} & r_3 I
  \end{array}
\right) \left(
  \begin{array}{lll}
    W_1 & 0 & 0 \\ 0 & W_2 & 0 \\ 0 & 0 & W_3
  \end{array}
\right),
\]
where the $r_i$ are scalars and $R$ and $W$ are symplectic matrices.
This implies
\[
r_i = \sqrt{|R_{ii}|}, \qquad
W_i = R_{ii} / r_i \rightarrow |W_i| = 1, \qquad 
R_{ik} = E_{ik} W_i^{-1}.
\]
The lattice functions are calculated from the diagonal blocks $W_i$.
Note that the matrix $R$ which defines the coupling between planes is
not available for output,
and that the formalism breaks down when $r_i=0$.
The default column set contains the element name,
the arc length, the beta, alpha and mu functions for the transverse
planes,
the closed orbit position, and the horizontal dispersion.

The Mais-Ripken functions have been introduced by
\index{Mais-Ripken}
\bibref{Mais and Ripken}{mais}.

The second moments (beam envelope) are defined in the sense of
\bibref{TRANSPORT}{transport}: 
\index{second moments}
\index{beam envelope}
\[
  \Sigma_{ik} = E_x ( E_{i1} E_{k1} + E_{i2} E_{k2}) +
                E_y ( E_{i3} E_{k3} + E_{i4} E_{k4}) +
                E_t ( E_{i5} E_{k5} + E_{i6} E_{k6}).
\]

\begin{table}[Ht]
  \caption{Eigenvector components in a \texttt{TWISS} table}
  \label{tab:twiss-eig}
  \begin{center}
    \begin{tabular}{|l|l|l|l|l|l|l|l|}
      \hline
      Eigenvector & &
      \multicolumn{2}{c|}{Mode 1} &
      \multicolumn{2}{c|}{Mode 2} &
      \multicolumn{2}{c|}{Mode 3} \\
      Component & Unit & real & imag & real & imag & real & imag \\
      \hline
      X-component & m &
      \texttt{E11}\index{E11} & \texttt{E12}\index{E12} &
      \texttt{E13}\index{E13} & \texttt{E14}\index{E14} &
      \texttt{E15}\index{E15} & \texttt{E16}\index{E16} \\
      PX-component & 1 &
      \texttt{E21}\index{E21} & \texttt{E22}\index{E22} &
      \texttt{E23}\index{E23} & \texttt{E24}\index{E24} &
      \texttt{E25}\index{E25} & \texttt{E26}\index{E26} \\
      Y-component & m &
      \texttt{E31}\index{E31} & \texttt{E32}\index{E32} &
      \texttt{E33}\index{E33} & \texttt{E34}\index{E34} &
      \texttt{E35}\index{E35} & \texttt{E36}\index{E36} \\
      PY-component & 1 &
      \texttt{E41}\index{E41} & \texttt{E42}\index{E42} &
      \texttt{E43}\index{E43} & \texttt{E44}\index{E44} &
      \texttt{E45}\index{E45} & \texttt{E46}\index{E46} \\
      T-component & m &
      \texttt{E51}\index{E51} & \texttt{E52}\index{E52} &
      \texttt{E53}\index{E53} & \texttt{E54}\index{E54} &
      \texttt{E55}\index{E55} & \texttt{E56}\index{E56} \\
      PT-component & 1 &
      \texttt{E61}\index{E61} & \texttt{E62}\index{E62} &
      \texttt{E63}\index{E63} & \texttt{E64}\index{E64} &
      \texttt{E65}\index{E65} & \texttt{E66}\index{E66} \\
      \hline                  
    \end{tabular}
  \end{center}
\end{table}

\begin{table}[Ht]
  \caption{Second moments \texttt{<X1,X2>} in a \texttt{TWISS} table}
  \label{tab:twiss-sig}
  \begin{center}
    \begin{tabular}{|l|l|l|l|l|l|l|}
      \hline
      Second moment & \multicolumn{6}{c|}{Second variable \texttt{X2}} \\
      First variable \texttt{X1} & \texttt{X} & \texttt{PX} &
          \texttt{Y} & \texttt{PY} & \texttt{T} & \texttt{PT} \\
      \hline
      \texttt{X} &
      \texttt{S11}\index{S11} & \texttt{S12}\index{S12} &
      \texttt{S13}\index{S13} & \texttt{S14}\index{S14} &
      \texttt{S15}\index{S15} & \texttt{S16}\index{S16} \\
      \texttt{PX} &
      \texttt{S21}\index{S21} & \texttt{S22}\index{S22} &
      \texttt{S23}\index{S23} & \texttt{S24}\index{S24} &
      \texttt{S25}\index{S25} & \texttt{S26}\index{S26} \\
      \texttt{Y} & 
      \texttt{S31}\index{S31} & \texttt{S32}\index{S32} &
      \texttt{S33}\index{S33} & \texttt{S34}\index{S34} &
      \texttt{S35}\index{S35} & \texttt{S36}\index{S36} \\
      \texttt{PY} & 
      \texttt{S41}\index{S41} & \texttt{S42}\index{S42} &
      \texttt{S43}\index{S43} & \texttt{S44}\index{S44} &
      \texttt{S45}\index{S45} & \texttt{S46}\index{S46} \\
      \texttt{T} & 
      \texttt{S51}\index{S51} & \texttt{S52}\index{S52} &
      \texttt{S53}\index{S53} & \texttt{S54}\index{S54} &
      \texttt{S55}\index{S55} & \texttt{S56}\index{S56} \\
      \texttt{PT} & 
      \texttt{S61}\index{S61} & \texttt{S62}\index{S62} &
      \texttt{S63}\index{S63} & \texttt{S64}\index{S64} &
      \texttt{S65}\index{S65} & \texttt{S66}\index{S66} \\
      \hline                  
    \end{tabular}
  \end{center}
\end{table}

\begin{table}[Ht]
  \caption{Transfer matrix elements \texttt{(X2,X1)} in a \texttt{TWISS} table}
  \label{tab:twiss-mat}
  \begin{center}
    \begin{tabular}{|l|l|l|l|l|l|l|}
      \hline
      Matrix Element & \multicolumn{6}{c|}{Second variable \texttt{X2}} \\
      First variable \texttt{X1} & X & PX & Y & PY & T & PT \\
      \hline
      \texttt{X} &
      \texttt{R11}\index{R11} & \texttt{R12}\index{R12} &
      \texttt{R13}\index{R13} & \texttt{R14}\index{R14} &
      \texttt{R15}\index{R15} & \texttt{R16}\index{R16} \\
      \texttt{PX} &
      \texttt{R21}\index{R21} & \texttt{R22}\index{R22} &
      \texttt{R23}\index{R23} & \texttt{R24}\index{R24} &
      \texttt{R25}\index{R25} & \texttt{R26}\index{R26} \\
      \texttt{Y} &
      \texttt{R31}\index{R31} & \texttt{R32}\index{R32} &
      \texttt{R33}\index{R33} & \texttt{R34}\index{R34} &
      \texttt{R35}\index{R35} & \texttt{R36}\index{R36} \\
      \texttt{PY} &
      \texttt{R41}\index{R41} & \texttt{R42}\index{R42} &
      \texttt{R43}\index{R43} & \texttt{R44}\index{R44} &
      \texttt{R45}\index{R45} & \texttt{R46}\index{R46} \\
      \texttt{T} &
      \texttt{R51}\index{R51} & \texttt{R52}\index{R52} &
      \texttt{R53}\index{R53} & \texttt{R54}\index{R54} &
      \texttt{R55}\index{R55} & \texttt{R56}\index{R56} \\
      \texttt{PT} &
      \texttt{R61}\index{R61} & \texttt{R62}\index{R62} &
      \texttt{R63}\index{R63} & \texttt{R64}\index{R64} &
      \texttt{R65}\index{R65} & \texttt{R66}\index{R66} \\
      \hline
    \end{tabular}
  \end{center}
\end{table}

\section{Listing Element Attributes}
\label{sec:attlist}
\index{ATTLIST}
\index{list|(}
Most element attributes can be listed for a beam line or table.
The elements for which the attributes should be listed are selected
by a \secref{\texttt{SELECT} command}{select}.
The attributes are specified by the command
\begin{verbatim}
ATTLIST,LINE=name,FILE=name,COLUMN={string,string,...,string};
\end{verbatim}
Most element attributes can be listed as columns.
At present, \opal does not recognize \texttt{ATTRIBUTES} or attributes
of a \texttt{BEAMBEAM} element.
Example:
\begin{verbatim}
 SELECT, LINE=LIN1, FULL;
 ATTLIST, LINE=LIN1, FILE=TERM,
   COLUMN={NAME,CLASS,KEYWORD,TYPE,LENGTH,K0L,UNKNOWN};
\end{verbatim}
This may produce the following output:
\begin{verbatim}
@ TYPE     %s  ATTRIBUTE
@ LINE     %s  LIN1
@ DATE     %s  11/10/1999
@ TIME     %s  14.08.14
@ ORIGIN   %s  OPAL_9.3
@ COMMENT  %s  ""
* NAME CLASS KEYWORD TYPE L K0L UNKNOWN
$ %s %s %s %s %le %le %le
  #S MARKER MARKER Null 0 0 0
  L4 DRIFT DRIFT Null 1.57 0 0
  B1 RBEND RBEND Null 35.09 0.0113061 0
  L2 DRIFT DRIFT Null 0.56 0 0
  SF SEXT SEXTUPOLE Null 0.4 0 0
  L3 DRIFT DRIFT Null 0.28 0 0
  QF QUAD QUADRUPOLE MQ 1.6 0 0
  L1 DRIFT DRIFT Null 1.21 0 7
  B1 RBEND RBEND Null 35.09 0.0113061 0
  L2 DRIFT DRIFT Null 0.56 0 0
  SD SEXT SEXTUPOLE Null 0.76 0 0
  L3 DRIFT DRIFT Null 0.28 0 0
  QD QUAD QUADRUPOLE MQ 1.6 0 0
  #E MARKER MARKER Null 0 0 0
\end{verbatim}
As it can be seen, empty strings are printed as the word ``Null''.
\clearpage
\section{Listing a Table}
\label{sec:list}
Five commands exist which permit listing of a previously generated
table. 
All of these have the option of printing on a file or on the terminal,
and using a human-readable or a TFS format.

\subsection{LIST Command}
\index{LIST}
The most general of these commans is
\begin{verbatim}
LIST,TABLE=name,FILE=name,ALL=logical,
          COLUMN={expression,...,expression};
\end{verbatim}
The attributes of this command are:
\begin{description}
\item[TABLE]
  \index{TABLE}
  The name of the table to be listed (no default). 
  The rows to be listed can be selected by a
  \secref{\texttt{SELECT}}{select} command.
\item[FILE]
  \index{FILE}
  The name of the file to be written (default \texttt{LIST}).
  The table is written in the \secref{TFS format}{TFS}.
  If the file name is \texttt{TERM}, output goes to the terminal.
\item[TFS]
  \index{TFS}
  If true, the format is in TFS format, otherwise in human-readable
  format. 
\item[ALL]
  \index{ALL}
  If true, the column selection is by default, and the column width
  and precision is selected by \opal .
  In this case \texttt{COLUMN} is ignored.
\item[COLUMN]
  \index{COLUMN}
  A list of column descriptors.
  Each column descriptor consists of a \secref{real expression}{areal},
  which may also contain one or more names of columns as operands.
  If a name refers both to a column and to a global parameter,
  the column takes precedence.
  The expression may optionally be followed by a format indication of
  the form:
\begin{verbatim}
:width
:width:precision
\end{verbatim}
  Both \texttt{width} and \texttt{precision} are unsigned integers.
  \texttt{width} denotes the column width in characters (default~12),
  and \texttt{precision} controls the number of digits in the column
  (default~\texttt{width}-4).
\end{description}
Example:
\begin{verbatim}
list,table=su1,file=term,column={x,y,z,s*z:16:12};
\end{verbatim}

\subsection{``Canned'' List Commands}
\label{sec:canned}
Certain predefined column sets are obtainable with the commands
\begin{verbatim}
EIGEN,TABLE=name,FILE=name,TFS=logical;
ENVELOPE,TABLE=name,FILE=name,TFS=logical;
MATRIX,TABLE=name,FILE=name,TFS=logical;
TWISS3,TABLE=name,FILE=name,TFS=logical;
\end{verbatim}
These commands provide the following output:
\begin{description}
\item[EIGEN]
  \index{EIGEN}
  The name of the element, the arc length, the closed orbit,
  and the \tabref{eigenvectors}{twiss-eig}.
\item[ENVELOPE]
  \index{ENVELOPE}
  The name of the element, the arc length, the closed orbit,
  and the \tabref{beam envelope}{twiss-sig}.
\item[MATRIX]
  \index{MATRIX}
  The name of the element, the arc length, the closed orbit,
  and the \tabref{accumulated matrix}{twiss-mat}.
\item[TWISS3]
  \index{TWISS3}
  The name of the element, the arc length, the closed orbit,
  and the \tabref{Mais-Ripken functions}{twiss-col}.
\end{description}
The attributes of these commands are:
\begin{description}
\item[TABLE]
  \index{TABLE}
  The name of the table to be listed (no default). 
  The rows to be listed can be selected by a
  \secref{\texttt{SELECT}}{select} command.
\item[FILE]
  \index{FILE}
  The name of the file to be written (default \texttt{LIST}).
  The table is written in the \secref{TFS format}{TFS}.
  If the file name is \texttt{TERM}, output goes to the terminal.
\item[TFS]
  \index{TFS}
  If true, the format is in TFS format, otherwise in human-readable
  format. 
\end{description}

\index{list|)}
\index{tables|)}
