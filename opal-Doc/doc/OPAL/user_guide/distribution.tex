\chapter{Distribution Command}
\label{chp:distribution}
\index{distr|(}
Particle distributions can be read in or generated by specifying rms beam quantities.
The allowed parameters are described in Table \ref{tab:distrparam}.

Particle distributions are generated separately in all three phase space planes.
There are no explicit correlations between planes e.g. between longitudinal and transverse.
Besides an efficient parallel Gaussian distribution generator based on a parallelized 
``Method of Rejection'',
a more general algorithm for generating
distributions is available \cite{JohoDist}. The shape of the binomial distribution is governed by
one parameter $m$. By varying this single parameter one obtains the most commonly 
used distributions for our
type of simulations as, listed in Table \ref{tab:binomdist}.

If NBIN is larger than one the distribution is binned in energy and for each bin a separate field solve is
performed when using the electrostatic solvers.
 
\begin{table}[h!]
  \begin{center} \footnotesize
    \caption{Different distributions specified by a single parameter $m$}
    \label{tab:binomdist}
    \begin{tabular}{|l|l|l|l|}
      \hline
      \bf m & \bf Distribution & \bf Density & \bf Profile \\
      \hline
      0.0 & Hollow shell  & $\frac{1}{\pi}\delta(1-r^2)$ &$\frac{1}{\pi}(1-r^2)^{-0.5}$\\
      \hline
      0.5 & Flat profile  & $\frac{1}{2\pi}(1-r^2)^{-0.5}$ & $\frac{1}{2}$\\
      \hline
      1.0 & Uniform  & $\frac{1}{\pi}$ & $\frac{2}{\pi}(1-x^2)^{0.5}$\\
      \hline
      1.5 & Elliptical  & $\frac{3}{2\pi}(1-r^2)^{0.5}$ & $\frac{1}{4}(1-x^2)$ \\
      \hline
      2.0 & Parabolic  & $\frac{2}{\pi}(1-r^2)$ & $\frac{3}{8\pi}(1-x^2)^{1.5}$ \\
      \hline
      $\rightarrow \infty$ & Gaussian  & $\frac{1}{2\pi\sigma_x\sigma_y}exp(-\frac{x^2}{2\sigma_x^2} -\frac{y^2}{2\sigma_y^2})$ & 
      $\frac{1}{\sqrt{2\pi}*\sigma_x}exp(-\frac{x^2}{2\sigma_x^2}) $ \\
      \hline
    \end{tabular}
  \end{center} 
\end{table}
There are also special distribution commands.
\begin{enumerate}
\item {\it GUNGAUSS} will create a distribution that is uniform in $x$ and $y$ (with the
radius in each plane given by SIGMAX and SIGMAY) and with a Gaussian distribution longitudinally (given by SIGMAT).
This distribution is cold in the transverse direction and has a uniform temperature given by the initial kinetic energy in
eV specified by the PT and the SIGMAPT variables.
\item {\it GUNGAUSS3D} is identical to {\it GUNGAUSS} except that it generates a distribution that is Gaussian in the
transverse planes as well.
\item {\it GUNUNIFORM} is the same as {\it GUNGAUSS} except that the longitudinal distribution is uniform. The result is a
uniformaly filled cylinder of charge. (SIGMAT gives the length of the cylinder in meters.)
\item {\it UNITUNIL} will create a 3D spatial Uniform distribution in transverse as well as longitudinal direction, cold
in transverse direction and with a uniform temperature given by the initial kinetic in eV specified by the PT and the SIGMAPT variables.
\item {\it GUNGAUSSFLATTOP} will create a distribution that is uniform transversely (identical to {\it GUNGAUSS}). The longitudinal profile
has a Guassian rise and fall with a flat top distribution in between. For this option one must specify the SIGMARISE and SIGMAFALL, the CUTOFFRISE and CUTOFFFALL and the
FLATTOPTIME. The units of these variables are in meters. The SIGAMRISE and SIGMAFALL variables define the Gaussian $\sigma$ for the
head and tail of the beam bunch respectively and the CUTOFF variables defines where the distribution is cutoff (specified in $\sigma$). The total emission time TEMISSION is $TFlatTop + (1+CRise) * SRise + (1+CFall) * SFall$.
\item {\it GUNGAUSSFLATTOPTH} will create a distribution similar to  {\it GUNGAUSSFLATTOP}  but with thermal emittance (see \ref{sec:them}). 

\end{enumerate}

\begin{table}[h!]
  \begin{center}
    \footnotesize
    \caption{Parameters for the DISTRIBUTION command}
    \begin{tabular}{|l|l|}
      \hline
      Parameter & Purpose \\
      \hline
      \mytabline{DISTRIBUTION}{\texttt{FROMFILE} or \texttt{BINOMINAL} or \texttt{GAUSS}  or \texttt{GUNGAUSS} or \texttt{GUNGAUSS3D}}   
      \mytabline{}{or \texttt{GUNUNIFORM} or \texttt{ROTSYMBINOMIAL} or \texttt{GUNGAUSSFLATTOP} or \texttt{UNIFORMXYZ}}
      \mytabline{FNAME}{Specifies the filename of a particle distribution to be read in}
      \mytabline{XMULT}{Scales the x coordinate: $x = XMULT*x$}	
      \mytabline{PXMULT}{Scales the px coordinate: $px = PXMULT*px$}
      \mytabline{YMULT}{Scales the y coordinate: $y = YMULT*y$}
      \mytabline{PYMULT}{Scales the py coordinate: $py = PYMULT*py$}
      \mytabline{TMULT}{Scales the t coordinate: $t = TMULT*t$}
      \mytabline{PTMULT}{Scales the pt coordinate: $pt = PTMULT*pt$}
      \hline                        
      \mytabline{$SIGMAX$}{$\rms{x}$ see Chapter on Notation }
      \mytabline{$SIGMAPX$}{$\rms{p}_x$ see Chapter on Notation }
      \mytabline{$SIGMAY$}{$\rms{y}$ see Chapter on Notation }
      \mytabline{$SIGMAPY$}{$\rms{p}_y$ see Chapter on Notation }
      \mytabline{$SIGMAT$}{$\rms{t}$ see Chapter on Notation }
       \mytabline{TRANSVCUTOFF}{Defines the transverse cut-off of \texttt{GUNGAUSS3D} in units of $\sigma$}
      \mytabline{$PT$}{$\langle p_t \rangle$ see Chapter on Notation }
      \mytabline{$SIGMAPT$}{$\rms{p}_t$ see Chapter on Notation }
      \hline
       \mytabline{mx} {Defines the transverse distribution (see Table \ref{tab:binomdist}) }
      \mytabline{my} {Defines the transverse distribution (see Table \ref{tab:binomdist}) }
      \mytabline{mt} {Defines the longitudinal distribution (see Table \ref{tab:binomdist}) }
      \hline
      \mytabline{CORRX} {Defines the $x$, $p_x$ correlation }
      \mytabline{CORRY} {Defines the $y$, $p_y$ correlation }
      \mytabline{CORRT} {Defines the $t$, $p_t$ correlation }
      \hline
    \end{tabular}
  \end{center}
 \end{table}
      
\begin{table}[h!]
  \footnotesize
  \caption{Parameters of the distribution command}
  \label{tab:distrparam}
  \begin{center}
    \begin{tabular}{|l|l|}
      \hline
      Parameter & Purpose \\
      \hline
      \mytabline{TEMISSION} {Defines the length of the emission process [s] }
      \mytabline{NBIN} {How many energy bins begin used }
      \mytabline{DEBIN} {Defines a energy band $dE$ [MeV].}
      \mytabline{} {If the maximal energy difference between all bins are}
      \mytabline{} {smaller than $dE$ all bins are merged into one bin.}
      \hline
      \mytabline{ELASER}{Laser energy (eV)}
      \mytabline{SIGLASER}{Sigma of (uniform) laser spot size (m)}
      \mytabline{W}{Workfunction of material (eV)}
      \mytabline{FE}{Fermi energy (eV)}
      \mytabline{AG}{Acceleration Gradient (MV/m)}
      \hline
    \end{tabular} 
  \end{center}
\end{table}



The following example creates a distribution during $39$ ps using $39$ energy bins.
\begin{verbatim}
Dist1:DISTRIBUTION, DISTRIBUTION=gungauss,
      sigmax=0.00030, sigmapx=0.0, corrx=0.0,
      sigmay=0.00030, sigmapy=0.0, corry=0.0,
      sigmat=lz, pt=1.0, sigmapt=0.000001, corrt=0.0, 
      TEMISSION=39.0e-12, NBIN=39;
\end{verbatim}

The following example reads in a distribution from a file and scales the coordinates:
\begin{verbatim}
DistFile:DISTRIBUTION, DISTRIBUTION=FROMFILE,
         FNAME="../Dist/inpdist1finitecur.dat",
         XMULT=0.06816207, YMULT=0.06816207,
         TMULT=1.0*beta*0.06816207,
         PXMULT=1/gambet, PYMULT=1/gambet,
         PTMULT=1.0/beta^2/gamma;
\end{verbatim}

The file with the data has to have the following format:\\
\\
$N$\\
$x_1$ $px_1$ $y_1$ $py_1$ $z_1$ $pz_1$\\
$x_2$ $px_2$ $y_2$ $py_2$ $z_2$ $pz_2$\\
.\\
.\\
$x_N$ $px_N$ $y_N$ $py_N$ $z_N$ $pz_N$,\\
\\
where $N$ is the number of particles, the vector $(x_i,y_i,z_i)$ describes the position of the i-th particle and the vector $(px_i, py_i, pz_i)$ its momentum in $\beta \gamma$.

\section{Correlations for Gaussian Distribution (Experimental)}

To generate gaussian initial distribution with dispersion, first we
generate the uncorrelated gaussian inputs matrix $R=(R1,...,R_n)$.
The mean of $R_i$ is $0$ and the standard deviation squared is 1. Then
we correlate $R$.
The correlation coefficient matrix $\sigma$ in $x,p_x,t,p_t$ phase space reads: \\

$ \hspace{0.5cm}x \hspace{0.8cm}  px  \hspace{0.9cm} t\hspace{1.0cm} pt\\$
$\sigma= \left[
\begin{array}{cccc}
1    &c_x&r51    &r61\\
c_x&1    &r52    &r62\\
r51  &r52  &1      &c_t\\
r61  &r62  &c_t  &1\\
\end{array}
\right].$ \\

The Cholesky decomposition of the symmetric positive-definite matrix $\sigma$ is $\sigma=C^TC$, then the correlated distribution is $C^TR$.

\textbf{Note}: This correlation works for the moment only with the gaussian distribution.

\subsection{Example}
Let the initial correlation coefficient matrix be:

$\sigma= \left[
\begin{array}{cccc}
1      &0.756  &0.023    &0.496\\
0.756  &1      &0.385    &-0.042\\
0.023  &0.385  &1        &-0.834\\
0.496  &-0.042 &-0.834   &1\\

\end{array}
\right]$ \\
then the corresponding distribution command read:
\begin{verbatim}
Dist:DISTRIBUTION, DISTRIBUTION=gauss,
     sigmax=4.796e-03, sigmapx=231.0585, corrx=0.756,
     sigmay=23.821e-03, sigmapy=1.6592e+03, corry=-0.999,
     t=0.466e-02, sigmat=0.466e-02, pt=72e6, 
     sigmapt=74.7, corrt=-0.834,
     r61=0.496, r62=-0.042, r51=0.023, r52=0.385;
\end{verbatim}

\section{Thermal Emittance}
\label{sec:them}
The thermal emittance calculation is based on \cite{flo:97, clen:2000} where  $P(E_f,E_{ph}=\hbar\omega)$ the probability for a photon of energy $E_{ph}$ exiting an electron to a final state energy $E_f$ is

\begin{equation} P(E_f,E_{ph}=\hbar\omega) \propto N_f(E_f) N_i(E_f - E_{ph}=\hbar\omega) \text{ with}\end{equation}


 $N_f(E_f)$ is the density of final state
and $N_i(E_f - E_{ph})$ is the density of initial state.

Two cases, no-scattering (non-equilibrum) and scattering (equilibrium, e-e and e-phonon collisions) can be distinguished. In \opal the non-equilibrum case is considered and a uniform radial distribution is assumed hence: $x_{rms} = \frac{r}{2}$ \footnote{Soon we can generate distributions form virtual cathode images}.

Photoemission from a metal involves fist the absorption of a photon with:
\begin{equation} 
\hbar\omega > \Phi_e
\end{equation}
where $\Phi_e = \Phi - \Delta$ is the reduced work function.
The reduction is a function of the applied electric field $E_c$:
\begin{equation}
\Delta = e \sqrt{e E_c / 4 \pi \epsilon_0}.
\end{equation}

Electrons are emitted isotropic into the half-sphere with: $E_{kin} = \eps_{f} + \hbar\omega$.

Particles with angel $\varphi$ larger than $\varphi_{max}=\arccos{\sqrt{(\eps_f + \Phi_e / E_{kin})}}$ will pass the potential barrier. 
\begin{equation} p_x = p \sin{\varphi} \cos{\theta},~ \varphi=[0\dots \varphi_{max}],~  \theta=[0\dots \pi] \end{equation}
and 
\begin{equation} p= m_0 c \sqrt{\gamma^2 - 1}. \end{equation}

The following parameters defines the thermal emittance:  $r_{rms}$, material such as Cu, Fe, Cs2Te $\rightarrow \Phi, \eps_f$, 
the laser energy given by $\hbar\omega$ and the electric field $E_c$ which enters in the Schottky effect calculation.

This is a example of an \opal distribution definition with thermal emittance similar to the example in \cite{clen:2000} p.199.

\begin{verbatim}
Dist1:DISTRIBUTION, DISTRIBUTION = GUNGAUSSFLATTOPTH,
      sigmax=0.00054, sigmapx=0.0, corrx=0.0,
      sigmay=0.00054, sigmapy=0.0, corry=0.0,
      sigmat=lzFlatTop, pt=1.0, sigmapt=0.0, corrt=0.0,
      sigmarise=0.5e-12, sigmafall=0.5e-12, 
      flattoptime=10e-12, cutoffrise=3.0, cutofffall=3.0,
      TEMISSION=TEmis, NBIN=50, DEBIN=80,
      ELASER=4.6, SIGLASER=0.001, W=4.6, FE=7.0, AG=84;  

\end{verbatim}


\index{distr|)}



