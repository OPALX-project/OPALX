\chapter{Distribution Command}
\label{sec:distribution}
\index{distr|(}
Particle distributions can be read in or generated by specifying rms beam quantities.
The allowed parameters are described in Table \ref{tab:distrparam}.

Particle distributions are generated separately in all three phase space planes.
There are no explicit correlations between planes e.g. between longitudinal and transverse.
Besides an efficient parallel Gaussian distribution generator based on a parallelized 
``Method of Rejection'',
a more general algorithm for generating
distributions is available \cite{JohoDist}. The shape of the binomial distribution is governed by
one parameter $m$. By varying this single parameter one obtains the most commonly 
used distributions for our
type of simulations as, listed in Table \ref{tab:binomdist}.

If NBIN is larger than one the distribution is binned in energy and for each bin a separate field solve is
performed when using the electrostatic solvers.
 
\begin{table}[h!]
\begin{flushleft} \footnotesize
 \begin{tabular}{|l|l|l|l|}
\hline
\bf m & \bf Distribution & \bf Density & \bf Profile \\
\hline
0.0 & Hollow shell  & $\frac{1}{\pi}\delta(1-r^2)$ &$\frac{1}{\pi}(1-r^2)^{-0.5}$\\
\hline
0.5 & Flat profile  & $\frac{1}{2\pi}(1-r^2)^{-0.5}$ & $\frac{1}{2}$\\
\hline
1.0 & Uniform  & $\frac{1}{\pi}$ & $\frac{2}{\pi}(1-x^2)^{0.5}$\\
\hline
1.5 & Elliptical  & $\frac{3}{2\pi}(1-r^2)^{0.5}$ & $\frac{1}{4}(1-x^2)$ \\
\hline
2.0 & Parabolic  & $\frac{2}{\pi}(1-r^2)$ & $\frac{3}{8\pi}(1-x^2)^{1.5}$ \\
\hline
$\rightarrow \infty$ & Gaussian  & $\frac{1}{2\pi\sigma_x\sigma_y}exp(-\frac{x^2}{2\sigma_x^2} -\frac{y^2}{2\sigma_y^2})$ & 
                       $\frac{1}{\sqrt{2\pi}*\sigma_x}exp(-\frac{x^2}{2\sigma_x^2}) $ \\
\hline
\end{tabular}
\end{flushleft} 
\caption{\label{tab:binomdist}{Different distributions specified by a single parameter $m$}}
\end{table}
There are also special distribution commands.
\begin{enumerate}
\item {\it GUNGAUSS} will create a distribution that is uniform in $x$ and $y$ (with the
radius in each plane given by SIGMAX and SIGMAX) and with a Gaussian distribution longitudinally (given by SIGMAT).
This distribution is cold in the transverse direction and has a uniform temperature given by the initial kinetic energy in
eV specified by the PT and the SIGMAPT variables.
\item {\it GUNGAUSS3D} is identical to {\it GUNGAUSS} except that it generates a distribution that is Gaussian in the
transverse planes as well.
\item {\it GUNUNIFORM} is the same as {\it GUNGUAS} except that the longitudinal distribution is uniform. The result is a
uniformaly filled cylinder of charge. (SIGMAT gives the length of the cylinder in meters.)
\item {\it UNITUNIL} will create a 3D spatial Uniform distribution in transverse as well as longitudinal direction, cold
in transverse direction and with a uniform temperature given by the initial kinetic in eV specified by the PT and the SIGMAPT variables.
\item {\it GUNGAUSSFLATTOP} will create a distribution that is uniform transversely (identical to {\it GUNGAUSS}). The longitudinal profile
has a Guassian rise and fall with a flat top distribution in between. For this option one must specify the RISETIME, the FALLTIME and the
FLATTOPTIME. The units of these variables are in meters. The RISETIME and FALLTIME variables define the Gaussian $\sigma$ for the
head and tail of the beam bunch respectively.   
\end{enumerate}

\begin{table}[h!]
 \footnotesize
  \begin{tabular}{|l|l|}
      \hline
      Parameter & Purpose \\
      \hline
      \mytabline{DISTRIBUTION}{\texttt{FROMFILE} or \texttt{BINOMINAL} or \texttt{GAUSS}  or \texttt{GUNGAUSS} or \texttt{GUNGAUSS3D} or \texttt{GUNUNIFORM}}   
      \mytabline{}{or \texttt{ROTSYMBINOMIAL} or \texttt{GUNGAUSSFLATTOP} or \texttt{ UNIFORMXYZ}}        
      \mytabline{FNAME}{Specifies the filename of a particle distribution to be read in}
      \mytabline{XMULT}{Scales the x coordinate: $x = XMULT*x$}	
      \mytabline{PXMULT}{Scales the px coordinate: $px = PXMULT*px$}
      \mytabline{YMULT}{Scales the y coordinate: $y = YMULT*y$}
      \mytabline{PYMULT}{Scales the py coordinate: $py = PYMULT*py$}
      \mytabline{TMULT}{Scales the t coordinate: $t = TMULT*t$}
      \mytabline{PTMULT}{Scales the pt coordinate: $pt = PTMULT*pt$}
      \hline                        
      \mytabline{$SIGMAX$}{$\rms{x}$ see Chapter on Notation }
      \mytabline{$SIGMAPX$}{$\rms{p}_x$ see Chapter on Notation }
      \mytabline{$SIGMAY$}{$\rms{y}$ see Chapter on Notation }
      \mytabline{$SIGMAPY$}{$\rms{p}_y$ see Chapter on Notation }
      \mytabline{$SIGMAT$}{$\rms{t}$ see Chapter on Notation }
       \mytabline{TRANSVCUTOFF}{Defines the transverse cut-off of \texttt{GUNGAUSS3D} in units of $\sigma$}
      \mytabline{$PT$}{$\langle p_t \rangle$ see Chapter on Notation }
      \mytabline{$SIGMAPT$}{$\rms{p}_t$ see Chapter on Notation }
      \hline
       \mytabline{mx} {Defines the transverse distribution (see Table \ref{tab:binomdist}) }
      \mytabline{my} {Defines the transverse distribution (see Table \ref{tab:binomdist}) }
      \mytabline{mt} {Defines the longitudinal distribution (see Table \ref{tab:binomdist}) }
      \hline
      \mytabline{CORRX} {Defines the $x$, $p_x$ correlation }
      \mytabline{CORRY} {Defines the $y$, $p_y$ correlation }
      \mytabline{CORRT} {Defines the $t$, $p_t$ correlation }
      \hline
       \mytabline{TEMISSION} {Defines the length of the emission process [s] }
        \mytabline{NBIN} {How many energy bins begin used }
        \mytabline{DEBIN} {Defines a energy band $dE$ [MeV].}
       \mytabline{} {If the maximal energy difference between all bins are}
       \mytabline{} {smaller than $dE$ all bins are merged into one bin.}
       \hline
    \end{tabular} 
     \caption{Parameters of the distribution command}
    \label{tab:distrparam}
\end{table}



The following example creates a distribution during $39$ ps using $39$ energy bins.
\begin{verbatim}
Dist1:DISTRIBUTION, DISTRIBUTION=gungauss,
sigmax=  0.00030, sigmapx=0.0, corrx=0.0,
sigmay=  0.00030, sigmapy=0.0, corry=0.0,
sigmat=  lz, pt=1.0, sigmapt=0.000001, corrt=0.0 , 
TEMISSION=39.0e-12, NBIN=39;
\end{verbatim}

The following example reads in a distribution from a file and scales the coordinates:
\begin{verbatim}
DistFile:DISTRIBUTION, DISTRIBUTION=FROMFILE,
			FNAME="../Dist/inpdist1finitecur.dat",
                   	XMULT=0.06816207,
                   	YMULT=0.06816207,
                   	TMULT=1.0*beta*0.06816207,
                   	PXMULT=1/gambet,
                   	PYMULT=1/gambet,
                   	PTMULT=1.0/beta^2/gamma;
\end{verbatim}

The file with the data has to have the following format:\\
\\
$N$\\
$x_1$ $px_1$ $y_1$ $py_1$ $z_1$ $pz_1$\\
$x_2$ $px_2$ $y_2$ $py_2$ $z_2$ $pz_2$\\
.\\
.\\
$x_N$ $px_N$ $y_N$ $py_N$ $z_N$ $pz_N$,\\
\\
where $N$ is the number of particles, the vector $(x_i,y_i,z_i)$ describes the position of the i-th particle and the vector $(px_i, py_i, pz_i)$ its momentum in $\beta \gamma$.
\index{distr|)}
