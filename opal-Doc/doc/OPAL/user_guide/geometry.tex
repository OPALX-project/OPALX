\chapter{Geometry}
\label{sec:geometry}
\index{geometry}

At present the GEOMETRY command only can be used to specify boundaries for the MG Solver. The command can be used in two modes:
\begin{enumerate}
\item specify a \textsc{H5Fed} file holding the surface mesh of a complicated boundary geometry
\item specify a cylinder with an elliptic base area
\end{enumerate}

\section{Geometry Command} 
\label{sec:geometrycmd}
\begin{table}[ht]
  \begin{center}
    \begin{tabular}{|l|p{0.6\textwidth}|l|}
      \hline
      Command &Purpose \\
      \hline
      \tabline{GEOMETRY}{Specify a geometry}{geometry}
      \tabline{FGEOM}{Specifies the \textsc{H5Fed} geometry file}{FGEOM}
      \tabline{LENGTH}{Specifies the length of the geometry}{LENGTH}
      \tabline{S}{Specifies the start of the geometry}{S}
      \tabline{A}{Specifies the semi-major axis of the elliptic base area}{A}
      \tabline{B}{Specifies the semi-minor axis of the elliptic base area}{B}
      \hline
    \end{tabular}
    \caption{Geometry command summary}
    \label{tab:geometrycmd}
  \end{center}
\end{table}

%\index{distr|)}
\section{Define the Geometry File}
\label{sec:FGEOM}
\index{FGEOM}
The \textsc{H5Fed} file containing the surface mesh of the geometry.

\section{Define the Length}
\label{sec:LENGTH}
\index{LENGTH}
The length of the specified geometry in [m].

\section{Define the Start}
\label{sec:S}
\index{S}
The start of the specified geometry in [m].

\section{Define the Semi-Major Axis}
\label{sec:A}
\index{A}
The semi-major axis of the ellipse in [m].

\section{Define the Semi-Minor Axis}
\label{sec:B}
\index{B}
The semi-minor axis of the ellipse in [m].
