\chapter{Tracking}
\label{sec:track}
\index{tracking|(}

\begin{table}[ht]
  \begin{center}
    \begin{tabular}{|p{0.3\textwidth}|p{0.6\textwidth}|}
      \hline
      Command & Purpose \\
      \hline
      \tabline{TRACK}{Enter tracking mode}{trackmode}
      \tabline{DT}{Initial time step for tracking} {trackmode}
      \tabline{MAXSTEPS}{The maximal number of timesteps}{trackmode} 
      \tabline{STEPSPERTURN}{The timsteps per revolution period}{trackmode} 
      \tabline{name=expression}{Parameter relation}{variable}
%      \tabline{NOISE}{Define power supply ripple}{tracknoise}
      \tabline{START}{Define initial conditions}{trackstart}
      \tabline{RUN}{Run particles for specified number of turns or steps}{trackrun}
      \tabline{TSAVE}{Save end conditions}{tracksave}
      \tabline{ENDTRACK}{Leave tracking mode}{trackmode}
      \hline
    \end{tabular}
    \caption{Commands accepted in Tracking Mode}
    \label{tab:trackcmd}
  \end{center}
\end{table}

\section{Track Mode}
\label{sec:trackmode}
\index{TRACK}
\index{ENDTRACK}

Before starting to track, a working \secref{beam line}{line} or
\secref{sequence}{sequence} and a \secref{beam}{beam} must be selected. 
The time step and  maximal timesteps should be set. 

This command causes \opal to enter ``tracking mode'',
in which it accepts only the \tabref{track commands}{trackcmd}.
The attributes of the command are:
\begin{description}
\item[LINE]
  \index{LINE}
  The label of a preceding \secref{\texttt{LINE}}{line} or
  \secref{\texttt{SEQUENCE}}{sequence} (no default).
\item[BEAM]
  \index{BEAM}
  The named \texttt{BEAM} command defines the particle mass, charge
  and reference momentum (default: \texttt{UNNAMED\_BEAM}).
  \index{UNNAMED\_BEAM}
\item[MAXSTEPS]
  \index{MAXSTEPS}
  The maximal number of timesteps,default value is 10.
\item[DT]
  \index{DT}
  Initial time step for tracking, default value is 1 ps.
\item[STEPSPERTURN]
  \index{STEPSPERTURN}
  The timsteps per revolution period. Only available for \opalcycl, default value is 720.
  
\end{description}

In \opalt and \opalmap, the commond format is:
\begin{verbatim}
  TRACK,LINE=name,BEAM=name,MAXSTEPS=value,DT=value;
\end{verbatim}

In \opalcycl, instead of setting time step, the timesteps per-turn should be set. 
The commond format is:  
\begin{verbatim}
  TRACK,LINE=name,BEAM=name,MAXSTEPS=value,STEPSPERTURN=value;
\end{verbatim}

Particles are tracked in parallel i.e. the coordinates of all particles
are transformed at each beam element as it is reached.

\opal leaves \textbf{track mode} when it sees the command
\begin{verbatim}
  ENDTRACK;
\end{verbatim}

%\section{Define Power Supply Ripple}
%\label{sec:tracknoise}
%\index{NOISE}

%One may define noise to be applied to magnet excitations
%with the statement
%\begin{verbatim}
%NOISE,VARIABLE=variable,AMPLITUDE=real_vector,
%      FREQUENCY=real_vector,PHASE=real_vector;
%\end{verbatim}
%The NOISE statement must be entered after the
%\secref{\texttt{TRACK} command}{trackmode},
%but before the \secref{\texttt{RUN} command}{trackrun}.
%It has the effect to apply several sinusoidal variations to the
%specified parameter.
%The command has four attributes.
%\begin{description}
%\item[VARIABLE]
%  Reference to the \secref{\texttt{variable}}{avariable} to be affected.
%\item[AMPLITUDE]
%  A \secref{\texttt{vector}}{avector} of \secref{\texttt{real}}{areal} 
%  noise amplitudes $A_i$.
%\item[FREQUENCY]
%  A \secref{\texttt{vector}}{avector} of \secref{\texttt{real}}{areal} 
%  noise frequencies $f_i$.
%\item[PHASE]
%  A \secref{\texttt{vector}}{avector} of \secref{\texttt{real}}{areal} 
%  noise phases $phi_i$.
%\end{description}
%Example:
%\begin{verbatim}
%NOISE,VARIABLE=QF[K1],AMPLITUDE={1.e-3,2.e-4},
%      FREQUENCY=table(2,50*(#+1)),PHASE=table(2,0);
%\end{verbatim}
%Before each turn is tracked, the noise is re-evaluated as
%\[
%\Delta A = \sum_{i=1}^N A_i \cos ( 2\pi (f_i t + \phi_i)).
%\]
%where $t$ is the real time.

\section{Define Initial Conditions}
\label{sec:trackstart}
\index{START}
The \texttt{START} command is not used in \noopalt and \noopalcycl . In 
\opalt and \opalcycl initial conditions are defined by generating a particle
distribution using the \texttt{DISTRIBUTION} command (see Section \ref{sec:distribution}) or
run \opal in the restart mode (see Section \ref{sec:Restart}). 

The \texttt{START} command defines the initial coordinates of
the particles to be tracked.
There may be many \texttt{START} statements, one for each particle.
Particles are always started with coordinates relative
to the computed closed orbit for the defined energy error.
The command format is:
\begin{verbatim}
START,X=real,PX=real,Y=real,PY=real,T=real,DELTAP=real;
START,FX=real,PHIX=real,FY=real,PHIY=real,FT=real,PHIT=real;
\end{verbatim}
\texttt{START} statements must be entered after the
\secref{\texttt{TRACK} command}{trackmode},
but before the \secref{\texttt{RUN} command}{trackrun}.

\subsection{Absolute Particle Positions}
The first form of the \secref{\texttt{START} command}{trackstart} defines 
\secref{absolute particle positions}{variables}:
\begin{description}
\item[X]
  \index{X}
  Horizontal position $x$, referred to the ideal orbit [m].
\item[PX]
  \index{PX}
  Horizontal canonical momentum, divided by the reference momentum [1].
\item[Y]
  \index{Y}
  Vertical position $y$, referred to the ideal orbit [m].
\item[PY]
  \index{PY}
  Vertical canonical momentum, divided by the reference momentum [1].
\item[T]
  \index{T}
  The negative time difference, 
  multiplied by the instantaneous velocity of the particle [m].
\item[PT]
  \index{PT}
  Momentum error, divided by the reference momentum [1].
\end{description}

\subsection{Normalised Particle Positions}
The second form of the \texttt{START} command defines 
\secref{normalised particle positions}{normal}:
\begin{description}
\item[FX]
  \index{FX}
  The normalised amplitude for mode 1 [1]:
\item[PHIX]
  \index{PHIX}
  The phase for mode 1 [1]:
  $\phi_x = - \arctan(p_{xn}/x_n) / 2 \pi$,
\item[FY]
  \index{FY}
  The normalised amplitude for mode 2 [1]:
\item[PHIY]
  \index{PHIY}
  The phase for mode 2 [1]:
  $\phi_y = - \arctan(p_{yn}/y_n) / 2 \pi$,
\item[FT]
  \index{FT}
  The normalised amplitude for mode 3 [1]:
\item[PHIT]
  \index{PHIT}
  The phase for mode 3 [1]:
  $\phi_t = + \arctan(p_{tn}/t_n) / 2 \pi$,
\end{description}

\subsection{Initial Conditions}
Mixing absolute and normalised positions is possible,
in this case the results are added.
Initial conditions $Z$ in unnormalised phase space are related 
to the closed orbit and the absolute and normalised coordinates as follows:
\[
\begin{array}{rcl}
  Z = Z_{co}&+& \sqrt{E_x} \hbox{\tt FX}
  (\Re V_k \cos \hbox{\tt PHIX} + \Im V_k \sin \hbox{\tt PHIX}) \\
  &+& \sqrt{E_y} \hbox{\tt FY}
  (\Re V_k \cos \hbox{\tt PHIY} + \Im V_k \sin \hbox{\tt PHIY}) \\
  &+& \sqrt{E_t} \hbox{\tt FT}
  (\Re V_k \cos \hbox{\tt PHIT} + \Im V_k \sin \hbox{\tt PHIT})
\end{array}
\]
where $Z_{co}$ is the closed orbit vector, and $Z$ is the vector 
\[
Z = (\texttt{X,PX,Y,PY,T,PT})^T,
\]
and $\Re V_k$ and $\Im V_k$ are the real and imaginary parts of the 
$k^{th}$ eigenvector,
which are computed in the \secref{\texttt{TRACK} command}{trackmode}.

\section{Track Particles for a Specified Number of Turns}
\label{sec:trackrun}
\index{RUN}

This command starts or continues the actual tracking:
\begin{verbatim}
RUN,METHOD=name,FILE=string,TURNS=integer;
\end{verbatim}
The \texttt{RUN} command initialises tracking and uses the most recent 
particle bunch for initial conditions.
The particle positions may be the result of previous tracking,
or they may have been generated by 
\secref{\texttt{START} commands}{trackstart}.
The command writes a file of particle positions.
Its attributes are:
\begin{description}
\item[METHOD]
  \index{METHOD}
  The name of the tracking method to be used.
  For the time being only one method is known:
  \begin{description}
  \item[THIN]
    \index{THIN}
    All elements are treated a s thin lenses.
    This is the fastest of the known method which do not lump elements.
    \item[PARALLEL-T]
    \index{PARALLEL-T}
    This methods finally puts \opal in \opalt mode.
     \item[CYCLOTRON-T]
    \index{CYCLOTRON-T}
    This methods finally puts \opal in \opalcycl mode.
     \end{description}  
    \item[FIELDSOLVER]
    \index{FIELDSOLVER}
     The used field solver see Section \ref{sec:fieldsolver} is specified.	
    
     \item[DISTRIBUTION]
    \index{DISTRIBUTION}
     The used particle distribution see Section \ref{sec:distribution} is specified.	
 
   \item[BEAM]
    \index{BEAM}
     The used particle beam Section \ref{sec:beam} is specified.	


\item[FILE]
  \index{FILE}
  The name of the file to be writen (default="\texttt{track}").
\item[TURNS]
  \index{TURNS}
  The number of turns (integer) to be tracked (default: 1, namely single bunch).

  In \opalcycl, this parameter represents the number of bunches those will be injected into the cyclotron. In restart mode, the code
  firstly read an attribute $NumBunch$ from $.h5$ file which records how many bunches have already been injected. If $NumBunch$
  $<$ $TURNS$, the last $TURNS$$ -$ $NumBunch$ bunches will be injected in sequence by reading the initial distribution from $.h5$ file.   

\item[MBMODE]
  \index{MBMODE}
  This defines which mode of multi-bunch runs. There are two options for it, namely, \texttt{AUTO} and \texttt{FORCE}. 
  See Section \ref{sec:opalcycl:MultiBunch} for their explanations in detail.
  
  For restarting run with \texttt{TURNS} large than one, if the existing bunches of the read-in step is large than one, 
  the mode is forcely set to \texttt{FORCE}. Otherwise, it is forcely set to \texttt{AUTO}.

  This argument is available for \opalcycl.
  
\item[PARAMB]
   \index{PARAMB}
   This is a control parameter to define when to start to transfer from single bunch to multi-bunches for \texttt{AUTO} mode (default: 5.0). 

   This argument is only available for \texttt{AUTO} mode multi-bunch run in \opalcycl.

\end{description}
Example:
\begin{verbatim}
run, file="table",turns=5,mbmode="AUTO",paramb=10.0,
     method="CYCLOTRON-T",beam=beam1,fieldsolver=Fs1,
     distribution=Dist1;
\end{verbatim}

This command tracks 5 bunches in cyclotron and writes the results on file \texttt{table}.

\section{Save Particle Positions}
\label{sec:tracksave}
\index{TSAVE}
The \texttt{TSAVE} command is not used in \noopalt and \noopalcycl . The command
\begin{verbatim}
TSAVE,FILE=string;
\end{verbatim}
saves the most recent particle bunch positions on the file named by 
\texttt{string}.
\index{FILE}
These will normally be the positions of surviving particles after the
most recent \secref{\texttt{RUN} command}{trackrun}.
If no \texttt{RUN} command has been seen yet,
the positions are the result of any 
\secref{\texttt{START} commands}{trackstart} seen.
The positions are written as \texttt{START} commands,
and the file may be read by a subsequent tracking run.

\index{tracking|)}

\section{Tracking Example}
\label{sec:trackxmpl}

\begin{verbatim}
L: LINE=(OCT,-OCT); 
// Misalignments and closed orbit correction may be done here. 
P0=60.0;
B:BEAM,ENERGY=60.0; 
TRACK,LINE=L,BEAM=B;
   START,X=0.001,Y=0.001,PT=0.001;
   RUN,TURNS=1024; 
ENDTRACK;
\end{verbatim}
