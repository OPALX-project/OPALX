\chapter{Installation}
\label{chp:installation}
\opal\ and all its flavours are based on several packages which are all installable using the configure-make-install trilogy. \\ \\
\opal\ is also preinstalled on the FELSIM cluster at PSI. The preinstalled version can be accessed 
using the module command: 
\begin{footnotesize}
\begin{verbatim}
module load opal
\end{verbatim}
\end{footnotesize}

\section{Build and install \opal\ on a Mac (Snow Leopard)}
\subsection{Supporting Libraries}
The following libraries and tools must be present before starting with the actual \opal\ installation process described in (\ref{sec:instmacclop}).
The following packages are maybe already installed. Please check the versions carefully and do not use older ones. For a Linux installation
refer to the instruction in the individual packages. 
\begin{itemize}
\item macport from www.macports.org
\item Xcode \url{http://developer.apple.com/TOOLS/Xcode/}
\item automake-1.10.2                
\item autoconf-2.64
\item libtool-2.2           
\end {itemize}


\subsection{Environment Variables} \label{subsec:envvar}
Assuming you install IPPL and \opal\ into {\tt \$HOME/svnwork}, the following 
environment variables must be set accordingly:
\begin{footnotesize}
\begin{verbatim}
export OPAL_ROOT=$HOME/svnwork/OPAL/
export DOOM_ROOT=$OPAL_ROOT/doom/
export CLASSIC_ROOT=$OPAL_ROOT/classic/5.0/

export H5Part_ROOT=$HOME/svnwork/H5Part-1.6.2
export IPPL_ROOT=$HOME/svnwork/ippl/
export IPPL_ARCH=MACOSX
\end{verbatim}
\end{footnotesize}


\subsubsection{Build/Install gcc (4.5.0)}
\begin{footnotesize}
\begin{verbatim}
sudo port install gcc45
\end{verbatim}
\end{footnotesize}

\subsubsection{Build/Install gsl (1.13)}
\begin{footnotesize}
\begin{verbatim}
sudo port install gsl
\end{verbatim}
\end{footnotesize}

\subsubsection{Build/Install OpenMPI (openmpi-1.3.3)}
\begin{footnotesize}
\begin{verbatim}
CC=gcc CXX=g++ F77=gfortran ./configure
\end{verbatim}
\end{footnotesize}

\subsubsection{Build/Install HDF5 1.6.10}
\begin{footnotesize}
\begin{verbatim}
./configure --enable-parallel --prefix=/usr/local

Libraries have been installed in:
   /usr/local/lib

If you ever happen to want to link against installed libraries
in a given directory, LIBDIR, you must either use libtool, and
specify the full pathname of the library, or use the `-LLIBDIR'
flag during linking and do at least one of the following:
   - add LIBDIR to the `DYLD_LIBRARY_PATH' environment variable
     during execution

See any operating system documentation about shared libraries 
for more information, such as the ld(1) and ld.so(8) manual 
pages.
\end{verbatim}
\end{footnotesize}

The last two libraries to install are H5Part  and IPPL
\subsubsection{H5Part (1.6.2)}
The tarball can be found at:
\begin{center}
\url{https://codeforge.lbl.gov/projects/h5part/} 
\end{center}
%or alternatively the following svn checkout
%\begin{center}
%svn checkout svn+ssh://svn.psi.ch/repos/H5Part/H5Part/1.4
%\end{center}
%will get you the trunk of the repository. 
Now we can build and install the package:
\begin{footnotesize}
\begin{verbatim}
./configure --enable-parallel --disable-tools --prefix=/usr/local/
make 
sudo make install
\end{verbatim}
\end{footnotesize}

\subsubsection{IPPL}
The following svn checkout
\begin{footnotesize}
\begin{verbatim} cd $IPPL_ROOT
\end{verbatim}
\end{footnotesize}
\begin{center}
svn checkout svn+ssh://savannah.psi.ch/repos/amas/amas/ippl/trunk
\end{center}
will get you the trunk of the repository \footnote{If you can not checkout the sources
send email to \url{andreas.adelmann@psi.ch}.}. Now we can build and install the package:
\begin{footnotesize}
\begin{verbatim}
CXX=mpicxx F77=gfortran ./configure --with-ippl-linuxgcc 
         --libdir=$IPPL_ROOT/lib/$IPPL_ARCH
\end{verbatim}
\end{footnotesize}
Maybe {\tt Makefile.def} must be adapted prior to the {\tt configure} step.
What remains is actually the installation of \classic\ and \opal.





%\subsubsection{FFTW (3.2.1 www.fftw.org ) }
%\begin{footnotesize}
%\begin{verbatim}
%./configure --enable-mpi --prefix=/usr/local

%Libraries have been installed in:
%   /usr/local/lib

%If you ever happen to want to link against installed libraries
%in a given directory, LIBDIR, you must either use libtool, and
%specify the full pathname of the library, or use the `-LLIBDIR'
%flag during linking and do at least one of the following:
%   - add LIBDIR to the `DYLD_LIBRARY_PATH' environment variable
%     during execution

%See any operating system documentation about shared libraries 
%for more information, such as the ld(1) and ld.so(8) manual pages.
%\end{verbatim}
%\end{footnotesize}


\subsection{Installing CLASSIC and OPAL} \label{sec:instmacclop}
The following svn checkout \footnote{If you can not checkout the sources
send an email to \url{andreas.adelmann@psi.ch}.}
\begin{footnotesize}
\begin{verbatim}cd $OPAL_ROOT
\end{verbatim}
\end{footnotesize}
\begin{center}
svn checkout svn+ssh://savannah.psi.ch/repos/amas/amas/OPAL/trunk OPAL
\end{center}
will get you the trunk of the repository. 

\subsubsection{Doom (H. Grothe CERN) }
Use the {\tt autogen.sh} if you install for the first time:
\begin{verbatim}
cd $DOOM_ROOT
autogen.sh
\end{verbatim}
Doom can be used in conjunction with \opal-map. Now  install \classic\:
\begin{footnotesize}
\begin{verbatim}
cd $CLASSIC_ROOT
\end{verbatim}
\end{footnotesize}
and then \opal :
\begin{footnotesize}
\begin{verbatim}
cd $OPAL_ROOT
\end{verbatim}
\end{footnotesize}
by using the available {\tt autogenXX.sh} in the respective directories mentioned above.

\section{Build and install OPAL on FELSIM (PSI)}
\subsection{Autotools}
This is a sample session for the PSI FELSIM Linux cluster, however it can be regarded as a prototype for any decent Linux cluster
installation. Set the environment variables as described in  (\ref{subsec:envvar}) but change
\begin{footnotesize}
\begin{verbatim}
export IPPL_ARCH=LINUX
\end{verbatim}
\end{footnotesize}
Make sure the following modules are loaded:
\begin{footnotesize}
\begin{verbatim}
Currently Loaded Modulefiles:
1) sge/ge62 4) mpi/openmpi-1.2.6-intel-11.1 7) trilinos/trilinos-10.2.0
2) defaultenvironment  5) hdf5/hdf5-1.6.10-openmpi-1.2.6-intel-11.18) gsl/gsl-1.12
3) intel/intel-11.1_64  6) superlu-dist/superlu_dist_2.0 9) gnuplot/gnuplot-4.2.3
\end{verbatim}
\end{footnotesize}
Now we can checkout OPAL from the repository: 
\begin{footnotesize}
\begin{verbatim}
cd ${HOME}/svnwork
svn checkout \
    svn+ssh://savannah.psi.ch/repos/amas/amas/OPAL/trunk OPAL
\end{verbatim}
\end{footnotesize}
To build OPAL we have to build DOOM and CLASSIC first.
\begin{footnotesize}
\begin{verbatim}
cd $DOOM_ROOT
autogen.sh

cd $CLASSIC_ROOT/src
autogen.sh
\end{verbatim}
\end{footnotesize}
Now it is time to build OPAL:
\begin{footnotesize}
\begin{verbatim}
cd $OPAL_ROOT/src
autogen.sh
\end{verbatim}
\end{footnotesize}

\subsection{CMake}
It is assumed that all necessary libraries are installed and referenced using the variables described in (\ref{subsec:envvar}).
Then we can checkout OPAL from the repository: 
\begin{footnotesize}
\begin{verbatim}
cd ${HOME}/svnwork
svn checkout \
    svn+ssh://savannah.psi.ch/repos/amas/amas/OPAL/trunk OPAL
\end{verbatim}
\end{footnotesize}

Now change to the directory where \opal\ should be build and issue the following commands:
\begin{footnotesize}
\begin{verbatim}
CXX=mpicxx cmake $OPAL_ROOT
make 
make install 
\end{verbatim}
\end{footnotesize}
At the end of the build process the \opal\ executable resides in {\tt src}, relative to the build directory.

\section{Build and install OPAL on the Cray XT5}
This is a sample session for the Cray XT5 "Rosa" at CSCS, Switzerland. First add the following commands to your {\tt .bashrc} if
you have not already done so. It is assumed that IPPL and H5Part are installed in {\tt \${HOME}/svnwork/}.

\begin{footnotesize}
\begin{verbatim}
module load subversion/1.4.2
module swap PrgEnv-pgi/1.4.48 PrgEnv-gnu/1.4.48
module swap gcc/4.1.1 gcc/3.2.3
module load craypat
module load hdf5
#
export IPPL_ROOT=$HOME/svnwork/ippl
export IPPL_ARCH=XT5

# OPAL stuff
export DOOM_ROOT=$HOME/svnwork/OPAL/doom
export CLASSIC_ROOT=$HOME/svnwork/OPAL/classic/5.0/
export OPAL_ROOT=$HOME/svnwork/OPAL/
export H5Part=$HOME/svnwork/H5Part
export HDF5HOME=/apps/hdf5-1.6.5

export MPICH_ROMIO_NO_RECORD_LOCKING=1

export CXX=CC
export CPP=cc
\end{verbatim}
\end{footnotesize}

Then login the shell again or activate it by 
\begin{footnotesize}
\begin{verbatim}
$ source ~/.bashrc
\end{verbatim}
\end{footnotesize}
Now we can checkout OPAL from the repository: 

\begin{footnotesize}
\begin{verbatim}
$ cd ${HOME}/svnwork
$ svn checkout \
           https://svn.psi.ch/amas/amas/OPAL/trunk ~/svnwork/OPAL
\end{verbatim}
\end{footnotesize}
To build OPAL we first have to build DOOM and CLASSIC. 

\begin{footnotesize}
\begin{verbatim}
cd $DOOM_ROOT
autogen-gele.sh

cd $CLASSIC_ROOT/src
autogen-gele.sh
\end{verbatim}
\end{footnotesize}
Now it is time to build OPAL.
\begin{footnotesize}
\begin{verbatim}
cd $OPAL_ROOT/src
autogen-gele.sh
\end{verbatim}
\end{footnotesize}

\section{Using pre-build Binaries}
Pre-build binaries are available Linux and Mac OS X at the following download page. 

\section{Enabling the Multigrid Space Charge Solver}

{\bf Please note:} The Multigrid space charge solver is not yet capable of emitting a beam from the cathode.  You are advised to use the FFT space charge solver for studies
with particle emission.  

The following packages must be pre-installed: {\tt mkl-10.0\_em64t, parmetis, SuperLUDist}.

If using CMake you can enable the solver with\\ {\tt cmake -D ENABLE\_ML\_SOLVER=TRUE \$OPAL\_ROOT} and make sure the {\tt TRILINOS\_INCLUDE\_PATH} environment variable points to the directory containing the Trilinos header files.

If no Trilinos version ($>$10.6) is available, download and build the source code from the Trilinos webpage.\footnote{\url{http://trilinos.sandia.gov}} The following Trilinos packages are required:
\begin{itemize}
  \item epetra and epetraext
  \item ml and ml\_parmetis3x
  \item amesos and amesos-superludist
  \item ifpack
  \item teuchos and teuchos-extended
  \item aztecco and aztecoo-teuchos
  \item galeri 
  \item belos
\end{itemize}
To enable these packages run {\tt cmake} with the following arguments:
\begin{footnotesize}
\begin{verbatim}

CC=mpicc
CXX=mpicxx
CPP="mpicxx -E" 
F77=mpif77 

cmake \
--prefix=/path/to/install \
-DCMAKE_INSTALL_PREFIX:PATH=/path/to/install \
-DCMAKE_CXX_FLAGS:STRING="-DMPICH_IGNORE_CXX_SEEK -fPIC" \
-DCMAKE_C_FLAGS:STRING="-DMPICH_IGNORE_CXX_SEEK -fPIC" \
-DCMAKE_Fortran_FLAGS:STRING="-fPIC" \
-D CMAKE_BUILD_TYPE:STRING=DEBUG \
-D TPL_ENABLE_SuperLUDist:BOOL=ON\
-D TPL_ENABLE_MPI:BOOL=ON \
-D TPL_ENABLE_BLAS:BOOL=ON \
-D TPL_ENABLE_LAPACK:BOOL=ON \
-D TPL_ENABLE_ParMETIS:BOOL=ON \
-D TPL_ENABLE_METIS:BOOL=OFF \
-D BLAS_LIBRARY_DIRS:PATH=${MKL_LIB_DIR} \
-D BLAS_INCLUDE_DIRS:PATH=${MKL_INCLUDE} \
-D BLAS_LIBRARY_NAMES:STRING="mkl_blas95_lp64;mkl_intel_lp64;mkl_intel_thread;mkl_core;pthread;guide" \
-D LAPACK_LIBRARY_DIRS:PATH=${MKL_LIB_DIR} \
-D LAPACK_INCLUDE_DIRS:PATH=${MKL_INCLUDE} \
-D LAPACK_LIBRARY_NAMES:STRING="mkl_lapack" \
-D ParMETIS_INCLUDE_DIRS:PATH="${PARMETIS_INCLUDE_PATH}" \
-D ParMETIS_LIBRARY_DIRS:PATH="${PARMETIS_LIBRARY_PATH}" \
-D ParMETIS_LIBRARY_NAMES:STRING="parmetis;metis" \
-D ParMETIS_LIBRARIES:STRING="parmetis;metis" \
-D TPL_BLACS_INCLUDE_DIRS:FILEPATH="/gpfs/homefelsim/kraus/include" \
-D BLACS_LIBRARY_DIRS:FILEPATH="${MKL_LIB_DIR}" \
-D TPL_BLACS_LIBRARIES="/opt/intel-mkl/mkl-10.2/lib/em64t/libmkl_blacs_lp64.a;/opt/intel-mkl/mkl-10.2/lib/em64t/libmkl_blacs_ilp64.a" \
-D SuperLUDist_INCLUDE_DIRS:FILEPATH="/gpfs/homefelsim/kraus/extlib/SuperLU_DIST_2.5/include" \
-D SuperLUDist_LIBRARY_DIRS:FILEPATH="/gpfs/homefelsim/kraus/extlib/SuperLU_DIST_2.5/lib" \
-D SuperLUDist_LIBRARY_NAMES:STRING="superlu_dist_2.5" \
-D SuperLUDist_LIBRARIES="superlu_dist_2.5" \
-D TPL_SuperLUDist_LIBRARIES="/gpfs/homefelsim/kraus/extlib/SuperLU_DIST_2.5/lib/libsuperlu_dist_2.5.a" \
-D SCALAPACK_INCLUDE_DIRS:FILEPATH="${MKL_INCLUDE}" \
-D SCALAPACK_LIBRARY_DIRS:FILEPATH="${MKL_LIB_DIR}" \
-D SCALAPACK_LIBRARY_NAMES:STRING="mkl_scalapack_lp64" \
-D Trilinos_ENABLE_Belos:BOOL=ON \
-D Trilinos_ENABLE_Epetra:BOOL=ON \
-D Trilinos_ENABLE_EpetraExt:BOOL=ON \
-D Trilinos_ENABLE_Ifpack:BOOL=ON \
-D Trilinos_ENABLE_ML:BOOL=ON \
-D Trilinos_ENABLE_Amesos:BOOL=ON \
-D Amesos_ENABLE_BLACS:BOOL=ON \
-D Amesos_ENABLE_SuperLUDist:BOOL=ON \
-D Amesos_ENABLE_SCALAPACK:BOOL=ON \
-D Trilinos_ENABLE_Amesos-superlu:BOOL=ON\
-D Trilinos_ENABLE_AztecOO:BOOL=ON \
-D Trilinos_ENABLE_Teuchos:BOOL=ON \
-D Trilinos_ENABLE_Aztecoo-Teuchos:BOOL=ON \
-D Trilinos_ENABLE_Teuchos-Extended:BOOL=ON \
-D Trilinos_ENABLE_Isorropia:BOOL=ON \
-D Trilinos_ENABLE_Isorropia-Epetraext:BOOL=ON \
-D Trilinos_ENABLE_Didasko:BOOL=OFF \
-D Didasko_ENABLE_TESTS=OFF \
-D Didasko_ENABLE_EXAMPLES=OFF \
-D Trilinos_ENABLE_TESTS:BOOL=OFF \
 $EXTRA_ARGS \
 $TRILINOS_ROOT
\end{verbatim}
\end{footnotesize}


\section{Debug Flags}\label{sec:debugflags}

\begin{table}[ht]\footnotesize
  \begin{center}
    \caption{Debug flags.}
    \label{tbl:debug_flags}
      \begin{tabular}{lll}
        \hline
        {\bf Name} & {\bf Description} & {\bf Default} \\
        \hline
        DBG\_SCALARFIELD & dumps scalar potential on the grid & not set \\
        DBG\_STENCIL & dumps stencil (MG solver) to a Matlab readable file & not set \\
        \hline
      \end{tabular}
    \end{center}
\end{table}

\paragraph{DBG\_SCALARFIELD} dumps the field to a file called rho\_scalar. The structure of the data can be deduced from the following Matlab script:

\begin{footnotesize}
\begin{verbatim}
function scalfield(RHO)

rhosize=size(RHO)
for i=1:rhosize(1)
  x = RHO(i,1);
  y = RHO(i,2);
  z = RHO(i,3);
  rhoyz(y,z) = RHO(i,4);
  rhoxy(x,y) = RHO(i,4);
  rhoxz(x,z) = RHO(i,4);
  rho(x,y,z) = RHO(i,4);
end
\end{verbatim}
\end{footnotesize}

\paragraph{DBG\_STENCIL} dumps the discretization stencil to a file (A.dat). The following Matlab code will read and store the sparse matrix in the variable 'A'.

\begin{footnotesize}
\begin{verbatim}
load A.dat;
A = spconvert(A);
\end{verbatim}
\end{footnotesize}


\section{Examples}
When checking out the \opal\ framework you will find the {\em opal-Tests} directory and moreover
a subdirectory called {\em RegressionTests}. There several input files can be found which are
run every day to check the validity of the current version of \opal. This is a good starting-point to learn how to
model accelerators with the various flavours of \opal. More examples will be given in subsequent chapters, enjoy!

%& IPPL \\
% & V 1.0


