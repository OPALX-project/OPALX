\chapter{Action Commands}
\label{sec:action}
\index{action}

\section{Normal-Form Analysis}
\index{normal form!dynamic}
\index{normal form!static}
\index{dynamic!normal form}
\index{static!normal form}
Please note Tables are not yet supported in \noopalt and \noopalcycl . Normal-Form Analysis does not make sense in a non map based computation. 
The two commands
\begin{verbatim}
DYNAMIC,LINE=name,BEAM=name,FILE=string,ORDER=integer;
STATIC, LINE=name,BEAM=name,FILE=string,ORDER=integer;
\end{verbatim}
both evaluate the truncated Taylor series map for one turn up to order
\texttt{ORDER} and perform normal-form analysis on the result.
Both have the following attributes:
\begin{description}
\item[LINE]
\index{LINE}
The label of a \secref{\textbf{beam line or sequence}}{lines} defined
previously (no default).  Its transfer map will be evaluated and
analysed to the desired order.
\item[BEAM]
\index{LINE}
The label of a \secref{\texttt{BEAM}~command}{beam} defined
previously (default = \texttt{UNNAMED\_BEAM}).
\index{UNNAMED\_BEAM}
It defines the charge, kinetic energy, rest mass,
and reference velocity for the reference
particle. 
\item[FILE]
\index{FILE}
The name of the file to be written 
(default = "\texttt{dynamic}" or "\texttt{static}" respectively).
\item[ORDER]
\index{ORDER}
The maximum order for the map.
\end{description}

\subsection{Normal-Form Analysis for Dynamic Map}
\label{sec:dynamic}
\index{DYNAMIC}
The command interprets the transfer map as dynamic,
i.e. there is synchrotron motion with an average momentum $p_s$.

\subsection{Normal-Form Analysis for Static Map}
\label{sec:static}
\index{STATIC}
The command assumes that there are no cavities,
and interprets the transfer map as static,
i.e. the particles have constant momentum.
The \texttt{STATIC} command also finds the fixed point of the map,
that is the variation of the closed orbit with momentum.
