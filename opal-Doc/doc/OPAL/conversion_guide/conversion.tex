\documentclass{article}
%\usepackage{a4p}
\usepackage{html}
\usepackage{epsfig}
\usepackage{epic}
\usepackage{eepic}
\usepackage{makeidx}

\makeindex
\renewcommand{\topfraction}{1.0}
\renewcommand{\bottomfraction}{1.0}
\renewcommand{\textfraction}{0.0}

\newcommand{\opal}{\textsc{OPAL }}
\newcommand{\opalt}{\textsc{OPAL-t }}
\newcommand{\opalcycl}{\textsc{OPAL-cycl }}
\newcommand{\opalmap}{\textsc{OPAL-map }}

\newcommand{\noopalt}{\textsc{\leftthumbsdown \opalt }}
\newcommand{\noopalmap}{\textsc{\leftthumbsdown \opal }}
\newcommand{\noopalcycl}{\textsc{\leftthumbsdown \opalcycl }}


\newcommand{\bibref}[2]{\hypercite[cite]{#1}{}{}{bib:#2}}

\newcommand{\tabline}[3]{\hyperref{\texttt{#1}}%
  {\texttt{#1} (}{)}{sec:#3} & #2 \\}
\newcommand{\figref}[2]{\hyperref{#1}{#1 (see Figure~}{)}{fig:#2}}
\newcommand{\secref}[2]{\hyperref{#1}{#1 (see Section~}{)}{sec:#2}}
\newcommand{\tabref}[2]{\hyperref{#1}{#1 (see Table~}{)}{tab:#2}}

\begin{htmlonly}
\bodytext{bgcolor = "#ffffff"}
\end{htmlonly}


\newcommand{\documentlabel}{PSI - GFA (AMAS)}

\begin{document}

\begin{titlepage}

\begin{htmlonly}
\begin{rawhtml}
<center>
PAUL SCHERRER INSTITUT
<br>
<h1>The OPAL Framework , Version 0.1</h1>
<h2>(Object Oriented Parallel Accelerator Library)</h2>
<h2>MAD-8 (X) to OPAL Conversion Guide </h2>
<br>
Andreas Adelmann
</center>
\end{rawhtml}
\end{htmlonly}
\begin{latexonly}
\begin{center}\normalsize
PSI
\end{center}
\vskip 0.7cm
\begin{flushright}
%\documentlabel \\                    % document label
\end{flushright}
\vskip 2.3cm
\begin{center}\LARGE                 % document title
{\bf The OPAL Framework } \\
(Object Oriented Parallel Accelerator Library) \\
Version 0.1 \\
{\bf MAD-8 (X) to OPAL Conversion Guide}
\end{center}
\vskip 1.5em
\begin{center}                       % author
Andreas Adelmann
\vskip 2em
{\large Abstract}
\end{center}
\end{latexonly}
\begin{quotation}
This guide is aimed at users of MAD version 8 and earlier.
It shows how to change input files to the OPAL MAD format.

Please send any comments grep hito 
\htmladdnormallink{\texttt{andreas.adelmann@psi.ch}}{mailto:andreas.adelmann@psi.ch}.
\end{quotation}
\vfill
\begin{center}
PSI, Villigen \\
\today
\end{center}
\end{titlepage}

\tableofcontents

\section{Basic Language Format}
The following changes have been made to the MAD input language in order
to reduce the risk of ambiguity and to have an input language closer to
well-known programming languages like Pascal, C, or C++:

\begin{description}
\item[Statement termination]
  All statements are now terminated by a semicolon (;).
  This removes the need for a continuation character when a statement is 
  longer than one line.

\item[Comments]
  Comments are now denoted as in C++, namely
  \begin{itemize}
  \item Text following with a double slash (\texttt{//}) is ignored up to
    the end of line (like in MAD-8 for the '!' character). 
    Example:
\begin{verbatim}
// This is a comment
\end{verbatim}
  \item Comment can also be delimited by "/*" (begin) and "*/" (end). 
    They may span several lines, and the begin/end markers may be nested
    (like in MAD-8 for COMMENT/ENDCOMMENT).
    Example:
\begin{verbatim}
/* This is also a comment */
\end{verbatim}
  \end{itemize}

\item[Line length]
  The input line length is no longer limited to 80 characters.

\item[Keywords]
  All command and attribute keywords must be entered in full.
  Abbreviations are no longer accepted.

\item[Attributes]
  On all commands, all attribute keywords are mandatory.
  This removes potential ambiguities between string values of attributes
  which could be mistaken for attribute names.

\item[Object attributes]
  Command attributes are now written in the form
\begin{verbatim}
<object-name> -> <attribute-name>
\end{verbatim}
  which replaces the MAD-8 syntax
\begin{verbatim}
<object-name>[<attribute-name>]
\end{verbatim}
\end{description}

\section{Conventions}
The different flavours of \opal also have different units to describe the phase-space.
\opalmap is following the CLASSIC standard, the phase-space coordinates are now:
\[
x, p_{x}, y, p_{y}, - v\cdot t, p_{t} = \delta p / p_{s}.
\]
where $p_{s}$ is the constant momentum of a "synchronous"
particle, and $v$ is the instantaneous velocity of the particle.

\opalt 

\opalcycl

\section{Multipoles}
A \texttt{MULTIPOLE} \textbf{Always has a finite length.}
Some algorithms, notably tracking and map accumulation,
allow to replace finite length multipoles by thin lenses.
Special integrators will be made available which approximate thick lenses 
by sets of thin lenses.

It has an open set of \textbf{multipole components.}
These are entered as two \secref{vectors}{vect} 
\texttt{knormal} and \texttt{kskew}.

\section{Variables and constants}
The format for variables (parameters) and constants has been unified,
and a data types have been introduced:
\begin{itemize}
\item \secref{logical variable}{logical},
\item \secref{real constant}{real},
\item \secref{real variable}{real},
\item \secref{real vector}{vect},
\item \secref{string variable}{string}.
\end{itemize}
\begin{verbatim}
REAL P=<logical expression>;          // a logical variable
BOOL B=<real expression>;             // a real variable
REAL CONST C=<constant expression>;   // a real constant
REAL VECTOR V=<vector expression>;    // a real vector
STRING S=<string expression>;         // a string constant
\end{verbatim}
The word \texttt{REAL} is optional.
For backward compatibility, a variable parameter can still be entered with
\begin{verbatim}
P:=<expression>
\end{verbatim}
A variable or vector also can be redefined by the above syntax.
To assigne a \textbf{constant} value to one of them, use
\begin{verbatim}
P=<expression>;
V=<vector-expression>;
\end{verbatim}

\subsection{Logical Expressions}
\label{sec:logical}
A logical expression has the same format as a logical expression in C.
It is built from \tabref{operators}{operators} and operands.
Operands may include \tabref{functions}{functions}:
\begin{verbatim}
   <relation> ::=
     true | false |
     <real expression> <relational operator> <real expression>
   <relational operator> ::= "==" | "!=" |
     "<" | ">" | ">=" | "<="
   <and expression> ::= <relation> |
     <and expression> "&&" <relation>
   <logical expression> ::= <and expression> |
     <logical> "||" <and expression>
\end{verbatim}

\subsection{Real Expressions}
\label{sec:real}

A real expression is built from operators \tabref{operators}{operators} and 
operands. Operands may include
\tabref{functions}{functions}:

\begin{verbatim}
   <primary> ::= <literal constant> |
                 <symbolic constant> |
                 <real variable> |
                 <real vector>"["<index>"]" |
                 <function name>"("<arguments>")" |
                 (<real expression>)
   <factor>  ::= <primary> |
                 <factor> "^" <primary>
   <term>    ::= <factor> |
                 <term> "*" <factor> |
                 <term> "/" <factor>
   <real expression> ::= <term> |
                 "+" <term> |
                 "-" <term> |
                 <real expression> "+" <term> |
                 <real expression> "-" <term> |
\end{verbatim}

\subsection{Vector Expressions}
\label{sec:vect}
A \texttt{<vector-expression>} can have one of the following formats:
\begin{verbatim}
   <real expression>
   {<real expression>, <real expression>, ... , <real expression>}
   TABLE(<max>, <real expression>)
   TABLE(<min>:<max>, <real expression>)
   TABLE(<min>:<max>:<step>, <real expression>)
\end{verbatim}
It can be entered as a single scalar value,
giving a vector of length one,
as a table of scalar values with an unlimited number of values,
or as a \texttt{TABLE} generator/
In the \texttt{TABLE} generator the \texttt{<count>} specifies how
many components are wanted,
and the \texttt{<real expression>} may contain the special value
\texttt{\#}.
It allws to generate arbitrary tables.
The generator fills the vector with zeros from \verb'1 ... <max>',
and the generates the expressions for all values \verb'<min> ... <max>'
with steps \verb'<step>'.
Thus 
\begin{verbatim}
TABLE(4, #*A+B)
\end{verbatim} 
generates the vector 
\begin{verbatim}
{1*A+B, 2*A+B, 3*A+B, 4*A+B},
\end{verbatim}
and 
\begin{verbatim}
TABLE(1,9,4, V[#+1])
\end{verbatim} 
returns the components \verb'2, 6, 10' of the vector \texttt{V} and
stores them in components \verb'1, 5, 9'.

\textbf{A vector component} is referred to by the syntax
\begin{verbatim}
<vector>[<index>]
\end{verbatim}
where the \texttt{index} must have a positive non-zero value not
exceeding the length of the vector.
This may be combined with the attribute notation like
\begin{verbatim}
MULTIPOLE->KNORMAL[1]
\end{verbatim}
which denotes the normal quadrupole component of \texttt{multipole}.

\subsection{String Expressions}
\label{sec:string}
All \textbf{string values}, including object names,
but not attribute keywords,
can be written using string concatenation:
\begin{verbatim}
   <simple string> :: " <litteral> " | <label> |
      STRING(<numeric expression>)
   <string expression> ::= <simple string> |
      <string expression> "&" <simple string>
\end{verbatim}
The '\&' is the concatenation operator.
\begin{verbatim}
QF & STRING(I): QUADRUPOLE, L=1.6, K1=KF & STRING(I);
\end{verbatim}
An operand of the form
\begin{verbatim}
STRING(<numeric expression>)
\end{verbatim}
generates a string representation of the <numeric expression>.

\subsection{Expressions}
All arithmetic is done in floating-point.
When an integer is required, the \textbf{last} operation consists in rounding
to the nearest integer.
A different behaviour can be requested by use of one of the functions
\texttt{ROUND(X), TRUNC(X), FLOOR(X), CEIL(X)}
All available operators are listed in \tabref{operators}{operators},
all available functions are listed in \tabref{functions}{functions},

The complete list of available operators in expressions is now:
\begin{table}[ht]
\begin{center}  
\begin{tabular}{|l|p{0.6\textwidth}|l|l|}
\hline
Operator & Meaning & result type & operand type \\
\hline
\multicolumn{4}{|c|}{\textbf{Operators with one operand}} \\
\hline
\texttt{+ X} & unary plus, returns \texttt{X} & real &real \\
\hline
\texttt{- X} & unary minus, returns the negative of \texttt{X} & real
&real \\
\hline
\multicolumn{4}{|c|}{\textbf{Operators with two operands}} \\
\hline
\texttt{X + Y} & add \texttt{X} to \texttt{Y} & real & real,real \\
\hline
\texttt{X - Y} & subtract \texttt{Y} from \texttt{X} & real & real,real \\
\hline
\texttt{X * Y} & multiply \texttt{X} by \texttt{Y} & real & real,real \\
\hline
\texttt{X / Y} & divide \texttt{X} by \texttt{Y} & real & real,real \\
\hline
\texttt{X \^\  Y} & power, return \texttt{X} raised to the power \texttt{Y > 0} &
real & real,real \\
\hline
\texttt{X \& Y} & concatenate the strings \texttt{X} and \texttt{Y} &
string & string,string \\
\texttt{X < Y} &
true, if \texttt{X} is less than \texttt{Y} & logical & real,real \\
\hline
\texttt{X <= Y} & true, if \texttt{X} is not greater than \texttt{Y} &
logical & real,real \\
\hline
\texttt{X > Y} &
true, if \texttt{X} is greater than \texttt{Y} & logical & real,real \\
\hline
\texttt{X >= Y} & true, if \texttt{X} is not less than \texttt{Y} &
logical & real,real \\
\hline
\texttt{X == Y} & true, if \texttt{X} is equal to \texttt{Y} & 
logical & real,real \\
\hline
\texttt{X != Y} & true, if \texttt{X} is not equal to \texttt{Y} & 
logical & real,real \\
\hline
\texttt{X \&\& Y} & true, if both \texttt{X} and \texttt{Y} are true &
logical & logical,logical \\
\hline
\texttt{X || Y} & true, if at least one of \texttt{X} and \texttt{Y} is true &
logical & logical,logical \\
\hline
\end{tabular}
\end{center}
\caption{Operators available in MAD-9 Expressions}
\label{tab:operators}
\end{table}

\begin{table}[ht]
\begin{center}
\begin{tabular}{|l|p{0.6\textwidth}|l|l|}
\hline
Function & Meaning & result type & operand type \\
\hline
\multicolumn{4}{|c|}{\textbf{Functions with no arguments}} \\
\hline
\texttt{RANF()} & random number, uniform distribution in [0,1) & real & \\
\hline
\texttt{GAUSS()} &
random number, Gaussian distribution with $sigma$=1 & real & \\
\hline
\texttt{USER0()} & random number, user-defined distribution & real & \\
\hline
\texttt{SI()} &
arc length from start of ring to the entry of the current element &
real & \\
\hline
\texttt{SC()} &
arc length from start of ring to the centre of the current element &
real & \\
\hline
\texttt{SO()} &
arc length from start of ring to the exit of current the element &
real & \\
\hline
\multicolumn{4}{|c}{\textbf{Functions with one argument}} \\
\hline
\texttt{TRUNC(X)} & 
truncate \texttt{X} towards zero (discard fractional part) & real & real \\
\hline
\texttt{ROUND(X)} & round \texttt{X} to nearest integer & real &real \\
\hline
\texttt{FLOOR(X)} & return largest integer not greater than \texttt{X} &
real & real \\
\hline
\texttt{CEIL(X)} &
return smallest integer not less than \texttt{X} & real & real \\
\hline
\texttt{SIGN(X)} & return sign of \texttt{X}
(+1 for \texttt{X} positive, -1 for \texttt{X} negative,
 0 for \texttt{X} zero) & real & real \\
\hline
\texttt{SQRT(X)} & return square root of \texttt{X} & real & real \\
\hline
\texttt{LOG(X)} &
return natural logartithm of \texttt{X} & real & real \\
\hline
\texttt{EXP(X)} &
return exponential to the base $e$ of \texttt{X} & real & real \\
\hline
\texttt{SIN(X)} &
return trigonometric sine of \texttt{X} & real & real \\
\hline
\texttt{COS(X)} & return trigonometric cosine of \texttt{X} & real & real \\
\hline
\texttt{ABS(X)} & return absolute value of \texttt{X} & real & real \\
\hline
\texttt{TAN(X)} & return trigonometric tangent of \texttt{X} & real & real \\
\hline
\texttt{ASIN(X)} & return inverse trigonometric sine of \texttt{X} &
real & real \\
\hline
\texttt{ACOS(X)} & return inverse trigonometric cosine of \texttt{X} &
real & real \\
\hline
\texttt{ATAN(X)} & return inverse trigonometric tangent of \texttt{X} &
real & real \\
\hline
\texttt{TGAUSS(X)} & random number, Gaussian distribution with $sigma$=1,
 truncated at \texttt{X} & real & real \\
\hline
\texttt{USER1(X)} & 
random number, user-defined distribution with one parameter &
real & real \\
\hline
\texttt{STRING(X)} &
return string representation of the value<br>
of the numeric expression \texttt{X} &
string & real \\
\hline
\multicolumn{4}{|c}{\textbf{Functions with two arguments}} \\
\hline
\texttt{ATAN2(X,Y)} & return inverse trigonometric tangent of \texttt{Y/X} &
real & real,real \\
\hline
\texttt{MAX(X,Y)} & return the larger of \texttt{X, Y} & real & real,real \\
\hline
\texttt{MIN(X,Y)} & return the smaller of \texttt{X, Y} & real & real,real \\
\hline
\texttt{MOD(X,Y)} & return the largest value less than \texttt{Y}
 which differs from \texttt{X} by a multiple of \texttt{Y} &
real & real,real \\
\hline
\texttt{USER2(X,Y)} & 
random number, user-defined distribution with two parameters &
real & real,real \\
\hline
\end{tabular}
\end{center}
\caption{Functions Available in MAD-9 Expressions}
\label{tab:functions}
\end{table}

\section{Program Flow Control}

\subsection{If Statements}
\textbf{\texttt{IF} statements} can now be written like in C:
\begin{verbatim}
IF (X < 10) {
   LAG = 0.0;
} ELSE {
   LAG = 0.5;
}
\end{verbatim}
The expression folllowing \texttt{IF} is a 
\secref{\texttt{<logical expression>}}{logical}

\subsection{While Statements}
\textbf{\texttt{WHILE} statements} can now be written like in C:
\begin{verbatim}
X = 0;
WHILE (X < 10) {
   SHOW, NAME = QF & STRING(X);
   X = X + 1;
}
\end{verbatim}
The expression folllowing \texttt{WHILE} is a 
\secref{\texttt{<logical expression>}}{logical}

\subsection{Macros}
The subroutines of MAD-8 have been replaced by the \textbf{\texttt{MACRO} 
construct}, which allows arguments to be handed over:
\begin{verbatim}
// MACRO DEFINITIONS:
SHOWIT(X): MACRO {
   SHOW, NAME = X;
}
DOIT(): MACRO {
   DYNAMIC,LINE=RING,FILE="DYNAMIC.OUT";
}

// MACRO CALLS:
SHOWIT(PI);
DOIT();
\end{verbatim}
An empty parameter list must be indicated by \texttt{()}.
Macros can occur anywhere in the input.
However, a macro expansion always produces zero or more \textbf{complete}
statements.

\section{Lines and Sequences}

\subsection{Sequences}
The length of a \texttt{SEQUENCE} must now be given either on its 
header line or on the \texttt{END} line:
\begin{verbatim}
LHC: SEQUENCE, L=...;
   ...
END, L=...; // Do not use both ways to specify the length!!!
\end{verbatim}
The preferred way is on the header line.

A \texttt{SEQUENCE} may now have arguments like an \texttt{LINE}:
\begin{verbatim}
S(X): SEQUENCE,L=10;
   SF.&X:SF,AT=2;
   X:QF,AT=5;
   OF.&X:OF,AT=8;
ENDSEQUENCE;
\end{verbatim}
This half-cell may be instantiated with
\begin{verbatim}
S1:S(Q1);
\end{verbatim}
giving the same effect as
\begin{verbatim}
S1: SEQUENCE,L=10;
   SF.Q1:SF,AT=2;
   Q1:QF,AT=5;
   OF.Q1:OF,AT=8;
ENDSEQUENCE;
\end{verbatim}
Note the generated names.

A \texttt{LINE} or \texttt{SEQUENCE} can now be nested within a
\texttt{SEQUENCE}.
If the nested objects have arguments,
their use causes an instantiation.
\begin{verbatim}
LHC: SEQUENCE, L=...;
   ...
   S(QF1),AT=...;
   S(QD1),AT=...;
   ...
END
\end{verbatim}

\section{Physical Actions}
\label{sec:action}

In order to make matching much more flexible,
the \texttt{USE} command has been removed.
In MAD-8 the example
\begin{verbatim}
USE,PERIOD=LINE1
BEAM,...
SURVEY
TWISS
TRACK
...
ENDTRACK
\end{verbatim}
allows only the use of one beam line at a time.
In MAD-9, the equivalent commands are:
\begin{verbatim}
BEAM1:BEAM,...
SU1:SURVEY,LINE=LINE1;
TW1:TWISS,LINE=LINE1,BEAM=BEAM1;
TR1:TRACK,LINE=LINE1,BEAM=BEAM1;
...
ENDTRACK
\end{verbatim}
The labels on the \texttt{SURVEY}, \texttt{TWISS}, and \texttt{TRACK}
commands are optional, but they may be referred to in matching.
The following should be noted:
\begin{enumerate}
\item
  If the \texttt{BEAM} command is not labelled, its name defaults to
  \texttt{UNNAMED\_BEAM}. 
  If it is labelled, different subsequent commands can refer to
  different \texttt{BEAM} commands.
\item
  If no \texttt{BEAM} command is given on \texttt{TWISS} or
  \texttt{TRACK}, the command \texttt{UNNAMED\_BEAM} is used.
  Its default definition is:
  \begin{verbatim}
UNNAMED_BEAM:BEAM,PARTICLE=PROTON,PC=1,EX=1,EY=1,ET=1;
  \end{verbatim}
\item
  The \texttt{TWISS} and \texttt{SURVEY} commands create a table
  object.
  All values contained in a table object can be referred to in
  matching with the syntax
  \begin{verbatim}
table-name@place->column-name
  \end{verbatim}
  To avoid re-computation of a table which does not change during
  matching, declare it as \texttt{STATIC}.
\end{enumerate}

\printindex

\end{document}
