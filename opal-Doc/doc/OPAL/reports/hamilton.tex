\documentclass{report}
%\usepackage{a4p}

\newcommand{\lieop}[1]
{{:}{#1}{:}}
\renewcommand{\arraystretch}{2.0}

\begin{document}

\title{Physical Methods used in OPAL}
\author{Andreas Adelmann}
\abstract{
  This paper describes the hamilitonian forms used the CLASSIC library;
  this formalism is also used in OPAL.

  The paper showns how the hamilitonian must be converted such as to
  use the relative momentum error without making it difficult to get
  at the time difference between the reference particle and the actual
  particle.

  The expansions for scalar and vector potentials to be plugged in are
  also described.
  }

\maketitle


\section{Introduction}
OPAL uses a hybrid description of the momentum error.
The \textit{average momentum} is defined by means of the
\textit{relative momentum error} $\delta_r$, with the definition
\begin{equation}
  p_r = p_0\;(1+\delta_r), \qquad \delta_r = \frac{p_r - p_0}{p_0}.
\end{equation}
where $p_r$ is the \textit{average (synchronous) momentum} and $p_0$
is the \textit{design momentum} of the machine.

In the presence of synchrotron motion and/or synchrotron radiation,
the \textit{actual momentum} oscillates around $p_r$.
It is defined by the \textit{energy error} scaled by the
\textit{momentum} 
\begin{equation}
  p_t = \frac{E-E_r}{p_rc},
\end{equation}
where $E_r$~is the energy corresponding to the momentum~$p_r$.

For strongly relativistic particles both $\delta_r$ and $p_t$ can be
interpreted as relative energy errors or as relative momentum errors.
However, for low energies the distinction is important.
With respect to the same momentum these definitions differ by powers
of~$\beta$: 
\begin{equation}
  \frac{\Delta E}{E_r} \approx p_t = \beta\;\frac{\Delta E}{p_r c}
  \approx \beta^2\;\frac{\Delta p}{p_r},
  \qquad \hbox{where\ } \beta = \frac{v}{c} = \frac{p_r c}{E_r}.  
\end{equation}
This paper describes the formalism used in the CLASSIC library and in
OPAL. 
It expresses both quantities as relative momentum errors.
To avoid problems when calculating the time difference and treat the
synchrotron motion correctly,
we apply some transformations to the hamilitonian.
Section~\ref{sec:hamilton} introduces the hamilitonian used and
describes the necessary transformations. 
Section~\ref{sec:potential} develops the potential expansions to be
used,
while Section~\ref{sec:motion} describes the motion for various
elements.  


\clearpage
\section{Hamiltonian for Magnetic Elements}
\label{sec:hamilton}


\subsection{Hamiltonian with Arc Length as Independent Variable}
The general hamilitonian for a magnetic element with a curved reference
system with the curvature~$h$ has been derived in several
references~\cite{COURANT,DRAGT}.
It has the form
\begin{equation}
  H =-(1+hx)
  \left(
    eA_s + \sqrt{\bar{E}^2-(mc)^2-(p_x-eA_x)^2-(p_y-eA_y)^2}
  \right).
\end{equation}
The independent variable is the arc length~$s$,
and the canonical pairs of variables are
\begin{equation}
  (x,\;p_x), \qquad (y,\;p_y), \qquad (\bar{t}=-ct,\;\bar{E}=E/c).
\end{equation}
The longitudinal variables are scaled with the velocity of light~$c$.
Thus all position coordinates have the dimension of a length,
and the hamilitonian and all canonical momenta have the same dimension.
The particle charge is~$e$, and 
$A_x$,~$A_y$, and~$A_s$ are the components of the magnetic vector
potential expressed in the curvilinear reference system.
If all three vector potential components are retained, 
arbitrary magnetic elements can be represented.


\clearpage
\subsection{Reference Particle and Time Difference}
The machine is laid out for a \textit{design momentum}~$p_0$
and has the circumference~$C$.
A \textit{design particle} has the design momentum and the velocity
\begin{equation}
  v_0 = (p_0 c) \Bigm/ \sqrt{p_0^2+(mc)^2}.
\end{equation}
It completes one turn in the time
\begin{equation}
  T_0 = C / v_0 = C \sqrt{p_0^2+(mc)^2} \Bigm/ (p_0 c).
\end{equation}
It defines a \textit{synchronous reference frame} which moves
along the reference orbit with the constant velocity $v_0$.
The \textit{synchronous revolution frequency} is defined as
\begin{equation}
f_0 = 1 / T_0 = v_0 / C.
\end{equation}
Given the harmonic number $h$, the RF~frequency is normally adjusted to
\begin{equation}
f_{RF} = h f_0,
\end{equation}
and the phase lags of the RF~cavities are defined with respect to the
arrival time of the design particle.

A change the average particle momentum is obtained by changing the RF
frequencies to a slightly different value
\begin{equation}
f_{RF} -> f_{RF} (1 + \Delta f/f).
\end{equation}
We postulate a ficticious \textit{off-momentum reference particle}
with the constant \textit{reference momentum}
\begin{equation}
  p_r = p_0 (1 + \delta_r)
\end{equation}
Its orbit length is to first order in~$\delta_r$
\begin{equation}
  C_r \approx \int_0^C (1 + h D_x \delta_r) ds = C\;(1 + \alpha_p\delta_r),
\end{equation}
where
\begin{equation}
  \alpha_p = \frac{1}{\delta_r} \cdot \frac{C_r - C}{C} =
  \frac{1}{C} \cdot \int_0^C h D_x ds
\end{equation}
is the momentum compaction factor.
The reference particle has the velocity
\begin{equation}
  v_r = p_r c \Bigm/ \sqrt{p_r^2+(mc)^2}.
\end{equation}
and takes the time
\begin{equation}
  T_r = C_r / v_r = 
  C (1+\alpha_p \delta_r) \sqrt{p_r^2+(mc)^2} \Bigm/ (p_r c).
\end{equation}
to complete one turn.
We also define a \textit{off-momentum reference frame} with moves with
the velocity 
\begin{equation}
  v_R = C / T_r = v_r / (1 + \alpha_p\delta_r).
\end{equation}
along the design orbit.
By definition it completes one revolution in the time~$T_r$.
It represents an approximation to the arriveal time of the reference
particle, and it can be used to define the RF~phase lags.

The velocity of the off-momentum reference frame can be expressed in
two forms: 
\begin{equation}
  v_R = v_0\;(1+\Delta f/f) = v_r / (1+\alpha_p \delta_r).
  \label{eq:alphap}
\end{equation}
This defines a functional relation between $\Delta f/f$ and
$\alpha_p\delta_r$.
Physically the momentum error and the off-momentum orbit are defined
by the RF~frequency difference~$\Delta f/f$,
and the average momentum deviation can be approximately computed from
the relation~(\ref{eq:alphap}).
Note that for $\Delta f/f = 0$ one has $v_R = v_0$.

\clearpage
\subsection{Introduction of Deviations in Time and Momentum}
We note that neither of the longitudinal variables is a small quantity.
We therefore first transform the variables such as to introduce the
time difference between the reference frame and the actual particle 
and the relative momentum error.
The transformation is derived from the generating function
\begin{equation}
  F_1 = -x p_{x1}-y p_{y1} - (\bar{t}+sc/v_R) \sqrt{(p_r+p_1)^2+(mc)^2}.
\end{equation}
The canonical variables for the transverse motion do not change.
The longitudinal variables are transformed according to
\begin{equation}\begin{array}{lclcl}
  t_1&=&-\frac{\partial F_1}{\partial p_1}
     &=&(\bar{t}+sc/v_R) (p_r+p_1)\Bigm/\sqrt{(p_r+p_1)^2+(mc)^2}, \\
  \bar{E}&=&-\frac{\partial F_1}{\partial \bar{t}}
         &=&\sqrt{(p_r+p_1)^2+(mc)^2}.
\end{array}\end{equation}
The new variables $t_1, p_1$ can be rewritten in terms of the original
variables $-t, \bar{E}$:
\begin{equation}
  \begin{array}{lclcl}
    t_1&=&(-t+s/v_R) \cdot c(p_r+p_1)\Bigm/\sqrt{(p_r+p_1)^2+(mc)^2}
    &=&(-t+s/v_R) \cdot v, \\
    p_1&=&\sqrt{\bar{E}^2-(mc)^2}-p_r
    &=&p-p_r.
  \end{array}
\end{equation}
The new time variable is the negative time deviation from the
reference frame, multiplied by the actual velocity~$v$ of the
particle,
and the new energy variable is the deviation of the momentum from the 
reference momentum~$p_r$.
The transformed hamilitonian expressed in the new variables is
\begin{equation}
  \begin{array}{lcl}
    H_1&=&H - \frac{\partial F_1}{\partial s}\\
    &=&- (1+h x_1)
    \left(
      e A_s +
      \sqrt{(p_r+p_1)^2-(p_{x1}-eA_x)^2-(p_{y1}-eA_y)^2}
    \right)\\
    &&+ \sqrt{(p_r+p_1)^2+(mc)^2} \cdot c/v_R.
  \end{array}
  \label{eq:ham}
\end{equation}
The last term expresses the motion of the reference frame.


\clearpage
\subsection{Scaling the Momentum Variables}
The equations of motion do not change if the hamilitonian and all 
canonical momenta are scaled with the constant momentum~$p_r$
as follows:
\begin{equation}
  \begin{array}{lcllcllcl}
    &&&H_2&=&H_1/p_r, \\
    p_{x2}&=&p_{x1}/p_r, \qquad &
    p_{y2}&=&p_{y1}/p_r, \qquad &
    p_2   &=&p_1/p_r, \\
    x_2&=&x_1, \qquad &
    y_2&=&y_1, \qquad &
    t_2&=&t_1, \qquad
  \end{array}
\end{equation}
The transverse momenta are now to first order equal to
the transverse angles,
and $p_2$ is the relative momentum error with respect to $p_r$.
The final hamilitonian is after renaming~$p_2$ to~$p_t$ and dropping
the index~2: 
\begin{equation}
  \begin{array}{lcl}
    H&=&- (1+h x)
    \left(
      e A_s / p_r +
      \sqrt{(1+p_t)^2-(p_x-e A_x/p_r)^2-(p_y-e A_y/p_r)^2}
    \right)+\\
    &&+(c/v_R) \sqrt{(1+p_t)^2+(mc/p_r)^2}.
  \end{array}
\end{equation}
In this form all canonical variables are small deviations from a
reference value and the hamilitonian can easily be expanded as a Taylor
series.
This can be done automatically by use of a 
\textit{Truncated Power Series Algebra Package},
and no further analytical expansion is required.
For vector potential independent of the time the exact equations of
motion can be written as: 
\begin{equation}
  \begin{array}{lcrcl}
    \frac{dx}{ds}&=&\frac{\partial H}{\partial p_x}&=&
    (1+hx)(p_x-e A_x/p_r) / p_z, \\
    \frac{dy}{ds}&=&\frac{\partial H}{\partial p_y}&=&
    (1+hx)(p_y-e A_y/p_r) / p_z, \\
    \frac{dt}{ds}&=&\frac{\partial H}{\partial p_t}&=&
    -(1+hx)(1+p_t) / p_z + v / v_R, \\ 
    \frac{p_x}{ds}&=&-\frac{\partial H}{\partial x}&=&(e/p_r)\left(
      \frac{\partial}{\partial x}((1+hx)A_s) +
      \frac{dx}{ds}\cdot\frac{\partial A_x}{\partial x} +
      \frac{dy}{ds}\cdot\frac{\partial A_y}{\partial x}
    \right) + h p_z, \\
    \frac{dp_y}{ds}&=&-\frac{\partial H}{\partial y}&=&(e/p_r)\left(
      \frac{\partial}{\partial y}((1+hx)A_s) +
      \frac{dx}{ds}\cdot\frac{\partial A_x}{\partial y} +
      \frac{dy}{ds}\cdot\frac{\partial A_y}{\partial y}\right), \\
    \frac{dp_t}{ds}&=&0.
  \end{array}
\end{equation}
where the relative longitudinal momentum
\begin{equation}
  p_z = \sqrt{(1+p_t)^2-(p_x-e A_x/p_r)^2-(p_y-e A_y/p_r)^2},
  \label{eq:plong}
\end{equation}
and the actual particle velocity
\begin{equation}
  v = c (1+p_t) \Bigm/ \sqrt{(1+p_t)^2+(mc/p_r)^2}.
  \label{eq:speed}
\end{equation}
have been introduced.
The latter is constant since $p_t$ does not change.
These equations model the dependency of the motion upon~$p_r$ exactly 
for all problems, even if the hamilitonian is expanded in terms of the
canonical variables.


\clearpage
\section{Potential Expansions}
\label{sec:potential}


\subsection{Expand the Scalar Potential from the Midplane Field}
Assume the magnetic field on the mid-plane is known to be
\begin{equation}
  B_x(x,0,z) = B_z(x,0,z) = 0, \qquad B_y(x,0,z) = B(x,z).
\end{equation}
Its scalar potential can be written immediately~\cite{SLAC-24} as
\begin{equation}
  V = \sum_{n=0}^\infty \frac{(-1)^n y^{2n+1}}{(2n+1)!}
  \left(\frac{\partial^2}{\partial x^2}+
    \frac{\partial^2}{\partial z^2}\right)^n B(x,z),
\end{equation}
and the field components are:
\begin{equation}
  B_x = \sum_{n=0}^\infty \frac{(-1)^n y^{2n+1}}{(2n+1)!}
  \left(\frac{\partial^2}{\partial x^2}+
    \frac{\partial^2}{\partial z^2}\right)^n
  \frac{\partial}{\partial x}B(x,z),
\end{equation}
\begin{equation}
  B_y = \sum_{n=0}^\infty \frac{(-1)^n y^{2n}}{(2n)!}
  \left(\frac{\partial^2}{\partial x^2}+
    \frac{\partial^2}{\partial z^2}\right)^n B(x,z),
\end{equation}
\begin{equation}
  B_z = \sum_{n=0}^\infty \frac{(-1)^n y^{2n+1}}{(2n+1)!}
  \left(\frac{\partial^2}{\partial x^2}+
    \frac{\partial^2}{\partial z^2}\right)^n
  \frac{\partial}{\partial z}B(x,z).
\end{equation}
One can easily verify these formulas by checking $\hbox{grad\ } V$
and the vanishing of $\hbox{curl\ } B$ and $\hbox{div\ } B$.


\clearpage
\subsection{Expand the Scalar Potential from Multipole Coefficients}
The Laplace equation in cylinder coordinates is
\begin{equation}
  \Delta V = \frac{1}{r}\frac{\partial}{\partial r}
  \left(r \frac{\partial V}{\partial r}\right) + 
  \frac{1}{r^2}\frac{\partial^2 V}{\partial \phi^2} +
  \frac{\partial^2 V}{\partial z^2} = 0.
\end{equation}
Putting
\begin{equation}
  V = V_1(\phi) V_2(r,z)
\end{equation}
the equation can be separated as
\begin{equation}
  \frac{1}{V_1} \frac{d^2 V_1}{d \phi^2} = -
  \frac{1}{V_2} \left(
    r^2 \left(
      \frac{\partial^2 V_2}{\partial r^2} +
      \frac{\partial^2 V_2}{\partial z^2}
    \right)
    + r \frac{\partial V_2}{\partial r}
  \right) = -n^2.
\end{equation}
The left-hand side depends only on $\phi$ and the right-hand side only
on $r$~and~$z$,
thus both must be constant.
For a smooth solution they must equal $-n^2$, where $n$~is an integer.
Putting the left-hand side equal to~$-n^2$:
\begin{equation}
  \frac{d^2V_1}{d\phi^2} + n^2 V_1 = 0 \quad \Rightarrow \quad
  V_1 = \exp(\pm j n \phi).
\end{equation}
Putting the right-hand side equal to~$-n^2$ the solution is found by Fourier
transformation along~$z$: 
\begin{equation}
  r^2 \frac{\partial^2 V_2}{\partial r^2} +
  r^2 \frac{\partial^2 V_2}{\partial z^2} +
  r \frac{\partial V_2}{\partial r} - n^2 V_2 = 0 \quad \Rightarrow \quad
  r^2 \frac{\partial^2 \hat{V_2}}{\partial r^2} +
  r \frac{\partial \hat{V_2}}{\partial r} - (r^2 \omega^2 + n^2) \hat{V_2} = 0.
\end{equation}
The transformed equation is a modified Bessel differential equation.
It has a unique solution which depends on $r$ only and remains finite
for~$r=0$, namely a modified Bessel function
\begin{equation}
  \hat{V_2}(r,\omega) = I_n(\omega r) = (- j \omega r / 2)^n
  \sum_{m=0}^\infty \frac{(\omega^2 r^2 / 4)^m}{m! (m+n)!}.
\end{equation}
This solution can be multiplied by an arbitrary function
$\hat{f}(\omega)$ without violating the equation. 
Now make the ansatz
\begin{equation}
  \hat{f}(\omega) = (- j\omega / 2)^{-n} \hat{K}_n(\omega),
\end{equation}
where $\hat{K}_n(\omega)$ is the Fourier transform of an arbitrary
function $K_n(z)$.
The back transformation gives
\begin{equation}
  V_2(r,z) = r^n \sum_{m=1}^\infty \frac{(- r^2 / 4)^m}{m! (m+n)!}
  \frac{d^{2m} K_n(z)}{d z^{2m}}.
\end{equation}
Summing over all $n$, the scalar potential is
\begin{equation}
  V = \sum_{n=-\infty}^\infty \exp(\pm j n \phi) r^n
  \sum_{m=0}^\infty \frac{(- r^2 / 4)^m}{m! (m+n)!}
  \frac{d^{2m} K_n(z)}{d z^{2m}}.
\end{equation}
The $K_n(z)$ can be identified as the multipole coefficients on the
magnet axis.
Indeed for $K_n(z)$ independent of~$z$ one finds the expected form
\begin{equation}
  V = \sum_{n=-\infty}^\infty K_n \exp(\pm j n \phi) \frac{r^n}{n!}.
\end{equation}
For variable $K_n$, the potential is corrected with a factor which
contains even derivatives of~$K_n$ and even powers of~$r$.
The potential can be rewritten in terms of the cartesian coordinates:
\begin{equation}
  V = \sum_{n=-\infty}^\infty (x \pm j y)^n 
  \sum_{m=0}^\infty \frac{(-(x^2+y^2)/4)^m}{m! (m+n)!}
  \frac{d^{2m} K_n(z)}{d z^{2m}}.
\end{equation}


\clearpage
\subsection{Vector Potential for a Sector Dipole}
\begin{figure}[ht]
  \begin{center}
    \setlength{\unitlength}{1pt}
    \begin{picture}(400,215)
                                % axes
      \thinlines
      \put(90,130){\vector(3,1){60}} % s_1 axis
      \put(84,135){\makebox(0,0){$s_1$}}

      \put(250,150){\vector(3,-1){60}} % s_2 axis
      \put(316,135){\makebox(0,0){$s_2$}}

      \put(200,0){\vector(-1,3){70}} % x_1 axis
      \put(118,206){\makebox(0,0){$x_1$}}

      \put(200,0){\vector(1,3){70}} % x_2 axis
      \put(282,206){\makebox(0,0){$x_2$}}

      \put(200,30){\vector(0,1){126}} % rho axis
      \put(210,75){\makebox(0,0){$\rho$}}

      \put(200,158.1){\circle*{3}} % y axis
      \put(210,168){\makebox(0,0){$y$}}

      \bezier{40}(190.5,28.5)(200,31.7)(209.5,28.5) % alpha
      \put(200,20){\makebox(0,0){$\alpha$}}

      \put(200,158.8){\circle*{4}} % r axis
      \put(200,158.8){\vector(0,1){50}}
      \put(210,200){\makebox(0,0){$r$}}

                                % magnet outline
      \thicklines
      \bezier{140}(150,150)(200,167)(250,150) % center line
      \put(238,154){\vector(3,-1){12}}
      \put(242,162){\makebox(0,0){$s$}}
      \put(160,120){\line(-1,3){20}}
      \put(240,120){\line(1,3){20}}
      \bezier{150}(140,180)(200,200)(260,180) % top line
      \bezier{130}(160,120)(200,134)(240,120) % bottom line
    \end{picture}
    \caption{Reference System for a Sector Bending Magnet}
    \label{fig:sbend}
  \end{center}
\end{figure}
For a sector dipole the coordinate system is shown in Fig.~\ref{fig:sbend}.
In the coordinate system~$(\rho,y,s)$ the vector potential can be normalized
so as to have only an $s$-component, $\vec{A} = (0, 0, A_s)$.
The magnetic field then takes the form
\begin{equation}
  B_\rho(\rho,y,s) = - \frac{\partial A_s}{\partial y},
  \qquad
  B_y(\rho,y,s) = \frac{1}{\rho}\frac{\partial (\rho A_s)}{\partial \rho}.
\end{equation}
and Maxwell's equation $\hbox{curl curl\ }\vec{A} = 0$ takes the form 
\begin{equation}
  \frac{\partial^2 A_s}{\partial \rho^2} + 
  \frac{1}{\rho}\frac{\partial A_s}{\partial \rho} -
  \frac{1}{\rho^2} A_s +
  \frac{\partial^2 A_s}{\partial y^2} = 0.
  \label{eq:laplace_cyl}
\end{equation}
We now replace $\rho$ by $\rho_0 (1 + h x)$,
where $h$ is the curvature of the reference orbit, and $\rho_0$ is the
corresponding radius of curvature. 
We also replace $A_s$ by $U = (1 + hx) A_s$.
The field components can then be written as
\begin{equation}
  (1 + hx) B_x = - \frac{\partial U}{\partial y}, \qquad
  (1 + hx) B_y = \frac{\partial U}{\partial x},
\end{equation}
and Eqn.~(\ref{eq:laplace_cyl}) takes the form
\begin{equation}
  \left(
    \frac{\partial^2 U}{\partial x^2} + 
    \frac{\partial^2 U}{\partial y^2} 
  \right) -
  \frac{h}{1 + hx} \frac{\partial U}{\partial x}  = 0.
\end{equation}
Let the magnetic field in the midplane be given by the polynomial
expansions
\begin{equation}
  B_x(x,0,s) = \sum_{n=0}^\infty A_n x^n, \qquad
  B_y(x,0,s) = \sum_{n=0}^\infty B_n x^n.
\end{equation}
and $U$ be developed in a series
\begin{equation}
  U(x,y) = \sum_{m=0}^\infty U_m(x) y^m.
  \label{eq:expand_U}
\end{equation}
It is then immedately seen that
\begin{equation}
  (1+hx)B_x(x,0,s)=-\lim_{y\rightarrow 0}\frac{\partial U}{\partial y} =
  -U_1(x),\qquad
  (1+hx)B_y(x,0,s)=\lim_{y\rightarrow 0}\frac{\partial U}{\partial x} =
  \frac{dU_0(x)}{dx}.
\end{equation}
From this the first two~$U_m(x)$ are deduced as
\begin{equation}\begin{array}{lclcl}
    U_0(x)&=&\int_0^x (1+hx)\sum_{n=0}^\infty B_n x^n dx
    &=&B_0x+\sum_{n=1}^\infty (B_n + n h B_{n-1}) \frac{x^{n+1}}{n+1},
    \\
    U_1(x)&=&-(1+hx)\sum_{n=0}^\infty A_n x^n
    &=& -A_0-\sum_{n=1}^\infty (A_n + n h A_{n-1}) x^n.
  \end{array}
\end{equation}
Substituting Eqn.~(\ref{eq:expand_U}) into the differential equation gives
\begin{equation}
  \sum_{m=0}^\infty \left(
    \frac{d^2 U_m}{dx^2} y^m + m(m-1)U_m y^{m-2}
    -\frac{h}{1+hx} \frac{dU_m}{dx} y^m
  \right) = 0,
\end{equation}
which leads to the recurrence relation
\begin{equation}
  (m+2)(m+1)U_{m+2}(x) =
    -\frac{d^2U_m}{dx^2} + \frac{h}{1+hx}\frac{dU_m}{dx},
  \qquad \hbox{for $m\ge 0.$}
\end{equation}
This relation can easily be solved using differential algebra.
In the case $h=0$ the~$U_k$ terms take the simple form
\begin{equation}
  U_0(x)=\sum_{i=0}^\infty B_n x^{n+1} / (n + 1),
  \qquad
  U_1(x)=-\sum_{i=0}^\infty A_n x^n,
\end{equation}
\begin{equation}
  (m+2)(m+1)U_{m+2}(x)=-\frac{d^2U_m}{dx^2},
  \qquad \hbox{for $m\ge 0.$}
\end{equation}


\clearpage
\section{Equations of Motion and their Solutions}
\label{sec:motion}


\subsection{Field-Free Regions}
For a field-free region (drift space),
using the definitions of~$p_z$ and~$v$ refer to Eqn.~(\ref{eq:plong})
and~(\ref{eq:speed}) respectively, the equations of motion are
\begin{equation}
  \begin{array}{lcllcl}
    x'&=&p_x/p_z,                 & p_x'&=&0, \\
    y'&=&p_y/p_z,                 & p_y'&=&0, \\
    t'&=&v/v_R-(1+p_t)/p_z, \quad & p_t'&=&0.
  \end{array}
\end{equation}
They have the exact solution
\begin{equation}
  \begin{array}{lcllcl}
    x_2&=&x_1+L\cdot p_{x1}/p_z,                   & p_{x2}&=&p_{x1}, \\
    y_2&=&y_1+L\cdot p_{y1}/p_z,                   & p_{y2}&=&p_{y1}, \\
    t_2&=&t_1+L\cdot (v/v_R-(1+p_{t1})/p_z), \quad & p_{t2}&=&p_{t1}.
  \end{array}
\end{equation}
Sometimes a first-order approximation is useful:
\begin{equation}
  \begin{array}{lcllcl}
    x_2&=&x_1+L\cdot p_{x1}, & p_{x2}&=&p_{x1}, \\
    y_2&=&y_1+L\cdot p_{y1}, & p_{y2}&=&p_{y1}, \\
    t_2&=&t_1+L\cdot p_{t1}/(\beta^2\gamma^2), \quad & p_{t2}&=&p_{t1}.
  \end{array}
\end{equation}


\clearpage
\subsection{Orbit Correctors}
In an orbit corrector the field is assumed to be homogeneous.
The hamilitonian is
\begin{equation}
  eA_s/p_r = fx - \sqrt{(1+p_t)^2-p_x^2-p_y^2} +
  (c/v_R) \sqrt{(1+p_t)^2+(mc/p_r)^2}.
\end{equation}
The curvature of the orbit is~$f = B_y / (B \rho)$.
With the definition 
\begin{equation}
  p_z = \sqrt{(1+p_t)^2-p_y^2-p_x^2}
\end{equation}
the equations of motion read
\begin{equation}
  \begin{array}{lcllcl}
    x'&=&p_x / p_z, \qquad & p_x'&=&-f, \\
    y'&=&p_y / p_z, \qquad & p_y'&=&0, \\
    t'&=&- p / p_z+v/v_R, \qquad & p_t'&=&0, \\
  \end{array}
\end{equation}
The momentum equations are trivially solved by
\begin{equation}
  p_{x2} = p_{x1}-fL, \qquad p_{y2} = p_{y1}, \qquad p_{t2} = p_{t1}.
\end{equation}
Substituting $p_x$ for the integration variable in the position
equations, the three non-trivial equations become:
\begin{equation}
  x_2 = x_1+\int_0^L (p_x / p_z) ds =
  x_1-\frac{1}{f} \int_{p_{x1}}^{p_{x2}} (p_x / p_z) dp_x,
\end{equation}
\begin{equation}
  y_2 = y_1+\int_0^L (p_{y1} / p_z) ds =
  y_1-\frac{p_{y1}}{f} \int_{p_{x1}}^{p_{x2}} (1 / p_z) dp_x,
\end{equation}
\begin{equation}
  t_2 = t_1-\int_0^L ((1+p_{t1}) / p_z) ds =
  t_1+\frac{1+p_{t1}}{f} \int_{p_{x1}}^{p_{x2}} (1 / p_z) dp_x.
\end{equation}
With the substitution $p_x = \sqrt{(1+p_t)^2-p_y^2} \sin u$ the
integrals have the solution
\begin{equation}
  x_2 = x_1+\frac{1}{f} (p_{z2}-p_{z1}), \qquad
  y_2 = y_1+p_y a, \qquad
  t_2 = t_1-Lv/v_R+(1+p_t) a,
\end{equation}
where
\begin{equation}
  a =-\frac{1}{f} \left(
    \arcsin\left(p_{x2}/\sqrt{(1+p_{t2})^2-p_{y2}^2}\right) -
    \arcsin\left(p_{x1}/\sqrt{(1+p_{t1})^2-p_{y1}^2}\right)
  \right)
\end{equation}
is the arc length of the projection of the orbit onto the mid-plane.  
The equations for~$x$ and~$a$ can be rewritten to avoid a difference
of two nearly equal quantities in the equation for~$x$,
and to use known quantities in the inverse trigonometrics.
\begin{equation}
  \begin{array}{lcl}
    x_2&=&x_1+\frac{1}{f} (p_{x2}+p_{x1})/(p_{z2}+p_{z1}),\\
    a&=&-\frac{1}{f}\left(\arctan(p_{x2}/p_{z2})-
      \arctan(p_{x1}/p_{z1})\right).
  \end{array}
\end{equation}
Often the deflections are small, and the thin-lens approximation is
sufficient:
\begin{equation}
  \begin{array}{lcllcl}
    x_2&=&x_1+L\cdot p_{x1}, & p_{x2}&=&p_{x1}, \\
    y_2&=&y_1+L\cdot (p_{y1} + fL/2), & p_{y2}&=&p_{y1} + fL, \\
    t_2&=&t_1+L\cdot p_{t1}/(\beta^2\gamma^2), \quad & p_{t2}&=&p_{t1}.
  \end{array}
\end{equation}


\clearpage
\subsection{Rectangular Bends}

\subsubsection{Fringing Field Regions}
Before entering and after exiting the dipole the reference system must
be rotated by half the bend angle $\alpha/2$.
Introducing the longitudinal momentum at entrance
\begin{equation}
  p_{z1} = \sqrt{(1+p_t)^2-p_{x1}^2-p_{y1}^2},
\end{equation}
we first rotate the momentum vector and obtain:
\begin{equation}
  \begin{array}{lcllcl}
    p_{x2}&=&p_{x1}\cos(\frac{\alpha}{2})+p_{z1}\sin(\frac{\alpha}{2}),
    \qquad p_{y2}&=&p_{y1}, \\
    p_{z2}&=&p_{z1}\cos(\frac{\alpha}{2})-p_{x1}\sin(\frac{\alpha}{2}),
    \qquad p_{t2}&=&p_{t1}.
  \end{array}
\end{equation}
Geometric considerations produce the transformed positions:
\begin{equation}
  x_2=x_1 \frac{p_{z1}}{p_{z2}}, \qquad
  y_2=y_1+\frac{p_{y1}}{p_{z2}} x_1\sin(\frac{\alpha}{2}), \qquad
  t_2=t_1-\frac{1+p_{t1}}{p_{z2}} x_1\sin(\frac{\alpha}{2}).
\end{equation}
The first-order approximation of these formulas gives:
\begin{equation}
  \begin{array}{ll}
    x_2=x_1, & p_{x2} = p_{x1} + \frac{\alpha}{2}, \\
    y_2=y_1, & p_{y2} = p_{y1}, \\
    t_2=t_1, & p_{t2} = p_{t1}.
  \end{array}
\end{equation}
Obviously it is only valid for small deflection angles $\alpha$.


\subsubsection{Body of the Dipole}
For constant field in the body of the dipole the solution is identical
to the orbit corrector equations:
\begin{equation}
  \begin{array}{lcllcl}
    x_2&=&x_1+\frac{1}{f} (p_{x2}+p_{x1})/(p_{z2}+p_{z1}), \quad &
    p_{x2}&=&p_{x1}-fL, \\
    y_2&=&y_1+p_{y1} a,            & p_{y2}&=&p_{y1}, \\
    t_2&=&t_1-Lv/v_R+(1+p_{t1}) a, & p_{t2}&=&p_{t1}. \\
  \end{array}
\end{equation}
where
\begin{equation}
  a=-\frac{1}{f}\left(\arctan(p_{x2}/p_{z2})-\arctan(p_{x1}/p_{z1})\right).
\end{equation}
If the magnet body contains multipole coefficients,
refer to section~\ref{sec:multi}.


\clearpage
\subsection{Solenoids}
In a solenoid the vector potential can be written as
\begin{equation}
  A_x(x,y,s) =-\frac{1}{2} B_0 y, \qquad
  A_y(x,y,s) =+\frac{1}{2} B_0 x, \qquad
  A_s(x,y,s) = 0.
\end{equation}
The hamilitonian of Eqn.~(\ref{eq:ham}) takes the form
\begin{equation}
  H = (c/v_R) \sqrt{(1+p_t)^2+(mc/p_r)^2} -
  \sqrt{(1+p_t)^2-(p_x+k y)^2-(p_y-k x)^2}
\end{equation}
where
\begin{equation}
  k = \frac{e B_0}{2 p_r c}.
\end{equation}
After introducing
\begin{equation}
  R = \sqrt{(1+p_t)^2+(mc)^2}, \qquad 
  p_z = \sqrt{(1+p_{t})^2-(p_{x}+ky)^2-(p_{y}-kx)^2}
\end{equation}
the equations of motion read:
\begin{equation}
  \begin{array}{lcllcllcl}
    x'&=&(p_x+k y) / p_z, \qquad &
    y'&=&(p_y-k x) / p_z, \qquad &
    t'&=&(v/v_R)-(1+p_t)/p_z, \\
    p_x'&=&  k (p_y-k x) / p_z, \qquad &
    p_y'&=&- k (p_x+k y) / p_z, \qquad &
    p_t'&=&0.
  \end{array}
\end{equation}
The hamilitonian~$H$ is a constant of motion, and~$p_t' = 0$.
Therefore $R$~and~$p_z$ are both constant,
and the equations of motion reduce to a linear system in the phase
space coordinates.
Using the definitions
\begin{equation}
  C = cos(kL/p_z), \qquad
  S = sin(kL/p_z),
\end{equation}
the exact motion through the body of the solenoid is
\begin{equation}
  \begin{array}{lcllcl}
    x_2   &=& C (C x_1 + S y_1) + (S/k) (C p_{x1} + S p_{y1}), \quad &
    p_{x2}&=& C (C p_{x1} + S p_{y1}) - k S (C x_1 + S y_1), \\
    y_2   &=& C (C y_1 - S y_1) + (S/k) (C p_{y1} - S p_{x1}), \quad &
    p_{y2}&=& C (C p_{y1} - S p_{x1}) - k S (C y_1 - S x_1), \\
    t_2   &=&t_1 + L\left(v/v_R - (1 + p_t) / p_z\right), \quad &
    p_{t2}&=&p_{t1}.
  \end{array}
  \label{eq:solo}
\end{equation}
This can easily be verified by substitution.
Note that the coefficients~$C$ and~$S$ depend on the initial
conditions. A first-order solution for the solenoid is easily found by
putting $p_z = 1$ in the equations of motion.
The coefficients $C$ and $S$ then become constants:
\begin{equation}
  C = cos(kL), \qquad S = sin(kL),
\end{equation}
and the equations used in TRANSPORT emerge:
\begin{equation}
  \begin{array}{lcllcl}
    x_2   &=& C (C x_1 + S y_1) + (S/k) (C p_{x1} + S p_{y1}), \quad &
    p_{x2}&=& C (C p_{x1} + S p_{y1}) - k S (C x_1 + S y_1), \\
    y_2   &=& C (C y_1 - S y_1) + (S/k) (C p_{y1} - S p_{x1}), \quad &
    p_{y2}&=& C (C p_{y1} - S p_{x1}) - k S (C y_1 - S x_1), \\
    t_2   &=&t_1 + L p_{t1} / (\beta^2 \gamma^2), \quad &
    p_{t2}&=&p_{t1}.
  \end{array}
\end{equation}

At the solenoid entrance there are two more transformations:
First, the particles traverse the radial component of the fringing
field (see Figure~\ref{fig:solo}).
\begin{figure}[ht]
  \centering
  \begin{picture}(400,100)
    \thinlines
    \put(0,50){\vector(1,0){400}}
    \put(410,50){\makebox(10,10){s}}
                                %
    \put(100,25){\vector(0,1){13}}
    \put(100,38){\vector(1,0){100}}
    \put(200,38){\line(1,0){100}}
    \put(300,38){\vector(0,-1){13}}
                                %
    \put(100,75){\vector(0,-1){13}}
    \put(100,62){\vector(1,0){100}}
    \put(200,62){\line(1,0){100}}
    \put(300,62){\vector(0,1){13}}
                                %
    \put(200,50){\vector(0,1){12}}
    \put(205,50){\makebox(10,10)[l]{$r$}}
                                %
    \thicklines
    \put(100,25){\line(0,-1){25}}
    \put(300,25){\line(0,-1){25}}
    \put(100,25){\line(1,0){200}}
                                %
    \put(100,75){\line(0,1){25}}
    \put(300,75){\line(0,1){25}}
    \put(100,75){\line(1,0){200}}
  \end{picture}
  \caption{Approximate Field Picture for a Solenoid}
  \label{fig:solo}
\end{figure}
As an approximation assume that the field lines bend sharply at the
entrance to continue in radial direction within a thin disk.
Within a tube of radius~$r$ the flux $\Phi(r) = \pi B_0 r^2$ leaves
the solenoid.
This flux distributes over the circumference~$2 \pi r$.
The integrated effect through the disk causes an azimuthal deflection 
\begin{equation}
  p_x \rightarrow p_x+k y, \qquad p_y \rightarrow p_y-k x.
\end{equation}
Second, the canonical momenta~$p_x$ and~$p_y$ differ from the
mechanical momenta by~$-ky$ and~$+kx$ respectively.
It turns out that this exactly compensates the deflection in the
radial flux.
In a ``hard-edge'' approximation the composition of these two
transformations cancels.
A similar consideration is valid at the solenoid exit.
The equations~(\ref{eq:solo}) can thus be used as they stand.


\clearpage
\subsection{Linear Maps}
Assume the linearized normalized field strength are known:
\begin{description}
\item[$h_x$:] 
  Constant part of the deflection in $x$.
\item[$h_y$:] 
  Constant part of the deflection in $y$.
\item[$K_x$:]
  Derivative w.r.t. $x$ of the deflection in $x$.
\item[$K_s$:] 
  Derivative w.r.t. $y$ of the deflection in $x$,
  equal to the derivative w.r.t. $x$ of the deflection in $y$.
\item[$K_y$:] 
  Derivative w.r.t. $y$ of the deflection in $y$.
\end{description}
The linear equation of motion can be written as
\begin{equation}
  x'' = h_x (1 - p_t) + K_x x + K_s y, \qquad
  y'' = h_y (1 - p_t) + K_s x + K_y y.
  \label{eq:linear}
\end{equation}

\subsubsection{Linear Map without Coupling}
For $k_s=0$ the solution is simple.
Both equations~(\ref{eq:linear}) have the form
\begin{equation}
  z'' - K z = h (1 - p_t).
\end{equation}
Following \cite{SLAC-75} define $k = \sqrt{|K|}$ and write the
solution as
\begin{equation}
  \begin{array}{lcrcrcr}
    z_2 &=& c_z z_1 &+& s_z p_{z1} &+& h d_z (1 - p_{t1}), \\
    p_{z2} &=& -K s_z z_1 &+& c_z p_{z1} &+& h s_z (1 - p_{t1}).
  \end{array}
\end{equation}
where
\begin{equation}
  \begin{array}{llll}
    c_z = \cos(kL), & s_z = \sin(kL)/k, &
    d_z = (1 - c_z) / K, & \hbox{if $K > 0$}, \\
    c_z = \cosh(kL), & s_z = \sinh(kL)/k, &
    d_z = (1 - c_z) / K, & \hbox{if $K < 0$}, \\
    c_z = 1, & s_z = L, & d_z = L^2/2, & \hbox{if $K = 0$}, \\
  \end{array}
\end{equation}

\subsubsection{Linear Map with Coupling}
In the coupled case $K_s \ne 0$ a rotation allows to recover an uncoupled
form. Letting 
\begin{equation}
  x = u \cos\alpha + v \sin\alpha, \qquad
  y = v \cos\alpha - u \sin\alpha, \qquad
  \tan 2\alpha = 2 K_s / (K_x - K_y)
\end{equation}
the equations of motion~(\ref{eq:linear}) can be transformed to
\begin{equation}
  \begin{array}{lcl}
  u'' &=& (h_x \cos\alpha - h_y \sin\alpha) (1 - p_t) + K_u u, \\
  v'' &=& (h_y \cos\alpha + h_x \sin\alpha) (1 - p_t) + K_v v.
  \end{array}
\end{equation}
where
\begin{equation}
  \begin{array}{lcl}
    K_u &=& \frac{1}{2}(K_x + K_y) +
    \bigg((K_x - K_y)^2 - 4 K_s^2\bigg)\bigg/\sqrt{(K_x - K_y)^2 + 4 K_s^2}, \\
    K_v &=& \frac{1}{2}(K_x + K_y) -
    \bigg((K_x - K_y)^2 - 4 K_s^2\bigg)\bigg/\sqrt{(K_x - K_y)^2 + 4 K_s^2}.
  \end{array}
\end{equation}
These equations can be solved like Eq.~(\ref{eq:linear}).
Substituting for the original coordinates produces the result
\begin{equation}
  \begin{array}{lcl}
    x_2 &=& \frac{1}{2}
    \bigg((c_u + c_v) x_1 + (c_u - c_v) u_1 \\
      &&\qquad+ (s_u + s_v) p_{x1} + (s_u - s_v) p_{u1} \\
      &&\qquad+ ((d_u + d_v) hx + (d_u - d_v) hu) (1 - p_{t1})\bigg) \\
    p_{x2} &=& \frac{1}{2}
    \bigg(- (K_u s_u + K_v s_v) x_1 - (K_u s_u - K_v s_v) u_1 \\
      &&\qquad+ (c_u + c_v) p_{x1} + (c_u - c_v) p_{u1} \\
      &&\qquad+ ((s_u + s_v) hx + (s_u - s_v) hu) (1 - p_{t1})\bigg) \\
    y_2 &=& \frac{1}{2}
    \bigg((c_u + c_v) y_1 - (c_u - c_v) v_1 \\
      &&\qquad+ (s_u + s_v) p_{y1} - (s_u - s_v) p_{v1} \\
      &&\qquad+ ((d_u + d_v) hy - (d_u - d_v) hv) (1 - p_{t1})\bigg) \\
    p_{y2} &=& \frac{1}{2}
    \bigg(- (K_u s_u + K_v s_v) y_1 + (K_u s_u - K_v s_v) v_1 \\
      &&\qquad+ (c_u + c_v) p_{y1} - (c_u - c_v) p_{v1} \\
      &&\qquad+ ((s_u + s_v) hy - (s_u - s_v) hv) (1 - p_{t1})\bigg) \\
    t_2 &=& t_1 + L p_{t1} / \beta^2\gamma^2;
  \end{array}
\end{equation}
where the following definitions are used:
\begin{equation}
  \begin{array}{lclcllclcl}
    u_1 &=& x_1 \cos 2 \alpha &-& y_1 \sin 2 \alpha \qquad &
    v_1 &=& y_1 \cos 2 \alpha &+& x_1 \sin 2 \alpha \\
    p_{u1} &=& p_{x1} \cos 2 \alpha &-& p_{y1} \sin 2 \alpha \qquad &
    p_{v1} &=& p_{y1} \cos 2 \alpha &+& p_{x1} \sin 2 \alpha \\
    h_u &=& h_x \cos 2 \alpha &-& h_y \sin 2 \alpha \qquad &
    h_v &=& h_y \cos 2 \alpha &+& h_x \sin 2 \alpha \\
  \end{array}
\end{equation}


\clearpage
\subsection{Other Multipole-like Elements}
\label{sec:multi}
For all elements which have no linear terms in the hamilitonian the Lie
series
\begin{equation}
  z_i^{(2)} = \sum_{k=0}^N \frac{(-L \lieop{H})^k}{k!} z_i^{(1)}
\end{equation}
converges very rapidly,
and can easily be evaluated in the framework of differential algebra.
The method can be applied to quadrupoles, sextupoles, octupoles, higher
order multipoles, and to any magnet containing combined multipoles of
different orders. 
It may however cause problems when the hamilitonian contains linear
terms, i.~e. when the map does not conserve the origin.

\subsubsection{Thin-Lens Approximation}
\label{thinlens}
In the thin-lens approximation the effect of a general multipole is found by
sandwiching a thin lens between two drifts whose length is half the length
of the multipole. 
Using the integrated normalized multipole coefficients $K_{n,i}$
(upright) and $K_{s,i}$ (skewed) respectively the effect of the thin
lens can be found recursively:
\begin{enumerate}
\item initialize: set $k_{x,N} = k_{y,N} = 0$,
\item recursion: for $i = N, N-1 \ldots 1$ do
\[
k_{x,i} = x_1 \cdot k_{x,i+1} - y_1 \cdot k_{y,i+1} + K_{n,i}, \qquad
k_{y,i} = x_1 \cdot k_{y,i+1} + y_1 \cdot k_{x,i+1} + K_{s,i}.
\]
\item apply result: set $x_2 = x_1 - k_{x,1}, \quad y_2 = y_1 + k_{y,1}$.
\end{enumerate}

\subsubsection{Linear Approximation with Finite Length}


\clearpage
\begin{thebibliography}{99}

\bibitem{SLAC-24}
  R.~H.~Helm.
  {\sl First and Second Order Beam Optics of a Curved, Inclined Magnetic
    Field Boundary in the Impulse Approximation}
  SLAC-0024, Nov 1963. 20pp. 

\bibitem{SLAC-75}
  Karl~L.~Brown.
  {\sl A First and Second Order Matrix Theory for the Design of Beam
  Transport Systems and Charged Particle Spectrometers}
  SLAC-75, Revision 3, Aug 1972. 117pp. 

\bibitem{COURANT}
  E.~D. Courant and H.~S. Snyder.
  {\sl Theory of the alternating gradient synchrotron}.
  Annals of Physics, 3:1--48, 1958.
  
\bibitem{DRAGT}
  A. Dragt.
  {\sl Lectures on Nonlinear Orbit Dynamics, 1981 Summer School on High
    Energy Particle Accelerators, Fermi National Accelerator Laboratory,
    July 1981}.
  American Institute of Physics, 1982.

\end{thebibliography}

\end{document}
