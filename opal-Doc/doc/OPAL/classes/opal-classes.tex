\documentclass[11pt,a4paper]{report}
%% ada \usepackage{here}
\usepackage{epic}

% ========================================================================

\setlength\hoffset{0pt}
\setlength\oddsidemargin{0pt}
\setlength\evensidemargin{\oddsidemargin}
\setlength\textwidth{160mm}

\setlength\headheight{12pt}
\setlength\headsep{24pt}
\setlength\voffset{-36pt}
\setlength\textheight{9in}
\setlength\footskip{36pt}

\setlength\parindent{0pt}
\setlength\partopsep{0pt}
\setlength\topsep{0pt}
\raggedbottom

\setcounter{secnumdepth}{2}
\setcounter{tocdepth}{2}

\newcommand{\ejectpage}{\clearpage}

% special macros for UML diagrams
% ------------------------------------------------------------------------

\def\dline{\dashline[30]{8}}


\def\updotarrow(#1,#2)(#3,#4){%
  \dashline[30]{8}(#1,#2)(#3,#4)%
  \put(#3,#4){%
    \drawline(-2,-8)(0,0)(2,-8)%
  }%
}


\def\leftdotarrow(#1,#2)(#3,#4){%
  \dashline[30]{8}(#1,#2)(#3,#4)%
  \put(#3,#4){%
    \drawline(8,-2)(0,0)(8,2)%
  }%
}


\def\downdotarrow(#1,#2)(#3,#4){%
  \dashline[30]{8}(#1,#2)(#3,#4)%
  \put(#3,#4){%
    \drawline(-2,8)(0,0)(2,8)%
  }%
}


\def\rightdotarrow(#1,#2)(#3,#4){%
  \dashline[30]{8}(#1,#2)(#3,#4)%
  \put(#3,#4){%
    \drawline(-8,-2)(0,0)(-8,2)%
  }%
}


\def\class(#1,#2)#3{%
  \put(#1,#2){\framebox(80,40){\texttt{#3}}}%
}


\def\bclass(#1,#2)(#3,#4)#5{%
  \put(#1,#2){\framebox(#3,#4){\texttt{#5}}}%
}


\def\template(#1,#2)#3#4{%
  \put(#1,#2){%
    \begin{picture}(90,50)%
      \drawline(80,30)(80,0)(0,0)(0,40)(50,40)%
      \put(0,0){\makebox(80,30){\texttt{#3}}}%
      \dashline[30]{8}(50,30)(90,30)(90,50)(50,50)(50,30)
      \put(50,30){\makebox(40,20){\texttt{#4}}}%
    \end{picture}%
  }%
}


\def\module(#1,#2)#3{%
  \put(#1,#2){%
    \begin{picture}(80,50)%
      \drawline(0,0)(80,0)(80,40)(0,40)(0,0)%
      \drawline(0,40)(0,50)(20,50)(20,40)%
      \put(0,0){\makebox(80,40){\texttt{#3}}}%
    \end{picture}%
  }%
}


\def\smodule(#1,#2)(#3,#4)#5{{%
  \put(#1,#2){%
    \begin{picture}(#3,#4)%
      \drawline(0,0)(#3,0)(#3,#4)(0,#4)(0,0)
      \put(0,#4){%
        \begin{picture}(40,0)
          \put(0,0){\makebox(40,20){\texttt{\tiny #5}}}
          \drawline(0,0)(0,20)(40,20)(40,0)
        \end{picture}
      }%
    \end{picture}%
  }%
}}


\newcounter{temp}
\def\upderive(#1,#2)(#3,#4){%
  \setcounter{temp}{#4}
  \addtocounter{temp}{-12}
  \drawline(#1,#2)(#3,\value{temp})
  \put(#3,#4){%
    \begin{picture}(,)%
      \drawline(0,0)(-6,-12)(6,-12)(0,0)%
    \end{picture}%
  }%
}


\def\leftderive(#1,#2)(#3,#4){%
  \setcounter{temp}{#3}
  \addtocounter{temp}{12}
  \drawline(#1,#2)(\value{temp},#4)
  \put(#3,#4){%
    \begin{picture}(,)%
      \drawline(0,0)(12,-6)(12,6)(0,0)%
    \end{picture}%
  }%
}


\def\downderive(#1,#2)(#3,#4){%
  \setcounter{temp}{#4}
  \addtocounter{temp}{12}
  \drawline(#1,#2)(#3,\value{temp})
  \put(#3,#4){%
    \begin{picture}(,)%
      \drawline(0,0)(6,12)(-6,12)(0,0)%
    \end{picture}%
  }%
}


\def\rightderive(#1,#2)(#3,#4){%
  \setcounter{temp}{#3}
  \addtocounter{temp}{-12}
  \drawline(#1,#2)(\value{temp},#4)
  \put(#3,#4){%
    \begin{picture}(,)%
      \drawline(0,0)(-12,-6)(-12,6)(0,0)%
    \end{picture}%
  }%
}


\def\upcompose(#1,#2)(#3,#4){%
  \setcounter{temp}{#4}
  \addtocounter{temp}{-16}
  \drawline(#1,#2)(#3,\value{temp})
  \put(#3,#4){%
    \begin{picture}(,)%
      \drawline(0,0)(-5,-8)(0,-16)(5,-8)(0,0)%
    \end{picture}%
  }%
}


\def\leftcompose(#1,#2)(#3,#4){%
  \setcounter{temp}{#3}
  \addtocounter{temp}{16}
  \drawline(#1,#2)(\value{temp},#4)
  \put(#3,#4){%
    \begin{picture}(,)%
      \drawline(0,0)(8,-5)(16,0)(8,5)(0,0)%
    \end{picture}%
  }%
}


\def\downcompose(#1,#2)(#3,#4){%
  \setcounter{temp}{#4}
  \addtocounter{temp}{16}
  \drawline(#1,#2)(#3,\value{temp})
  \put(#3,#4){%
    \begin{picture}(,)%
      \drawline(0,0)(-5,+8)(0,+16)(5,+8)(0,0)%
    \end{picture}%
  }%
}


\def\rightcompose(#1,#2)(#3,#4){%
  \setcounter{temp}{#3}
  \addtocounter{temp}{-16}
  \drawline(#1,#2)(\value{temp},#4)
  \put(#3,#4){%
    \begin{picture}(,)%
      \drawline(0,0)(-8,-5)(-16,0)(-8,5)(0,0)%
    \end{picture}%
  }%
}


\newcounter{tempx}
\newcounter{tempy}
\def\note(#1,#2)(#3,#4)#5{%
  \put(#1,#2){%
    \begin{picture}(#3,#4)%
      \setcounter{tempx}{#3}
      \setcounter{tempy}{#4}
      \addtocounter{tempx}{-10}
      \addtocounter{tempy}{-10}
      \drawline(#3,\value{tempy})(#3,0)(0,0)(0,#4)(\value{tempx},#4)%
        (#3,\value{tempy})(\value{tempx},\value{tempy})(\value{tempx},#4)%
    \end{picture}%
  }%
  \put(#1,#2){\makebox(#3,#4){#5}}
}


% ------------------------------------------------------------------------
\begin{document}

\begin{center}
PSI Paul Scherrer Institut\\
\vspace{3cm}
{\huge {\bf OPAL Class Reference Manual}}

\vspace{1cm}
Andreas Adelmann

\vspace{2cm}
{\bf Abstract}
\end{center}

\begin{quote}
  This report describes the classes implemented in the OPAL program
  and their relationship with the CLASSIC classes.
\end{quote}

\clearpage

\setlength{\unitlength}{0.8pt}

\tableofcontents
\listoffigures

% ========================================================================

\chapter{Overview}
The class diagrams below do not show all classes nor all dependencies,
but they do explain the general structure of the program.

%\clearpage
\section{Module Structure}
The  OPAL classes are grouped in several modules.
The principal dependencies between the modules are shown in
Fig.~\ref{fig:modules}.
\begin{figure}[H]
  \begin{center}
    \begin{picture}(580,440)
      \module(300,400){\vbox{Abstract-\\Objects}}
      \module(0,400){Algorithms}
      \module(500,400){Utilities}
      \dline(40,340)(40,370)(540,370)(540,340)
      \module(0,300){Structure}
      \dline(50,240)(50,270)(90,270)(90,370)
      \updotarrow(30,240)(30,300)
      \module(0,200){\vbox{Physics-\\Actions}}
      \dline(140,340)(140,370)
      \module(100,300){\vbox{Value-\\Definitions}}
      \dline(240,340)(240,370)
      \module(200,300){Elements}
      \updotarrow(340,340)(340,400)
      \module(300,300){\vbox{Mad-\\Parser}}
      \dline(440,340)(440,370)
      \module(400,300){\vbox{Basic-\\Actions}}
      \dline(250,140)(250,170)(530,170)(530,140)
      \updotarrow(550,140)(550,200)
      \updotarrow(230,140)(230,300)
      \module(200,100){Lines}
      \updotarrow(340,140)(340,300)
      \module(300,100){Track}
      \dline(440,140)(440,170)
      \module(400,100){Match}
      \module(500,100){Errors}
      \module(500,300){Expressions}
      \updotarrow(550,240)(550,300)
      \dline(530,240)(530,270)(490,270)(490,370)
      \module(500,200){Attributes}
      \updotarrow(230,40)(230,100)
      \dline(250,40)(250,70)(290,70)(290,170)
      \module(200,0){Editor}
    \end{picture}
  \end{center}
  \caption{CLASSIC Modules and their Dependences}
  \label{fig:modules}
\end{figure}

% ========================================================================

\clearpage
\section{Hierarchy of Command Objects}
All OPAL commands are represented by object classes.
At start-up time OPAL creates a ``model'' object of each class, known
as an ``exemplar''.
When a parser reads a command, it clones the named object, reads its
attributes, and executes the clone object by calling its execute()
method.
For details refer to the description of the parsers.
The top level of the ``Object'' hierarchy is shown in Fig.~\ref{fig:objects}.
\begin{figure}[H]
  \begin{center}
    \begin{picture}(580,240)
      \thinlines
      \small
      \class(60,200){Directory}
      \rightdotarrow(140,220)(200,220)
      \put(150,225){\makebox(0,0)[b]{1}}
      \put(190,225){\makebox(0,0)[b]{*}}
      \class(200,200){Object}
      \leftcompose(340,220)(280,220)
      \class(340,200){Attribute}
      \put(290,225){\makebox(0,0)[b]{1}}
      \put(330,225){\makebox(0,0)[b]{m}}
      \drawline(40,140)(40,170)(540,170)(540,140)
      \class(0,100){Action}
      \drawline(140,140)(140,170)
      \class(100,100){Editor}
      \upderive(240,140)(240,200)
      \class(200,100){Element}
      \upderive(240,40)(240,100)
      \class(200,0){\vbox{Beam-\\Sequence}}
      \upcompose(260,70)(260,100)
      \put(265,90){\makebox(0,0)[l]{1}}
      \drawline(260,70)(340,70)(340,40)
      \put(345,50){\makebox(0,0)[l]{1}}
      \class(300,0){\vbox{Element-\\Base}}
      \drawline(340,140)(340,170)
      \class(300,100){Definition}
      \drawline(440,140)(440,170)
      \class(400,100){\vbox{Value-\\Definition}}
      \class(500,100){Table}
    \end{picture}
    \caption{Hierarchy for Command Objects}
    \label{fig:objects}
  \end{center}
\end{figure}

% ========================================================================

\clearpage
\section{Command Attributes}
Each attribute is represented by an object of class ``Attribute'',
arranged in an array contained in the command object.  Each attribute
points to
\begin{enumerate}
\item
  An AttributeBase.
  This contains the attribute value,
  and is normally set to a default value in the exemplar object.
  The clone process for the object creates a new reference to the
  value in the exemplar,
  and any new value overrides this value.
  The values are templates, and may return several data types
  (\texttt{double, string, PlaceRep, RangeRep, vector<>}, etc.).
\item
  An AttributeHandler.
  This contains the name and the help text for the attribute.
  It takes care of parsing and printing the attribute.
  The attribute handlers for the same attribute of all object of a
  given class are shared.
\end{enumerate}
The relationship of attribute objects is shown in
Fig.~\ref{fig:attribute},
the hierarchy for attribute values in Fig.~\ref{fig:values},
and the hierarchy for attribute values in Fig.~\ref{fig:handlers}.

Expressions entered by the user are kept in tree form.
Expression trees are built from the classes in the hierarchy shown in
Fig.~\ref{fig:express}.
The expression classes are templates, and may return several data types
(\texttt{double, string, PlaceRep, RangeRep, vector<>}, etc.).

\begin{figure}[H]
  \begin{center}
    \begin{picture}(180,140)
      \small
      \thinlines
      \class(0,0){\vbox{Attribute-\\Base\\``abstract''}}
      \class(100,0){\vbox{Attribute-\\Handler\\``abstract''}}
      \class(50,100){Attribute}
      \upcompose(90,70)(90,100)
      \drawline(40,40)(40,70)(140,70)(140,40)
    \end{picture}
    \caption{Relationships between Attribute Objects}
    \label{fig:attribute}
  \end{center}
\end{figure}

\clearpage

\begin{figure}[H]
  \begin{center}
    \begin{picture}(580,310)
      \thinlines
      \small
      \template(100,10){SAutomatic}{T}
      \dline(80,30)(100,30)
      \note(0,0)(80,60)%
      {\vbox{automatic\\expression,\\returns \texttt{T}}}
      \upderive(140,50)(140,90)
      \template(100,90){SDeferred}{T}
      \dline(80,110)(100,110)
      \note(0,80)(80,60)%
      {\vbox{deferred\\expression,\\returns \texttt{T}}}
      \upderive(140,130)(140,170)
      \template(100,170){SValue}{T}
      \dline(80,190)(100,190)
      \note(0,160)(80,60){\vbox{value,\\returns \texttt{T}}}
      \template(200,170){SRefAttr}{T}
      \dline(240,130)(240,170)
      \note(200,70)(80,60)%
      {\vbox{reference\\to scalar,\\returns \texttt{T}}}
      \template(300,170){ARefAttr}{T}
      \dline(340,130)(340,170)
      \note(300,70)(80,60)%
      {\vbox{reference\\to vector,\\returns\\\texttt{vector<T>}}}
      \template(400,10){AAutomatic}{T}
      \dline(480,30)(500,30)
      \note(500,0)(80,60)%
      {\vbox{automatic\\expression,\\returns\\\texttt{vector<T>}}}
      \upderive(440,50)(440,90)
      \template(400,90){ADeferred}{T}
      \dline(480,110)(500,110)
      \note(500,80)(80,60)%
      {\vbox{deferred\\expression,\\returns\\\texttt{vector<T>}}}
      \upderive(440,130)(440,170)
      \template(400,170){AValue}{T}
      \dline(480,190)(500,190)
      \note(500,160)(80,60)%
      {\vbox{value,\\returns\\\texttt{vector<T>}}}
      \drawline(140,210)(140,240)(440,240)(440,210)
      \drawline(240,210)(240,240)
      \drawline(340,210)(340,240)
      \upderive(290,240)(290,270)
      \class(250,270){\vbox{Attribute-\\Base}}
    \end{picture}
    \caption{Hierarchy for Attribute Values}
    \label{fig:values}
  \end{center}
\end{figure}

\begin{figure}[H]
  \begin{center}
    \begin{picture}(380,340)
      \thinlines
      \class(150,300){\vbox{Attribute-\\Handler}}
      \upderive(190,70)(190,300)
      \drawline(40,240)(40,270)(340,270)(340,240)
      \drawline(140,240)(140,270)
      \drawline(240,240)(240,270)
      \class(0,200){Place}
      \class(100,200){Range}
      \class(200,200){Reference}
      \class(300,200){TableRow}
      \drawline(40,140)(40,170)(340,170)(340,140)
      \drawline(140,140)(140,170)
      \drawline(240,140)(240,170)
      \class(0,100){Bool}
      \class(100,100){Real}
      \class(200,100){String}
      \class(300,100){TokenList}
      \drawline(40,40)(40,70)(340,70)(340,40)
      \drawline(140,40)(140,70)
      \drawline(240,40)(240,70)
      \class(0,0){\vbox{Bool-\\Array}}
      \class(100,0){\vbox{Real-\\Array}}
      \class(200,0){\vbox{String-\\Array}}
      \class(300,0){\vbox{TokenList-\\Array}}
    \end{picture}
    \caption{Hierarchy for Attribute Handlers}
    \label{fig:handlers}
  \end{center}
\end{figure}

\begin{figure}[H]
  \begin{center}
    \begin{picture}(440,740)(-20,-10)
      \small
      \thinlines
      \dline(-20,-10)(-20,320)(420,320)(420,-10)(-20,-10)
      \template(0,430){SRefExpr}{T}
      \dline(40,400)(40,410)
      \note(0,360)(80,40){\vbox{reference\\to scalar}}
      \drawline(40,470)(40,500)(360,500)(360,470)
      \template(100,430){SCell}{T}
      \dline(140,400)(140,430)
      \note(100,360)(80,40){\vbox{cell from\\a table}}
      \drawline(140,470)(140,500)
      \upderive(200,140)(200,700)
      \template(220,430){SHash}{T}
      \dline(260,400)(260,430)
      \note(220,360)(80,40){\vbox{\texttt{\#} sign in\\\texttt{TABLE()}}}
      \drawline(260,470)(260,500)
      \template(320,430){Indexer}{T}
      \dline(360,400)(360,430)
      \note(320,360)(80,40){\vbox{Index into\\\texttt{vector<T>}}}
      \template(0,600){SConstant}{T}
      \dline(40,570)(40,600)
      \note(0,530)(80,40){\vbox{scalar\\constant}}
      \drawline(40,640)(40,670)(360,670)(360,640)
      \template(100,600){SNull}{T}
      \dline(140,570)(140,600)
      \note(100,530)(80,40){\vbox{expression,\\no scalars}}
      \drawline(140,640)(140,670)
      \template(220,600){SUnary}{T}
      \dline(260,570)(260,600)
      \note(220,530)(80,40){\vbox{expression,\\one scalar}}
      \drawline(260,640)(260,670)
      \template(320,600){SBinary}{T}
      \dline(360,570)(360,600)
      \note(320,530)(80,40){\vbox{expression,\\two scalars}}
      \template(160,700){Expression}{T}

      \dline(-20,350)(-20,680)(420,680)(420,350)(-20,350)
      \template(0,70){ARefExpr}{T}
      \dline(40,40)(40,70)
      \note(0,0)(80,40){\vbox{reference\\to vector}}
      \drawline(40,110)(40,140)(360,140)(360,110)
      \template(100,70){AUnary}{T}
      \dline(140,40)(140,70)
      \note(100,0)(80,40){\vbox{expression\\one vector}}
      \drawline(140,110)(140,140)
      \template(220,70){ABinary}{T}
      \dline(260,40)(260,70)
      \note(220,0)(80,40){\vbox{expression\\two vectors}}
      \drawline(260,110)(260,140)
      \template(320,70){ASUnary}{T}
      \dline(360,40)(360,70)
      \note(320,0)(80,40){\vbox{expression\\vector to\\scalar}}
      \template(0,240){AColumn}{T}
      \dline(40,210)(40,240)
      \note(0,170)(80,40){\vbox{expression\\table column}}
      \drawline(40,280)(40,310)(360,310)(360,280)
      \template(100,240){ARow}{T}
      \dline(140,210)(140,240)
      \note(100,170)(80,40){\vbox{expression,\\table row}}
      \drawline(140,280)(140,310)
      \template(220,240){ATable}{T}
      \dline(260,210)(260,240)
      \note(220,170)(80,40){\vbox{expression,\\\texttt{TABLE()}}}
      \drawline(260,280)(260,310)
      \template(320,240){AList}{T}
      \dline(360,210)(360,240)
      \note(320,170)(80,40){\vbox{expression\\\texttt{\{...\}}}}
    \end{picture}
  \end{center}
  \caption{Hierarchy for scalar (top) and vector (bottom) Expressions}
  \label{fig:express}
\end{figure}

% ========================================================================

\clearpage
\section{Active Elements}
The OPAL element classes serve as an interface to the CLASSIC library.
Each OPAL element contains a pointer to an instance of a corresponding
CLASSIC element.
The CLASSIC element handles the interface to the algorithms,
whilst the OPAL element deals with the attributes in MAD language,
and updates the CLASSIC element when its \texttt{update()} function is
called.
This happens before execution of any executable command,
so as to ensure that the data structure is up to date.

The OPAL element hierarchy is illustrated by two typical examples,
namely the MadDrift and MadBend elements in Fig.~\ref{fig:elements}.
\begin{figure}[H]
  \begin{center}
    \begin{picture}(320,550)
      \thinlines
      \smodule(75,390)(120,140){\vbox{Abstract-\\Objects}}
      \smodule(0,110)(320,260){Elements}
      \smodule(0,0)(320,70){\vbox{Beamline-\\Core}}

      \thicklines
      \class(95,480){Object}
      \upderive(135,450)(135,480)
      \class(95,410){Element}
      \upderive(135,360)(135,410)
      \class(95,320){\vbox{Mad-\\Element}}
      \upderive(135,290)(135,320)
      \drawline(60,260)(60,290)(210,290)(210,260)
      \class(20,220){\vbox{Mad-\\Drift}}
      \put(60,220){\vector(0,-1){170}}
      \class(20,10){DriftRep}

      \class(170,220){\vbox{Mad-\\Bend}}
      \upderive(210,190)(210,220)
      \drawline(160,160)(160,190)(260,190)(260,160)
      \class(120,120){MadRBend}
      \class(220,120){MadSBend}
      \put(160,120){\vector(0,-1){70}}
      \class(120,10){RBendRep}
      \put(260,120){\vector(0,-1){70}}
      \class(220,10){SBendRep}
    \end{picture}
  \end{center}
  \caption{Hierarchy for Drift and Bend Elements}
  \label{fig:elements}
\end{figure}

% ========================================================================

\clearpage
\section{Beam Lines}
The OPAL beam lines are treated like the elements.
Again, a Line or Sequence points to a CLASSIC beam line.
This is shown in Fig.~\ref{fig:lines}. 
This same figure also shows the templates used for lines and sequences with
arguments. 
\begin{figure}[H]
  \begin{center}
    \begin{picture}(460,560)
      \thinlines
      \smodule(10,0)(200,70){Beamlines}
      \smodule(250,0)(200,70){Parser}
      \smodule(10,110)(440,150){Lines}
      \smodule(60,300)(220,240){\vbox{Abstract-\\Objects}}
      \smodule(300,350)(100,70){MadParser}

      \thicklines
      \class(20,10){\vbox{TBeamline\\<...>}}
      \put(60,160){\vector(0,-1){110}}
      \class(20,160){Line}
      \drawline(60,200)(60,280)(160,280)(160,200)
      \class(120,10){\vbox{TBeamline\\<...>}}
      \put(160,160){\vector(0,-1){110}}
      \class(120,160){Sequence}
      \upderive(110,280)(110,320)
      \class(70,320){\vbox{Beam-\\Sequence}}
      \upderive(110,360)(110,390)
      \class(70,390){Element}

      \class(310,10){\vbox{Token-\\Stream}}
      \downderive(350,90)(350,50)
      \drawline(300,130)(300,90)(400,90)(400,130)
      \class(260,130){\vbox{Macro-\\Stream}}
      \upcompose(300,170)(300,210)
      \class(260,210){\vbox{Line-\\Template}}
      \class(360,130){\vbox{Macro-\\Stream}}
      \upcompose(400,170)(400,210)
      \class(360,210){\vbox{Sequence-\\Template}}
      \drawline(300,250)(300,280)(400,280)(400,250)
      \upderive(350,280)(350,370)
      \class(310,370){Macro}

      \drawline(110,430)(110,460)(350,460)(350,410)
      \upderive(230,460)(230,490)
      \class(190,490){Object}
    \end{picture}
  \end{center}
  \caption{Hierarchy for OPAL Beam Lines and Sequences}
  \label{fig:lines}
\end{figure}

% ========================================================================

\clearpage
\section{Line Structure}
The structure of a beam line or sequence follows the layout shown in
Fig.~\ref{fig:structure}.
For non-shared elements, all the objects except the concrete
representations for \textbf{ideal} elemenst (\texttt{XXXRep}) are
unique.
For shared elements, also the wrappers (\texttt{XXXWrapper}) are
unique. 
\begin{figure}[H]
  \begin{center}
    \begin{picture}(380,350)
      \class(150,310){\vbox{Line \textrm{or}\\Sequence}}
      \put(190,310){\vector(0,-1){30}}
      \bclass(130,240)(120,40){\vbox{TBeamline\\<FlaggedElmPtr>}}
      \drawline(190,210)(190,240)
      \drawline(40,210)(340,210)
      \multiput(40,210)(100,0){4}{\vector(0,-1){30}}
      \class(0,140){MarkerRep}
      \class(100,140){\vbox{Align-\\Wrapper}}
      \class(200,140){\vbox{Align-\\Wrapper}}
      \class(300,140){\vbox{Align-\\Wrapper}}
      \multiput(140,140)(100,0){3}{\vector(0,-1){30}}
      \class(100,70){\vbox{Solenoid-\\Rep}}
      \class(200,70){\vbox{Multipole-\\Wrapper}}
      \class(300,70){\vbox{RBend-\\Wrapper}}
      \multiput(240,70)(100,0){2}{\vector(0,-1){30}}
      \class(200,0){\vbox{Multipole-\\Rep}}
      \class(300,0){\vbox{RBend-\\Rep}}
    \end{picture}
  \end{center}
  \caption{Structure of a OPAL Beam Line or Sequence}
  \label{fig:structure}
\end{figure}

% ========================================================================

\clearpage
\section{Algorithms}
The algorithm hierarchy is shown in Fig.~\ref{fig:algorithms}.
An algorithms is constructed with a beam line as a parameter,
and normally applied to that line by its \texttt{execute()} method.
Most algorithms can also be applied to a table by simply looping over
the table and using the \texttt{accept(BeamlineVisitor \&)} method of
each contained \texttt{ElmPtr}.

The integrator hierarchy is shown in Fig.~\ref{fig:integrators}.

\begin{figure}[H]
  \begin{center}
    \begin{picture}(580,410)
      \thinlines
      \class(325,370){\vbox{Beamline-\\Visitor}}
      \upderive(365,340)(365,370)
      \class(325,300){\vbox{Default-\\Visitor}}
      \upderive(365,240)(365,300)
      \drawline(140,240)(140,270)(490,270)(490,240)
      \class(100,200){Flagger}
      \drawline(240,240)(240,270)
      \class(200,200){\vbox{Abstrac-\\Mapper}}
      \upderive(240,140)(240,200)
      \class(200,100){\vbox{Lie-\\Mapper}}
      \drawline(140,140)(140,170)(340,170)(340,140)
      \class(300,100){\vbox{Linear-\\Mapper}}
      \upderive(340,40)(340,100)
      \class(300,0){\vbox{Ideal-\\Mapper}}
      \class(100,100){Mapper}
      \upderive(140,40)(140,100)
      \drawline(40,40)(40,70)(240,70)(240,40)
      \class(0,0){\vbox{Thin-\\Mapper}}
      \class(100,0){\vbox{Thick-\\Mapper}}
      \class(200,0){\vbox{Transport-\\Mapper}}
      \class(325,200){Surveyor}
      \class(450,200){\vbox{Abstract-\\Tracker}}
      \upderive(490,170)(490,200)
      \drawline(440,140)(440,170)(540,170)(540,140)
      \class(400,100){Tracker}
      \upderive(440,40)(440,100)
      \class(400,0){\vbox{Thin-\\Tracker}}
      \class(500,100){\vbox{Orbit-\\Tracker}}
    \end{picture}
  \end{center}
  \caption{Hierarchy for OPAL and CLASSIC Algorithms}
  \label{fig:algorithms}
\end{figure}

\begin{figure}[H]
  \begin{center}
    \begin{picture}(410,110)
      \thinlines
      \class(0,0){Integrator}
      \leftderive(110,20)(80,20)
      \class(110,0){\vbox{Track-\\Integrator}}
      \leftderive(220,20)(190,20)
      \class(220,0){\vbox{Map-\\Integrator}}
      \leftderive(330,20)(300,20)
      \class(330,0){\vbox{MPSplit-\\Integrator}}
      \dline(370,40)(370,70)
      \note(210,70)(200,40)
      {\vbox{replaces current element\\
          by N thin multipoles}}
    \end{picture}
  \end{center}
  \caption{Hierarchy for OPAL and CLASSIC Integrators}
  \label{fig:integrators}
\end{figure}

% ========================================================================

\clearpage
\section{OPAL Parser and Streams}
OPAL uses the class \texttt{MadParser} for parsing in the normal mode.
When entering one of the special modes, it switches to another parser,
derived from \texttt{MadParser}.
The hierarchy for parsers and input streams is shown in
Fig.~\ref{fig:parser}. 

\begin{figure}[H]
  \begin{center}
    \begin{picture}(480,340)
      \thinlines
      \class(0,100){\vbox{Edit-\\Parser}}
      \dline(40,70)(40,100)
      \note(0,30)(80,40){edit mode}
      \class(100,100){\vbox{Error-\\Parser}}
      \dline(140,70)(140,100)
      \note(100,30)(80,40){error mode}
      \class(200,100){\vbox{Match-\\Parser}}
      \dline(240,70)(240,100)
      \note(200,30)(80,40){match mode}
      \class(300,100){\vbox{Sequence-\\Parser}}
      \dline(340,70)(340,100)
      \note(300,30)(80,40){\vbox{sequence\\parsing}}
      \class(400,100){\vbox{Track-\\Parser}}
      \dline(440,70)(440,100)
      \note(400,30)(80,40){track mode}
      \drawline(40,140)(40,170)(440,170)(440,140)
      \drawline(140,140)(140,170)
      \drawline(340,140)(340,170)
      \upderive(240,140)(240,200)
      \class(200,200){\vbox{Mad-\\Parser}}
      \upderive(240,240)(240,300)
      \drawline(140,240)(140,270)(240,270)
      \class(100,200){\vbox{Classic-\\Parser}}
      \class(200,300){Parser}
    \end{picture}

    \begin{picture}(380,330)
      \thinlines
      \class(0,100){FileStream}
      \dline(40,70)(40,100)
      \note(0,0)(80,70){\vbox{reading\\from a\\file}}
      \class(100,100){\vbox{Terminal-\\Stream}}
      \dline(140,70)(140,100)
      \note(100,0)(80,70){\vbox{reading\\from the\\terminal}}
      \drawline(40,140)(40,170)(140,170)(140,140)
      \upderive(90,170)(90,200)
      \class(50,200){\vbox{Abstract-\\FileStream}}
      \drawline(90,240)(90,270)(340,270)(340,240)
      \upderive(240,240)(240,300)
      \class(200,300){TokenStream}
      \class(200,200){\vbox{Macro-\\Stream}}
      \dline(240,170)(240,200)
      \note(200,100)(80,70){\vbox{reading\\from a\\macro}}
      \class(300,200){\vbox{String-\\Stream}}
      \dline(340,170)(340,200)
      \note(300,100)(80,70){\vbox{reading\\from a\\string}}
    \end{picture}
  \end{center}
  \caption{Hierarchy for Parsers (top) and Streams (bottom)}
  \label{fig:parser}
\end{figure}


% ========================================================================

% include the file created by docker
\chapter{Detailed Class Descriptions}
The detailed class description can be found at the AMAS documentation web site.

% ========================================================================

\clearpage
\begin{thebibliography}{99}

\bibitem{Gamma}
  E.~Gamma, R.~Helm, R.~Johnson, and J.~Vlissides,
  \textit{Design Patterns, Elements of Reusable Object-Oriented
    Software},
  Addison-Wesley, 1995.

\end{thebibliography}

% ========================================================================

\end{document}
\bye
