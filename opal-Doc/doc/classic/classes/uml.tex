% special macros for UML diagrams
% ------------------------------------------------------------------------

\def\dline{\dashline[30]{8}}


\def\updotarrow(#1,#2)(#3,#4){%
  \dashline[30]{8}(#1,#2)(#3,#4)%
  \put(#3,#4){%
    \drawline(-2,-8)(0,0)(2,-8)%
  }%
}


\def\leftdotarrow(#1,#2)(#3,#4){%
  \dashline[30]{8}(#1,#2)(#3,#4)%
  \put(#3,#4){%
    \drawline(8,-2)(0,0)(8,2)%
  }%
}


\def\downdotarrow(#1,#2)(#3,#4){%
  \dashline[30]{8}(#1,#2)(#3,#4)%
  \put(#3,#4){%
    \drawline(-2,8)(0,0)(2,8)%
  }%
}


\def\rightdotarrow(#1,#2)(#3,#4){%
  \dashline[30]{8}(#1,#2)(#3,#4)%
  \put(#3,#4){%
    \drawline(-8,-2)(0,0)(-8,2)%
  }%
}


\def\class(#1,#2)#3{%
  \put(#1,#2){\framebox(80,40){\texttt{#3}}}%
}


\def\bclass(#1,#2)(#3,#4)#5{%
  \put(#1,#2){\framebox(#3,#4){\texttt{#5}}}%
}


\def\template(#1,#2)#3#4{%
  \put(#1,#2){%
    \begin{picture}(90,50)%
      \drawline(80,30)(80,0)(0,0)(0,40)(50,40)%
      \put(0,0){\makebox(80,30){\texttt{#3}}}%
      \dashline[30]{8}(50,30)(90,30)(90,50)(50,50)(50,30)
      \put(50,30){\makebox(40,20){\texttt{#4}}}%
    \end{picture}%
  }%
}


\def\module(#1,#2)#3{%
  \put(#1,#2){%
    \begin{picture}(80,50)%
      \drawline(0,0)(80,0)(80,40)(0,40)(0,0)%
      \drawline(0,40)(0,50)(20,50)(20,40)%
      \put(0,0){\makebox(80,40){\texttt{#3}}}%
    \end{picture}%
  }%
}


\def\smodule(#1,#2)(#3,#4)#5{{%
  \put(#1,#2){%
    \begin{picture}(#3,#4)%
      \drawline(0,0)(#3,0)(#3,#4)(0,#4)(0,0)
      \put(0,#4){%
        \begin{picture}(40,0)
          \put(0,0){\makebox(40,20){\texttt{\tiny #5}}}
          \drawline(0,0)(0,20)(40,20)(40,0)
        \end{picture}
      }%
    \end{picture}%
  }%
}}


\newcounter{temp}
\def\upderive(#1,#2)(#3,#4){%
  \setcounter{temp}{#4}
  \addtocounter{temp}{-12}
  \drawline(#1,#2)(#3,\value{temp})
  \put(#3,#4){%
    \begin{picture}(,)%
      \drawline(0,0)(-6,-12)(6,-12)(0,0)%
    \end{picture}%
  }%
}


\def\leftderive(#1,#2)(#3,#4){%
  \setcounter{temp}{#3}
  \addtocounter{temp}{12}
  \drawline(#1,#2)(\value{temp},#4)
  \put(#3,#4){%
    \begin{picture}(,)%
      \drawline(0,0)(12,-6)(12,6)(0,0)%
    \end{picture}%
  }%
}


\def\downderive(#1,#2)(#3,#4){%
  \setcounter{temp}{#4}
  \addtocounter{temp}{12}
  \drawline(#1,#2)(#3,\value{temp})
  \put(#3,#4){%
    \begin{picture}(,)%
      \drawline(0,0)(6,12)(-6,12)(0,0)%
    \end{picture}%
  }%
}


\def\rightderive(#1,#2)(#3,#4){%
  \setcounter{temp}{#3}
  \addtocounter{temp}{-12}
  \drawline(#1,#2)(\value{temp},#4)
  \put(#3,#4){%
    \begin{picture}(,)%
      \drawline(0,0)(-12,-6)(-12,6)(0,0)%
    \end{picture}%
  }%
}


\def\upcompose(#1,#2)(#3,#4){%
  \setcounter{temp}{#4}
  \addtocounter{temp}{-16}
  \drawline(#1,#2)(#3,\value{temp})
  \put(#3,#4){%
    \begin{picture}(,)%
      \drawline(0,0)(-5,-8)(0,-16)(5,-8)(0,0)%
    \end{picture}%
  }%
}


\def\leftcompose(#1,#2)(#3,#4){%
  \setcounter{temp}{#3}
  \addtocounter{temp}{16}
  \drawline(#1,#2)(\value{temp},#4)
  \put(#3,#4){%
    \begin{picture}(,)%
      \drawline(0,0)(8,-5)(16,0)(8,5)(0,0)%
    \end{picture}%
  }%
}


\def\downcompose(#1,#2)(#3,#4){%
  \setcounter{temp}{#4}
  \addtocounter{temp}{16}
  \drawline(#1,#2)(#3,\value{temp})
  \put(#3,#4){%
    \begin{picture}(,)%
      \drawline(0,0)(-5,+8)(0,+16)(5,+8)(0,0)%
    \end{picture}%
  }%
}


\def\rightcompose(#1,#2)(#3,#4){%
  \setcounter{temp}{#3}
  \addtocounter{temp}{-16}
  \drawline(#1,#2)(\value{temp},#4)
  \put(#3,#4){%
    \begin{picture}(,)%
      \drawline(0,0)(-8,-5)(-16,0)(-8,5)(0,0)%
    \end{picture}%
  }%
}


\newcounter{tempx}
\newcounter{tempy}
\def\note(#1,#2)(#3,#4)#5{%
  \put(#1,#2){%
    \begin{picture}(#3,#4)%
      \setcounter{tempx}{#3}
      \setcounter{tempy}{#4}
      \addtocounter{tempx}{-10}
      \addtocounter{tempy}{-10}
      \drawline(#3,\value{tempy})(#3,0)(0,0)(0,#4)(\value{tempx},#4)%
        (#3,\value{tempy})(\value{tempx},\value{tempy})(\value{tempx},#4)%
    \end{picture}%
  }%
  \put(#1,#2){\makebox(#3,#4){#5}}
}
