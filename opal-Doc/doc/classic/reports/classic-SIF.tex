\documentclass{article}

\begin{document}

\title{
  Standard Input Language \\
  for the CLASSIC Parser
}
\author{
  F. Christoph Iselin (CERN)
}

\maketitle

\section{Motivation}
With the growing size of modern particle accelerators and charged
particle beam lines the input data sets needed to describe these
devices to computer programs become increasingly bulky.
This makes data preparation a laborious and error-prone process.
The situation gets even worse if data are to be prepared for several
computer programs with widely different input formats.

The {\em Standard Input Format\/} (SIF) was designed to tackle these
problems.
It is based on the language defined in~\cite{snowmass},
and it has the following aims:
\begin{itemize}
\item
  The accelerator structure must be described such that data can be
  exchanged easily between all programs conforming to the standard.
\item
  The input must be easily readable by a human reader, and avoid
  unnecessary work like repetition and expansion of symmetry.
\end{itemize}

\section{Definition of the Standard Language Statements}
Normally the information is presented to the computer program by means
of definitions.
A definition introduces or redefines a beam line element, a sequence
of elements, or a parameter,
and must be entered {\em before} it is used in another definition.

Definitions are entered sequentially,
they are separated by semicolons ``;''.
Like in \verb'C++' comments are marked by a double slash ``//'',
and all characters from the ``//'' to the end of line are ignored.
A comment {\em does not} terminate a definition.

Examples for definition formats:
\begin{verbatim}
   definition-1;
   definition-2; definition-3; // comment
   definition-4; // comment
\end{verbatim}

\subsection{Keywords and Labels}
Keywords denote element types and their parameters,
as well as names of elementary functions.
Labels denote beam elements, sequences of beam elements, and
user-defined parameters.
Either consists of a letter followed by a sequence of letters, digits,
and/or underscores.
The case of letters is not significant.
In opposition to original SIF, abbreviations are {\em not} permitted.

\subsection{Element Definitions}
The general format for an element definition is
\begin{verbatim}
   label: type-keyword {, parameter-keyword=value};
\end{verbatim}
Subsequent definitions refer to an element by its label.
The {\em type-keyword} selects the element type (dipole, quadrupole,
drift space, etc.) to be used.
When a label is already defined, a ``derived'' element can be defined
as 
\begin{verbatim}
   label: defined-label {, parameter-keyword=value};
\end{verbatim}
It inherits all unspecified attributes from {\tt defined-label}.
A program is allowed to define additional element types.
An element whose {\em type-keyword} is unknown to a program is
flagged with a message.

When an element is redefined, all existing beam line definitions are
updated to use the {\em new} definition.

The table of elements is open-ended.
The standard elements are shown in Table~\ref{elements}.
\begin{table}[htb]
  \centering
  \caption{Standard beam elements}
  \label{elements}
  \vspace{1em}
  \begin{tabular}{|lp{0.7\textwidth}|}
    \hline
    DRIFT       & Drift Space \\
    RBEND       & Rectangular (parallel-faced) bending magnet
                  with a cartesian reference system \\
    SBEND       & Sector (normal-entry) bending magnet
                  with a curvilinear reference system \\
    QUADRUPOLE  & Quadrupole \\
    SEXTUPOLE   & Sextupole \\
    OCTUPOLE    & Octupole \\
    MULTIPOLE   & General Multipole \\
    SOLENOID    & Solenoid \\
    ELSEPARATOR & Electrostatic Separator \\
    RFCAVITY    & RF Cavity \\
    HKICK       & Horizontal Orbit Corrector Kicker \\
    VKICK       & Vertical Orbit Corrector Kicker \\
    SROT        & Change of reference by rotation on longitudinal axis \\
    ECOL        & Elliptic Collimator \\
    RCOL        & Rectangular Collimator \\
    HMONITOR    & Monitor for horizontal plane \\
    VMONITOR    & Monitor for vertical plane \\
    MONITOR     & Monitor for both planes \\
    MARKER      & Marker (serves to mark a position) \\
                & more to come... \\
    \hline
  \end{tabular}
\end{table}
Element parameters are given with {\em parameter-keywords}.
The standard {\em parameter-keywords} are listed in
Table~\ref{keywords}.
\begin{table}[htb]
  \centering
  \caption{Standard parameter keywords}
  \label{keywords}
  \vspace{1em}
  \begin{tabular}{|lp{0.7\textwidth}|}
    \hline
    L     & Element length \\
    ANGLE & Bending angle or rotation angle \\
    K0    & Dipole strength \\
    K1    & Quadrupole strength \\
    K2    & Sextupole strength \\
    K3    & Octupole strength \\
    K4    & Decapole strength \\
    K5    & Dodecapole strength \\
    KS    & Solenoid strength \\
    E1    & Entrance edge angle for bending magnet \\
    E2    & Exit edge angle for bending magnet \\
    KICK  & Kick angle for orbit correctors \\
    FREQ  & RF frequency \\
    VOLT  & RF voltage \\
    LAG   & RF phase lag \\
    XSIZE & Horizontal half-aperture \\
    YSIZE & Vertical half-aperture  \\
    E     & Electrostatic field \\
    \hline
  \end{tabular}
\end{table}
Parameters accepted for various elements and their default values are
shown in Table~\ref{values}.
\begin{table}[htb]
  \centering
  \caption{Standard element parameters and default values}
  \label{values}
  \vspace{1em}
  \begin{tabular}{|lp{0.7\textwidth}|}
    \hline
    DRIFT    & L=0.0 \\
    RBEND    & L=0.0, ANGLE=0.0, K1=0.0, E1=0.0, E2=0.0, K2=0.0 \\
    SBEND    & L=0.0, ANGLE=0.0, K1=0.0, E1=0.0, E2=0.0, K2=0.0 \\
    QUADRUPO & L=0.0, K1=0.0 \\
    SEXTUPOL & L=0.0, K2=0.0 \\
    OCTUPOLE & L=0.0, K3=0.0 \\
    MULTIPOL & L=0.0, multipole components to be defined \\
    SOLENOID & L=0.0, KS=0.0 \\
    ELSEPARA & L=0.0, E=0.0 \\
    RFCAVITY & L=0.0, FREQ=0.0, VOLT=0.0, LAG=0.0 \\
    HKICK    & L=0.0, KICK=0.0 \\
    VKICK    & L=0.0, KICK=0.0 \\
    SROT     & ANGLE=0.0 \\
    ECOL     & L=0.0, XSIZ=0.0, YSIZ=0.0 \\
    RCOL     & L=0.0, XSIZ=0.0, YSIZ=0.0 \\
    HMONITOR & L=0.0 \\
    VMONITOR & L=0.0 \\
    MONITOR  & L=0.0 \\
    MARKER   & \\
             & more to come... \\
    \hline
  \end{tabular}
\end{table}
If a {\em parameter-keyword} is unknown to a program, the parameter
is ignored with a message.

Examples for beam elements:
\begin{verbatim}
   D1:  DRIFT, L=0.2
   QF:  QUADRUPOLE, L=1.6
   QF7: QF, K2=0.012834
   B:   SBEND, L=5, ANGLE=0.0001
\end{verbatim}

\subsection{Coupling of Element Parameters}
Often certain parameters are defined in terms of other paramters.
This may be readily achieved by allowing expressions for element
parameters involving parameters of other elements and/or globally
known values.
Expressions may contain any of the elementary functions listed in
Table~\ref{functions}.
\begin{table}[htb]
  \centering
  \caption{Standard functions allowed in expressions}
  \label{functions}
  \vspace{1em}
  \begin{tabular}{|lp{0.7\textwidth}|}
    \hline
    SQRT($x$)    & Square root $\sqrt{x}$ \\
    LOG($x$)     & Natural logarithm $\log(x)$ \\
    EXP($x$)     & Exponential $e^x$ \\
    SIN($x$)     & Trigonometric sine $\sin(x)$ \\
    COS($x$)     & Trigonometric cosine $\cos(x)$ \\
    ABS($x$)     & Absolute value $|x|$ \\
    TAN($x$)     & Trigonometric tangent $\tan(x)$ \\
    ASIN($x$)    & Inverse trigonometric sine $\arcsin(x)$ \\
    ACOS($x$)    & Inverse trigonometric cosine $\arccos(x)$ \\
    ATAN($x$)    & Inverse trigonometric tangent $\arctan(x)$ \\
    ATAN2($x,y$) & Inverse trigonometric tangent $\arctan(x/y)$ \\
    MAX($x,y$)   & Maximum of two values $\max(x,y)$ \\
    MIN($x,y$)   & Minimum of two values $\min(x,y)$ \\
    \hline
  \end{tabular}
\end{table}
Global values can be introduced by a statement
\begin{verbatim}
   label = value
\end{verbatim}
where \verb'value' can be an expression involving defined quantities.
All expressions are evaluated {\em immediately} when read;
a future change of an operand {\em does not} affect their results.

Examples for coupled parameters:
\begin{verbatim}
   KF  = 0.0128
   QF: QUAD, L=1.6, K1=KF
   QD: QUAD, L=1.6, K2=-KF

   L1  = 2
   D1: DRIFT, L=L1
   D2: DRIFT, L=10-L1
\end{verbatim}

\subsection{Beam Line Definitions}
Beam lines are described by the construct
\begin{verbatim}
   label: LINE=(member-1, member-2, ... member-n)
\end{verbatim}
Each {\em member-i} may be one of the following:
\begin{itemize}
\item
  A beam element name,
\item
  The name of another beam line,
\item
  A sequence of names, separated by commas and enclose in parentheses.
\item
  One of the above, preceded by
  \begin{itemize}
  \item
    a repetition count and an asterisk to indicate repetition,
  \item
    a minus sign to indicate reflection,
  \item
    a minus sign, a repetition count and an asterisk.
  \end{itemize}
\end{itemize}
Examples for beam lines:
\begin{verbatim}
   CELL:   LINE = (QF,D,B,D,QD,D,B,D)
   PERIOD: LINE = (3*CELL,QF,-3*CELL,D,B,D)
   INSERT: LINE = (QF,D,B,D,QF)
\end{verbatim}

\subsection{Convention for Multipole Signs}
The sign conventions for quadrupoles and higher multipoles and the
numeric factors involved, and the orientation of coordinates are now
the same in different programs.
They may be described as follows:
Let the $B_y$ field on the midplane be
\[
B_y(x) = B_0 + B_1 x + B_2 x^2 + B_3 x^3 + \ldots
\]
Then the field coefficients are
\[\begin{array}{lcl}
  h &=& + B_0/(B\rho), \\
  K_1 &=& -nh^2 = + B_1 / (B\rho), \\
  K_2 &=& +\beta h^3 = + B_3 / (B\rho), \\
  K_k &=& + B_k / (B\rho) \qquad \hbox{general case.}
\end{array}\]
and the horizontal axis points to the left.
A positive bend angle bends towards negative x,
and a positive multipole coefficient always creates a positive field
on the positive x-axis.
\clearpage
\section{Examples}
\begin{verbatim}
   !
   ! DRIFT SPACES:
   !
   L1:   DRIF,L=0.5
   L2:   DRIF,L=2.8
   L3:   DRIF,L=47.5083
   !
   ! BENDING MAGNETS:
   BW:   RBEN,L=23.4,ANGLE=0.753741D-03
   B4:   RBEN,L=23.4,ANGLE=0.753741D-02
   B6:   RBEN,L=35.09,ANGLE=0.11306116D-01
   !
   ! QUADRUPOLES:
   !
   QS1:  QUAD,TYPE=QS,L=5.0,K1=0.51741915D-01
   QS2:  QUAD,TYPE=QS,L=3.0,K1=-0.4574919D-01
   QS3:  QUAD,TYPE=QS,L=2.0,K1=0.37887276D-01
   QF:   QUAD,TYPE=MQ,L=1.6,K1=-0.2268553D-01
   QD:   QUAD,TYPE=MQ,L=1.6,K1=0.22683642D-01
   !
   ! SEXTUPOLES:
   !
   SF1:  SEXT,L=0.4,K2=0.13129
   SD1:  SEXT,L=0.74,K2=0.26328
   !
   ! BEAM LINES:
   !
   CELL: LINE=(L24,B6,L22,SF,L23,QF,L21,B6,L22,SD,L23,QD)
   LOBS: LINE=(L1,QS1,L2,QS2,L3,QS3,L4,QS4)
   RFS:  LINE=(L5,QS5,L5,QS6,L5,2*(QS7,L5,QS8,L5))
   DISS: LINE=(QS11,L25,BW,L2X,QS12,L25,B4,L2X,QS13,L25,B4,L2X,&
               QS14,L25,B4,L31,AS15,L25,B4,L32,SF2,L23,QS16)
   ARC:  LINE=(L21,B6,L22,SD2,L23,QD,15*CELL,L24,B6,L41,QF,&
               L21,B6,L22,SD4,L23,QD,15*CELL,L24,B6,L22,SF3,L23)
   DISL: LINE=(QL16,L34,B4,L2X,QL15,L33,B4,L2X,QL14,L25,B4,L2X,&
               QL13,L25,B4,L2X,QL12,L25,BW,L2X,QL11)
   RFL:  LINE=(2*(L5,QL8,L5,QL7),L5,QL6,L5,QL5,L5)
   LOBL: LINE=(QL4,L14,QL3,L13,1L2,L12,QL1,L11)
   OCT:  LINE=(LOBS,RFS,DISS,ARC,DISL,RFL,LOBL)
\end{verbatim}

\clearpage
\begin{thebibliography}{99}
\bibitem{snowmass}
David C.~Carey and F.~Christoph Iselin,
{\em A Standard Input Language for Particle Beam and Accelerator
  Computer Programs}.
Submitted to the 1984 Summer Study on the Design and Utilisation of
the Superconducting Super Collider, Snowmass, Colorado, June~23 --
July~13, 1984.
\end{thebibliography}

\end{document}
