\documentclass[a4paper,11pt]{article}
\usepackage{amsmath} 
\usepackage{amsfonts}
\usepackage{graphicx}
\usepackage{algorithm,algorithmic}
\usepackage{url}
\usepackage{color}
\usepackage{tikz}
\usepackage{multirow}
\usepackage{verbatim}
\usepackage{hyperref}
\usepackage{float}
\usepackage{geometry}
\usepackage{indentfirst}
\usepackage{amssymb}


\usepackage{listings}
\usepackage{makeidx}

\usepackage{times}
\usepackage{graphicx}
\usepackage[latin1]{inputenc}
\usepackage{longtable}
\usetikzlibrary{shapes,arrows}
\geometry{left=3.17cm,right=3.17cm,top=2.54cm,bottom=2.54cm}
%\begin{comment}
\pagestyle{empty}
%\end{comment}
\begin{document}
%We write this latex/pdf file for better denote the symbols used in M. A. Furman and M. Pivi's secondary emission model. \\

We have implemented the secondary emission model according to M. A. Furman and M. Pivi, Phys. Rev. ST Accel. Beams 5, 124404 (2002), pp 124404-14 in the code OPAL. We benchmark against the  and TxPhysics library for secondary emission and found some discrepancy. We guess, that the problem may be a tiny difference between the model itself and its implementation in the TxPhysics library. When we calculate the energy $E$ of emitted electrons according to appendix D in the mentioned paper, for the case $n=1$, the SEY reads:
\begin{equation} 
\delta_1(E_0) \equiv \delta_e(E_0)+\delta_r(E_0)+P_{1,ts}(E_0) %\ equiv P_1(E_0)
\end{equation}
 which is in agreement with the model from the paper. However, in TxPhysics the SEY is implemented like: 
 \begin{equation} 
 \delta_1(E_0) \equiv \delta_e(E_0)+\delta_r(E_0)+P_{1,ts}(E_0) = \delta_e(E_0)+\delta_r(E_0)+P_1(E_0), 
 \end{equation}
 which implies that $P_{1,ts}(E_0) = P_1(E_0)$ which is in our understanding not in accordance with the model of M. A. Furman and M. Pivi.\\
\end{document}