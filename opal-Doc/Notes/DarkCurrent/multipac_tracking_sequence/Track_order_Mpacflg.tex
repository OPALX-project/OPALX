\documentclass[a4paper,11pt]{article}
\usepackage{amsmath} 
\usepackage{amsfonts}
\usepackage{graphicx}
\usepackage{algorithm,algorithmic}
\usepackage{url}
\usepackage{color}
\usepackage{tikz}
\usepackage{multirow}
\usepackage{verbatim}
\usepackage{hyperref}
\usepackage{float}
\usepackage{geometry}
\usepackage{indentfirst}
\usepackage{amssymb}

%\usepackage{draftwatermark}
\usepackage{listings}
\usepackage{makeidx}

\usepackage{times}
\usepackage{graphicx}
\usepackage[latin1]{inputenc}
\usepackage{longtable}
\usetikzlibrary{shapes,arrows}
\geometry{left=3.17cm,right=3.17cm,top=2.54cm,bottom=2.54cm}
%\begin{comment}
\pagestyle{empty}
%\end{comment}
\begin{document}
\begin{center}
{\large The secondary emission model used in ParallTTracker} \\
Chuan Wang \\
\today\\
\end{center}
\begin{figure}[H]
\begin{center}

% Define block styles
\tikzstyle{decision} = [draw,diamond, fill=blue!20, 
    text width=4.5em, text badly centered, node distance=2.2cm, inner sep=0pt]
\tikzstyle{block} = [draw,rectangle, fill=blue!20, 
    text width=7.5em, text badly centered, inner sep=3pt, rounded corners, minimum height=4em]
\tikzstyle{line} = [draw, -latex']
\tikzstyle{cloud} = [draw, \ellipse,fill=red!20, node distance=2.2cm,
    minimum height=2em]
\tikzstyle{every node}=[font=\small]
\scalebox{0.75}{
\begin{tikzpicture}[node distance = 2.2cm, auto, every node/.style={anchor=base,font=\small}]
    % Place nodes
     \node [block] (first) {Setup track method, multipacting flag(mpacflg), etc.};
     % \node [decision, below of=first, node distance=3.2cm] (d1) {method\\="PARALLEL-T"};
     \node [block, below of=first, node distance=3.2cm] (second) {Setup beam, fieldsolver; initial solver BCond};
     \node [decision, below of=second, node distance=2.5cm] (d2) {mpacflg};
     \node [block, left of=d2, node distance=3.5cm] (n1) {Normal initialization of primary particle, set dT for tracking};
     \node [block, right of=d2, node distance=3.5cm] (m1) {Only calc charge per macroparticle, set dT for tracking};
     \node [block, below of=n1, node distance=2.5cm] (n2) {Set charge for each particle in bunch};
     \node [block, below of=m1, node distance=2.5cm] (m2) {Set qi\_m for empty bunch};
     \node [block, below of=n2, node distance=2.5cm] (n3) {Set coupling const, calc beam parameter and print distribution of primary bunch};
     \node [block, below of=m2, node distance=2.5cm] (m3) {Calc beam parameter for empty bunch};
     \node [block, right of=n3, node distance=3.5cm] (third) {Call constructor of ParalelTTracker};
     \node [block, below of=third, node distance=2.5cm] (forth) {Pass multipacting flag value to ParallelTTracker};
     \node [block, below of=forth, node distance=2.5cm] (fifth) {Call execute method in ParallelTTracker};
    % Draw edges
     \path [line] (first) -- (second);
     \path [line] (second) --  (d2);		
     \draw[->] (d2) -- node [midway,above=2pt] {false} (n1);
     \draw[->] (d2) -- node [midway,above=2pt] {true} (m1);	
     \path [line] (n1) -- (n2);
     \path [line] (m1) -- (m2);
     \path [line] (m2) -- (m3);
     \path [line] (third) -- (forth);
     \path [line] (n2) -- (n3);
     \path [line] (n3) -- (third); 
     \path [line] (m3) -- (third);  
     \path [line] (forth) -- (fifth);
\end{tikzpicture}
}

\end{center}
\caption{  Basic order in TrackRun.\label{fig:T-R}}
\end{figure}

\begin{figure}[H]
\begin{center}

% Define block styles
\tikzstyle{decision} = [draw,diamond, fill=blue!20, 
    text width=4.5em, text badly centered, node distance=2.2cm, inner sep=0pt]
\tikzstyle{block} = [draw,rectangle, fill=blue!20, 
    text width=7.5em, text badly centered, inner sep=3pt, rounded corners, minimum height=4em]
\tikzstyle{line} = [draw, -latex']
\tikzstyle{cloud} = [draw, \ellipse,fill=red!20, node distance=2.2cm,
    minimum height=2em]
\tikzstyle{every node}=[font=\small]
\scalebox{0.85}{
\begin{tikzpicture}[node distance = 2.2cm, auto, every node/.style={anchor=base,font=\small}]
    % Place nodes
     \node [block] (first) {Accept beamline, prepare sections, update RF cavities and do autophase if required.};
      \node [block, below of=first, node distance=3.2cm] (second) {Set total particle number in bunch, starting time and time steps.};
     \node [decision, below of=second, node distance=3.0cm] (d1) {mpacflg};
     \node [block, left of=first, node distance=4.5cm] (n1) {Set emission steps if required and set time step for all particles already in simulation};
     \node [block, below of=n1, node distance=3.5cm] (n2) {Set section and proper local coordinate system for each partcles};
     \node [block, below of=n2, node distance=3.0cm] (n3) {Do binary repartition if required};
     \node [block, below of=n3, node distance=3.0cm] (n4) {Calc beam parameters, update space orientation and activate elements};
     \node [block, below of=d1, node distance=3.3cm] (third) {Find BoundaryGeometry and attached distributions, generate particles according to geometry};
     \node [decision, below of=third, node distance=3.5cm] (d2) {mpacflg};
     \node [block, right of=d2, node distance=4.5cm] (m1) {Set section and proper local coordinate system for each partcles};
     \node [block, below of=m1, node distance=2.5cm] (m2) {Calc beam parameter and activate all elements in simulation};
     \node [block, below of=d2, node distance=2.5cm] (forth) {Set parameters for field emission and secondary emission model};
     \node [block, below of=forth, node distance=2.5cm] (fifth) {Set variables for particle bins};
     \node [block, below of=fifth, node distance=2.5cm] (sixth) {Track in each step};
    % Draw edges
     \path [line] (0,1) -- (first);
     \path [line] (first) -- (second);
     \path [line] (second) -- (d1);
     \draw [->] (d1) -- +(-2.2,0)-- +(-2.2,6.2) -- node [near start,above=2pt] {false} (n1);
     \path [line] (forth) --  (fifth);		
     \draw[->] (d2) -- node [midway,right=2pt] {false} (forth);
     \draw[->] (d2) -- node [midway,above=2pt] {true} (m1);	
      \path [line] (n1) -- (n2);
      \path [line] (m1) -- (m2);
      \path [line] (m2) -- (forth);
      \path [line] (fifth) -- (sixth);
      \path [line] (n2) -- (n3);
      \path [line] (n3) -- (n4);
      \path [line] (n4) --  (third); 
      \path [line] (d1) -- node [midway,right=2pt] {true} (third);  
     \path [line] (third) -- (d2);
    \end{tikzpicture}
}

\end{center}
\caption{  Basic order in ParallelTTracker.\label{fig:T-T}}
\end{figure}

\begin{figure}[H]
\begin{center}
%\usepackage[latin1]{inputenc}
%\usepackage{tikz}
%\usetikzlibrary{shapes,arrows}

%<
%\usepackage{verbatim}
%\usepackage[active,tightpage]{preview}
%\PreviewEnvironment{tikzpicture}
%\setlength\PreviewBorder{5pt}%
%



% Define block styles
\tikzstyle{decision} = [diamond, draw, fill=blue!20, 
    text width=4.5em, text badly centered, node distance=2.2cm, inner sep=0pt]
\tikzstyle{block} = [rectangle, draw, fill=blue!20, 
    text width=7.5em, text badly centered, inner sep=3pt, rounded corners, minimum height=4em]
\tikzstyle{newblock} = [rectangle, draw, fill=red!20, 
    text width=7.5em, text badly centered, inner sep=3pt, rounded corners, minimum height=4em]
\tikzstyle{line} = [draw, -latex']
\tikzstyle{cloud} = [draw, ellipse,fill=red!20, node distance=2.2cm,
    minimum height=2em]
\tikzstyle{every node}=[font=\small]  
 \scalebox{0.85}{ 
\begin{tikzpicture}[node distance = 2.2cm, auto, every node/.style={anchor=base,font=\small}]
    % Place nodes
    \node [block] (inte1) {Integration first half step};
    \node [newblock, above of=inte1, node distance=3.2cm] (additional) {Additional collision test for newly generated secondaries};
    \node [newblock, right of=additional, node distance=6.2cm] (mark1) {Mark the collision particles};
   % \node [cloud, left of=incident] (expert) {expert};
   % \node [cloud, right of=incident] (system) {system};
   % \node [block, below of=inte1, node distance=2.2cm] (boundp) {Do boundp and calculate beam parameters};
    \node [block, below of=inte1, node distance=2.2cm] (multi) {Calculate space charge, bins, external field};
    \node [block, below of=multi, node distance=2.2cm] (delete) {Delete particles};
    \node [block, below of=delete, node distance=2.2cm] (kick) {Kick particles};
   
    \node [newblock, below of=kick, node distance=2.2cm] (collision) {Collision test before integration in second half step};
    \node [block, below of=collision, node distance=3.2cm] (inte2) {Integration second half step};
    \node [block, below of=inte2, node distance=2.2cm] (main) {Do main collision test};
    \node [newblock, right of=main, node distance=6.2cm] (mark3) {Mark the collision particles};
    \node [block, below of=main, node distance=2.8cm] (main1) {Do field emission  and update time to the next time step};
    \node [newblock, right of=collision, node distance=6.2cm] (mark2) {Mark the collision particles};
    \node [newblock, right of=main1, node distance=4.2cm] (secondary) {Call secondary emission model};
    % Draw edges
    \path [line] (0,5) -- (additional);
    \path [line] (inte1) -- (multi);
    \path [line] (multi) --  (delete);		
    \path [line] (delete) -- (kick);
    \path [line] (kick) --  (collision);
    \path [line] (collision) -- node [near start, right=4pt] {not hit} (inte2);
    \path [line] (collision) -- node [near start,above=2pt] {hit} (mark2);	
     \path [line] (inte2) -- (main);
     \path [line] (additional) -- node [near start, right=4pt] {not hit} (inte1);
     \path [line] (main) --  node [near start, right=4pt] {not hit} (main1);
     %\path [line] (main) --  node [near start] {not hit} (inte1);
    \draw[->] (main1) -| +(-2.5,3) |- (additional);
    \path [line] (additional) -- node [near start,above=2pt] {hit} (mark1);
    \path [line] (mark1) -- (mark2);
    %\draw[->] (main) -| node [near start,above=2pt] {hit} (secondary);
    %\path [line] (mark2) |- (secondary);
    \path [line] (secondary) -- (main1);
    %\draw[->] (secondary) -- +(-0,-2) -- +(-8,-2) |- (additional);
    \path [line] (mark3) |- (secondary);
    \path [line] (mark2) -- (mark3);
    \path [line] (main) -- node [near start,above=2pt] {hit} (mark3);
\end{tikzpicture}
}
\end{center}
\caption{  Basic order in ParallTTracker for each time step .\label{fig:se}}
\end{figure}
Here we need two additional collision tests, as is shown in Fig.\ref{fig:se}.\\

The first collision test is only on the newly generated secondaries before we do integration in the first time step. This is because secondary emission happens after the main collision test in the previous time step, and we can not make sure that these newly generated secondaries will remain inside boundary after integration in the first half time step.\\

The second collision test is before the integration in the second half time step and on all the particles except for those already have collisions with boundary in the first collision test.\\

During both of the above mentioned collision tests, we mark the collision particles by set their triangle number: $itsBunch->TriID[i]$ to the ID of collision triangles. We using these informations to do secondary emission after the main collision test.     

\end{document}