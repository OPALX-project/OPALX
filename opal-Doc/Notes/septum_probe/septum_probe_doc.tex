\documentclass{article}
\usepackage{graphicx}% Include figure files
\usepackage{float}
\usepackage{tikz}
\usetikzlibrary{arrows}

\begin{document}

\section{Septum}
\subsection{The input parameters}
The particles hitting on the septum is removed from the bunch. There are five parameters to describe a septum:\\
xstart: The x cooradinate of start point.[mm]\\
xend: The x cooradinate of end point.[mm]\\
ystart: The y cooradinate of start point.[mm]\\
yend: The y cooradinate of end point.[mm]\\
width:The width of the septum.[mm]\\
yend1:Not used since it is treated as a string in the input file which should be a real argument.

example:
eec2:Septum,xstart=4100.0,xend=4300.0,ystart=-1200.0,yend=-150.0,width=0.05;

\begin{tikzpicture}[scale=1.5,axis/.style={very thick, ->, >=stealth'}]
    % Draw axes
    \draw [->,thick] (1,-2.0) -- (1,2.0) node (yaxis) [above] {$y$};
    \draw [->,thick] (-1.2,1.0)  -- (4.0,1.0) node (xaxis) [right] {$x$};
    \draw (3.5,-0.2) --(2.0,-1.5)
          (2.0,-1.5) --(2.1,-1.6)
          (2.1,-1.6) --(3.6,-0.3)
          (3.6,-0.3) --(3.5,-0.2);
    \fill[red] (2.05,-1.55)  circle (1pt) node (start)[below] {(xstart,ystart)};
    \fill[red] (3.55,-0.25)  circle (1pt) node (end)[above] {(xend,yend)};;
    \node (steptum) at (2.55,-0.55)[anchor=mid] {septum};
    \draw [<->,thick,red] (2.75,-0.85)--(2.85,-0.95) node [right] {width}; 
    
   

\end{tikzpicture}


\section{Probe}
\subsection{The input parameters}

There are five parameters to describe a probe:\\
xstart: The x cooradinate of start point.[mm]\\
xend: The x cooradinate of end point.[mm]\\
ystart: The y cooradinate of start point.[mm]\\
yend: The y cooradinate of end point.[mm]\\
width:The width of the probe.[mm] \\
This argument is not used in the current version of probe since
the width is calculated in the code, and the width is equal to the length of one step.\\
yend1:Not used since it is treated as a string in the input file which should be a real argument.

\begin{tikzpicture}[scale=1.5,axis/.style={very thick, ->, >=stealth'}]
    % Draw axes
    \draw [->,thick] (0.5,-2.0) -- (0.5,2.0) node (yaxis) [above] {$y$};
    \draw [->,thick] (-1.2,1.0)  -- (4.0,1.0) node (xaxis) [right] {$x$};
    \draw (1.5,-0.5) --(3.0,-1.5)
          (3.0,-1.5) --(3.1,-1.4)
          (3.1,-1.4) --(1.6,-0.4)
          (1.6,-0.4) --(1.5,-0.5);
    \fill[red] (1.55,-0.45)  circle (1pt) node (start)[above] {(xstart,ystart)};
    \fill[red] (3.05,-1.45)  circle (1pt) node (end)[below] {(xend,yend)};;
    \node (steptum) at (2.55,-0.55)[anchor=mid] {probe};
    \draw [<->,thick,red] (2.25,-1.0)--(2.35,-0.9) node [right] {width};



\end{tikzpicture}


The algorithm:
After defining the edge line of the probe, the width of it is calculated using the following method:\\
1)Calculate the length of one step for each turn.\\
2)Calculate the angle $\alpha$ between particle path and the perpendicular line of the probe. \\
3)The width of the probe is equal to stepsize*$\cos(\alpha)$. \\
4)Record particles inside the width of the probe.\\
5)Using the recording coordinates and momentum of the particle, we trace back to the position of the line of the probe, and get the beam
parameters on this line.

example:

prob1:Probe,xstart=4166.16,xend=4250,ystart=-1226.85,yend=-1241.3;

\end{document}


