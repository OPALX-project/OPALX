\documentclass{article}
\usepackage{graphicx}% Include figure files
\usepackage{float}
\begin{document}



\section{The Implement of Sbend in OPAL-t}
Sbend is a sector bending magnet. To describe a sbend, we use the
parameters: entrance angle $E1$, exit angle $E2$, orbit radius
$\rho$ , gradient $K1$. Fig. \ref{fig:posi} and Fig. \ref{fig:neg}
give the view of Sbend with $B>0$ and $B<0$.
\begin{figure}[H]
\begin{center}
\includegraphics*[width=0.8\textwidth]{posi}
\end{center}
\caption{Sbend with $B>0$ } \label{fig:posi}
\end{figure}

\begin{figure}[H]
\begin{center}
\includegraphics*[width=0.8\textwidth]{neg}
\end{center}
\caption{Sbend with $B<0$ } \label{fig:neg}
\end{figure}

\subsection{The Local Coordinate}
At the entrance of the Sbend, a local coordinate is set as shown in
Fig. \ref{fig:posi} and Fig. \ref{fig:neg}. The point $z=0$ means
the element edge. The region between Line 1 and Line 3 is the
entrance fringe field region, the region between Line 4 and Line 6
is the exit fringe field region and the region between Line 3 and
Line 4 is main field region with field gradient.
\subsection{The Fringe Field}
The fringe field of Sbend is described using three linear equations.
For the
entrance pole face, \\
Line1: $z=zbegin\_entry$,\\
Line2: $z=polynomial\_entry$,\\
Line3: $z=zend\_entry$. \\For the
exit pole face, \\Line4: $z=k_3x+b_3$, \\Line5: $z=k_{34} x+b_{34}$,\\
Line6: $z=k_4x+b_4$.

In most cases, the slope of the exit pole face $k_3=K_{34}=k_4$, and is given in the input file as EXITANGLE. Other parametars are given in the fieldmap file.


If $B>0$,
$k_{34}=\tan(\phi-E1-E2)$,
\\$b_{34}=\rho\sin(\phi-E1)+\rho\sin
E1-k_{34}[\rho\cos(\phi-E1)-\rho\cos E1]$.


If $B<0$, $k_{34}=\tan(\phi-E1-E2)$,\\ $b_{34}=\rho\sin(\phi-E1)+\rho\sin
E1-k_{34}[-\rho\cos(\phi-E1)+\rho\cos E1]$.

\subsection{The Field Map 1DProfile1}
(First line) \\
string: 1DProfile1\\
integer: The number of Enge coefficients for entry fringe field.\\
integer: The number of Enge coefficients for exit fringe field.\\
float: gap heigth [cm]\\
(Second line) \\
float: $zbegin\_entry$ [cm]\\
float: $polynomial\_entry$ [cm]\\
float: $zend\_entry$ [cm]\\
integer: not used here\\
(Third line)\\
float: $b_3$ [cm]\\
float: $b_{34}$ [cm]\\
float: $b_4$ [cm]\\
integer: not used here\\
The following lines are the coefficients of the Enge function.

\subsection{The Field Gradient}
The parameter $K1$ in the description of Sbend is the field gradient
$n$, where n is defined by the equation
\begin{equation}
\label{eq:gradient} B_y(x,0,t)=B_y(0,0,t)(1-nx/\rho+...),
\end{equation}
and
\begin{equation}
\label{eq:gradient2} B_x(x,0,t)=B_y(0,0,t)(-ny/\rho),
\end{equation}


\section{The Result}

Fig. \ref{fig:envelope} shows the comparison of envelope IW2 line between
opal-t, transport and measurements.


\begin{figure}[H]
\begin{center}
\includegraphics*[width=1.0\textwidth]{envelopefit}
\end{center}
\caption{The comparison of envelope between opal-t, transport and measurements. }
\label{fig:envelope}
\end{figure}
\end{document}
