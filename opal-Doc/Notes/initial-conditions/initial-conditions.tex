\documentclass{article}
\usepackage{graphicx}% Include figure files
\begin{document}



\section{The Initial Distribution}
\subsection{The variables}
The parameters used in the initial distribution of OPAL-t are:\\
\textbf{sigmax} the maximum half-width of the beam envelope in the x
plane. [m] \\
\textbf{sigmay} the
maximum half-width of the beam envelope in the y plane. [m]\\
\textbf{sigmat} one-half the longitudinal extent of the bunch of particles. [m]\\
\textbf{pt} the average momentum of the bunch. [eV]\\
\textbf{sigmapx} the maximum half-angular divergence of
the beam envelope in the x plane. [eV]\\
\textbf{sigmapy} the maximum half-angular divergence of the beam
envelope in the y plane. [eV] \\
\textbf{sigmapt} the half-width of the momentum interval. [eV] \\
\textbf{corrx} the
correlation coefficient between x and px. \\
\textbf{corry} the correlation coefficient between y and py. \\
\textbf{t} the longitudinal position of the beam center. [m]\\
\textbf{corrt} the correlation coefficient between t and pt.\\
\textbf{r61} the correlation coefficient between x and pt. \\
\textbf{r62} the correlation coefficient between px and pt. \\
\textbf{r51} the correlation coefficient between x and t. \\
\textbf{r52} the correlation coefficient between px and t. \\

To generate gaussian initial distribution with dispersion, first we
generate the uncorrelated gaussian inputs matrix $R=(R1,...,R_n)$.
The mean of $R_i$ is 0 and the standard deviation squared is 1. Then
we correlate $R$.

The correlation coefficient matrix in x and t phase space is


$ \hspace{1.5cm}x \hspace{0.8cm}  px  \hspace{0.9cm} t\hspace{1.0cm}
pt
\\$

$\sigma= \left[
\begin{array}{cccc}
1    &corrx&r51    &r61\\
corrx&1    &r52    &r62\\
r51  &r52  &1      &corrt\\
r61  &r62  &corrt  &1\\

\end{array}
\right]$

 The Cholesky decomposition of the symmetric
positive-definite matrix $\sigma$ is $\sigma=C^TC$, then the
 correlated distribution is $C^TR$.

\textbf{Note}: This correlation is only done for gaussian distribution now.

\subsection{Unit Conversion}
Convert the unit of momentum from $\beta \gamma$ to
$mrad$.

\begin{equation}
\label{eq:betagamma1}
(\beta\gamma)_z=\frac{P}{m_0c}=\frac{Pc}{m_0c^2}.
\end{equation}
\begin{equation}
\label{eq:betagamma2} P_x[
mrad]=1000\frac{P_x[\beta\gamma]}{(\beta\gamma)_z}
\end{equation}

Convert the unit from eV to $\beta\gamma$.
\begin{equation}
\label{eq:eVtobetagamma}
\sqrt{(\frac{T}{m_0c^2}+1)^2-1}=\beta\gamma.
\end{equation}


\subsection{Example}
If the initial  correlation coefficient matrix

$\sigma= \left[
\begin{array}{cccc}
1      &0.756  &0.023    &0.496\\
0.756  &1      &0.385    &-0.042\\
0.023  &0.385  &1        &-0.834\\
0.496  &-0.042 &-0.834   &1\\

\end{array}
\right]$

The corresponding input file is\\
Distzs3:DISTRIBUTION, DISTRIBUTION=gauss,\\
sigmax= 4.796e-03, sigmapx=231.0585, corrx=0.756,\\
sigmay=  23.821e-03, sigmapy=1.6592e+03, corry=-0.999,\\
t=0.466e-02, sigmat=  0.466e-02, pt=72e6,sigmapt=74.7, corrt=-0.834,\\
r61=0.496,r62=-0.042,r51=0.023,r52=0.385;


\end{document}
